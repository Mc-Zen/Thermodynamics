% !TeX root = Theo_IV.tex
\chapter{Erste Anwendungen des großkanonischen Ensembles}
\section{Mittlere Energie und Teilchenzahl}
Bislang haben wir das chemische Potential $\mu$ pro Mol betrachtet und mit 
$N=\langle N_s \rangle$ die mittlere Molzahl bezeichnet. Jetzt soll $\mu$ das 
chemische Potential pro Teilchen ($\mu/L$) und $N=\langle \nu \rangle$ die 
mittlere Teilchenzahl ($NL$) kennzeichnen. \footnote{$L=6\cdot 10^{23}$ bezeichnet hier die Avogadro-Konstante.} Bezeichnet $s(\nu)$ einen Zustand $s$ 
mit Teilchenzahl $\nu$, so können wir die große Zustandssumme (\ref{eq:grosse_Zustandssumme}) wie folgt schreiben:

\begin{align*}
    Z_G(T,V,\mu)&=\sum_{\nu=0}^{\infty}\sum_{s(\nu)} e^{\beta(\mu \nu -U_s(\nu))}\\
    &=e^{-\beta \Omega(T,V,\mu)}.
\end{align*}
Für die Einzelwahrscheinlichkeiten (\ref{eq:Wahrscheinlichkeit}) ergibt sich analog die äquivalente Formulierung:
\begin{align*}
    P(s(\nu))=\frac{1}{Z_G}e^{\beta(\mu \nu-U_{s(\nu)})}.
\end{align*}

Alle großen S durch kleine s ersetzen!
