% !TeX root = Theo_IV.tex

\chapter{Mikroskopische Erklärung der Entropie}

\section{Quantenmechanische Zustände}

Die Systeme, die in der Thermodynamik und der statistischen Physik betrachtet werden, bestehen i.d.R. aus einer großen Zahl von Teilchen. Aus quantenmechanischer Sicht bedeutet das auch einen hohen Entartungsgrad der Energie. Dieser lässt sich mit der Entropie ins Verhältnis setzen, was in diesem Kapitel erfolgen soll.

Zunächst führen wir eine Bezeichnung für die Energie eines Zustands ein,
\begin{align*}
    U_s(N)
\end{align*}
mit Quantenzahlen $s$ und Teilchenzahlen $N$.

\section{Binäres Modellsystem}

Das Ziel soll es sein, eine Grundlage für die Beschreibung der Entropie auszuarbeiten. Dazu werden mikroskopische Zustände abgezählt, wobei zur Vereinfachung nur binäre Zustände verwendet werden. Als Modell werden $N$ unabhängige Elementarmagnete (Spins) gewählt, die zwei mögliche Zustände, $\uparrow$ (up, $+m$) und $\downarrow$ (down, $-m$) annehmen können.

Aus den $N$ unabhängigen Spins wird dann ein Vielteilchenzustand nach der Art $\uparrow_1\downarrow_2\uparrow_3\uparrow_4\downarrow_5\dots\downarrow_N$ gebildet, bei dem alle Spins nummeriert sind, sodass insgesamt $2^N$ verschiedene Zustände möglich sind.



\paragraph*{Magnetisches Moment}

Bei einer makroskopischen Messung werden nicht die einzelnen Spins, sondern ein magnetisches Gesamtmoment
\begin{align*}
    M=m(N_\uparrow-N_\downarrow)
\end{align*}
gebildet. Dabei bezeichnet $N_\uparrow$ die Anzahl der Spins mit Zustand $\uparrow$, $N_\downarrow$ die Anzahl der Spins mit Zustand $\downarrow$ und $m$ das elementare magnetische Moment. Das Gesamtmoment $M$ hat, wie man sich leicht überzeugen kann, einen Wertebereich von
\begin{align*}
    M\in\{Nm,(N-1)m,(N-4)m,\ldots ,-Nm\},
\end{align*}
kann also $(N+1)$ Werte annehmen. Dabei können aber $2^N$ mikroskopische Zustände angenommen werden. Für ein großes System $N \gg 1$ ist $2^N \gg N+1$, die Anzahl der möglichen makroskopischen Zustände ist damit wesentlich geringer. Folglich gibt es i.d.R. mehrere mikroskopische Zustände, die zum gleichen makroskopischen Zustand führen. Wie viele das sind, wird durch die Entartungsfunktion beschrieben.


\paragraph*{Entartungsfunktion}

Wir wählen ohne Beschränkung der Allgemeinheit $N$ gerade. Wir führen zwei neue Größen ein, mithilfe derer die Verteilung beschrieben werden soll:
\begin{align*}
    \left.
    \begin{aligned}
        N_\uparrow   & \equiv \frac{1}{2} N+n \\
        N_\downarrow & \equiv \frac{1}{2}N-n
    \end{aligned}
    \right\} \leftrightarrow
    \left\{
    \begin{aligned}
        N  & = N_\uparrow + N_\downarrow \\
        2n & = N_\uparrow-N_\downarrow
    \end{aligned}
    \right. \:.
\end{align*}
$N$ ist wie gehabt die Gesamtzahl der Spins und wir nennen $2n$ den \emph{Spinüberschuss}. Der letztere bestimmt allein das makroskopische Gesamtmoment
\begin{align*}
    M=2n\cdot m.
\end{align*}
$n$ liegt im Wertebereich $n\in\{-\frac{1}{2}N,-\frac{1}{2}N+1,\dots,\frac{1}{2}N\}$. Im Falle einer Gleichverteilung, $N_\uparrow=N_\downarrow$ ist $n=0$.

Wir wollen nun untersuchen, wie viele Zustände auf den gleichen Spinüberschuss $n$ führen \textendash{} das entspricht genau der Entartungsfunktion $g(n,N)$. Die exakte Lösung liefert die Kombinatorik:
\begin{align}
    \label{eq:entartungsfunktion_exakt}
    g(n,N) = \frac{N!}{N_\uparrow!\cdot N_\downarrow!} = \frac{N!}{\left(\frac{1}{2}N+n\right)!\cdot \left( \frac{1}{2}N-n \right)!}.
\end{align}

Summiert man über alle Zustände, so findet man für die Normierung 
\begin{align*}
    \sum_{n=-\frac{N}{2}}^{\frac{N}{2}}g(n,N) = 2^N. 
\end{align*}


\paragraph*{Eigenschaften der Entartungsfunktion für große Systeme}

Für die Thermodynamik und statistische Physik sind vor allem Systeme mit $N\gg 1$ relevant. Dann ist im Allgemeinen $|n|\ll N$ und $g\gg 1$. Mithilfe eines Tricks kann die Entartungsfunktion, wie in Gleichung \eqref{eq:entartungsfunktion_exakt} beschrieben genähert werden, was eine wesentlich bessere Handhabung ermöglicht.

Wir nehmen zunächst den natürlichen Logarithmus von \eqref{eq:entartungsfunktion_exakt}
\begin{align*}
    \ln g(n,N) = \ln \left( \frac{N!}{\left(\frac{1}{2}N+n\right)!\cdot \left( \frac{1}{2}N-n \right)!} \right) = \ln N! - \ln\left( \frac{N}{2}+n \right)!- \ln\left( \frac{N}{2}-n \right)!
\end{align*}
und wenden die Stirling-Näherung für große Argumente $N$
\begin{align*}
    \ln (N!) \approx \left( N+\frac{1}{2} \right)\ln N-N+\frac{1}{2}\ln(2\pi )
\end{align*}
an:
\begin{align*}
    \ln g(n,N) & \approx \frac{1}{2}\ln\frac{2}{N\pi}+N\ln 2-\frac{2n^2}{N} \\
    g(n,N)     & \approx \sqrt{\frac{2}{N\pi }}2^N e^{-\frac{2n^2}{N}}.
\end{align*}
Der Vorfaktor entspricht $g(0,N)$, sodass
\begin{align}
    \label{eq:entartungsfunktion_naeherung}
    g(n,N) \approx g(0,N) e^{-\frac{2n^2}{N}}, \quad g(0,N) =  \sqrt{\frac{2}{N\pi }}2^N.
\end{align}
Die Funktion ist gerade eine Gaußverteilung mit der Normierung
\begin{align*}
    \int_{-\infty}^{\infty} g(n,N)\diff n=2^N
\end{align*}
($n/N$ ist aufgrund der großen $N$ praktisch kontinuierlich und eine Integration bis unendlich statt bis $\pm 1/2$ trägt nur einen verschwindend geringen Fehler bei). Außerdem ist die Breite (für einen Abfall auf $g(0,N)/e$) $n_n$ durch $\sqrt{N/2}$ gegeben und die relative Breite durch
\begin{align}
    \label{eq:relative_breite}
    \frac{n_n}{N} = \frac{1}{\sqrt{2N}}.
\end{align}
In \Abbref{fig:DegeneracyFunctionGauss} ist die Entartungsfunktion $g(n,N)$ relativ (über $n/N$) aufgetragen.

\begin{figure}[htbp]
    \centering
    \tfigDegeneracyFunctionGauss
    \caption{Entartungsfunktion $g(n,N)$ über $n/N$ von $-1/2$ bis $+1/2$ aufgetragen. Die relative Halbwertsbreite $n_n/N$ ist bei $g(0,N)/e$ markiert. }
    \label{fig:DegeneracyFunctionGauss}
\end{figure}

Für $N=10^{22}$ ist die relative Halbwertsbreite bereits sehr schmal ($n_n/N\approx 10^{-11}$), sodass sich ein sehr scharfer Peak um $n=0$ (bzw. $M=0$) ausbildet. Insgesamt kann man für große Systeme sagen:
\begin{itemize}
    \item Spins sind gleichverteilt.
    \item Es wird ein eindeutiger Zustand mit $n=0$ bzw. $M=0$ angenommen mit kaum wahrnehmbaren Fluktuationen.
    \item Das System hat wohldefinierte physikalischen Eigenschaften.
\end{itemize}
Dies gilt insbesondere im thermodynamischen Grenzwert $N\rightarrow\infty$.


Die Tatsache, dass sich aus dem Aufaddieren vieler Teilzustände eine Gaußsche Normalverteilung ergibt, ist auch Gegenstand des zentralen Grenzwertsatzes der Statistik:
\begin{formal}
    \textbf{Zentraler Grenzwertsatz:}

    Bei der Summation sehr vieler unabhängiger Zufallsvariablen ergibt sich eine Normalverteilung (Gauß-Funktion).
\end{formal}


\paragraph*{Energie im Magnetfeld}

Ein Einzelspin hat die Energie $U=-mB$, ein System aus $N$ Spins also
\begin{align*}
    U(n)=-MB = -2nmB \equivalence n = -\frac{U}{2mB}. 
\end{align*}
Dabei ist der Abstand zwischen zwei Niveaus äquidistant,
\begin{align*}
    \Delta U = U(n) - U(n+1) = 2mB.
\end{align*}
Aus \eqref{eq:entartungsfunktion_naeherung} folgt, dass die Entartungsfunktion für $U$
\begin{align*}
    g(U,N) = g(0,N) e^{-\frac{U^2}{2m^2B^2N}} \quad \mathrm{mit}\quad g(0,N) = \sqrt{\frac{2}{N\pi }} 2^N
\end{align*}
ist.


\section{Grundannahmen: Thermodynamik und statistische Mechanik}

\paragraph*{Postulate und Bemerkungen}

Wir definieren zunächst einen makroskopischen Gleichgewichtszustand:
\begin{formal}
    Ein makroskopischer Gleichgewichtszustand entspricht vielen mikroskopischen (Quanten-) Zuständen, die alle mit seinen Kenngrößen (z.B. Energie, Teilchenzahl, Volumen, Magnetisierung usw.) verträglich sind. Diese nennen wir auch zugängliche Zustände.
\end{formal}

Bei dem zuvor betrachtetem Spinsystem legt beispielsweise $g(U,N)$ die Zahl der zugänglichen Zustände bei festem $N$ und $U\propto M \propto 2n$ fest. Die letzteren sind dabei die makroskopischen Kenngrößen.

Wie aber sind diese mikroskopischen Zustände selbst besetzt? In der statistischen Mechanik verwendet man als Grundannahme die Ergodenhypothese, die besagt, dass alle diese Zustände gleich wahrscheinlich sind:
\begin{postulate}[Erdogenhypothese]
    Ein abgeschlossenes System wird (in der Messzeit) in jedem ihm zugänglichen mikroskopischen Zustand mit gleicher Wahrscheinlichkeit angetroffen.
\end{postulate}

Es gibt auch nicht-ergodische Systeme, für die zugängliche mikroskopische Zustände in der Messzeit nicht angenommen werden oder ungleiche Besetzungswahrscheinlichkeiten besitzen.
Für ergodische Systeme gilt jedoch, dass das Zeitmittel äquivalent zum Scharmittel ist:
\begin{formal}
    Zeitlich versetzte Messungen der durchlaufenen mikroskopischen Zustände (\emph{zeitliches Nacheinander}) werden in der statischen Mechanik durch ein Ensemble von Kopien des Einzelsystems ersetzt (\emph{räumliches Nebeneinander}).

    Jedes Ensemblemitglied befindet sich genau in einem mikroskopischen Zustand. Für $g$ Zustände gibt es also $g$ Ensemblemitglieder.
\end{formal}


\paragraph*{Wahrscheinlichkeit}

Für ergodische Systeme mit $g$ erreichbaren Zustände ist die Wahrscheinlichkeit eines einzelnen Zustands $s$ sehr leicht zu bestimmen,
\begin{align*}
    P(s) = \frac{1}{g}
\end{align*}
und es gilt die Normierung
\begin{align*}
    \sum_{s=1}^g P(s) = 1.
\end{align*}
Das entspricht genau der Ergodenhypothese.
Der Zeitmittelwert als Ensemblemittelwert der physikalischen Messgröße $X(s)$ ist
\begin{align*}
    \left\langle X\right\rangle = \sum_s P(s)X(s) = \sum_s \frac{X(s)}{g}.
\end{align*}

\paragraph*{Beispiel: Ensemble eines Spinsystems}

Für $N=4$ und einen Spinüberschuss $2n=0$, also auch $U=0$ und $M=0$ ist die Entartungsfunktion durch
\begin{align*}
    g(n=0,N=4) = \frac{4!}{2!\cdot 2!} = 6
\end{align*}
gegeben. Das Ensemble besteht aus den möglichen Mikrozuständen
\begin{align*}
    \left\{ \uparrow\uparrow\downarrow\downarrow,\uparrow\downarrow\uparrow\downarrow,\uparrow\downarrow\downarrow\uparrow,\downarrow\uparrow\uparrow\downarrow,\downarrow\uparrow\downarrow\uparrow,\downarrow\downarrow\uparrow\uparrow \right\}.
\end{align*}



\section{Spinsysteme im thermischen Kontakt}

Wir betrachten nun zwei Spinsysteme mit den Kenngrößen $N_1,U_1\propto n_1$ bzw. $N_2,U_2\propto n_2$, die in thermischen Kontakt gebracht werden, also Energie austauschen können. Das Gesamtsystem ist insgesamt thermisch isoliert, sodass $U=U_1+U_2=\mathrm{const}$.
Wir wollen untersuchen, welche Aufteilung in $U_1$ und $U_2$ die wahrscheinlichste ist und welche Eigenschaften das gekoppelte System aufweist.


\paragraph*{Wahrscheinlichste Konfiguration}

Wir definieren zunächst einige Größen, die das Gesamtsystem beschreiben, nachdem beide Untersysteme in Kontakt gebracht wurden,
\begin{align*}
    N    & = N_1+N_2                                 \\
    n    & = n_1+n_2                                 \\
    U(n) & =U_1(n_1)+U_2(n_2) = -2mB(n_1+n_2)=-2mnB.
\end{align*}
Vorher sind $n_1$ und $n_2$ fest und es gibt $g_1(n_1,N_1)\cdot g_2(n_2,N_2)$ erreichbare Zustände. Nach dem Zusammenbringen ist $n=n_1+n_2$ fest und $n_1$ bzw. $n_2$ sind variabel. Die Anzahl der möglichen Zustände ist allerdings sehr viel größer. Die Entartungsfunktion ist jetzt durch
\begin{align*}
    g(n,N) = \sum_{n_1=-\frac{N_1}{2}}^{\frac{N_1}{2}} g_1(n_1,N_1)g_2(n-n_1,N_2)
\end{align*}
gegeben, während die Wahrscheinlichkeit für einen bestimmten Spinüberschuss $n_1$
\begin{align*}
    P_n(n_1) = \frac{g_1(n_1,N_1)g_2(n-n_1,N_2)}{g(n,N)}
\end{align*}
beträgt. Der wahrscheinlichste Spinüberschuss $n_1$ liegt beim Maximum dieser Funktion,
\begin{align}
    \label{eq:maximum_spinueberschuss_notwendige_bedingung}
    \diff \left(g_1(n_1,N_1)g_2(n_2,N_2)\right) \overset{!}{=} 0. 
\end{align}
Da nun $g_1\cdot g_2$ strikt größer als $0$ ist und ferner $\diff n_1=-\diff n_2$, kann ein kleiner Trick angewandt werden, um \eqref{eq:maximum_spinueberschuss_notwendige_bedingung} zu 
\begin{align*}
    \frac{\diff(g_1\cdot g_2)}{g_1\cdot g_2}=\diff(\ln(g_1\cdot g_2)) \overset{!}{=} 0
\end{align*}
und weiter zu 
\begin{align*}
    \diff(\ln g_1) + \diff(\ln g_2) = \frac{\partial \ln g_1}{\partial n_1} \diff n_1 + \frac{\partial \ln g_2}{\partial n_2} \diff n_2 = \left(\frac{\partial \ln g_1}{\partial n_1} - \frac{\partial \ln g_2}{\partial n_2}\right) \diff n_1  \overset{!}{=} 0
\end{align*}
umzuformulieren. Es ist also 
\begin{align}
    \label{eq:bedingung_wahrscheinlichster_spinueberschuss}
    \boxed{\frac{\partial \ln g_1}{\partial n_1}  = \frac{\partial \ln g_2}{\partial n_2} .}
\end{align}
Setzen wir nun die bereits bestimmten Funktionen $g_i=g_i(0,N_i) e^{-2n_i^2/N_i}$ mit $i\in\{1,2\}$ ein und benennen ferner die wahrscheinlichsten Spinüberschüsse mit $\hat{n}_1$ und $\hat{n}_2$, so erhalten wir 
\begin{align*}
    \frac{\hat{n}_1}{N_1} = \frac{\hat{n}_2}{N_2} . 
\end{align*}
Am Maximum ist der relative Spinüberschuss demzufolge für beide Systeme gleich. Man kann sich überzeugen, dass es sich wirklich um ein Maximum der Funktion $P_n(n_1)$ handelt, denn die zweite Ableitung ist kleiner als $0$:
\begin{align*}
    \frac{\partial ^2\ln(g_1\cdot g_2)}{\partial ^2n_1} = -4\left( \frac{1}{N_1}+\frac{1}{N_2} \right) < 0.
\end{align*}
Mithilfe von $n_i= -U_i/(2mB)$ kann \eqref{eq:bedingung_wahrscheinlichster_spinueberschuss} umgeschrieben werden zu
\begin{align}
    \label{eq:bedingung_wahrscheinlichster_spinueberschuss_U}
    \boxed{\frac{\partial \ln g_1}{\partial U_1}  = \frac{\partial \ln g_2}{\partial U_2} .}. 
\end{align}
Doch wohin führt uns diese Gleichung? In Kapitel \ref{sec:thermisches_gleichgewicht} haben wir gesehen, dass 
\begin{align*}
    \diff S = \frac{\partial S^{(1)}}{\partial U^{(1)}}\diff U^{(1)}+ \frac{\partial S^{(2)}}{\partial U^{(2)}}\diff U^{(2)} \overset{!}{=}0. 
\end{align*}
Wegen $\diff U^{(1)} = -\diff U^{(2)}$ ist also 
\begin{align*}
    \frac{\partial S^{(1)}}{\partial U^{(1)}} = \frac{\partial S^{(2)}}{\partial U^{(2)}}. 
\end{align*}
Diese Gleichung sieht aber \eqref{eq:bedingung_wahrscheinlichster_spinueberschuss_U} sehr ähnlich und es liegt nahe, die Identifizierung $S^{(i)}\propto \ln g_i$ zu machen. 


\paragraph*{Schärfe des Maximums}

An dieser Stelle soll ein kurzer Einschub zu der Schärfe des soeben berechneten Maximums erfolgen. 

Die maximale Zustandszahl wird wie im vorigen Abschnitt bestimmt bei $\hat{n}_1$ bzw. $\hat{n}_2$ erreicht und beträgt 
\begin{align*}
    (g_1g_2)_\mathrm{max} = g_1(\hat{n}_1,N_1)g_2(n-\hat{n}_1,N_1) = g_1(0,N_1)g_2(0,N_2) e^{-\frac{2\hat{n}_1^2}{N_1}-\frac{2\hat{n}_2^2}{N_2}},
\end{align*}
bzw. mit $\hat{n}_1+\hat{n}_2=n$ und $N_1+N_2=N$
\begin{align*}
    (g_1g_2)_\mathrm{max} = g_1(0,N_1)g_2(0,N_2) e^{-\frac{2n^2}{N}}. 
\end{align*}
Die Breite dieser Kurve 