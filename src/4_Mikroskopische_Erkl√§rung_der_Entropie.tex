% !TeX root = Theo_IV.tex

\chapter{Mikroskopische Erklärung der Entropie}
Wir wollen in diesem Kapitel die Bedeutung der Entropie und ihrer Zunahme im mikroskopischen Kontext nachvollziehen und modellisieren.
\section{Quantenmechanische Zustände}

Die Systeme, die in der Thermodynamik und der statistischen Physik betrachtet werden, bestehen i.d.R. aus einer großen Zahl von Teilchen. Aus quantenmechanischer Sicht bedeutet das auch einen hohen Entartungsgrad der Energie. Dieser lässt sich mit der Entropie ins Verhältnis setzen, was in diesem Kapitel erfolgen soll.

Zunächst führen wir eine Bezeichnung für die Energie eines Zustands ein,
\begin{align*}
    U_s(N)
\end{align*}
mit Quantenzahlen $s$ und Teilchenzahlen $N$.

\section{Binäres Modellsystem}

Das Ziel soll es sein, eine Grundlage für die Beschreibung der Entropie auszuarbeiten. Dazu werden mikroskopische Zustände abgezählt, wobei zur Vereinfachung nur binäre Zustände verwendet werden. Als Modell werden $N$ unabhängige Elementarmagnete (Spins) gewählt, die zwei mögliche Zustände, $\uparrow$ (up, $+m$) und $\downarrow$ (down, $-m$) annehmen können.

Aus den $N$ unabhängigen Spins wird dann ein Vielteilchenzustand nach der Art $\uparrow_1\downarrow_2\uparrow_3\uparrow_4\downarrow_5\dots\downarrow_N$ gebildet, bei dem alle Spins nummeriert sind, sodass insgesamt $2^N$ verschiedene Zustände möglich sind.



\paragraph*{Magnetisches Moment}

Bei einer makroskopischen Messung werden nicht die einzelnen Spins, sondern ein magnetisches Gesamtmoment
\begin{align*}
    M=m(N_\uparrow-N_\downarrow)
\end{align*}
gebildet. Dabei bezeichnet $N_\uparrow$ die Anzahl der Spins mit Zustand $\uparrow$, $N_\downarrow$ die Anzahl der Spins mit Zustand $\downarrow$ und $m$ das elementare magnetische Moment. Das Gesamtmoment $M$ hat, wie man sich leicht überzeugen kann, einen Wertebereich von
\begin{align*}
    M\in\{Nm,(N-1)m,(N-4)m,\ldots ,-Nm\},
\end{align*}
kann also $(N+1)$ Werte annehmen. Dabei können aber $2^N$ mikroskopische Zustände angenommen werden. Für ein großes System $N \gg 1$ ist $2^N \gg N+1$, die Anzahl der möglichen makroskopischen Zustände ist damit wesentlich geringer. Folglich gibt es i.d.R. mehrere mikroskopische Zustände, die zum gleichen makroskopischen Zustand führen. Wie viele das sind, wird durch die Entartungsfunktion beschrieben.


\paragraph*{Entartungsfunktion}

Wir wählen ohne Beschränkung der Allgemeinheit $N$ gerade. Wir führen zwei neue Größen ein, mithilfe derer die Verteilung beschrieben werden soll:
\begin{align*}
    \left.
    \begin{aligned}
        N_\uparrow   & \equiv \frac{1}{2} N+n \\
        N_\downarrow & \equiv \frac{1}{2}N-n
    \end{aligned}
    \right\} \leftrightarrow
    \left\{
    \begin{aligned}
        N  & = N_\uparrow + N_\downarrow \\
        2n & = N_\uparrow-N_\downarrow
    \end{aligned}
    \right. \:.
\end{align*}
$N$ ist wie gehabt die Gesamtzahl der Spins und wir nennen $2n$ den \emph{Spinüberschuss}. Der letztere bestimmt allein das makroskopische Gesamtmoment
\begin{align*}
    M=2n\cdot m.
\end{align*}
$n$ liegt im Wertebereich $n\in\{-\frac{1}{2}N,-\frac{1}{2}N+1,\dots,\frac{1}{2}N\}$. Im Falle einer Gleichverteilung, $N_\uparrow=N_\downarrow$ ist $n=0$.

Wir wollen nun untersuchen, wie viele Zustände auf den gleichen Spinüberschuss $n$ führen \textendash{} das entspricht genau der Entartungsfunktion $g(n,N)$. Die exakte Lösung liefert die Kombinatorik:
\begin{align}
    \label{eq:entartungsfunktion_exakt}
    g(n,N) = \frac{N!}{N_\uparrow!\cdot N_\downarrow!} = \frac{N!}{\left(\frac{1}{2}N+n\right)!\cdot \left( \frac{1}{2}N-n \right)!}\,.
\end{align}

Summiert man über alle Zustände, so findet man für die Normierung 
\begin{align*}
    \sum_{n=-\frac{N}{2}}^{\frac{N}{2}}g(n,N) = 2^N. 
\end{align*}


\paragraph*{Eigenschaften der Entartungsfunktion für große Systeme}

Für die Thermodynamik und statistische Physik sind vor allem Systeme mit $N\gg 1$ relevant. Dann ist im Allgemeinen $|n|\ll N$ und $g\gg 1$. Mithilfe eines Tricks kann die Entartungsfunktion, wie in Gleichung \eqref{eq:entartungsfunktion_exakt} beschrieben genähert werden, was eine wesentlich bessere Handhabung ermöglicht.

Wir nehmen zunächst den natürlichen Logarithmus von \eqref{eq:entartungsfunktion_exakt}
\begin{align*}
    \ln g(n,N) = \ln \left( \frac{N!}{\left(\frac{1}{2}N+n\right)!\cdot \left( \frac{1}{2}N-n \right)!} \right) = \ln N! - \ln\left( \frac{N}{2}+n \right)!- \ln\left( \frac{N}{2}-n \right)!
\end{align*}
und wenden die Stirling-Näherung für große Argumente $N$
\begin{align*}
    \ln (N!) \approx \left( N+\frac{1}{2} \right)\ln N-N+\frac{1}{2}\ln(2\pi )
\end{align*}
an:
\begin{align*}
    \ln g(n,N) & \approx \frac{1}{2}\ln\frac{2}{N\pi}+N\ln 2-\frac{2n^2}{N} \\
    g(n,N)     & \approx \sqrt{\frac{2}{N\pi }}2^N e^{-\frac{2n^2}{N}}.
\end{align*}
Der Vorfaktor entspricht $g(0,N)$, sodass
\begin{align}
    \label{eq:entartungsfunktion_naeherung}
    g(n,N) \approx g(0,N) e^{-\frac{2n^2}{N}}, \quad g(0,N) =  \sqrt{\frac{2}{N\pi }}2^N.
\end{align}
Die Funktion ist gerade eine Gaußverteilung mit der Normierung
\begin{align*}
    \int_{-\infty}^{\infty} g(n,N)\diff n=2^N
\end{align*}
($n/N$ ist aufgrund der großen $N$ praktisch kontinuierlich und eine Integration bis unendlich statt bis $\pm 1/2$ trägt nur einen verschwindend geringen Fehler bei). Außerdem ist die Breite (für einen Abfall auf $g(0,N)/e$) $n_n$ durch $\sqrt{N/2}$ gegeben und die relative Breite durch
\begin{align}
    \label{eq:relative_breite}
    \frac{n_n}{N} = \frac{1}{\sqrt{2N}}.
\end{align}
In \Abbref{fig:DegeneracyFunctionGauss} ist die Entartungsfunktion $g(n,N)$ relativ (über $n/N$) aufgetragen.

\begin{figure}[htbp]
    \centering
    \tfigDegeneracyFunctionGauss
    \caption{Entartungsfunktion $g(n,N)$ über $n/N$ von $-1/2$ bis $+1/2$ aufgetragen. Die relative Halbwertsbreite $n_n/N$ ist bei $g(0,N)/e$ markiert. }
    \label{fig:DegeneracyFunctionGauss}
\end{figure}

Für $N=10^{22}$ ist die relative Halbwertsbreite bereits sehr schmal ($n_n/N\approx 10^{-11}$), sodass sich ein sehr scharfer Peak um $n=0$ (bzw. $M=0$) ausbildet. Insgesamt kann man für große Systeme sagen:
\begin{itemize}
    \item Spins sind gleichverteilt.
    \item Es wird ein eindeutiger Zustand mit $n=0$ bzw. $M=0$ angenommen mit kaum wahrnehmbaren Fluktuationen.
    \item Das System hat wohldefinierte physikalischen Eigenschaften.
\end{itemize}
Dies gilt insbesondere im thermodynamischen Grenzwert $N\rightarrow\infty$.


Die Tatsache, dass sich aus dem Aufaddieren vieler Teilzustände eine Gaußsche Normalverteilung ergibt, ist auch Gegenstand des zentralen Grenzwertsatzes der Statistik:
\begin{formal}
    \textbf{Zentraler Grenzwertsatz:}

    Bei der Summation sehr vieler unabhängiger Zufallsvariablen ergibt sich eine Normalverteilung (Gauß-Funktion).
\end{formal}


\paragraph*{Energie im Magnetfeld}

Ein Einzelspin hat die Energie $U=-mB$, ein System aus $N$ Spins also
\begin{align*}
    U(n)=-MB = -2nmB \equivalence n = -\frac{U}{2mB}. 
\end{align*}
Dabei ist der Abstand zwischen zwei Niveaus äquidistant,
\begin{align*}
    \Delta U = U(n) - U(n+1) = 2mB.
\end{align*}
Aus \eqref{eq:entartungsfunktion_naeherung} folgt, dass die Entartungsfunktion für $U$
\begin{align*}
    \boxed{g(U,N) = g(0,N) e^{-\frac{U^2}{2m^2B^2N}} \quad \mathrm{mit}\quad g(0,N) = \sqrt{\frac{2}{N\pi }} 2^N}\:
\end{align*}
ist.


\section{Grundannahmen: Thermodynamik und statistische Mechanik}

\paragraph*{Postulate und Bemerkungen}

Wir definieren zunächst einen makroskopischen Gleichgewichtszustand:
\begin{formal}
    Ein makroskopischer Gleichgewichtszustand entspricht vielen mikroskopischen (Quanten-) Zuständen, die alle mit seinen Kenngrößen (z.~B. Energie, Teilchenzahl, Volumen, Magnetisierung usw.) verträglich sind. Diese nennen wir auch zugängliche Zustände.
\end{formal}

Bei dem zuvor betrachtetem Spinsystem legt beispielsweise $g(U,N)$ die Zahl der zugänglichen Zustände bei festem $N$ und $U\propto M \propto 2n$ fest. Die letzteren sind dabei die makroskopischen Kenngrößen.

Wie aber sind diese mikroskopischen Zustände selbst besetzt? In der statistischen Mechanik verwendet man als Grundannahme die Ergodenhypothese, die besagt, dass alle diese Zustände gleich wahrscheinlich sind:
\begin{postulate}[Ergodenhypothese]
    \label{post:Ergodenhypothese}
    Ein abgeschlossenes System wird (in der Messzeit) in jedem ihm zugänglichen mikroskopischen Zustand mit gleicher Wahrscheinlichkeit angetroffen.
\end{postulate}

Es gibt auch nicht-ergodische Systeme, für die zugängliche mikroskopische Zustände in der Messzeit nicht angenommen werden oder ungleiche Besetzungswahrscheinlichkeiten besitzen.
Für ergodische Systeme gilt jedoch, dass das Zeitmittel äquivalent zum Scharmittel ist:
\begin{formal}
    Zeitlich versetzte Messungen der durchlaufenen mikroskopischen Zustände (\emph{zeitliches Nacheinander}) werden in der statischen Mechanik durch ein Ensemble von Kopien des Einzelsystems ersetzt (\emph{räumliches Nebeneinander}).

    Jedes Ensemblemitglied befindet sich genau in einem mikroskopischen Zustand. Für $g$ Zustände gibt es also $g$ Ensemblemitglieder.
\end{formal}


\paragraph*{Wahrscheinlichkeit}

Für ergodische Systeme mit $g$ erreichbaren Zustände ist die Wahrscheinlichkeit eines einzelnen Zustands $s$ sehr leicht zu bestimmen,
\begin{align*}
    P(s) = \frac{1}{g}
\end{align*}
und es gilt die Normierung
\begin{align*}
    \sum_{s=1}^g P(s) = 1.
\end{align*}
Das entspricht genau der Ergodenhypothese.
Der Zeitmittelwert als Ensemblemittelwert der physikalischen Messgröße $X(s)$ ist
\begin{align*}
    \left\langle X\right\rangle = \sum_s P(s)X(s) = \sum_s \frac{X(s)}{g}.
\end{align*}

\paragraph*{Beispiel: Ensemble eines Spinsystems}

Für $N=4$ und einen Spinüberschuss $2n=0$, also auch $U=0$ und $M=0$ ist die Entartungsfunktion durch
\begin{align*}
    g(n=0,N=4) = \frac{4!}{2!\cdot 2!} = 6
\end{align*}
gegeben. Das Ensemble besteht aus den möglichen Mikrozuständen
\begin{align*}
    \left\{ \uparrow\uparrow\downarrow\downarrow,\uparrow\downarrow\uparrow\downarrow,\uparrow\downarrow\downarrow\uparrow,\downarrow\uparrow\uparrow\downarrow,\downarrow\uparrow\downarrow\uparrow,\downarrow\downarrow\uparrow\uparrow \right\}.
\end{align*}



\section{Spinsysteme im thermischen Kontakt}

Wir betrachten nun zwei Spinsysteme mit den Kenngrößen $N_1,U_1\propto n_1$ bzw. $N_2,U_2\propto n_2$, die in thermischen Kontakt gebracht werden (also Energie austauschen können). Das Gesamtsystem ist insgesamt thermisch isoliert, sodass $U=U_1+U_2=\mathrm{const}$.
Wir wollen untersuchen, welche Aufteilung in $U_1$ und $U_2$ die wahrscheinlichste ist und welche Eigenschaften das gekoppelte System aufweist.


\paragraph*{Wahrscheinlichste Konfiguration}

Wir definieren zunächst einige Größen, die das Gesamtsystem beschreiben, nachdem beide Untersysteme in Kontakt gebracht wurden,
\begin{align*}
    N    & = N_1+N_2                                 \\
    n    & = n_1+n_2                                 \\
    U(n) & =U_1(n_1)+U_2(n_2) = -2mB(n_1+n_2)=-2mnB.
\end{align*}
Vorher sind $n_1$ und $n_2$ fest und es gibt $g_1(n_1,N_1)\cdot g_2(n_2,N_2)$ erreichbare Zustände. Nach dem Zusammenbringen ist $n=n_1+n_2$ fest und $n_1$ bzw. $n_2$ sind variabel. Die Anzahl der möglichen Zustände ist allerdings sehr viel größer. Die Entartungsfunktion ist jetzt durch
\begin{align*}
    g(n,N) = \sum_{n_1=-\frac{N_1}{2}}^{\frac{N_1}{2}} g_1(n_1,N_1)g_2(n-n_1,N_2)
\end{align*}
gegeben, während die Wahrscheinlichkeit für einen bestimmten Spinüberschuss $n_1$
\begin{align*}
    P_n(n_1) = \frac{g_1(n_1,N_1)g_2(n-n_1,N_2)}{g(n,N)}
\end{align*}
beträgt. Der wahrscheinlichste Spinüberschuss $n_1$ liegt beim Maximum dieser Funktion,
\begin{align}
    \label{eq:maximum_spinueberschuss_notwendige_bedingung}
    \diff \left(g_1(n_1,N_1)g_2(n_2,N_2)\right) \overset{!}{=} 0. 
\end{align}
Da nun $g_1\cdot g_2$ strikt größer als $0$ ist und ferner $\diff n_1=-\diff n_2$, kann ein kleiner Trick angewandt werden, um \eqref{eq:maximum_spinueberschuss_notwendige_bedingung} als 
\begin{align*}
    \frac{\diff(g_1\cdot g_2)}{g_1\cdot g_2}=\diff(\ln(g_1\cdot g_2)) \overset{!}{=} 0
\end{align*}
und weiter als 
\begin{align*}
    \diff(\ln g_1) + \diff(\ln g_2) = \frac{\partial \ln g_1}{\partial n_1} \diff n_1 + \frac{\partial \ln g_2}{\partial n_2} \diff n_2 = \left(\frac{\partial \ln g_1}{\partial n_1} - \frac{\partial \ln g_2}{\partial n_2}\right) \diff n_1  \overset{!}{=} 0
\end{align*}
zu schreiben. Es gilt also 
\begin{align}
    \label{eq:bedingung_wahrscheinlichster_spinueberschuss}
    \boxed{\frac{\partial \ln g_1}{\partial n_1}  = \frac{\partial \ln g_2}{\partial n_2} }\:.
\end{align}
Setzen wir nun die bereits bestimmten Funktionen $g_i=g_i(0,N_i) e^{-2n_i^2/N_i}$ mit $i\in\{1,2\}$ ein und benennen ferner die wahrscheinlichsten Spinüberschüsse mit $\hat{n}_1$ und $\hat{n}_2$, so erhalten wir 
\begin{align*}
    \frac{\hat{n}_1}{N_1} = \frac{\hat{n}_2}{N_2} . 
\end{align*}
Am Maximum ist der relative Spinüberschuss demzufolge für beide Systeme gleich. Man kann sich leicht überzeugen, dass es sich wirklich um ein Maximum der Funktion $P_n(n_1)$ handelt, denn die zweite Ableitung,
\begin{align*}
    \frac{\partial ^2\ln(g_1\cdot g_2)}{\partial ^2n_1} = -4\left( \frac{1}{N_1}+\frac{1}{N_2} \right) < 0,
\end{align*}
ist kleiner als $0$.
Mithilfe von $n_i= -U_i/(2mB)$ kann \eqref{eq:bedingung_wahrscheinlichster_spinueberschuss} umgeschrieben werden:
\begin{align}
    \label{eq:bedingung_wahrscheinlichster_spinueberschuss_U}
    \boxed{\frac{\partial \ln g_1}{\partial U_1}  = \frac{\partial \ln g_2}{\partial U_2} }\:.
\end{align}
Doch wohin führt uns diese Gleichung? In Kapitel \ref{sec:thermisches_gleichgewicht} haben wir gesehen, dass für Gleichgewichtszustände
\begin{align*}
    \diff S = \frac{\partial S^{(1)}}{\partial U^{(1)}}\diff U^{(1)}+ \frac{\partial S^{(2)}}{\partial U^{(2)}}\diff U^{(2)} \overset{!}{=}0 
\end{align*}
gilt. Wegen $\diff U^{(1)} = -\diff U^{(2)}$ ist also 
\begin{align*}
    \frac{\partial S^{(1)}}{\partial U^{(1)}} = \frac{\partial S^{(2)}}{\partial U^{(2)}}. 
\end{align*}
Diese Gleichung sieht \eqref{eq:bedingung_wahrscheinlichster_spinueberschuss_U} sehr ähnlich und es liegt nahe, die Identifizierung $S^{(i)}\propto \ln g_i$ zu machen. 


\paragraph*{Schärfe des Maximums}

An dieser Stelle soll ein kurzer Einschub zu der Schärfe des soeben berechneten Maximums erfolgen. 

Die maximale Zustandszahl wird \textendash{} wie im vorigen Abschnitt bestimmt \textendash{} bei $\hat{n}_1$ bzw. $\hat{n}_2$ erreicht und beträgt 
\begin{align*}
    (g_1g_2)_\mathrm{max} = g_1(\hat{n}_1,N_1)g_2(n-\hat{n}_1,N_1) = g_1(0,N_1)g_2(0,N_2) e^{-\frac{2\hat{n}_1^2}{N_1}-\frac{2\hat{n}_2^2}{N_2}},
\end{align*}
bzw. mit $\hat{n}_1+\hat{n}_2=n$ und $N_1+N_2=N$:
\begin{align*}
    (g_1g_2)_\mathrm{max} = g_1(0,N_1)g_2(0,N_2) e^{-\frac{2n^2}{N}}. 
\end{align*}
Schwankungen um $\hat{n}_i$ mit $n_i=\hat{n_i}+\delta$ führen wieder auf eine Gaußverteilung mit:
\begin{align*}
    g_1(\hat{n_1}+\delta,N_1)g_2(\hat{n_2}+\delta,N_2)=(g_1g_2)_\mathrm{max}e^{-2\delta^2 (\frac{1}{N_1}+\frac{1}{N_2})}. 
\end{align*}
Die Halbwertsbreite dieser Kurve beträgt:
\begin{align*}
    \frac{\delta_h}{N_1+N_2}=\frac{1}{N_1+N_2}\sqrt{\frac{N_1N_2}{2(N_1+N_2)}}. 
\end{align*}
Für große Systeme ergibt sich damit, wie bereits erläutert, eine sehr scharfe Verteilung mit kleiner Halbwertsbreite (für $N_1=N_2=10^{22}$ ist $\delta_h\approx 10^{11}$ und damit $\frac{\delta_h}{N_i}=10^{-11}$). Um ein noch besseres Gefühl für die Verteilungsfunktion in der genannten Größenordnung zu bekommen, schätzen wir die Zustandswahrscheinlichkeit eines Zustandes $g_1g_2$, welcher mit $\delta=10^{12}$ vom Maximum bei $\hat{n}_1$ abweicht ab:
\begin{align*}
    P_n(n_1)\propto\frac{g_1g_2}{(g_1g_2)_\mathrm{max}}&=e^{-2\delta^2(\frac{1}{N_1}+\frac{1}{N_2})}\\&=e^{-400}=10^{-174},
\end{align*}
er wird somit praktisch nicht realisiert. \\
Die Anzahl aller möglichen Realisierungen eines Systemzustandes mit einem konstantem Spinüberschuss $n$ im gekoppelten System der Größe $N$ ergibt sich nach Integration über alle möglichen Terme,
\begin{align*}
    g(n,N)&=\int_{-\infty}^{\infty}g_1(\hat{n_1}+\delta,N_1)g_2(\hat{n_2}+\delta,N_2)\diff\delta\\&=
    (g_1g_2)_\mathrm{max}\sqrt{\frac{\pi}{2}\frac{N_1N_2}{N_1+N_2}},
\end{align*}
und lässt sich in Abhängigkeit der maximalen Zustandszahl ausdrücken. Die letzte Gleichheit folgt aus der maximalen Zustandszahl mit Normierungsfaktor der Form:
\begin{align*}
    (g_1g_2)_\mathrm{max}=\frac{2}{\pi}\sqrt{\frac{1}{N_1N_2}}2^Ne^{-2n^2/N}.
\end{align*}
Die Wahrscheinlichkeit für den maximalen Spinüberschusszustand beträgt:
\begin{align*}
    P(\hat{n_1})=\frac{(g_1g_2)_\mathrm{max}}{g(n,N)}=\sqrt{\frac{2}{\pi}}\sqrt{\frac{1}{N_1}+\frac{1}{N_2}}\underset{N_i=N/2}{\approx}\frac{1}{\sqrt{N}}.
\end{align*}
Die letzte Näherung folgt aus der Annahme, $N_i$ entspräche $N/2$. Die Wahrscheinlichkeit ist also vergleichsweise sehr klein. Da die Verteilungsfunktion jedoch sehr stark gepeakt ist, folgt, dass in einer kleinen Umgebung $\delta_h$ um das Maximum herum die Wahrscheinlichkeitsnormierung,
\begin{align*}
    1=\int_\mathrm{\delta_h} P_n(\hat{n_1}+\delta)\diff \delta,
\end{align*}
erhalten ist.
Betrachtet man das Verhältnis des Logarithmus des Maximums der Entartungsfunktion zum Logarithmus der Entartungsfunktion\footnote{Warum betrachten wir erneut den Logarithmus? Wir haben bereits einen Zusammenhang zwischen diesem und der Entropie motiviert und zeichnen weiter vor, wie sich dieser Vergleich bewähren kann.},
\begin{align*}
    \frac{\ln(g_1g_2)_\mathrm{max}}{\ln g(n,N)}\approx 1-\mathcal{O}\left(\frac{\ln N}{N}\right)\underset{N\rightarrow\infty}{\rightarrow}1,
\end{align*}
wird wieder ersichtlich, dass die wahrscheinlichste Konfiguration den Term dominiert, denn der Ausdruck geht für den thermodynamischen Limes gegen $1$. Das thermodynamische System wird folglich durch seine wahrscheinlichsten Mikrozustände, welche sich im Spinüberschuss nur leicht voneinander unterscheiden, dominiert. 
\section{Thermisches Gleichgewicht und Entropie}
\paragraph*{Grundpostulat der statistischen Mechanik}
Wir wollen nun den verallgemeinerten Fall zweier Untersysteme im thermischen Kontakt betrachten. Die Zahl der Zustände der Untersysteme wird mit $g_i(U_i,N_i,...)$ gekennzeichnet und hängt von der inneren Energie und Teilchenzahl des Systems (sowie gegebenenfalls weiteren Variablen) ab. Die Entartungsfunktion des Gesamtsystems setzt sich (unter der Bedingung, dass die Gesamtenergie des Systems erhalten bleibt) als Summe über alle möglichen Produkte der Entartungsgrade der Untersysteme zusammen: 
\begin{align*}
    g(U,N)=\sum_{U_1+U_2=U}g_1(U_1,N)g_2(U_2,N_2).
\end{align*}
Der größte Summand kann analog zum Vorgehen mit Spinsystemen mittels des Differentials wie folgt ermittelt werden:
\begin{align*}
    \diff(g_1g_2)&=\diff(g_1)g_2+g_1\diff g_2\\
    &=\left(\frac{\partial g_1}{\partial U_1}\right)_{N_1}g_2\diff U_1+\left(\frac{\partial g_2}{\partial U_2}\right)_{N_2}g_1\diff U_2=0.
\end{align*}
Aufgrund der Nebenbedingung, dass die Gesamtenergie des Systems erhalten bleibt, gilt für die Energieänderungen der Teilsysteme $\diff U_1=-\diff U_2$.
Damit ergibt sich 
\begin{align*}
    \frac{1}{g_1}\left(\frac{\partial g_1}{\partial U_1}\right)_{N_1}=\frac{1}{g_2}\left(\frac{\partial g_2}{\partial U_2}\right)_{N_2}
\end{align*}
als thermische Gleichgewichtsbedingung, aus welcher ferner auch
\begin{align*}
    \boxed{\left(\frac{\partial \ln g_1}{\partial U_1}\right)_{N_1}=\left(\frac{\partial \ln g_2}{\partial U_2}\right)_{N_2}}\:
\end{align*}
folgt.
\begin{postulate}[Grundpostulat der statistischen Mechanik]
    \label{post:GrundpostulatStatMech}
     Das Grundpostulat der statischen Mechanik ordnet einem System eine Entropie zu, welche mit der Zahl der erreichbaren Zustände, $g(U,N)$ zusammenhängt. Dabei sind jene Zustände erreichbar, welche für das System unter Einhaltung der makroskopischen Zustandsgrößen ($U,N$ und gegebenenfalls weitere) zugänglich sind.
    \begin{align*}
        \boxed{S(U,N)=k_\mathrm{B}\ln g(U,N)}
    \end{align*}
    Die auftretende Boltzmannsche Konstante, $k_\mathrm{B}=R/L$, kann durch die ideale Gasgleichung in der statistischen Mechanik hergeleitet werden.
\end{postulate}
\paragraph*{Mikroskopische Erklärung der Entropiezunahme}
Wir wolllen nun auf mikroskopischer Ebene erörtern, weshalb die Entropie eines Systems bei Lösen seiner Zwangsbedingungen zunimmt (Postulat \ref{post:entropie_maximierung}).

Vor dem thermischen Kontakt gilt für zwei Teilsysteme mit inneren Energien $U=U^0_1+U^0_2$:
\begin{align*}
    (g_1g_2)_\mathrm{Zwang}=g_1(U^0_1,N_1)g_2(U^0_2,N_2).
\end{align*}
Nach thermischem Kontakt hingegen folgt für den wahrscheinlichsten Zustand mit $\hat{U}_1$:
\begin{align*}
    (g_1g_2)_\mathrm{max}=g_1(\hat{U}_1,N_1)g_2(\hat{U}_2=U-\hat{U}_1,N_2)
\end{align*}
Die relativen Fluktuationen um diesen Zustand herum sind minimal:
\begin{align*}
    \frac{\Delta U_1}{U}=\frac{U_1-\hat{U}_1}{U}\propto \frac{1}{\sqrt{N}}.
\end{align*}
Das System hat nach Lösen der Zwangsbedingungen viel mehr Konfigurationsmöglichkeiten und befindet sich in jenen Zuständen, welche am wahrscheinlichsten sind \textendash{} also in denjenigen, welche den höchsten Entartungsgrad besitzen. Damit und aufrgund der Schärfe der Verteilungsfunktion (Fluktuationen sind minimal), folgt, dass die Zahl der Zustände vor thermischem Kontakt, $(g_1g_2)_\mathrm{Zwang}$, in der Regel wesentlich kleiner ist als die Zahl der Zustände $(g_1g_2)_\mathrm{max}$ nach thermischem Kontakt. Dies entspricht nach Postulat \ref{post:GrundpostulatStatMech} der Entropiemaximierung.

Auch die Additivität der Entropie kann über die Zustandszahlen hergeleitet werden. Im thermodynamischen Limes wird das System durch die wahrscheinlichste Konfiguration eindeutig beschrieben, womit 
\begin{align*}
    S&=k_\mathrm{B}\ln (g_1g_2)_\mathrm{max}\\
    &=k_\mathrm{B}\ln g_1(\hat{U}_1,N_1)+k_\mathrm{B}\ln g_2(\hat{U}_2,N_2)\\
    &=S_1(\hat{U}_1,N_1)+S_2(\hat{U}_2,N_2)
\end{align*}
gilt und die Additivität der Entropie der Teilsysteme hergeleitet werden kann.

\paragraph*{Irreversible Prozesse:} Betrachten wir zum Schluss dieses Kapitels noch die irreversiblen Vorgänge:
Nach Newton sind Prozesse, welche zu einer Verringerung der Entropie führen aufgrund der Zeitumkehrinvarianz von $\pmb{F}(\pmb{r})=\diff\pmb{p}/\diff t$ prinzipiell erlaubt. Sie sind allerdings aufgrund der soeben erarbeiteten Zustandszahlen höchst unwahrscheinlich. Die Kombination dieser beiden Aussagen führt auf den Poincaréschen Wiederkehrsatz.
\begin{formal}
    \formalemph{Poincaréscher Wiederkehrsatz:} Jedes noch so große endliche System nimmt nach der Wiederkehrzeit $\tau$ seinen Anfangszustand in periodischen Abständen wieder ein. 
\end{formal}
Abschätzungen von $\tau$ für Systeme der Teilchengröße $N=10^{23}$ führen jedoch auf Zeiten, welche viel größer sind als das Erdzeitalter. Der Wiederkehrvorgang findet somit auf unvorstellbar großen Zeitskalen jenseits unserer Erfahrungsmöglichkeiten statt.
\begin{summary}
    Wir haben in diesem Kapitel die \formalemph{Zustandszahlen} eines Systems kennengelernt und sie mit der \formalemph{Entropie} in Relation gesetzt:

    Ausgangspunkt bildete die Betrachtung eines großen binären Modellsystems mit hohem Entartungsgrad (beispielsweise ein Modell, welches einzelne Spinzustände beschreibt).
    Ein solches System hat $2^N$ mögliche mikroskopische Konfigurationen, jedoch \textendash{} bezogen auf die makroskopische physikalische Größe \textendash{} nur $N+1$ unterscheidbare Zustände. Letzteres folgt aus der einfachen Beschreibung des magnetischen Gesamtmoments des Systems, 
     \begin{align*}
        M=(N_\uparrow-N_\downarrow)m=2nm,
    \end{align*}
    welches bei einer makroskopischen Messung ermittelt wird und lediglich Werte zwischen $Nm$ und $-Nm$ annehmen kann. $2n$ entspricht dem eingeführten Spinüberschuss des Systems.

    Somit ist offensichtlich, dass sich viele mögliche mikroskopische Zustände auf wenige makroskopische Zustände des Systems verteilen. Die Anzahl der möglichen mikroskopischen Zustände eines makroskopischen Zustandes werden durch den \formalemph{Entartungsgrad} 
    \begin{align*}
        g(n,N)&=\frac{N!}{N_\uparrow!-N_\downarrow!}\\&\approx g(0,N)e^{-\frac{2n^2}{N}}\\&=\sqrt{\frac{2}{N\pi}}2^Ne^{-\frac{2n^2}{N}}
    \end{align*}
    in Abhängigkeit von Spinüberschuss und Systemgröße beschrieben. Die beschriebene Entartungsfunktion entspricht einer Gauß-Verteilung mit Normierung $2^N$ und relativer Breite $n_n/N=1/\sqrt{2N}$. (Sie kann auch mittels der Relation $U=-MB=-2nmB$ in Abhängigkeit der inneren Energie angegeben werden.)

    Diese Verteilung deckt sich auch mit dem \formalemph{zentralen Grenzwertsatz}: Bei Summation vieler unabhängiger Zufallsvariablen ergibt sich eine Normalverteilung der Wahrscheinlichkeiten.

    Die Verteilung der Entartungsfunktion besitzt folglich für große Systeme ein scharfes Maximum. Daraus folgt, dass diese einen eindeutigen makroskopischen Zustand mit gleichverteilten Spins und minimalen Fluktuationen (bei $n=0$ und $M=0$) annehmen und wohldefinierte physikalische Eigenschaften besitzen. Die wahrscheinlichste makroskopische Konfiguration (die Konfiguration mit dem größten Entartungsgrad) bestimmt die Mittelwerte aller physikalischen Größen des Systems vollkommen. 
    Kleine Systeme hingegen haben große Fluktuationen und lassen sich nicht analog zu den großen Systemen durch die wahrscheinlichste makroskopische Konfiguration modellieren.
    Im thermodynamischen Limes (unendliche Teilchenzahl bzw. Systemgröße) geht die Entartungsfunktion in eine scharfe Linie mit verschwindender Halbwertsbreite über. 
      
    Allgemein gilt die \formalemph{Ergodenhypothese}. Diese besagt, dass sich ein abgeschlossenes System mit gleicher Wahrscheinlichkeit in jedem ihm zugänglichen (sich mit den makroskopischen Größen deckenden) mikroskopischen Zustand befindet.

    Die statistische Mechanik führt in diesem Kontext das sogenannte \formalemph{Ensemble} ein. Dabei wird das zeitliche Nacheinander von mikroskopischen Konfigurationen durch ein räumliches Nebeneinander an Systemkopien ersetzt. Jedes Ensemblemitglied entspricht dabei einer zugänglichen mikroskopischen Konfiguration des Systems. Der Zeitmittelwert einer physikalischen Messgröße eines ergodischen Systems entspricht damit seinem Ensemblemittelwert.
   
    Wir haben auch \formalemph{zwei große Spinsysteme im thermischen Kontakt} (mit Energiefluss zwischen den Systemen) betrachtet. Die folgenden Einsichten lassen sich auch für beliebige abgeschlossene Systeme und ihre makroskopischen Größen im thermischen Kontakt verallgemeinern:

    Vor Kontakt sind die einzelnen wahrscheinlichsten Spinüberschüsse der Teilsysteme festgelegt und die Zustandszahlen des Gesamtsystems ergeben sich als Produkt der einzelnen Zustandszahlen der Teilsysteme. Nach dem thermischen Kontakt sind die einzelnen Spinüberschüsse variabel und der Gesamtspinüberschuss fest. Die \formalemph{Entartungsfunktion des zusammengesetzten Systems} ist dann über 
    \begin{align*}
        g(n,N) = \sum_{n_1=-\frac{N_1}{2}}^{\frac{N_1}{2}} g_1(n_1,N_1)g_2(n-n_1,N_2),
    \end{align*}
    die Summation über alle möglichen Spinüberschusskombinationskonfigurationen der Teilsysteme gegeben. 
    Letztere nehmen dann die Teilzustände an, welche die wahrscheinlichste Konfiguration des Gesamtsystems konstituieren. Dies liegt vor, wenn die relativen Spinüberschüsse beider Teilsysteme gleich sind und kann auch durch folgende \formalemph{Gleichgewichtsbedingung} zusammengefasst werden:
    \begin{align}
        \frac{\partial \ln g_1}{\partial U_1}  = \frac{\partial \ln g_2}{\partial U_2}.
    \end{align}
    Wir haben beobachtet, dass dieser Zusammenhang die Identifizierung von $\ln g_i$ mit der Entropie $S^{(i)}$ der Teilsysteme nahelegt.
    
    Das \formalemph{Grundpostulat der statistischen Mechanik} beschreibt diese Relation zwischen Entropie und Entartungsfunktion über: 
    \begin{align*}
        S(U,N)=k_\mathrm{B}\ln g(U,N).
    \end{align*}

    Eben diese Relation liefert auch eine mikroskopische Erklärung für die Entropiezunahme von Systemen, deren Zwangsbedingungen gelöst werden. Das Lösen der Zwangsbedingungen führt zu einer Zunahme an Konfigurationsmöglichkeiten für das System. Es befindet sich dann in aller Regel in einem sehr wahrscheinlichen Zustand (also entsprechend der Zustandsverteilung in einem Zustand mit hohem Entartungsgrad), der den vorigen Entartungsgrad des eingenommenen Zustandes und damit dessen Entropie übersteigt.

    Im Zusammenhang mit dieser Beobachtung steht auch die Begründung für die Unwahrscheinlichkeit irreversibler Prozesse. Obwohl diese nach Newton nicht ausgeschlossen werden können und dem \formalemph{Poincaréschen Wiederkehrsatz} zu Folge nach einer Wiederkehrzeit $\tau$, welche ein System in periodischen Abständen in ihren Anfangszustand zurückführt, erfolgen, sind sie unvorstellbar unwahrscheinlich und $\tau$ jenseits der erfahrbaren Zeitskalen.
\end{summary}
