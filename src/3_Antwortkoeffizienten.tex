% !TeX root = Theo_IV.tex
\chapter{Antwortkoeffizienten und Beispielsysteme}
Im Folgenden wollen wir uns mit der Anwendung des bislang erarbeiteten Formalismus der Wärmelehre beschäftigen. Dazu erarbeiten wir uns die Herleitungen verschiedener Antwortkoeffizienten wie der spezifischen Wärme, beleuchten die sogenannten Maxwellbeziehungen und beschreiben das einkomponentige, ideale Gas.
\section{Antwortkoeffizienten: spezifische Wärme und andere Ableitungen}
Die zugrunde liegende Frage, welche die Beschäftigung mit sogenannten \emph{Antwortkoeffizienten} motiviert, ist die Frage nach der Reaktivität eines Systems auf äußere Einflüsse. Dem entstammt auch die Benennung dieser \textendash{} im Folgenden genauer erörterten \textendash{} Materialkonstanten als Antwortkoeffizienten.
Ein prominentes Beispiel einer solchen Materialkonstante ist die spezifische Wärme, die wir uns ebenfalls im Laufe dieses Unterkapitels näher anschauen wollen.


Ausgangspunkt unserer Betrachtungen bilden die Temperatur $T$ und der Druck $P$ in Energiedarstellung (also als Funktionen von Entropie $S$ und Volumen $V$), wobei wir annehmen, dass wir mit konstanten Stoffmengen (Molzahlen) arbeiten.  Die möglichen Wege im Zustandsraum können wir mit den genannten Größen wie folgt charakterisieren:


\paragraph*{Isochore}
Der Begriff \emph{Isochore} bezeichnet Zustandsänderungen bei gleichbleibenden Volumina. Damit folgt für die Größen die Abhängigkeit:
\begin{align*}
    V=V_0=\mathrm{const}.
\end{align*}
Die Größen $T$ und $P$ sind damit nur noch Funktionen von $S$ und damit ist $T=T(P)_{V_0}$ bzw. $P=P(T)_{V_0}$ (dazu wird die Entropie $S$ durch Umkehren einer der Zustandsfunktionen $T$ und $P$ und Einsetzen in die andere eliminiert).


\paragraph*{Isentrope}
Analog gehen wir bei der Betrachtung der \emph{Isentrope} vor, welche Zustandsänderungen bezeichnen, bei denen die Entropie gleich bleibt:
\begin{align*}
    S=S_0=\mathrm{const}.%\quad\rightarrow\quad T=T(S_0,V) \quad\text{und}\quad P=(S_0,V) \quad\rightarrow\quad T=T(P)|_{S_0}.
\end{align*}
Wir motivieren diese Betrachtung durch den Verweis darauf, dass sie bei der Beschreibung der Schallausbreitung essentiell ist. Dies ist dadurch bedingt, dass die dort ablaufenden Kompressionsprozesse derartig schnell verlaufen, dass keine Wärmezufuhr oder -abfuhr stattfinden kann und diese somit isentropisch sind. Unter den allgemeineren, \emph{adiabatischen} Zustandsänderungen versteht man Prozesse, welche ohne Wärmeaustausch mit der Umgebung ablaufen. Isentrope Prozesse sind adiabatisch, der Umkehrschluss gilt jedoch nicht. 


Wir wollen uns auch weiterhin auf die Zustandsvariablen Temperatur und Druck konzentrieren, wobei wir im Allgemeinen kontrollierbare (bzw. im Experiment kontrollierte) Zustandsvariablen wie diese beiden nun auch als \emph{Koordinaten} bezeichnen werden. Es sei darauf verwiesen, dass die Entropie $S$ keine solche direkt kontrollierbare Größe ist.

Durch Invertierung der energetischem Zustandsfunktionen erhalten wir die neuen Zustandsfunktionen
\begin{align*}
    \boxed{S=S(T,P) \quad\mathrm{und}\quad V=V(T,P)}\:.
\end{align*}
Erneut wollen wir mögliche Wege im Zustandsraum bei gleichbleibenden Parametern betrachten.


\paragraph*{Isotherme}
Zum einen können wir die Temperatur festhalten \textendash{} wir sprechen dann von \emph{Isothermen}, für welche
\begin{align*}
    T=T_0=\mathrm{const}%\quad\rightarrow\quad S=S(T_0,P) \quad\text{und}\quad V=(T_0,P) \quad\rightarrow\quad S=S(V)|_{T_0}
\end{align*}
gilt.



\paragraph*{Isobare}
Zum anderen können wir den Druck konstant halten \textendash{} hier sprechen wir von \emph{Isobaren}, für welche wiederum
\begin{align*}
    P=P_0=\mathrm{const}%\quad\rightarrow\quad S=S(T,P_0) \quad\text{und}\quad V=(T,P_0) \quad\rightarrow\quad S=S(V)|_{P_0}
\end{align*}
folgt.
Wir fragen uns nun, wie Änderungen der Größen $T$ und $P$ auf das Volumen $V$ wirken. Dazu schreiben wir
\begin{align*}
    \diff V (T,P)= \left( \frac{\partial V}{\partial T}\right)_P \diff T+\left( \frac{\partial V}{\partial P}\right)_T \diff P
\end{align*}
und erhalten die \emph{relative Volumenänderung}
\begin{align*}
    \frac{\diff V}{V}=\frac{1}{V}\left( \frac{\partial V}{\partial T}\right)_P\diff T+\frac{1}{V}\left( \frac{\partial V}{\partial P}\right)_T\diff P \equiv \alpha(T,P)\diff T-\kappa_T(T,P)\diff P.
\end{align*}
Zum einen haben wir den neuen Koeffizienten
\begin{align*}
    \boxed{\alpha=\frac{1}{V}\left( \frac{\partial V}{\partial T}\right)_P}
\end{align*}
als den \emph{thermischen Ausdehnungskoeffizienten} identifiziert, welcher das Maß der Systemausdehnung mit der Temperatur bei konstantem Druck beschreib. Zum anderen identifizieren wir die \emph{isotherme Kompressibilität}
\begin{align*}
    \boxed{\kappa_T=-\frac{1}{V}\left( \frac{\partial V}{\partial P}\right)_T}\:,
\end{align*}
welche die Reaktion des Systems auf entsprechende Drücke und bei konstanter Entropie\footnote{Hier betrachten wir die Entropie und nicht die Temperatur, da sich letztere bei den schnell ablaufenden Prozessen unvermeidlich verändert. Nichtsdestotrotz bleibt ein Wärmeaustausch aus, die Entropie ist also konstant.} charakterisiert.


Nun wollen wir die Materialkonstanten herleiten, welche bei Prozessen mit Entropieänderung relevant sind.
Dazu betrachten wir das entropische Differential
\begin{align*}
    \diff S (T,P)= \left( \frac{\partial S}{\partial T}\right)_P \diff T+\left( \frac{\partial S}{\partial P}\right)_T \diff P
\end{align*}
und erhalten darüber den \emph{quasistatischen Wärmefluss pro Mol}
\begin{align*}
    \frac{T}{\sm}\diff S=\frac{\udiff Q}{\sm}=\frac{T}{\sm}\left( \frac{\partial S}{\partial T}\right)_P \diff T+\frac{T}{\sm}\left( \frac{\partial S}{\partial P}\right)_T \diff P\equiv c_p(T,P)\diff T+\ldots %.
\end{align*}
Wieder identifizieren wir einen neuen Koeffizienten (der nicht ausgeschriebene zweite Term führt hier auf keinen besonderen Koeffizienten): Die \emph{molare spezifische Wärme} bei konstantem Druck
\begin{align*}
    \boxed{c_p=\frac{T}{\sm} \left( \frac{\partial S}{\partial T} \right)_P= \frac{1}{\sm} \left( \frac{\udiff Q}{\diff T} \right)_P }\:.
\end{align*}
Die Größe $\udiff Q$ ist natürlich eigentlich ein unvollständiges Differential, sodass der letzte Term nur beschränkt gültig ist. 
Die insgesamt zugeführte Wärme lässt sich darüber nun leicht durch eine Aufintegration der Form
\begin{align*}
    Q(T)=\int_{T_0}^{T} \udiff Q = \int_{T_0}^{T}\Tilde{T} \diff S = \int_{T_0}^{T}\sm c_p\diff \Tilde{T}
\end{align*}
bestimmen.

Es existiert eine weitere Herleitung der spezifischen Wärme, welche kurz skizziert werden soll. Ausgangspunkt bildet das uns inzwischen gut bekannte unvollständige Differential
\begin{align*}
    \udiff Q=\diff U +P\diff V= \diff(U+PV)|_{P=\mathrm{const}}.
\end{align*}
Den letzten Ausdruck erhalten wir unter der Voraussetzung, dass der Druck $P$ konstant ist. Schreiben wir die innere Energie als Funktion der altbekannten extensiven Größen, so wird ersichtlich, dass die innere Energie hier letztlich von den intensiven Größen $T$ und $P$ abhängt:
\begin{align*}
    U=U(S(T,P),V(T,P))=U(T,P).
\end{align*}
Es sei darauf hingewiesen, dass es sich bei letzterer Formulierung nicht um eine energetische Fundamentalbeziehung handelt, denn die Abhängigkeiten von $S$ und $V$ fehlen.


Wir führen nun das neue \emph{thermodynamische Potential} der \emph{Enthalpie}
\begin{align*}
    \boxed{H=U+PV=H(T,P)}
\end{align*}
ein \textendash{} wir werden zu einem späteren Zeitpunkt genauer darauf zurückkommen \textendash{} und schreiben die spezifische Wärme mittels der neuen Größe als
\begin{align*}
    c_p=\frac{1}{\sm}\left( \frac{\partial H}{\partial T}\right)_P.
\end{align*}
Eine kleine Randnotiz für den Leser: Die Inhalte, die wir hier erarbeiten, sind in der Regel mathematisch einfach. Entscheidend sind die Überlegungen zur Festlegung und Betrachtung konstanter Größen und der daraus folgenden Zusammenhänge.


Wir ändern nun die zu betrachtenden Koordinaten und konzentrieren uns auf Systeme mit kontrollierbarer Temperatur $T$ und kontrollierbarem Volumen $V$.
Analog zu vorigem Vorgehen schreiben wir die Zustandsfunktionen
\begin{align*}
    S=S(T,V)\quad\mathrm{und}\quad P=P(T,V)
\end{align*}
in Abhängigkeit der gewählten Koordinaten und schreiben den sich daraus ergebenden Ausdruck für den molaren quasistatischen Wärmefluss
\begin{align*}
    \frac{T}{\sm}\diff S= \frac{\udiff Q}{\sm}=\frac{T}{\sm}\left(\frac{\partial S}{\partial T}\right)_V \diff T +\frac{T}{\sm}\left(\frac{\partial S}{\partial V}\right)_T \diff V\equiv c_v(T,V)\diff T+\ldots.
\end{align*}
Erneut erhalten wir damit einen Ausdruck für die molare spezifische Wärme, diesmal bei konstantem Volumen:
\begin{align*}
    \boxed{c_v=\frac{T}{\sm}\left(\frac{\partial S}{\partial T}\right)_V=\frac{1}{\sm}\left(\frac{\udiff Q}{\diff T}\right)_V}\:,
\end{align*}
obwohl $\udiff Q$ eigentlich wieder ein unvollständiges Differential ist. 
Dank der Konstanz des Volumens können wir die Relation $\udiff Q=\diff U$ nutzen und für die spezifische Wärme alternativ
\begin{align*}
    \boxed{c_v=\frac{1}{\sm}\left(\frac{\partial U}{\partial T}\right)_V}
\end{align*}
schreiben.

Wir wollen zum Abschluss dieses Unterkapitels die beiden definierten Größen $c_p$ und $c_v$ für die spezifische Wärme noch in Relation setzen. Dazu machen wir folgende Überlegung: Betrachten wir eine Erwärmung bei konstantem Druck $P$ mit $c_p$,
so wird zusätzlich zur Temperaturerhöhung auch mechanische Arbeit zur Ausdehnung des Volumens verrichtet. Mit unseren Definitionen der spezifischen Wärme folgt daraus \textbf{$c_p >c_v$}.


Wir greifen nun auf die bereits eingeführte Integrabilitätsbedingung zurück und erörtern am Beispiel der inneren Energie die Relationen zwischen den zweiten Ableitungen, welche wir auch als \emph{Maxwellbeziehungen} bezeichnen.
Fangen wir wieder beim bekannten, vollständigen Energiedifferential
\begin{align*}
    \diff U = T\diff S-P\diff V+\mu \diff \sm 
\end{align*}
an. Mithilfe der Integrabilitätsbedingung
\begin{align*}
    \frac{\partial^2U}{\partial V \partial S}=\frac{\partial^2U}{\partial S \partial V}
\end{align*}
gelangen wir wegen $T=\partial U/\partial S$ und $P=-\partial U/\partial V$ zu folgenden Maxwell-Relationen:
\begin{align*}
    \boxed{\left(\frac{\partial T}{\partial V}\right)_{S,\sm }=-\left(\frac{\partial P}{\partial S}\right)_{V,\sm }=-T\left(\frac{\partial P}{\udiff Q}\right)_{V,\sm }}\;
\end{align*}
(natürlich ist $\udiff Q$ erneut ein unvollständiges Differential). 
Dabei charakterisiert der erste Term die Temperaturänderung bei isentropischer Volumenänderung. Der letzte Term definiert die Druckänderung bei isochorem Wärmezufluss. An dieser Stelle wollen wir bereits als Ausblick erwähnen, dass alle bildbaren zweiten Ableitungen der Größen eines betrachteten Systems mittels eines Minimalsatzes dreier Antwortkoeffizienten dargestellt werden können.

Für einfache Einkomponentensysteme gilt folgende Relation für die bisher eingeführten Antwortkoeffizienten (ohne Beweis):
\begin{align*}
    \boxed{c_p=c_v+\frac{TV\alpha^2}{\sm \kappa_T}}\:.
\end{align*}
Besonders nützlich  ist, dass die Koeffizienten experimentell sehr gut zugänglich sind. Die genaue Herleitung der Relation soll später erfolgen.

\section{Beispiel: Das einkomponentige ideale Gas}
Wir wollen das bisher Erarbeitete nun im Rahmen des Modells der \emph{idealen Gase} anwenden. Ideal heißt in diesem Kontext, dass wir die Annahme machen, Moleküle verhalten sich wie Punktteilchen, ohne \textendash{} mit Ausnahme von Stößen \textendash{} miteinander zu wechselwirken. Die Realisierung eines solchen idealen Gases erfolgt durch ausreichende Verdünnung eines tatsächlichen Gases.
Welche Zustandsgleichungen gelten nun für dieses ideale Gas?
Die prominenteste Gleichung dürfte die \emph{ideale Gasgleichung} sein:
\begin{align*}
    \boxed{PV=\sm RT}\:.
\end{align*}
Dabei bezeichnet $R$ die \emph{universelle Gaskonstante} und ist das Produkt der \emph{Avogadrokonstante} (welche die Anzahl der Teilchen in einem Mol angibt) und der \emph{Boltzmann-Konstante} (deren Bedeutung wir später noch erörtern werden):
\begin{align*}
    R=\avogadro k_\mathrm{B}=\qty{ 8,314 462 618 153 24}{\joule\per\mole\per\kelvin}
\end{align*}
mit $\avogadro=\qty{6,02214076e23}{\per\mole}$ und $k_\mathrm{B}=\qty{1,38064852e23}{\joule\per\kelvin}$. 

Im Allgemeinen gilt unter konstantgehaltener Temperatur das \emph{Boyle-Mariottsche Gesetz},
\begin{align*}
    PV=\mathrm{const}. 
\end{align*}
und unter konstantgehaltenem Druck das \emph{Gay-Lussacsche Gesetz},
\begin{align*}
    \frac{V}{T}=\mathrm{const},
\end{align*}
welche beide bekannte Ergebnisse experimenteller Untersuchungen sind.
Ferner gibt es die \emph{kalorische Zustandsgleichung}:
\begin{align*}
    \boxed{U=\frac{f}{2}\sm RT}\:.
\end{align*}
$f$ bezeichnet dabei die Zahl der Freiheitsgrade. Die innere Energie ist also für einfache\footnote{Die Kennzeichnung als \anf{einfaches} Gas wird im Laufe der nächsten Kapitel näher erläutert. Wir wollen bereits vorgreifen und darauf verweisen, dass damit Zustände mit höheren Quantenzahlen beschrieben werden, welche sich entsprechend \anf{klassisch} verhalten.} ideale Gase proportional zur Temperatur des Systems.
Die Gleichung ist ein direktes Ergebnis der statistischen Mechanik und ferner des dort genutzten \emph{Gleichverteilungssatzes}.
\begin{formal}
    Der \formalemph{Gleichverteilungssatz} der statistischen Mechanik besagt, dass ein System pro Freiheitsgrad $f$ eine molare innere Energie $u=RT/2$ annimmt.
\end{formal}
Wir wollen dies am Beispiel eines Atoms illustrieren, welches drei Freiheitsgrade für die Translation im dreidimensionalen Raum besitzt. Für diatomare Moleküle kommen noch weitere Freiheitsgrade hinzu: Zwei für Rotationen entlang der Achsen senkrecht zur Molekülachse und zwei für Schwingungen, bzw. für die gespeicherte potentielle Energie und die vorliegende kinetische Energie des Moleküls.
Die das Molekül beschreibende Hamiltonfunktion sieht wie folgt aus:
\begin{align*}
    H=\frac{p_x^2+p_y^2+p_z^2}{2M}+\frac{p_{\phi}^2+p_{\psi}^2}{2\Theta}+\frac{\mu^2}{2m}+\frac{1}{2}m\omega^2d^2.
\end{align*}
Der erste Summand beschreibt die Translation, der zweite die Rotation und die letzten beiden die Schwingung.
Der Freiheitsgrad $f$ entspricht hier auch der Zahl der quadratischen Einzelterme der Hamiltonfunktion.

An dieser Stelle machen wir einen kleinen Exkurs in die Quantenmechanik:
Nicht alle Freiheitsgrade sind zu einem beliebigen Zeitpunkt in einem System angeregt. Begründen können wir dies mithilfe der Quantelung der Energieeigenwerte in der Quantenmechanik.
Für Oszillatoren nehmen die Energieeigenwerte beispielsweise die Werte $E=(n+1/2)\hbar\omega$ mit $n=0,1,2,\ldots$ an, für Rotatoren $E=J(J+1)\hbar^2/(2\Theta)$ mit $J=0,1,2,\ldots$.
Bedingt durch diese Quantelungen ist das klassische, kontinuierliche Verhalten von physikalischen Systemen erst bei höheren Temperaturen beobachtbar, nämlich dann, wenn hohe Quantenzahlen vorliegen. Für Oszillatoren und Rotatoren liegen die Größenordnungen bei $k_\mathrm{B}T\gg\hbar\omega $, respektive $k_\mathrm{B}T\gg\hbar^2/\Theta$.
Das dadurch eintretende \anf{Auftauen} der Freiheitsgrade kann ab entsprechenden Schwellenwerten festgestellt werden. Für unsere zwei Beispiele liegen die Auftaubereiche für Wasserstoff bei über \qty{1000}{\kelvin} für Oszillationen und bei \qty{200}{\kelvin} für Rotationen.
Anfangs dominieren also translatorische Prozesse, während mit Zunahme der Temperatur langsam auch der Rotationsbereich und noch später der Schwingungsbereich \anf{auftaut} (siehe auch \Abbref{fig:ThawingOfDegreesOfFreedom}).
Diese Freiheitsgradverteilungen in Abhängigkeit der Temperatur werden wir später mittels der Beschreibung des wärmeabhängigen Verlaufs der spezifischen Wärme illustrieren.

Wir wollen zunächst das Gelernte nutzen, um sowohl das chemische Potential, als auch die Entropie aufzuschreiben. Aus den zwei eingeführten Zustandsgleichungen folgt
\begin{align}
    \label{eq:entropischeZustandsgleichungenIdealeGase}
    \boxed{\frac{P}{T}=\frac{R}{v} \quad\text{und}\quad \frac{1}{T}=\frac{f}{2}\frac{R}{u}}
\end{align}
in der Entropiedarstellung. %(??).
Das chemische Potential bestimmen wir wie folgt über die \emph{Gibbs-Duhem-Gleichung}
\begin{align*}
    \diff \left(\frac{\mu}{T}\right) & =u\diff\left(\frac{1}{T}\right)+v\diff \left(\frac{P}{T}\right)                                    \\
                                     & =-\frac{f}{2}\frac{R}{u}\diff u-\frac{R}{v}\diff v                                                 \\
                                     & =\frac{\partial \frac{\mu}{T}}{\partial u}\diff u+\frac{\partial \frac{\mu}{T}}{\partial v}\diff v
\end{align*}
und anschließender Integration
\begin{align*}
    \frac{\mu}{T}-\left(\frac{\mu}{T}\right)_0=-\frac{f}{2}R\ln\frac{u}{u_0}-R\ln\frac{v}{v_0}.
\end{align*}
Wir können dabei $\mu_0$ und $v_0$ als dem Referenzzustand zugehörige Größen auffassen.
Die Integrationskonstante $(\mu/T)_0$ bildet dabei das chemische Potential des Referenzzustandes.
Mithilfe der Eulergleichung lässt sich die Entropie $S$ nun als Funktion von innerer Energie $U$, Volumen $V$ und Stoffmenge $\sm$ bestimmen. Wir wollen jedoch einen alternativen Weg einschlagen und die molare Entropie mittels des Differentials
\begin{align*}
    \diff s & =\frac{1}{T}\diff u+\frac{P}{T}\diff v            \\
            & =\frac{f}{2}\frac{R}{u}\diff u+\frac{R}{v}\diff v
\end{align*}
bestimmen.
Dabei haben wir im letzten Schritt die kürzlich hergeleiteten Beziehungen \eqref{eq:entropischeZustandsgleichungenIdealeGase} genutzt.
Wir integrieren und erhalten analog zu voriger Vorgehensweise für die beiden nicht gemischten Terme
\begin{align}
    \label{eq:EntropieEinkomponentigesIdealesGas}
    \boxed{s=s_0+\frac{f}{2}R\ln\frac{u}{u_0}+R\ln\frac{v}{v_0}}\:.
\end{align}
Noch ist die Entropie jedoch nicht vollständig, da die unbestimmte Integrationskonstante $s_0$ vorliegt. Wir wissen jedoch, dass die Entropie \textendash{} bedingt durch den absoluten Temperaturnullpunkt \textendash{} ebenfalls einen absoluten Nullpunkt bei $T=0$ haben muss.
Die statistische Mechanik berücksichtigt diesen Zusammenhang und liefert eine Vervollständigung des Entropieausdrucks für monoatomare ideale Gase in Form der \emph{Sackur-Tetrode-Formel}, welche jedoch hier nicht weiter ausgeführt werden soll.

Wir wenden uns den Antwortkoeffizienten für das einkomponentige ideale Gas zu. Bereits kennengelernt haben wir die isotherme Kompressibilität
\begin{align*}
    \kappa_T=-\frac{1}{V}\left(\frac{\partial V}{\partial P}\right)_T=\frac{1}{P}
\end{align*}
und den thermischen Ausdehnungskoeffizienten
\begin{align*}
    \alpha=\frac{1}{V}\left(\frac{\partial V}{\partial T}\right)_P=\frac{1}{T},
\end{align*}
wobei wir uns die ideale Gasgleichung $PV=\sm RT$ zunutze gemacht haben, um die partiellen Ableitungen auszuschreiben.
Aus der Maxwell-Relation folgt für die spezifischen Wärmekoeffizienten
\begin{align*}
    c_p=c_v+\frac{TV\alpha^2}{\sm \kappa_T}.
\end{align*}
Setzen wir die eben formulierten Koeffizienten ein, erhalten wir für das ideale Gas die Relation
\begin{align*}
    \boxed{c_p=c_v+R}\:.
\end{align*}
Dabei bezeichnet der Koeffizient
\begin{align*}
    c_v=\frac{1}{\sm }\left(\frac{\partial U}{\partial T}\right)_V=\frac{f}{2}R
\end{align*}
den spezifischen Wärme-Antwortkoeffizienten des idealen Gases bei konstantem Volumen und
\begin{align*}
    c_p=\frac{f+2}{2}R
\end{align*}
den Antwortkoeffizienten bei konstantem Druck. Die Ausdrücke erhalten wir jeweils mithilfe der kalorischen Zustandsgleichung.
Das Verhältnis
\begin{align*}
    \frac{c_p}{c_v}=\frac{f+2}{f}=\gamma
\end{align*}
der beiden Koeffizienten bezeichnet man als \emph{Adiabatenexponent}.
Je größer $f$, die Zahl der Freiheitsgrade, desto weniger weicht der Exponent von eins ab.
Er spielt vor allem für die Beschreibung der Wege im Zustandsraum eine wichtige Rolle, welche wir nun näher beleuchten wollen.

Wir betrachten vorerst isentrope ($s=\mathrm{const}$) Prozesse:
Aus der Umstellung von \eqref{eq:EntropieEinkomponentigesIdealesGas} nach $e^{(s-s_0)/R}$ folgt für diese Prozesse die Konstanz des Terms $u^{(f/2)}v$.  Mit den Relationen
\begin{align*}
    u \propto T \quad\text{und}\quad \frac{f}{2}=\frac{1}{\gamma -1}\quad\rightarrow\quad TV^{\gamma-1}=\mathrm{const}
\end{align*}
und der Beziehung $T\propto PV$ der idealen Gasgleichung folgt ferner die Konstanz
\begin{align*}
    \boxed{PV^{\gamma}=\mathrm{const}}\:.
\end{align*}
Alternativ kann diese Gleichung mit $V\propto T/P$ auch als
\begin{align*}
    P^{1-\gamma}T^\gamma=\mathrm{const}
\end{align*}
geschrieben werden.
Die drei vorgestellten Ausdrücke charakterisieren alle eine äquivalente Quantifizierung isentroper Prozesse.
Für $\gamma = 1$, also für Systeme großer Freiheitsgrade bzw. Dimensionen, erhalten wir den Grenzfall der bekannten idealen Gasgleichung.
Analog können die Quantifizierungen
\begin{itemize}
    \item isochorer Prozesse: ${P}/{T}=\mathrm{const}$,
    \item isobarer Prozesse: ${V}/{T}=\mathrm{const}$ und
    \item isothermer Prozesse $PV=\mathrm{const}$
\end{itemize}
hergeleitet werden.

Wir wollen uns nun überlegen, ob wir auch einen alternativen Zugang zur Beschreibung der inneren Energie finden können, welcher nicht der statistischen Mechanik und ihrer Beziehung
\begin{align*}
    U=\frac{f}{2}\sm RT
\end{align*}
entspringt, sondern phänomenologischer Natur ist.
Dabei steht uns die messbare Relation $PV=\sm RT$ der idealen Gasgleichung zur Verfügung.
In der Experimentalphysik begegnet uns der \emph{Gay-Lussacsche Überstromversuch}. Dieser isoliert ein zu untersuchendes Gas in einer Hälfte eines zweigeteilten Kastens. Im nächsten Schritt öffnen wir einen begrenzten Bereich der Trennwand und ermöglichen dadurch die freie Expansion des Gases in die zweite Hälfte. Nun beobachten wir, dass die Temperatur des Systems konstant bleibt.
Aufgrund der Isolation des Systems bleibt dessen innere Energie erhalten und es gilt:
\begin{align*}
    \diff u(T,V)=0=\left(\frac{\partial u}{\partial T}\right)_v\diff T+\left(\frac{\partial u}{\partial v}\right)_T\diff v.
\end{align*}
Ferner folgt mit unserem Experimentablauf und der beobachteten Konstanz der Temperatur
\begin{align*}
    \diff v \neq 0,\quad\diff T=0\quad\rightarrow\quad \left(\frac{\partial u}{\partial v}\right)_T=0
\end{align*}
und führt uns somit zu der Einsicht
\begin{align*}
    \boxed{u=u(T)}\:.
\end{align*}
\begin{formal}
    Die innere Energie eines abgeschlossenen Systems eines allgemeinen idealen Gases ist ausschließlich von dessen Temperatur abhängig.
\end{formal}
Damit folgt auch, dass die spezifische Wärme $c_v$ \textendash{} wie erwartet \textendash{} ausschließlich temperaturabhängig ist\footnote{Wir erinnern an dieser Stelle an den gemachten Vorgriff bezüglich der \anf{Auftaubereiche} der Freiheitsgrade, welche im quantenmechanischen Formalismus verankert sind},
\begin{align*}
    c_v=\left(\frac{\partial u}{\partial T}\right)_v=c_v(T),
\end{align*}
und wir erhalten zum Abschluss unserer Betrachtungen die molare innere Energie über Aufintegration der inneren Wärme der Form:
\begin{align*}
    u=u_0+\int_{T_0}^Tc_v(\Tilde{T})\diff \Tilde{T}.
\end{align*}
Invertieren wir die gefundene Darstellung der inneren Energie, können wir mittels der Entropiedarstellung
\begin{align*}
    \frac{1}{T}=f(u)=\left(\frac{\partial s}{\partial u}\right)_v
\end{align*}
schreiben. Die ideale Gasgleichung führt uns bekanntermaßen auf
\begin{align*}
    \frac{P}{T}=\frac{R}{v}=\left(\frac{\partial s}{\partial v}\right)_u.
\end{align*}
Damit haben wir zwei Zustandsgleichungen gefunden, welche von unseren extensiven Größen abhängen.
Wir können diese nun nutzen, um die molare Entropie mittels Integration aufzuschreiben:
\begin{align*}
    s=s_0+\int_{u_0}^u\frac{1}{\Tilde{T}}(u')\diff u'+R\ln\frac{v}{v_0}\quad\text{mit}\quad\diff u'=c_v(\Tilde{T})\diff \Tilde{T},
\end{align*}
wobei mit $\diff u=c_v(T)\diff T$ abschließend
\begin{align*}
    \boxed{s=s_0+\int_{u_0}^u\frac{c_v(\Tilde{T})}{\Tilde{T}}\diff \Tilde{T}+R\ln\frac{v}{v_0}}
\end{align*}
folgt.
Dabei handelt es sich um eine parametrische Darstellung der entropischen Fundamentalgleichung.

Wir wollen nun ein konkretes Beispiel für den Verlauf der spezifischen Wärme $c_v$ geben und betrachten dazu ein aus zweiatomigen Molekülen bestehendes ideales Gas. Für dieses gilt, wie wir bereits erläutert haben,
\begin{align*}
    u=\frac{f}{2}RT\quad\text{mit}\quad f=7\quad\rightarrow\quad c_v=\frac{f}{2}R=\frac{7}{2}R,
\end{align*}
sodass der spezifische Wärmeverlauf in Abhängigkeit der Temperatur $T$ wie in Abbildung~\Abbref{fig:ThawingOfDegreesOfFreedom} aussieht. 
\begin{figure}[htbp]
    \centering
    \tfigThawingOfDegreesOfFreedom
    \caption{Temperaturabhängigkeit der spezifischen Wärme eines idealen Gases zweiatomiger Moleküle.}
    \label{fig:ThawingOfDegreesOfFreedom}
\end{figure}

Die angekündigten, von der Besetzbarkeit der Zustände abhängigen, Auftaubereiche der Freiheitsgrade schlagen sich wie erwartet hier in der spezifischen Wärme nieder.
Wir beobachten ferner, dass für Festkörper ein zu $T^3$ proportionaler Verlauf vorliegt, welcher durch die Anschauung von Phononen und ihren Schwingungen plausibilisiert werden kann \textendash{} darauf wollen wir im Rahmen der statistischen Mechanik erneut zu sprechen kommen.

Wir wollen uns nun erweiterte Systeme vorstellen, in denen eine Mischung einfacher idealer Gase vorliegen. Es stellt sich uns die Frage, wie die Fundamentalgleichungen aussehen, die dieses neue System beschreiben. Für die innere Energie,
\begin{align*}
    \boxed{U=\left(\sum_j\frac{f_j}{2}\sm _j\right)RT}\:,
\end{align*}
ist die Herleitung naheliegend, da wir die neuen, unterschiedlichen Freiheitsgrade und Stoffmengen in die altbekannte Fundamentalgleichung einsetzen können und über die hinzukommenden Terme nur summieren müssen (dies ist uns deshalb erlaubt, da wir für ideale Gase davon ausgehen, dass keine Wechselwirkungen zwischen den Gasmolekülen vorliegen). Auch die entropische Fundamentalgleichung ergibt sich als Summation über die Einzelentropien der unterschiedlichen Gase:
\begin{align*}
    \boxed{S=\sum_j\sm _js_j=\sum_j\sm _j\left(s_{j0}+\frac{f_j}{2}R\ln\frac{T}{T_0}+R\ln\frac{V}{\sm _jv_0}\right)}
\end{align*}
(wieder charakterisieren die Größen $T_0,v_0,s_{j0}$ den Referenzzustand).
Dies ist eine Konsequenz des additiven Charakters der Entropie und wird durch das \emph{Gibbs'sche Theorem} beschrieben:
\begin{align*}
    \boxed{S(\text{Mischung idealer Gase})=\sum_jS_j(\text{Einzelgase bei } V,T)}.
\end{align*}
\begin{formal}
    Das \formalemph{Gibb'sche Theorem} besagt, dass sich die Gesamtentropie eines Systems gemischter idealer Gase als die Summe der Einzelentropien der Gase schreiben lässt.
\end{formal}
Veranschaulichen können wir es, indem wir uns einen sogenannten \emph{Mischapparat} vorstellen, welcher zwei unterschiedliche Gase und semipermeable Membranen, die für je eine Gassorte durchlässig sind, bei gleichem Druck $P=P_1=P_2$ und gleicher Temperatur $T=T_1=T_2$ beinhaltet. 

\begin{figure}[htbp]
    \centering
    \begin{subfigure}[b]{.45\textwidth}
        \centering
        \tfigMixingMachineOne        
        \caption{}
        \label{fig:MixingMachineOne}
    \end{subfigure}
    \begin{subfigure}[b]{.45\textwidth}
        \centering
        \tfigMixingMachineTwo           
        \caption{}
        \label{fig:MixingMachineTwo}
    \end{subfigure}
    \caption{Dartstellung eines Mischapparates}
    \label{fig:MixingMaschine}
\end{figure}
Wir stellen uns vor, dass die Ausgangssituation der \Abbref{fig:MixingMachineOne} entspricht \textendash{} Membran (1) ist nur für das erste Gas durchlässig, Membran (2) nur für das zweite, sodass beide Gase noch in getrennten Kammern vorliegen. Nun kann der Kolben gemeinsam mit
Membran (1) quasistatisch%(??) 
bewegt werden, bis die Gase \textendash{} immer noch je im selben Volumen enthalten \textendash{} vollständig miteinander vermischt sind (\Abbref{fig:MixingMachineTwo}). Es ist ersichtlich, dass dieser Vorgang ohne weitere Prozesse abläuft, Volumina $V_1,V_2$, Stoffmengen $\sm _1,\sm _2$, Druck $P_1,P_2$und Temperatur $T_1,T_2$ also durchweg konstant bleiben und letztlich auch die innere Energie $U$ erhalten wird. Folglich ist auch die Gesamtentropie $S$ vor und nach der Mischung hier gleich groß.

Wir wollen darauf verweisen, dass \emph{Mischentropien} sehr wohl existieren, d.~h., dass Mischprozesse einen Zuwachs der Entropie verursachen können. Dies setzt jedoch einen anderen Versuchsaufbau voraus, für welchen beispielsweise keine Erhaltung der Volumina $V_1,V_2$ gilt. 
\begin{figure}[htbp]
    \centering
    \tfigMixTwoGases
    \caption{<caption>}
    \label{fig:MixTwoGases}
\end{figure}

Das simpelste Beispiel wird durch eine Kammer beschrieben, welche durch eine Trennwand geteilt wird. Links und rechts der Trennwand befinden sich bei gleicher Temperatur $T$ und Druck,
\begin{align*}
    P\propto\frac{\sm _1}{V_1}=\frac{\sm _2}{V_2}=\frac{\sm }{V}=\frac{\sm _1+\sm _2}{V_1+V_2},
\end{align*}
zwei unterschiedliche Gase. Wird die Trennwand nun gemäß \Abbref{fig:MixTwoGases} entfernt, so unterscheiden sich die anfangs eingenommenen Volumina der Gase, $V_1$, respektive $V_2$, von dem zuletzt eingenommenen Gesamtvolumen $V_1+V_2$.
Wir wollen die Entropie des Endzustandes beschreiben und mit der Entropie vor Mischung in Relation setzen. Dazu schreiben wir zuerst die Zustandsgleichung
\begin{align*}
    S=\sum_j\sm _js_j=\sum_j\sm _j\left(s_{j0}+\frac{f_j}{2}R\ln\frac{T}{T_0}+R\ln\frac{V}{\sm _jv_0}\right)
\end{align*}
auf. Wir wollen nun den letzten Term wie folgt umschreiben:
\begin{align*}
    \ln\frac{V}{\sm _jv_0}=\ln\frac{V}{\sm v_0}-\ln\frac{\sm _j}{\sm }\quad\text{mit}\quad\frac{\sm _j}{\sm }=x_j\quad\text{und}\quad \sm =\sum_j\sm _j.
\end{align*}
Nach Einsetzen erhalten wir
\begin{align*}
    \boxed{
        \begin{aligned}
            S & =\sum_j\sm _j\left(s_{j0}+\frac{f_j}{2}R\ln\frac{T}{T_0}+R\ln\frac{V}{\sm v_0}\right)-\sm R\sum_jx_j\ln x_j                 \\
              & =\sum_jS_j+S_m
        \end{aligned}
    }\:.
\end{align*}
Dies entspricht der Entropie des Systems vor Mischung (die Einzelgase bei Temperatur $T$ und Druck $P \propto n_j/v_j=n/V$) und einem zusätzlichen positiven Term für die Mischungsentropie $S_m$.
\paragraph*{Gibbs'sches Paradoxon}
Wir haben bislang jedoch noch eine Anwendung des Theorems übersehen, die zum Widerspruch führt und auch als \emph{Gibbs'sches Paradoxon} bekannt ist.
Betrachten wir einmal den Grenzfall gleicher Gaskomponenten \textendash{} wir werden feststellen müssen, dass dies beim beschriebenen Mischprozess zu einer unsinnigen Entropiezunahme führt. Schließlich können wir uns vorstellen, die Trennwand zu einem beliebigen Zeitpunkt wieder einzusetzen und damit die Anfangsbedingungen wieder hergestellt zu haben \textendash{} für die natürlich die anfängliche Entropie vorherrscht. Vertrauen wir der Herleitung des positiven Mischterms der Entropie bedingungslos, so haben wir gerade einen simplen Prozess der Entropieverringerung beschrieben, welcher sich äußerst unphysikalisch anfühlt. 

Die Quantenmechanik hilft uns hier glücklicherweise das Paradoxon begründet aufzulösen: Die Systeme vor und nach der Trennung unterscheiden sich physikalisch nicht (natürlich setzen wir erneut gleiche Anfangstemperaturen und Drücke voraus). Alle Gasmoleküle werden als zueinander identisch angesehen, sodass keine zusätzlichen Systemkonfigurationen bei Mischung vorliegen, weshalb auch der Entropie-Mischterm plausiblerweise in diesem Fall null bleibt.

\section{Das Van-der-Waals-Gas}
Wir wollen uns jetzt ausgehend von den idealen Gasen den realen Gasen und der Van-der-Waals-Gleichung zuwenden.

Die Wechselwirkungen der Atome und Moleküle idealer Gase sind aufgrund ihrer großen mittleren Abstände verschwindend gering und werden deshalb nicht betrachtet.
Anders verhält es sich bei realen Gasen. Tragen wir das Wechselwirkungspotential in Abhängigkeit der Abstände wie in [Abb??] auf, so erkennen wir in unmittelbarer Radiusgrößenordnung starke Abstoßungen; bei größeren Abständen hingegen Anziehung (auch Kohäsion genannt). 

Für reale Gase beobachten wir zwei wesentliche Effekte: Zum einen die sogenannte Volumenreduktion, welche das effektiv vom Gas einnehmbare Volumen, $v_\mathrm{eff}=v-b$, beschreibt ($v$ bezeichnet das Behältervolumen und $b$ das Eigenvolumen eines Moles der Gasmoleküle).
Zum anderen liegt eine Druckkorrektur, $P=P_\mathrm{kin}-a/v^2$, vor. Der erste Term beschreibt den kinetischen Druck, der der thermischen Bewegung entspringt, der zweite die intermolekulare Anziehung, welche den Druck auf die Behälterwand reduziert. 
Da letztere Anziehung von den Molekülwechselwirkungen abhängt, ist sie proportional zur Molekülpaaranzahl ($\propto N^2/V^2$).

Wir setzen die für das reale Gas diskutierten Größen in die ideale Gasgleichung ein:
\begin{align*}
    P_\mathrm{kin} v_\mathrm{eff}=RT\\
    (P+\frac{a}{v^2})(v-b)=RT
\end{align*}
und erhalten die sogenannte \emph{Van-der-Waals-Gleichung},
\begin{align*}
    \boxed{ P=\frac{RT}{v-b}-\frac{a}{v^2}}\:
\end{align*}
(in der statistischen Mechanik lässt sich diese Gleichung ebenfalls herleiten).
Da $a$ und $b$ empirische Konstanten sind, liefert die Gleichung keine strenge quantitative Beschreibung realer Gase.
Sie erlaubt jedoch eine qualitative Beschreibung ihrer Eigenschaften und der gas-flüssig Phasenübergänge.

Bevor wir zum nächsten Unterkapitel übergehen, wollen wir (ohne Beweis) noch zwei interessante Zusammenhänge für reale Gase vorstellen.
Zum einen gilt für die Ableitung der inneren molaren Energie nach dem molaren Volumen unter konstanter Temperatur: 
\begin{align*}
    \left( \frac{\partial u}{\partial v} \right)_T =\frac{a}{v^2}>0.
\end{align*}
Wird das Volumen verringert, so nehmen die Wechselwirkungen zwischen den Molekülen und damit die innere Energie zu.
Ferner folgt aus dieser Beziehung überraschenderweise, dass die spezifische Wärme volumenunabhängig ist, $c_v=c_v(T)$, denn es gilt
\begin{align*}
    \frac{\partial c_v}{\partial v}=\frac{\partial}{\partial v}\frac{\partial u}{\partial T}=\frac{\partial}{\partial T}\frac{a}{v^2}=0.
\end{align*} 

\begin{summary}
    In diesem Kapitel haben wir uns verschiedene Koeffizienten \textendash{} sogenannte Antwortkoeffizienten kennengelernt und untersucht. 


\end{summary}