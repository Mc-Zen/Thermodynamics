% !TeX root = Theo_IV.tex

\chapter{Einleitung\label{einleitung}}



\section{Inhalt}

Der Inhalt dieser Vorlesung gliedert sich in zwei Teile, die Thermodynamik und die statistische Physik:

Bei der Thermodynamik geht es im Allgemeinen um Vielteilchensysteme, für die genauen mikroskopischen Positionen interessiert man sich allerdings nicht, sondern vielmehr um makroskopische Größen und Verteilungen.
Die Vielteilchensysteme werden durch den Minimalsatz von makroskopischen Variablen beschrieben, z.B. durch Energie, Volumen und Entropie.
Man spricht tatsächlich bei der Thermodynamik auch von der Lehre der Entropie.

Das Ziel ist es, allgemeine und modellunabhängige Aussagen und Prinzipien zu formulieren und häufig geht es nicht um die Berechnung von Systemparametern, sondern um die Relationen zwischen ihnen.
Damit hat die Thermodynamik einen sehr breiten Anwendungsbereich und findet Verwendung in vielen 
Wissenschaften\footnote{Beispiele: Gase, Magnetismus, Supraleitung, chemische Reaktionen, Phasenübergänge, Schwarzkörperstrahlung, Neutronensterne, schwarze Löcher, Biologie, soziologische Systeme, Ottomotor, Klima, ...}.

Die Thermodynamik stellt auch die Basis für einige Weiterentwicklungen dar, wie z.B. die Nichtgleich\-gewichts-Thermodynamik (in dieser Vorlesung wird eine reine thermostatische Beschreibung behandelt).

Die statistische Physik fasst Parameter von Vielteilchensysteme mithilfe der mikroskopischen Bewegungsgleichungen und statistischen Methoden mit makroskopischen Größen zusammen.



\begin{itemize}
	\item Herbert B. Callen: Thermodynamics and an Introduction to Thermostatistics, John Wiley \& Sons, New York 1985
	\item Gerhard Adam, Otto Hittmaier: Wärmetheorie, Vieweg, Wiesbaden 1992
	\item Charles Kittel, Herbert Kroemer: Thermal Physics, W.H. Freeman and Company, 1980
\end{itemize}

Folgende weiterführende Literatur kann zurate gezogen werden:
\begin{itemize}
	\item Arnold Sommerfeld: Theoretische Physik V: Thermodynamik und Statistik, Harri Deutsch, 1977
	\item Franz Schwabl: Statistische Mechanik, Springer, Berlin 2000
	\item Torsten Fließbach: Statistische Physik - Lehrbuch zur Theoretischen Physik IV, Springer Spektrum, Berlin 2010
	\item Wolfgang Nolting: Grundkurs Theoretische Physik 4/2: Thermodynamik, Springer Spektrum, Berlin 2016
	\item Wolfgang Nolting: Theoretische Physik 6: Statistische Physik, Springer Spektrum, Berlin 2013
	\item Mehran Kardar: Statistical Physics of Particles, Cambridge University Press, Cambridge 2007
\end{itemize}



\section{Grundlegende Konstanten der Thermodynamik}

Für Konstanten, deren Wert per Definition festgelegt wurde, die also exakt sind, wird ein $\equiv $-Zeichen verwendet.


\begin{table}[H]
	\centering
	\begin{tabular}{|l|l|} \hline
		\textbf{Konstante}       & \textbf{Wert}                                                                            \\
		\hline

		Boltzmannkonstante       & \centering\arraybackslash{} $k_\mathrm{B} \equiv \SI{1,38064852e-23}{\joule\per\kelvin}$ \\
		Universelle Gaskonstante & \centering\arraybackslash{} $R \equiv \SI{8,31446261815324}{\joule\per\kelvin\per\mole}$ \\
		Avogadro-Konstante        & \centering\arraybackslash{} $N_\mathrm{A} \equiv \SI{6,02214076e23}{\per\mole}$          \\
		Atomare Masseneinheit    & \centering\arraybackslash{} $u= \SI{1,6605390666050e-27}{\kg}$                           \\
		\hline
	\end{tabular}
\end{table}




\section{Grundlegende Formeln der Thermodynamik}
