% !TeX root = Theo_IV.tex

\chapter{Einleitung\label{einleitung}}



\section{Inhalt}

Der Inhalt dieser Vorlesung gliedert sich in folgende Abschnitte:

\begin{itemize}
	\item
\end{itemize}



\section{Grundlegende Konstanten der Thermodynamik}

Für Konstanten, deren Wert per Definition festgelegt wurde, die also exakt sind, wird ein $\equiv $-Zeichen verwendet.


\begin{table}[H]
	\centering
	\begin{tabular}{|l|l|} \hline
		\textbf{Konstante}         & \textbf{Wert}                                                     \\\hline
		
		Boltzmannkonstante & \centering\arraybackslash{}$k_\mathrm{B}\equiv \SI{1,38064852e-23}{\joule\per\kelvin}$ \\
		Universelle Gaskonstante & \centering\arraybackslash{}$R\equiv \SI{8,31446261815324}{\joule\per\kelvin\per\mole}$ \\
		Avogadrokonstante & \centering\arraybackslash{}$N_\mathrm{A}\equiv \SI{6,02214076e23}{\per\mole}$ \\
		Atomare Masseneinheit & \centering\arraybackslash{}$u= \SI{1,6605390666050e-27}{\kg}$ \\
		\hline
	\end{tabular}
\end{table}




\section{Grundlegende Formeln der Thermodynamik}
