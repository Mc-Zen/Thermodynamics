% !TeX root = Theo_IV.tex


\chapter{Das Prinzip minimaler Energie und thermodynamische Potentiale}

In diesem Kapitel wollen wir den Formalismus der Thermodynamik noch ein wenig weiter ausführen.
Bisher haben wir stets entweder in der Energiedarstellung oder in der Entropiedarstellung gearbeitet. Technisch können beide aber nur schlecht kontrolliert werden. Dagegen lässt sich beispielsweise die Temperatur sehr gut kontrollieren. Zu diesem Zweck werden in diesem Kapitel neue thermodynamische Potentiale eingeführt, die an die physikalische Situation angepasst sind.

Außerdem wird die Frage behandelt, wie sich das durch das Postulat \ref{post:entropie_maximierung} beschriebene Extremalprinzip für die Entropie auf andere thermodynamische Potentiale übertragen lässt.


\section{Das Prinzip der minimalen Energie}

Zur Wiederholung soll zunächst erinnert werden, wie sich die Entropie nach dem Postulat~\ref{post:entropie_maximierung} maximiert:
\begin{formal}
    Bei konstanter innerer Energie eines System ist der Wert einer ungehemmten Variablen $X$ im Gleichgewichtszustand durch ein Maximum der Entropie ausgezeichnet:
    \begin{align*}
        \left( \frac{\partial S}{\partial X}  \right)_U = 0, \qquad \left( \frac{\partial ^2S}{\partial X^2}  \right)_U < 0.
    \end{align*}
\end{formal}




Alternativ können wir auch sagen:

\begin{formal}
    Bei konstanter Entropie eines Systems ist der Gleichgewichtswert einer ungehemmten (frei einstellbaren) internen Variable $X$ durch ein Minimum der inneren Energie ausgezeichnet:
    \begin{align*}
        \left( \frac{\partial U}{\partial X}  \right)_S = 0, \qquad \left( \frac{\partial ^2U}{\partial X^2}  \right)_S > 0.
    \end{align*}
\end{formal}