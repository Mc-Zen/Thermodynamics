% !TeX root = Theo_IV.tex

\chapter{Thermodynamische Prozesse und Maschinen}
Im Laufe dieses Kapitels wollen wir uns mit unterschiedlichen thermodynamischen Prozessen und Maschinen auseinandersetzen. Ein Beispiel für letztere sind Wärmekraft- und Carnot-Maschinen, welche deshalb für uns interessant sind, weil die Entwicklung des Entropiebegriffes historisch durch ihre Beschreibung begründet wurde.
Die Realisierung thermodynamischer Maschinen baut auf zwei wesentlichen Prinzipien auf: der Energieerhaltung (und den damit verbundenen physikalischen Gesetzen der Mechanik) und der Monotonie der Entropiezunahme ($\Delta S \geq 0$) in abgeschlossenen Systemen.
Wir wollen zunächst mit einigen Charakterisierungen thermodynamischer Prozesse anfangen.


\section{Quasistatische Prozesse}
Die quasistatischen Prozesse sind uns mittlerweile sehr gut bekannt. Sie bilden Prozessführungen, welche zu jedem Zeitpunkt Gleichgewichtszustände des Systems beschreiben.
\paragraph*{Der thermodynamische Konfigurationsraum}
Wir wollen zur Veranschaulichung und Charakterisierung allgemeiner Prozesse den sogenannten thermodynamischen Konfigurationsraum einführen. Dieser wird durch die extensiven Variablen eines betrachteten Systems aufgespannt und ist exemplarisch in Abb. [Abb][Ref] dargestellt.
Die Menge aller Gleichgewichtszustände eines quasistatischen Prozesses liegen im Konfigurationsraum auf einer Hyperfläche $S=S(U,V,X_j$), die der Beziehung 
\begin{align*}
    \left(\frac{\partial S}{\partial U}\right)_{V,X_j}=\frac{1}{T}>0
\end{align*}
(welche aus Postulat \ref{post:eigenschaften_entropie} folgt) unterliegt.  
Wir können auch zusammengesetzte, abgeschlossene Systeme leicht im Konfigurationsraum darstellen, da sich die extensiven Größen eines Teilsystemes aus der Additivität und der Kenntnis der Größen des anderen Teilsystems ergeben. Formal ist die Beschreibung der Hyperfläche durch $S=S(U^{(1)},X_j^{(1)},U^{(2)},X_j^{(2)})$ in diesem Fall also zu 
\begin{align*}
    S&=S(U^{(1)},X_j^{(1)},U=U^{(1)}+U^{(2)},X_j=X_j^{(1)}+X_j^{(2)})\\
    &=S(U^{(1)},X_j^{(1)},U,X_j)
\end{align*}
äquivalent.
Nichtgleichgewichtszustände hingegen haben, bedingt durch ihre Dynamik, viel mehr Dimensionen und spannen einen bedeutend größeren Raum auf. Dieser beinhaltet z.~B. Inhomogenitäten, Turbulenzen, Flussfelder und viele mehr. 

\paragraph*{Reale und quasistatische Prozesse}
Wir wissen, dass es sich bei der quasistatischen Prozessführung um eine Idealisierung handelt, welche reale Elemente - wie Geschwindigkeiten, Flüsse, Raten und weitere - vernachlässigt. 
Nichtsdestotrotz ist diese Idealisierung für uns nützlich, da die Thermodynamik sehr genaue (Vor-)Aussagen über nur eben solche Prozessführungen erlaubt.
Für die Abbildung realer Prozesse müssen wir die Diskrepanz zur Idealisierung jedoch berücksichtigen. (I.~d.~R. liegen für reale Prozesse wie in Abb. [Abb](??) [Ref] illustriert ein Anfangszustand $A$ und ein Endzustand $Z$ auf der Hyperfläche vor, welche über einen beliebigen Weg im Konfigurationsraum (über Nichtgleichgewichtszustände) miteinander verbunden sind).  
Eine mögliche Annäherung, welche den Übergang von Idealisierung zu Realität überbrückt, ist die Darstellung realer Prozesse als dichte Abfolge von Gleichgewichtszuständen über Nichtgleichgewichtszustände. Anschaulich gesprochen entspricht dies einer Abfolge von Zuständen auf einer Hyperfläche $S$ im Konfigurationsraum, die über Wege - welche nicht auf der Hyperfläche selbst liegen - miteinander verbunden sind. Auch dies haben wir in Abb. [Abb2 neben Abb1][Ref](??) illustriert.    
\begin{formal}
     \formalemph{Reale Prozesse} entwickeln sich i.~d.~R. ausgehend von einem Punkt $A$ auf der Hyperfläche $S$ über Nichtgleichgewichtszustände des Konfigurationsraumes (außerhalb von $S$) zu einem Endzustand $Z$, welcher wiederum als Gleichgewichtszustand in $S$ liegt.

     \formalemph{Quasistatische (auch quasistationäre) Prozesse} entwickeln sich über eine dichte Abfolge von Gleichgewichtszuständen in $S$, welche über Nichtgleichgewichtszustände verbunden sind.
 \end{formal}

\paragraph*{Zeitkonstanten}
Nach Erläuterung dieser Näherung stellt sich uns die Frage, durch welche quantitative Größe quasistatische Prozesse von nicht-quasistatischen Prozessen unterschieden werden können.
Wir wollen dies anhand eines Beispiels erörtern.

\begin{figure}[htbp]
    \centering
    \tfigAdiabaticGasCompression
    \caption{Adiabatische Kompression eines Gases}
    \label{fig:AdiabaticGasCompression}
\end{figure}

In \Abbref{fig:AdiabaticGasCompression} haben wir eine adiabatische Gaskompression mittels eines Kolbens in einer Kammer abgebildet. Drücken wir den Kolben schnell genug ein, so entstehen Verwirbelungen des Gases, für welche wir die folgende Vereinfachung treffen: Sie breiten sich vom Kolben über die gesamte Kammerlänge $l$ bis zur gegenüberliegenden Kammerwand mit Schallgeschwindigkeit $c$ aus. Damit folgt für die \anf{Lebensdauer} dieser Störung $\tau =l/c$. (Um ein Gefühl für die Größenordnung zu gewinnen, können wir eine Kammerlänge $l=\SI{1}{\m}$ einsetzen und erhalten $\tau=\SI{0.003}{\s}$.) Wollen wir den Kolben also quasistatisch eindrücken, so muss für die Laufzeit $\Delta t \gg\tau$ gelten.
\begin{formal}
    \formalemph{Quasistatische Prozesse} werden über die Beziehung $\Delta t \gg\tau$ charakterisiert: Ihre Prozesszeiten $\Delta t$, welche zwischen Gleichgewichtszuständen liegen, müssen wesentlich größer als die charakteristischen Relaxationszeiten $\tau$ der Systemstörungen sein.
    Es gilt $\Delta S\geq\Delta Q_{\mathrm{mess}}/T$.
\end{formal}  

\section{Reversible und irreversible Prozesse}

\paragraph*{Definitionen und Bemerkungen }
Wir wollen im Folgenden eine genaue Unterscheidung reversibler und irreversibler Prozesse geben. 
Dazu betrachten wir ein abgeschlossenes, zusammengesetztes System, welches durch die Lockerung einer Zwangsbedingung von Zustand $A$
in Zustand $B$ übergeht. Dem Postulat \ref{post:eigenschaften_entropie} zufolge ist die Entropie des Endzustandes größer als die Entropie des Anfangszustandes.
Folglich ist der umgekehrte, spontane Übergang von $B$ nach $A$ verboten und wir bezeichnen den Zustandsübergang von $A$ nach $B$ als irreversibel. Nicht-quasistatische Prozesse sind im Folgeschluss immer irreversibel, der Umkehrschluss gilt nicht.
Reversible Prozesse hingegen sind umkehrbare Prozesse, d.~h. die mit ihnen verknüpfte Entropieänderung ist gleich null.
Sie sind quasistatisch, wobei der Umkehrschluss im Gegenzug wieder nicht immer gilt. %[Grafik??]
\begin{formal}
    Ein \formalemph{reversibler Prozess} ist ein quasistatischer Prozess, welcher keine Entropieänderung verursacht ($\Delta S=0$) und damit auch umkehrbar ist. 
    Ein \formalemph{irreversibler Prozess} ist ein Prozess, welcher bedingt durch eine Entropiezunahme ($\Delta S>0$) in abgeschlossenen Systemen nicht umkehrbar ist. Nicht-quasistatische Prozesse sind immer irreversibel, aber auch quasistatische Prozesse können irreversibel sein.
\end{formal}
Wir wollen die eingeführten Begriffe anhand einiger Beispielprozesse veranschaulichen:
Ausgangspunkt bildet in allen Fällen ein \emph{thermisch isolierter} Kasten, in dessen linken Hälfte sich ein Gas befindet. Damit sind alle Ausgangszustände identisch.
\begin{figure}[htbp]
    \centering
    \begin{subfigure}[b]{.32\textwidth}
        \centering
        \tfigProcessReversibleQuasistationary        
        \caption{}
        \label{fig:ProcessReversibleQuasistationary}
    \end{subfigure}
    \begin{subfigure}[b]{.32\textwidth}
        \centering
        \tfigProcessIrreversibleQuasistationary           
        \caption{}
        \label{fig:ProcessIrreversibleQuasistationary}
    \end{subfigure}
    \begin{subfigure}[b]{.32\textwidth}
        \centering     
        \tfigProcessIrreversibleNonquasistationary
        \caption{}
        \label{fig:ProcessIrreversibleNonquasistationary}
    \end{subfigure}
    \caption{Vergleich unterschiedlicher Prozessführungen}
    \label{fig:ProcessReversibleIrreversibleQuasistationary}
\end{figure}
\paragraph*{Reversibel und quasistatisch}
Im ersten Aufbau, \Abbref{fig:ProcessReversibleQuasistationary}, wird der Kasten durch einen Kolben geteilt, welcher frei beweglich ist. Es läuft eine quasistatische, adiabatische Gasexpansion ab, welche den Kolben verschiebt. Die Verrichtung dieser mechanischen Arbeit führt zur Verringerung der inneren Energie und damit auch der Temperatur des Systems. Die Entropie hat sich insgesamt jedoch nicht verändert.
\paragraph*{Irreversibel und quasistatisch}
Im zweiten Aufbau, \Abbref{fig:ProcessIrreversibleQuasistationary}, wird der Kasten durch viele dicht beieinander liegende, entfernbare Trennwände geteilt. Wir können durch sukzessives Entfernen der Wände eine quasistatische Prozessführung anleiten, jedoch ist diese mit einer Entropiezunahme verbunden - der Prozess ist irreversibel.
\paragraph*{Irreversibel und nicht-quasistatisch}
Im dritten Aufbau, \Abbref{fig:ProcessIrreversibleNonquasistationary}, wird der Kasten durch eine einzige entfernbare Trennwand geteilt. Ziehen wir diese heraus, so findet eine nicht-quasistatische Zustandsänderung statt. Zwar gelangen wir zum selben Endzustand wie beim vorigen Experiment, jedoch laufen wir aufgrund der entstehenden Strömungen keinen Weg innerhalb der definierten Hyperfläche $S$ im Konfigurationsraum ab. Auch hier handelt es sich folglich um einen irreversiblen Prozess, welcher mit einer zu \ref{fig:ProcessIrreversibleQuasistationary} identischen Entropiezunahme verknüpft ist.

Es sei angemerkt, dass wir irreversible Prozesse durch die Ankopplung eines weiteren Systemes (also in offenen Systemen) realisieren können. Dazu muss die Entropie in Form eines Wärmeflusses vom Teilsystem abgegeben und vom angekoppelten System aufgenommen werden. 
Ein Beispiel einer derartigen offenen Systemkopplung ist die Erde. Sie beherbergt Prozesse, welche mit einer Entropieabnahme (geordneten Lebensformen) verbunden sind. Jedoch wird zugleich Entropie in Form von Wärme an ihre Umgebung abgegeben.

\paragraph*{Reversible Quellen und Reservoire}
Wir wollen einige weitere nützliche idealisierte Bausteine einführen, die uns später bei der Beschreibung kontrollierter Systemparameter, der Charakterisierung neuer thermodynamischer Potentiale sowie der Erarbeitung der Ensembletheorie der statistischen Mechanik helfen werden.
%ggf Tabelle anlegen (links bild, rechts text, vice versa, vice versa) 
\begin{itemize}
    \item \textbf{Reversible Arbeitsquelle (RAQ):} Ein idealisiertes System, mit welchem (mechanische) Arbeit reversibel ($\Delta S=\Delta Q/T=0$) ausgetauscht werden kann, nennt man eine \emph{reversible Arbeitsquelle}, kurz \emph{RAQ}. Eine derartige Arbeitsquelle/-senke ist also vollständig wärmeisoliert und hat eine konstante Entropie $S^{\mathrm{RAQ}}=\mathrm{const}$. Ein Beispiel für eine RAQ ist ein mechanisches System ohne Reibung.
    \item \textbf{Reversible Wärmequelle (RWQ):} Ein idealisiertes System, mit welchem Wärme reversibel ausgetauscht werden kann, nennt man eine \emph{reversible Wärmequelle}, kurz \emph{RWQ}. Eine derartige Wärmequelle/-senke ist von starren Wänden umgeben (keine mechanische Arbeit kann daran verrichtet werden). Für die Änderung ihrer inneren Energie gilt: $\diff U^{\mathrm{RWQ}}=\udiff Q^{\mathrm{RWQ}}=T\diff S=c(T)\diff T$.
    \item \textbf{Volumenreservoir:} Eine sehr große RAQ, deren Druck $P$ unabhängig von Volumen- und innerer Energieänderung konstant bleibt, bezeichnen wir als \emph{Volumenreservoir}.\footnote{Mathematisch beschreiben wir mit $(\partial P/\partial U)_{V,N_i}=(\partial P/\partial V)_{U,N_i}=0$ eine homogene Funktion $-1$-ten Grades ($\partial P/\partial \lambda U=\lambda ^{-1}\partial P/\partial U$), welche für unendlich groß-dimensionierte Systeme ($\lambda\rightarrow \infty$) somit gegen 0 strebt.}
    \item \textbf{Wärmereservoir:} Eine sehr große RWQ, deren Temperatur $T$ unabhängig von Änderungen der inneren Energie konstant bleibt, bezeichnen wir als \emph{Wärmereservoir}.\footnote{Analog zum Volumenreservoir beschreibt $(\partial T/\partial U)_{V,N_i}=0$ eine homogene Funktion $-1$-ten Grades.}
\end{itemize}
Ein Beispiel für ein Wärme- und Volumenreservoir ist die Atmosphäre (in bestimmten änderungsfreien Zeitabschnitten). Bedingt durch ihre Größenordnung wirken sich lokale Temperatur- und Druckänderungen auf der Erde nicht messbar auf sie aus.
\section{Prozesse maximaler Arbeit}
Welche Arbeitsleistung kann nun maximal von einem System verrichtet werden, wenn es von einem Zustand in einen anderen übergeht? 
Um diese Frage zu beantworten nutzen wir die eingeführten, reversiblen Quellen und betrachten Arbeit- und Wärmefluss separat. 

\paragraph*{Das Theorem maximaler Arbeit}
\begin{figure}[htbp]
    \centering
    \tfigTheoremMaximizedWork
    \caption{Kopplung eines Teilsystems mit Wärme- und Arbeitsquelle}
    \label{fig:MaximaleArbeit}
\end{figure}
In \Abbref{fig:MaximaleArbeit} veranschaulichen wir die (insgesamt abgeschlossene) Kopplung des Teilsystems TS mit einer RWQ und einer RAQ. Die Änderung der inneren Energie des Teilsystems beim Zustandsübergang ($\Delta U^{\mathrm{TS}}_{AB}=U^{\mathrm{TS}}_{B}-U^{\mathrm{TS}}_{A}$) wird in Form von Wärme ($\Delta Q^{\mathrm{RWQ}}$) an die RWQ und in Form von Arbeit ($\Delta W^{\mathrm{RAQ}}$) reversibel an die RAQ abgegeben. (Positive Änderungsterme kennzeichnen hier Größenzunahmen.) Wir fassen das Ganze entsprechend dem ersten Hauptsatz der Thermodynamik (\ref{hs:erster}) zusammen:
\begin{align}
    \label{eq:maximaleArbeit}
    -\Delta U^{\mathrm{TS}}_{AB}=\Delta Q^{\mathrm{RWQ}}+\Delta W^{\mathrm{RAQ}}
\end{align}
Für reversible und irreversible Prozesse des Systems ergeben sich folgende Größenbilanzen:
\renewcommand*{\arraystretch}{1.3}
\arrayrulecolor{formalshade!90!black}
\begin{table}[b]
    \centering
    \caption{Entropiebilanz eines abgeschlossenen gekoppelten Systems}
    \begin{tabularx}{.5\textwidth}{|l|X|l|}
        \hline
        \rowcolor{formalshade!98!blue}
        &Reversibel &Irreversibel\\
        \hline
        \rowcolor{formalshade}
        Gesamtsystem&$\Delta S^\mathrm{Ges}=0$&$\Delta S^\mathrm{Ges}>0$\\
        \rowcolor{formalshade!80!white}
        Teilsystem TS&$\Delta S^{\mathrm{TS}}_{\mathrm{AB}}$&$\Delta S^{\mathrm{TS}}_{\mathrm{AB}}$\\
        \rowcolor{formalshade}
        RAQ&$0$&$0$\\
        \rowcolor{formalshade!80!white}
        RWQ&$-\Delta S^{\mathrm{TS}}_\mathrm{AB}$&$-\Delta S^{\mathrm{TS}}_\mathrm{AB}+\Delta S^\mathrm{Ges}$\\
        \hline
    \end{tabularx} 
    \label{tab:Entropiebilanz}
\end{table}
\begin{itemize}
    \item \textbf{Entropie (der RWQ):} Bei reversiblen Prozessen innerhalb eines abgeschlossenen Systems ist die Gesamtentropie erhalten ($\Delta S^\mathrm{Ges}=0$). Die Entropieänderung des Teilsystems entspricht $\Delta S^{\mathrm{TS}}_{\mathrm{AB}}$. Da die RAQ per Definition keine Entropieänderung erfährt, schließen wir daraus, dass $\Delta S^\mathrm{RWQ}=-\Delta S^{\mathrm{TS}}_\mathrm{AB}$ ist.
    
    Für irreversible Prozesse verhalten sich das Teilsystem und die RAQ gleich. (Wir gehen hier von irreversiblen Prozessen aus, die den selben Anfangs- und Endzustand, $A$ und $B$, wie die dazu verglichenen reversiblen Prozesse haben.)
    Die Gesamtentropie nimmt in diesem Fall jedoch mit $\Delta S^\mathrm{Ges}>0$ zu, sodass für die RWQ $\Delta S^\mathrm{RWQ}=-\Delta S^{\mathrm{TS}}_\mathrm{AB}+\Delta S^\mathrm{Ges}$ folgt.
    (Der Vergleich ist in Tabelle \ref{tab:Entropiebilanz} noch einmal zusammengefasst.) 

    Die Entropiezunahme der RWQ ist folglich für reversible Prozesse kleiner als für irreversible:
    \begin{align*}
        \boxed{\Delta S^\mathrm{RWQ}_\mathrm{rev}<\Delta S^\mathrm{RWQ}_\mathrm{irr}}\:.
    \end{align*}
    
    \item \textbf{Wärme (der RWQ):} Bekanntermaßen gilt für die RWQ 
    \begin{align*}
        \Delta U^\mathrm{RWQ}=\Delta Q^\mathrm{RWQ}=\int_{\Delta S^\mathrm{RWQ}}T^\mathrm{RWQ}\diff S^\mathrm{RWQ},
    \end{align*}
    womit für die Wärmeaufnahmen 
    \begin{align*}
        \boxed{\Delta Q_\mathrm{rev}^\mathrm{RWQ}<\Delta Q_\mathrm{irr}^\mathrm{RWQ}}
    \end{align*}
    folgt. Die Wärmezunahme der RWQ ist (analog zur Entropiezunahme) für reversible Prozesse kleiner als für irreversible Prozesse.
    \item \textbf{Arbeit (der RAQ):} Über Umstellen von \ref{eq:maximaleArbeit},
    \begin{align*}
        \Delta W^\mathrm{RAQ}=-\Delta U_\mathrm{AB}^\mathrm{TS}-\Delta Q^\mathrm{RWQ},
    \end{align*}
    schließen wir somit auch darauf, dass die Arbeitsaufnahme für reversible Prozesse größer ist als für irreversible:
    \begin{align*}
        \boxed{\Delta W^\mathrm{RAQ}_\mathrm{rev}>\Delta W^\mathrm{RAQ}_\mathrm{irr}}\:.
    \end{align*}
\end{itemize} 
\begin{formal}
    \formalemph{Theorem maximaler Arbeit:} Der Arbeitsübertrag auf die RAQ ist für reversible Zustandsänderungen maximal, der Wärmefluss in die RWQ zugleich minimal. 
\end{formal}
Die mechanische Energieausbeute eines Prozesses lässt sich wie erwartet über die Minimierung von Wärmeverlusten maximieren. Idealisierte, reversible Prozesse gleicher Anfangs- und Endzustände realer Prozesse liefern dabei deren gültige Schranken. Bei irreversiblen Prozessen dissipiert ein Teil der möglichen Arbeitsleistung in Form von Wärme (beispielsweise über Reibung).

Eine \emph{Wärmepumpe} ist eine Maschine, welche Arbeit verrichtet ($\Delta W^\mathrm{RAQ}<0$) um Wärme der RWQ ins System zu pumpen ($\Delta Q^\mathrm{RWQ}<0$).
Für die Wärme- und Entropiebilanz folgt damit:
\begin{align*}
    \boxed{
    \begin{aligned}
            -\Delta Q^\mathrm{RWQ}_\mathrm{rev}>-\Delta Q^\mathrm{RWQ}_\mathrm{irr}\\
            -\Delta W^\mathrm{RAQ}_\mathrm{rev}<-\Delta W^\mathrm{RAQ}_\mathrm{irr}
    \end{aligned}
    }\:.
\end{align*}
Für reversible Prozesse ist die Wärmeaufnahme aus der RWQ also maximal und die Arbeit der RAQ minimal.






% \section{Prozesse maximaler Arbeit}

% Wir wollen nun untersuchen, welche Prozesse die maximale Arbeit verrichten und wie sich diese verhalten. 

% \paragraph*{Theorem maximaler Arbeit}

% \begin{formal}
%     Von allen Prozessen $A\rightarrow B$ ist der Arbeitsübertrag auf die reversible Arbeitsquelle maximal (und der Wärmefluss in die reversible Wärmequelle minimal), wenn der Ablauf reversibel ist. 
% \end{formal}

% Das entspricht einer Optimierung der mechanischen Energieausbeute durch Minimierung der Wärmeverluste. 


\paragraph*{Spezialfall}

Für ein Teilsystem TS, das die Wärme $\Delta Q^{\mathrm{RWQ}}$ aus der reversiblen Wärmequelle zugeführt bekommt und an das keine reversible Arbeitsquelle gekoppelt ist, lässt sich der zweite Hauptsatz der Thermodynamik ableiten. 
Im reversiblen Ablauf ändert sich die Entropie nämlich mit 
\begin{align*}
    \Delta S^{\mathrm{TS}} = \int \diff S^\mathrm{TS} = -\int \diff S^\mathrm{RWQ} = - \int \frac{\udiff Q^\mathrm{RWQ}}{T} = \int \frac{\udiff Q^\mathrm{TS}_\mathrm{rev}}{T} .
\end{align*}
Ohne das Teilsystem TS ist dann 
\begin{align}
    \label{eq:2.HS_teil1}
    \diff S = \frac{\udiff Q_\mathrm{rev}}{T},
\end{align}
was dem ersten Teil des zweiten Hauptsatzes der Thermodynamik entspricht (Gleichheit). 

Im irreversiblen Ablauf ist dagegen 
\begin{align*}
    \Delta S^{\mathrm{TS}} = \int \diff S^\mathrm{TS} = -\int \diff S^\mathrm{RWQ} +\Delta S = - \int \frac{\udiff Q^\mathrm{RWQ}}{T^\mathrm{RWQ}} + \Delta S = \int \frac{\udiff Q^\mathrm{TS}_\mathrm{irr}}{T^\mathrm{RWQ}} +\Delta S
\end{align*}
mit $\Delta S>0$. Insgesamt ist dann 
\begin{align}
    \label{eq:2.HS_teil2}
    \diff S > \frac{\udiff Q_\mathrm{irr}}{T},
\end{align}
was dem zweiten Teil des zweiten Hauptsatzes der Thermodynamik entspricht (strenge Ungleichheit). 


Kombiniert man die Gleichungen \eqref{eq:2.HS_teil1} und \eqref{eq:2.HS_teil2} und verwendet, dass $\udiff Q=\diff U-\udiff W_\mathrm{mech}-\udiff W_\mathrm{chem}$, so erhält man die Grundrelation der Thermodynamik:
\begin{align}
    \label{eq:grundrelation_der_TD}
    \boxed{T\diff S\geq \diff U-\udiff W_\mathrm{mech}-\udiff W_\mathrm{chem}}\:.
\end{align}



\section{Wirkungsgrad von Maschinen}

Eine thermodynamische Maschine besteht i.d.R. aus einem Teilsystem TS (später auch Hilfssystem genannt), das an zwei reversible Wärmereservoire angekoppelt wird \textendash{} ein heißes (h) und ein kaltes (k) \textendash{} sowie an eine reversible Arbeitsquelle RAQ. TS könnte z.B. ein Kolben sein.  

Insgesamt ist das System idealerweise nach außen hin abgeschlossen, $\diff U=0$. Ferner soll sich der Zustand des Teilsystems TS nach einem vollständigen Arbeitszyklus nicht geändert haben. 

Nach dem ersten Hauptsatz gilt ($\diff U=0$)
\begin{align}
    \label{eq:maschine_HS1}
    \udiff Q_\mathrm{h} +\udiff Q_\mathrm{k} + \udiff W^\mathrm{RAQ} = 0
\end{align}
und nach dem zweiten für einen reversiblen Ablauf 
\begin{align}
    \label{eq:maschine_HS2}
    \diff S_\mathrm{h} + \diff S_\mathrm{k} = 0 \equivalence \frac{\udiff Q_\mathrm{h}}{T_\mathrm{h}} + \frac{\udiff Q_\mathrm{k}}{T_\mathrm{k}} = 0. 
\end{align}
Dabei ist natürlich $\udiff Q_\mathrm{h}=-\udiff Q_\mathrm{TS}$ u.s.w. 

\paragraph*{Thermodynamische Maschine}

Bei einer thermodynamischen Maschine wird allgemein Wärme $\udiff Q_\mathrm{h}$ aus einer Energiequelle (z.B. ein Ofen oder Dampfkessel) mit hoher Temperatur $T_\mathrm{h}$ über das Teilsystem (z.B. ein Kolben) in Form von Arbeit $\udiff W^\mathrm{RAQ}$ an die Maschinerie und in Form von Wärme $\udiff Q_\mathrm{k}$ an ein Kühlsystem mit niedrigerer Temperatur abgegeben (siehe \Abbref{fig:ThermodynamicMaschine}). 

\begin{figure}[htbp]
    \centering
    \tfigThermodynamicMaschine
    \caption{Allgemeine thermodynamische Maschine: Eine Energiequelle (z.B. ein Ofen oder Dampfkessel) mit Temperatur $T_\mathrm{h}$ gibt Wärme an das Hilfssystem TS ab, welches die aufgenommene Energie in Form von Wärme an ein Kühlsystem mit der Temperatur $T_\mathrm{k}$ und in Form von mechanischer Arbeit an eine Maschinerie verteilt. }
    \label{fig:ThermodynamicMaschine}
\end{figure}

Nach Gleichung \eqref{eq:maschine_HS2} gilt für den reversiblen Ablauf
\begin{align*}
    \udiff Q_\mathrm{k} = \frac{T_\mathrm{k}}{T_\mathrm{h}} (-\udiff Q_\mathrm{h})
\end{align*}
und nach Gleichung \eqref{eq:maschine_HS1}
\begin{align*}
    \udiff W^\mathrm{RAQ} = \left( 1-\frac{T_\mathrm{k}}{T_\mathrm{h}} \right) (-\udiff Q_\mathrm{h}).
\end{align*}
Der Wirkungsgrad $\eta$ beschreibt das Verhältnis der Energie, die in (i.~d.~R.mechanische) Arbeit umgesetzt wird:
\begin{align*}
    \eta = \frac{\udiff W^\mathrm{RAQ}}{-\udiff Q_\mathrm{h}} = 1-\frac{T_\mathrm{k}}{T_\mathrm{h}}=\frac{T_\mathrm{h}-T_\mathrm{k}}{T_\mathrm{h}}. 
\end{align*}
Dieser Wirkungsgrad ist unabhängig von dem genauen Ablauf des Prozess und daher universell gültig. 

Der Wirkungsgrad wird verbessert, wenn die Temperatur $T_\mathrm{h}$ des heißen Wärmereservoirs hoch und die Temperatur $T_\mathrm{k}$ des kalten Reservoirs niedrig ist. Für $T_\mathrm{k}\rightarrow 0$ wird der Wirkungsgrad $1$. 
Dann ist $\udiff W^\mathrm{RAQ}=-\udiff Q_\mathrm{h}$. 
Betrachtet man sich die Gleichung \eqref{eq:maschine_HS2}, so sieht man, dass für Temperaturen nahe dem absoluten Nullpunkt nur wenig Wärme in das kalte Wärmereservoir fließen muss, damit die Entropieänderung des heißen Wärmereservoirs kompensiert wird (die gesamte Entropieänderung muss durch das kalte Wärmereservoir aufgenommen werden, da die Maschinerie als reversible Arbeitsquelle keine Wärme und damit auch keine Entropie aufnimmt). 

Sind die Temperaturen $T_\mathrm{h}$ und $T_\mathrm{k}$ gleich, so ist der Wirkungsgrad $0$. 
\begin{formal}
    Ein Prozess, dessen einziges Resultat ist, dass Wärme aus eine Wärme aus einer Wärmequelle in mechanische Arbeit umgewandelt wird, ist nicht möglich. Es gibt folglich kein Perpetuum mobile zweiter Art. 
\end{formal}


Im irreversiblen Fall ist der Umsatz in mechanische Arbeit sogar noch schlechter, 
\begin{align*}
    \Delta W_\mathrm{rev}^\mathrm{RAQ} > \Delta W_\mathrm{irr}^\mathrm{RAQ}
\end{align*}
und damit der Wirkungsgrad kleiner als im reversiblen Fall,
\begin{align*}
    \eta_\mathrm{rev}>\eta_\mathrm{irr}. 
\end{align*}



\paragraph*{Kühlschrank}

Beim Kühlschrank wird das oben beschriebene Prinzip der thermodynamischen Maschine umgekehrt. Mithilfe eines Motors wird über das Hilfssystem TS der Umgebung Wärme zugeführt. Dabei erhöht sich aber die Entropie der Umgebung. Für einen reversiblen Ablauf muss sich daher die Entropie und somit die Temperatur des ebenfalls an das Hilfssystem gekoppelten Kühlschranks verringern. 


\begin{figure}[htbp]
    \centering
    \tfigFridge
    \caption{Prinzip eines Kühlschranks: Einem Kühlschrank wird Wärme entzogen, indem ein Motor (reversible Arbeitsquelle) über ein Hilfssystem TS der Umgebung Wärme zuführt, deren Entropie sich dabei erhöht, sodass sich die Entropie des Kühlschranks verringern muss. }
    \label{fig:fridge}
\end{figure}

Idealerweise wird $-\udiff Q_\mathrm{k}$ maximal und $-\udiff W^\mathrm{RAQ}$ minimal (wegen der Stromrechnung). Der Wirkungsgrad der Kühlleistung ist gegeben durch 
\begin{align*}
    \eta = \frac{-\udiff Q_\mathrm{k}}{-\udiff W^\mathrm{RAQ}} = \frac{T_\mathrm{k}}{T_\mathrm{h}-T_\mathrm{k}}. 
\end{align*}
Ist die Umgebungstemperatur gleich der Kühlschranktemperatur, so geht die Effizienz gegen unendlich, da überhaupt keine Kühlleistung nötig ist. Für $T_\mathrm{k}\rightarrow 0$ geht $\eta$ gegen $0$, sodass der absolute Nullpunkt $T_\mathrm{k}$ nicht erreicht werden kann. 

Beim irreversiblen Ablauf ist wiederum
\begin{align*}
    \eta_\mathrm{rev}>\eta_\mathrm{irr}. 
\end{align*}


\paragraph*{Wärmepumpe}

Eine Wärmepumpe hat genau die gleiche Funktionsweise wie ein Kühlschrank, nur dass hier statt dem kalten das heiße System betrachtet wird. Der Wärmeleistungskoeffizient ist analog zum vorigen Abschnitt durch 
\begin{align*}
    \eta = \frac{T_\mathrm{h}}{T_\mathrm{h}-T_\mathrm{k}}
\end{align*}
gegeben, wobei jetzt im Zähler aber $T_\mathrm{h}$ anstelle von $T_\mathrm{k}$ steht. 



\section{Der Carnot-Zyklus}

Wir haben gesehen, dass eine thermodynamische Maschine Energie in Form von Wärme und Arbeit zwischen verschiedenen Wärmereservoirs und Arbeitsquellen transportiert. Dazu ist i.~d.~R. ein Hilfssystem nötig, dessen Aufgabe es ist, eine reversible Verteilung von $\Delta Q_\mathrm{h}$ auf $\Delta W^\mathrm{RAQ}$ und $\Delta Q_\mathrm{k}$ zu verwirklichen. Dabei darf sich das Hilfssystem selbst aber netto nicht verändern. 


Dies wird durch Kreisprozesse wie den Carnot-Zyklus realisiert. In \Abbref{fig:CarnotCycleIndicatorDiagram} ist für diesen das Indikatordiagramm als Temperatur über die Entropie dargestellt. Im Schritt AB findet eine isotherme Expansion statt, in der sich die Wärme um $\Delta Q_{AB}=(S_B-S_A)T_\mathrm{h} = \Delta S T_\mathrm{h}$ ändert . Im Schritt BC wird isentrop expandiert. Es folgt eine isotherme Kompression von C nach D, bei der die Wärme $\Delta Q_{CD}=-\Delta S T_\mathrm{k}$ abgegeben wird und zuletzt eine isentrope Kompression von D zurück nach A. Dabei wird ständig an die reversible Arbeitsquelle gekoppelt (Volumenarbeit durch Expansion und Kompression). 

Die verrichtete Arbeit ist nach dem ersten Hauptsatz der Thermodynamik durch 
\begin{align*}
    \Delta W^\mathrm{RAQ}=\Delta Q_{AB}+\Delta Q_{CD} = \Delta S(T_\mathrm{h}-T_\mathrm{k})
\end{align*}
und der Wirkungsgrad wie im vorigen Abschnitt beschrieben durch 
\begin{align}
    \nu = \frac{\Delta W^\mathrm{RAQ}}{\Delta Q_{AB}} = \frac{T_\mathrm{h}-T_\mathrm{k}}{T_\mathrm{h}}
\end{align}
gegeben. In der Realität wird aufgrund von mechanischer Reibung und durch Abweichung von quasistationären Zuständen jedoch meist nur ein Wirkungsgrad erzielt, der etwa einem Drittel des idealen Wirkungsgrades entspricht. 

\begin{figure}[htbp]
    \centering
    \tfigCarnotCycleIndicatorDiagram
    \caption{Der Carnot-Zyklus}
    \label{fig:CarnotCycleIndicatorDiagram}
\end{figure}

