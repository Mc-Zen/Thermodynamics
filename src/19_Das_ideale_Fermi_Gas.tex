% !TeX root = Theo_IV.tex
\chapter{Das ideale Fermi-Gas}
Betrachten identische, nicht-wechselwirkende Teilchen mit halbzahligem Spin 
Bsp.:
\section{Extensive thermodynamische Variable}
\paragraph*{Zustandsdichte}
Wir wollen ebene Wellen betrachten, welche eine Energie 
\begin{align*}
    .
\end{align*}
haben. Diese ist durch den Wellenvektor charakterisiert. Damit folgt für die Zustandsdichte:
\begin{align*}
    .
\end{align*}
\paragraph*{Großkanonisches Potential} Wir haben das große Potential bereits als Summe über diskrete Zustände formuliert.  Nun schreiben wir es als Integral über kontinuierliche Zustände als:
\begin{align*}
    \Omega&=-k_B T \int_0^\infty \ln \left[1+e^{\beta (\mu - \epsilon)}\right]D(\epsilon)d\epsilon\\
    &=-k_B T D_0 \int_0^\infty \ln \left[1+e^{\beta (\mu - \epsilon)}\right]\sqrt{\epsilon}d\epsilon.
\end{align*}
\paragraph*{Mittlere Teilchenzahl} Wir erinnern an die mittlere Besetzungszahl eines Zustandes $k$ mit der Energie $\epsilon_k$:
\begin{align*}
    N_k^{FD}=\frac{1}{e^{\beta(\epsilon_k-\mu)}+1}.
\end{align*}
Wir charakterisieren die mittlere Besetzungszahl für eine kontinuierliche Energieverteilung mit Hilfer der Fermi-Verteilungsfunktion:
\begin{align*}
    f(\epsilon).
\end{align*}
Damit folgt für die Besetzungszahl
\begin{align*}
    beide Gleichungen .
\end{align*}
\paragraph*{Innere Energie} Für die mittlere innere Energie $U=\langle\epsilon\rangle$ ergibt sich mit der mittleren Besetzungszahl:
\begin{align*}
    .
\end{align*}
Wir wollen das große Potential ausgehend von der inneren Energie berechnen.
Dazu integrieren wir partiell mit zugehörigen Faktoren 
\begin{align*}
   a(\epsilon)&=\epsilon^{3/2}\\
   b'(\epsilon)&= \\
   b(\epsilon)&=:
\end{align*}
\begin{align*}
    U&=D_0 a(\epsilon)b(\epsilon)|_0^\infty+\frac{3}{2}D_0\int\sqrt{\epsilon}\frac{1}{\beta}\ln\left[e^{\beta(\mu-\epsilon)}+1\right]d\epsilon\\
    &=0-\frac{3}{2}\Omega
\end{align*}
Nach dem großen Potential umgestellt erhalten wir 
\begin{align*}
    \Omega=-\frac{2}{3}U=-PV
\end{align*}
und damit 
\begin{align*}
    P=\frac{2}{3}\frac{U}{V},
\end{align*}
die Zustandsgleichung für das Fermigas.
Wir vergleichen das Ergebnis mit den klassischen idealen Gasen, für welche 
\begin{align*}
    PV=N k_B T
\end{align*}
gilt. Diese Gleichung wird durch $N k_B T=2U/3$ eingeschlossen.
\section{Klassischer Grenzfall}
Im klassischen Grenzfall ist die Fugazität $e^{\beta \mu}$ sehr viel kleiner als eins. Die Erklärung dieses Tatbestandes folgt später. Damit folgt für die Verteilungsfunktion ein neuer Ausdruck:
\begin{align*}
    ,
\end{align*}
die korrigierte Boltzmann-Statistik. Arbeiten wir mit der korrigierten Boltzmann-Statistik folgt für die mittlere Teilchenzahl
\begin{align*}
    .
\end{align*}
mit der de-Broglie'schen Wellenlänge $\lambda_B$. Damit beschreibt $\lambda_B^3$ im übertragenen Sinne das Volumen, welches ein Teilchen einnimmt.
Die Quantenkonzentration $n_Q=\lambda_B^{-3}$
Kriterium für den klassischen Grenzfall: Wenn die Dichte $n$ sehr viel kleiner ist als $n_Q$ spüren die Teilchen ihren Quantencharakter nicht.???
\section{Quanten-Grenzfall}
Den Quanten-Grenzfall beispielsweise das entartete Elektronengas verhält sich anders als das ideale Gas.
Für das Elektron gibt es zwei Einstellungsmöglichkeiten: [F] und eine Dichte:[F]
Diese seien feste Parameter.
Das Fermi-Gas bei $T=0$
Für $T=0$ läuft $\beta$ gegen unendlich. Um die Verteilungsfunktion für diesen Fall auszuwerten nehmen wir eine Fallunterscheidung vor:
\begin{align*}
    .
\end{align*} 
Zustände mit $\mu>\epsilon$ sind vollständig besetzt, alle anderen Zustände mit $\epsilon>\mu$ sind nicht besetzt.
Für $T=0$ sind also alle Zustände bis $\epsilon_F=\mu(T=0,N,V)$ mit maximal einem Teilchen pro Zustand besetzt. Alle Zustände darüber sind unbesetzt, weshalb $\epsilon_F=\mu$ auch als Fermienergie oder Fermikante bezeichnet wird.

Wir wollen $\mu$ berechnen.
Ausgehend von der Teilchenzahl erhalten wir:
\begin{align*}
    N=\int_0^{\mu_0} D(\epsilon)d\epsilon\\
    =\frac{2V}{4\pi^2}\left(\frac{2m}{\hbar^2}\right)^{3/2}\int_0 ^{\mu_0} \sqrt{\epsilon}d\epsilon\\
    =\frac{V}{3\pi^2}\left(\frac{2m}{\hbar^2}\right)^{3/2}\mu_0 ^{3/2}
\end{align*}
mit $n=N/V$ folgt damit für $\mu_0$:
\begin{align*}
    \mu_0=:\epsilon_F=:k_B T_F=\frac{\hbar^2}{2m}\left(3\pi^2 n\right)^{2/3}.
\end{align*}
Wir illustrieren dies durch folgendes Beispiel.
Mit für Elektronen typischen Werten erhalten wir eine Fermi-Temperatur von $T_F\thickapprox 10^5 \si{K}$.

Berechnen wir die Teilchenzahl in Abhängigkeit von $\mu$, so können wir $\mu$ in Abhängigkeit einer Teilchendichte ($n=N/V$) schreiben.
\begin{align*}
    N
\end{align*}
Entwickeln wir nach $T/T_F$ erhalten wir zwei neue Ausdrücke für $\mu$ und $U$:
\begin{align*}
    .
\end{align*}
