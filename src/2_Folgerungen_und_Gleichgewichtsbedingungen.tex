% !TeX root = Theo_IV.tex

\section{Folgerungen und Gleichgewichtsbedingungen}

In diesem Kapitel soll die Entwicklung der thermodynamischen Theorie sowie die Auswertung des zweiten Postulates erfolgen. 

\subsection{Konjugierte Variablen (Energiedarstellung)}

Zunächst wollen wir die innere Energie in der differentiellen Form aufschreiben: 
\begin{align*}
    \diff U = \left(\frac{\partial U}{\partial S}\right)_{V,\sm_1,\ldots,\sm_r} \diff S + \left(\frac{\partial U}{\partial V}\right)_{S,\sm_1,\ldots,\sm_r} \diff V + \sum_{j = 1}^r \left(\frac{\partial U}{\partial \sm_j}\right)_{S,V,\sm_1,\ldots,\sm_{j - 1},\sm_{j + 1},\ldots,\sm_r} \diff \sm_j
\end{align*}
Die Faktoren vor den einzelnen Differentialen müssen nun gedeutet werden, indem sie als Größen definiert werden, die jeweils konjugiert zu der Größe aus dem Differential sind. 

Wir erhalten die Temperatur 
\begin{align}
    \label{eq:def_temperatur}
    T \equiv \left(\frac{\partial U}{\partial S}\right)_{V,\sm_1,\ldots,\sm_r} \geq 0
\end{align}
als konjugierte Größe zu $S$, den Druck 
\begin{align}
    \label{eq:def_druck}
    P \equiv -\left(\frac{\partial U}{\partial V}\right)_{S,\sm_1,\ldots,\sm_r} 
\end{align}
als konjugierte Größe zu $V$ und die chemischen Potentiale
\begin{align}
    \label{eq:def_chemische_potentiale}
    \mu_j \equiv \sum_{j = 1}^r \left(\frac{\partial U}{\partial \sm_j}\right)_{S,V,\sm_1,\ldots,\sm_{j - 1},\sm_{j + 1},\ldots,\sm_r},
\end{align}
die jeweils zu $\sm_j$ konjugiert sind. 

Die innere Energie lässt sich also in differentieller Form folgendermaßen schreiben:
\begin{align}
    \label{eq:innere_energie_differentielle_form}
    \diff U = T\diff S - P \diff V + \mu_1 \diff \sm_1 + \cdots + \mu_r \diff \sm_r
\end{align}

Bemerkungen:
\begin{itemize}
    \item Die Größen $U, S,V,\sm_k$ sind extensiv, während die dazu konjugierten Variablen $T,P,\mu_k$ intensive Größen sind. 
    \item $J\diff X$ mit intensiver Variable $J$ und extensiver Variable $X$ muss die Einheit einer Energie haben. Folglich ist $[\mu_j] = [U]$ und $[S]=[U/T]$. 
\end{itemize}

Es soll nun eine erste Interpretation der Terme in \eqref{eq:innere_energie_differentielle_form} gemacht werden. 

Zur Vereinfachung setzen wir zunächst alle Terme $\diff \sm_k=0$. 
Der erste Term ist 
\begin{align*}
    T\diff S = \diff U+ P\diff V = \udiff Q,
\end{align*}
entspricht also der quasistatischen Wärmezufuhr $\udiff Q$. 

Der zweite Term $-P\diff V$ entspricht der bereits eingeführten quasistatischen mechanischen Arbeit $\udiff W_\mathrm{mech}=-P\diff V$.


Die Deutung als Temperatur wird später weiter ausgeführt (siehe Kapitel ??[2.5])

Zuletzt können wir die Terme 
\begin{align*}
    \udiff W_ \mathrm{C} = \sum_{i = 1}^r \mu_i\diff \sm_i
\end{align*}
als quasistatische chemische Arbeit festlegen. Sie beschreibt die Energiezunahme beim Hinzufügen von Materie zu dem System. 

Insgesamt kann die innere Energie also geschrieben werden als
\begin{align}
    \label{eq:erster_HS_TD}
    \diff U = \udiff Q + \udiff W_\mathrm{mech} + \udiff W_\mathrm{C}.
\end{align}
Das ist der erste Hauptsatz der Thermodyanmik. 
Dabei ist zu beachten, dass ... unvollständige Differentiale sind. 

Zuletzt soll noch einige Aussagen zu $\udiff Q$ zusammengefasst werden:
\begin{itemize}
    \item Für quasistatische Prozesse gilt $\udiff Q=T\diff S$ und damit insbesondere $\udiff Q>0\implication T\diff S>0$. Eine Zufuhr von Wärme führt also zu einem Zuwachs der Entropie. 
    \item Durch Umstellen erhält man 
    \begin{align*}
        \diff S = \frac{1}{T}\udiff Q.
    \end{align*}
    Diese Gleichung kann so gedeutet werden, dass durch einen sogenannten integrierenden Faktor \textendash{} hier $1/T$ \textendash{} aus einem unvollständigen Differential ein vollständiges wird. 
    \item Nicht quasistatische Prozesse sind solche, bei denen sich das System im Nicht-Gleichgewicht entwickelt. So ist beim abgeschlossenen System z.B. $\udiff Q=0$ und folglich nach dem zweiten Postulat $\diff S >0$. Es handelt sich also um einen irreversiblen Prozess. Für ein nicht abgeschlossenenes System kann $\udiff Q\neq 0$ sein, sodass 
    \begin{align*}
        \diff S\geq \frac{\udiff Q}{T}. 
    \end{align*}
    Nur für Gleichheit ist dieser Prozess reversibel. 
\end{itemize}

