% !TeX root = Theo_IV.tex

\section{Folgerungen und Gleichgewichtsbedingungen}

In diesem Kapitel soll die Entwicklung der thermodynamischen Theorie sowie die Auswertung des zweiten Postulates erfolgen.

\subsection{Konjugierte Variablen (Energiedarstellung)}

Zunächst wollen wir die innere Energie in der differentiellen Form aufschreiben:
\begin{align*}
    \diff U = \left(\frac{\partial U}{\partial S}\right)_{V,\sm_1,\ldots,\sm_r} \diff S + \left(\frac{\partial U}{\partial V}\right)_{S,\sm_1,\ldots,\sm_r} \diff V + \sum_{j = 1}^r \left(\frac{\partial U}{\partial \sm_j}\right)_{S,V,\sm_1,\ldots,\sm_{j - 1},\sm_{j + 1},\ldots,\sm_r} \diff \sm_j
\end{align*}
Die Faktoren vor den einzelnen Differentialen müssen nun gedeutet werden, indem sie als Größen definiert werden, die jeweils konjugiert zu der Größe aus dem Differential sind.

Wir erhalten die Temperatur
\begin{align}
    \label{eq:def_temperatur}
    T \equiv \left(\frac{\partial U}{\partial S}\right)_{V,\sm_1,\ldots,\sm_r} \geq 0
\end{align}
als konjugierte Größe zu $S$, den Druck
\begin{align}
    \label{eq:def_druck}
    P \equiv -\left(\frac{\partial U}{\partial V}\right)_{S,\sm_1,\ldots,\sm_r}
\end{align}
als konjugierte Größe zu $V$ und die chemischen Potentiale
\begin{align}
    \label{eq:def_chemische_potentiale}
    \mu_j \equiv \sum_{j = 1}^r \left(\frac{\partial U}{\partial \sm_j}\right)_{S,V,\sm_1,\ldots,\sm_{j - 1},\sm_{j + 1},\ldots,\sm_r},
\end{align}
die jeweils zu $\sm_j$ konjugiert sind.

Die innere Energie lässt sich also in differentieller Form folgendermaßen schreiben:
\begin{align}
    \label{eq:innere_energie_differentielle_form}
    \diff U = T\diff S - P \diff V + \mu_1 \diff \sm_1 + \cdots + \mu_r \diff \sm_r
\end{align}

Bemerkungen:
\begin{itemize}
    \item Die Größen $U, S,V,\sm_k$ sind extensiv, während die dazu konjugierten Variablen $T,P,\mu_k$ intensive Größen sind.
    \item $J\diff X$ mit intensiver Variable $J$ und extensiver Variable $X$ muss die Einheit einer Energie haben. Folglich ist $[\mu_j] = [U]$ und $[S]=[U/T]$.
\end{itemize}

Es soll nun eine erste Interpretation der Terme in \eqref{eq:innere_energie_differentielle_form} gemacht werden.

Zur Vereinfachung setzen wir zunächst alle Terme $\diff \sm_k=0$.
Der erste Term ist
\begin{align*}
    T\diff S = \diff U+ P\diff V = \udiff Q,
\end{align*}
entspricht also der quasistatischen Wärmezufuhr $\udiff Q$.

Der zweite Term $-P\diff V$ entspricht der bereits eingeführten quasistatischen mechanischen Arbeit $\udiff W_\mathrm{mech}=-P\diff V$.


Die Deutung als Temperatur wird später weiter ausgeführt (siehe Kapitel \ref{sec:thermisches_gleichgewicht})

Zuletzt können wir die Terme
\begin{align*}
    \udiff W_ \mathrm{C} = \sum_{i = 1}^r \mu_i\diff \sm_i
\end{align*}
als quasistatische chemische Arbeit festlegen. Sie beschreibt die Energiezunahme beim Hinzufügen von Materie zu dem System (siehe auch Kapitel \ref{sec:gleichgewicht_bei_materiefluss}).

Insgesamt kann die innere Energie also geschrieben werden als
\begin{align}
    \label{eq:erster_HS_TD}
    \diff U = \udiff Q + \udiff W_\mathrm{mech} + \udiff W_\mathrm{C}.
\end{align}
Das ist der erste Hauptsatz der Thermodyanmik. 
Dabei ist zu beachten, dass die Prozessgrößen ($Q$,$W_{mech}$ und $W_c$) unvollständige Differentiale bilden. Letztere sollen im nächsten Kapitel näher erläutert werden. 

Zuletzt soll noch einige Aussagen zur Wärmezufuhr $\udiff Q$ zusammengefasst werden:
\begin{itemize}
    \item Für quasistatische Prozesse gilt $\udiff Q=T\diff S$ und damit insbesondere $\udiff Q>0\implication T\diff S>0$. Eine Zufuhr von Wärme führt also zu einem Zuwachs der Entropie. 
    \item Durch Umstellen erhält man 
    \begin{align*}
        \diff S = \frac{1}{T}\udiff Q.
    \end{align*}
    Da das unvollständige Differential $\udiff Q$ mittels des Faktors $1/T$ das vollständige Differential $\diff S$ charakterisiert, nennt man diesen Faktor einen integrierenden Faktor.
    \item Betrachten wir die Wärmezufuhr für nicht quasistatische Prozesse, so heißt das, dass das System keine Gleichgewichtszustände durchläuft, sondern sich im Nichtgleichgewicht entwickelt. Für ein abgeschlossenes System liegt keine Wärmezufuhr vor. Mit $\udiff Q=0$ folgt nach Postulat \ref{post:entropie_maximierung} $\diff S >0$. Die Entropie nimmt für sogenannte irreversible Prozesse zu.\footnote[4]{Irreversibel heißt hier, dass das System nicht mehr spontan in den Ausgangszustand übergeht, also die Entropie nicht abnimmt.} In nicht abgeschlossenenen Systemen, also in Systemen, in denen eine Wärmezufuhr stattfinden kann, nimmt die Entropie für irreversible Prozesse entsprechend zu.
    \begin{align*}
        \diff S\geq \frac{\udiff Q}{T}
    \end{align*}
    Gleichheit gilt dabei ausschließlich für reversible Prozesse. Es sei angemerkt, dass damit ferner folgt, dass die Entropiezunahme auch auf Anteile zurückzuführen ist, die nicht mit der Wärmezufuhr $\udiff Q$ zusammenhängen.
\end{itemize}

\subsection{Unvollständige Differentiale}

Wir wollen nun näher auf die mehrfach erwähnten unvollständigen Differentiale zurückkommen und eine mathematische Definition nachliefern. Ausgehend von einer Funktion $f(x_1,...,x_r)$ bilden wir das Differential 
\begin{align}
    \label{eq:erster_HS_TD}
    \diff f = \sum_{i=1}^r \frac{\partial f}{\partial x_i} \diff x_i.
\end{align}
Für die zweiten Ableitungen gilt für die gemischten Differentiale mit dem Satz von Schwarz Vertauschbarkeit der Form:
\begin{align}
    \frac{\partial ^2 f}{\partial x_i \partial x_j} = \frac{\partial ^2 f}{\partial x_j \partial x_i}.
\end{align}
Wir konstruieren eine Darstellung $\phi$, auch Pfaffsche oder 1-Form genannt
\begin{align}
    \phi (x_1,...,x_r)=\sum _{i=1}^r \phi_i(x_1,...,x_r)\diff x_i
\end{align}
mit dem Differential 
\begin{align}
    \diff f = \phi = \sum_i \phi_i \diff x_i \text{ mit } \phi_i = \frac{\partial f}{\partial x_i}. 
\end{align}
Für einfach zusammenhängende Definitionsbereiche \footnote[5]{Es sei angemerkt, dass für nicht zusammmenhängende Gebiete die Integrabilitätsbedingung jedoch nur eine notwendige Bedingung ist.} ist die Erfüllung des Satzes von Schwarz in Form folgender Integrabilitätsbedingung
\begin{align}
    \frac{\partial \phi_i}{\partial x_j}= [\frac{\partial^2f}{\partial x_j\partial x_i}=\frac{\partial^2f}{\partial x_i\partial x_j}]=\frac{\partial \phi_j}{\partial x_i}
\end{align}
hinreichend dafür, dass dieses Differential vollständig ist. 
Wir wollen daran erinnern, dass uns diese Integrabilitätsbedingung bereits aus der Mechanik bekannt ist. Dort ist ein betrachtetes Kraftintegral genau dann wegunabhängig, wenn die Kraft als negativer Gradient eines Potentials geschrieben werden kann. In einfach zusammenhängenden Gebieten ist dies gleichbedeutend mit der Aussage, dass die Rotation der Kraft gleich null ist.
\begin{align}
    \vec{K}=-\nabla U \leftrightarrow \text{rot}\vec{K}=0 \leftrightarrow \epsilon_{ijk}\partial_jK_k
\end{align} 
Der letzte Ausdruck ist äquivalent zu unserer Integrabilitätsbedingung.
Erfüllt das Differential diese Bedingung nicht, ist es unvollständig und wird mittels der Notation $\udiff$ gekennzeichnet.
Wir wollen dies am Beispiel der Variablen $S$ und $V$ illustrieren:
\begin{align}
    \phi = \udiff Q = T\diff S+0 \diff V\equiv \phi_1\diff S+\phi_2 \diff V
\end{align}
Wir identifizieren die zwei Funktionen $\phi_1$ und $\phi_2$ und prüfen die Integrabilitätsbedingung
\begin{align}
    \frac{\partial \phi_1}{\partial V} = \frac{\partial T}{\partial V}\neq \frac{\partial \phi_2}{\partial S}=\frac{\partial0}{\partial S}=0
\end{align}
und stellen fest, dass diese nicht erfüllt ist. Die Wärmezustandsfunktion $Q(S,V)$ existiert nicht, stattdessen definiert $Q$ als Prozessgröße ein unvollständiges Differential.

\subsection{Zustandsgleichungen (in Energiedarstellung)}
Wir möchten nun weitere Definitionen einführen und den Begriff der \textit{Zustandsgleichung} erläutern. Dafür erinnern wir an das Energiedifferential der Form 
\begin{align}
    \diff U = T\diff S -P\diff V + \sum\mu_j\diff N_j,
\end{align}
welches von den intensiven Variablen $T$,$P$ un $\mu_j$ mit ihren respektiven Zustandsgleichungen 
\begin{align}
    T=T(S,V,N_1,...,N_r)\\
    P=P(S,V,N_1,...,N_r)\\
    \mu_j=\mu_j(S,V,N_1,...,N_r)
\end{align}
abhängen. Die Kenntnis dieser Zustandsgleichungen ist äquivalent zur Kenntnis der einzelnen Zustandsfunktion $U(S,V,N_1,...,N_r)$, da wir die innere Energie über Aufintegration des Differentials erhalten.

Kommen wir erneut auf die Definition der extensiven und intensiven Größen zurück. Eine extensive Variable, wie z.B. die innere Energie, ist eine homogene Funktion ersten Grades, d.h. es gilt folgender Zusammenhang
\begin{align}
    U(\lambda S,\lambda V,\lambda N_1,...,\lambda N_r) = \lambda U(S,V,N_1,...,N_r).
\end{align}
In anderen Worten, die Variable skaliert mit der größe des Systems.
Für intensive Variablen, wie z.B. die Temperatur, gilt jedoch 
\begin{align}
    T(\lambda S,\lambda V,\lambda N_1,...,\lambda N_r)=T(S,V,N_1,...,N_r),
\end{align}
denn mit der Extensivität von $U$ folgt 
\begin{align}
    T(\lambda S,...) = \frac{\partial U(\lambda S,...)}{\partial(\lambda S)}=\frac{\partial U(S,...)}{\partial S}=T(S,...).
\end{align}
Es handelt sich dabei, wie auch beim Druck und beim chemischen Potential, um eine homogene Funktion nullten Grades, die sich bei Skalierung des Systems nicht ändert.

Wir wollen uns nun zwei Verallgemeinerungen anschauen, in entropischer und energetischer Darstellung.
Die Energiedarstellung kann wie folgt geschrieben werden:
\begin{align}
    U(S,V,N_1,...,N_r) \rightarrow U(S,X_1,X_2,...,X_t)
\end{align} 
Wir benutzen also verallgemeinerte energetisch extensive Parameter um die Fundamentalgleichung zu schreiben. Dabei erhalten wir die verallgemeinerten Ableitungen und das Energiedifferential
\begin{align}
    (\frac{\partial U}{\partial S})_{X_1,...,X_t} \equiv T = T(S,X_1,...,X_t)\\
    (\frac{\partial U}{\partial X_j})_{S,...,X_{k\neq j},...} \equiv P_j = P_j(S,X_1,...,X_t)\\
    \rightarrow \diff U = T\diff S + \sum_{j=1}^tP_j\diff X_j.
\end{align}
Diese Abstraktion ermöglicht die Beschreibung ganz anderer Systeme, für welche weiterhin das eingeführte Energie- bzw. Arbeitsdifferential $\diff U = P_j\diff X_j$ gilt.
Dabei bezeichnen $X_j$ die verallgemeinerten Volumina oder Wege und $P_j$ Drücke oder Kräfte.
Anmerkung: Den Druck erhalten wir folglich nur über eine Energiefunktion, welche von S abhängt.

Betrachtet man ein einkomponentiges System, so liegt die Einführung neuer Größen, wie der molaren inneren Energie $u$ nahe, welche wir nun einführen wollen:
\begin{align}
    u=\frac{U}{N}=u(\frac{S}{N},\frac{V}{N},\frac{N}{N})\equiv u(s,v)
\end{align}
Sie hängt folglich von der molaren Entropie $s$ und dem Molvolumen $v$ ab.
Wir wollen nun das Differential dieser Größe berechnen. Formal schreiben wir
\begin{align}
    \diff u = (\frac{\partial u}{\partial s})_v \diff s + (\frac{\partial u}{\partial v})_s\diff v
\end{align}
mit
\begin{align}
    (\frac{\partial u}{\partial s})_v=(\frac{\partial u}{\partial s})_{V,N}=(\frac{\partial U}{\partial S})_{V,N}=T\\
    (\frac{\partial u}{\partial v})_s = -P.
\end{align}

Das Konstanthalten des Molvolumens ist dabei äquivalent zum Konstanthalten des Volumens und der Mohlzahl (Stoffmenge). Die partielle Ableitung nach $s$ kann ferner nach $S$ umgeschrieben werden, womit wir auf die Temperatur $T$ kommen.
Analog betrachten wir die partielle Ableitung nach $v$ und erhalten den negativen Druck $P$.
Insgesamt erhalten wir das Differential
\begin{align}
    \diff u = T\diff s -P\diff v.
\end{align}
Bei einem einkomponentigem System reicht also die Kenntnis über die molaren Größen, um das gesamte System zu beschreiben.

\subsection{Entropiedarstellung}
Analog zur Energiedarstellung betrachten wir nun eine verallgemeinerte Entropiedarstellung.
\begin{align}
    S(U,V,N_1,...,N_t) \rightarrow S(X_0,X_1,...,X_t)
\end{align}
Dabei benutzen wir wieder verallgemeinerte, diesmal entropische, extensive Parameter, um die entropische Fundamentalgleichung aufzuschreiben. Analog zur vorigen energetischen Beschreibung erhalten wir das Differential der Form
\begin{align}
    \diff S = \sum_{j=0}^tF_j\diff X_j \text{ mit }F_j:=(\frac{\partial S}{\partial X_j})_{...X_{k\neq j}}
\end{align}
mit den entropischen, intensiven Parametern
\begin{align}
    F_0=\frac{1}{T}(U,X_1,...,X_t)\\
    F_1=\frac{P}{T}(U,X_1,...,X_t)\text{ (für $X_1=V$)}\\
    F_k=-\frac{P_k}{T}(U,X_1,...,X_t)\\
    F_r=-\frac{\mu_r}{T}(U,X_1,...,X_t) \text{ (für $X_r=N_r$)}. 
\end{align}
Die energetische und entropische Beschreibung der Thermodyanmik sind äquivalent, jedoch werden wir bei Gleichgewichtsbetrachtungen, die aus dem zweiten Postulat folgen, mit der entropischen Fundamentalbeziehung arbeiten.

\subsection{Thermisches Gleichgewicht\label{sec:thermisches_gleichgewicht}}

Nun soll es darum gehen, den Inhalt der Postulate \ref{post:entropie_maximierung} und \ref{post:eigenschaften_entropie} auszuwerten.
Es wird folgen, dass sich $T$ so verhält, wie man es von einer Temperatur erwartet.

\paragraph*{Temperatur}

Wir starten wieder mit einem Modellsystem, das insgesamt abgeschlossen ist und aus zwei Untersystemen $(1)$ und $(2)$ besteht, wie in \Abbref{fig:DoppelsystemUSfesteWaermeleitendeWand} darstellt.
Die beiden Untersysteme sind durch eine feste Wand getrennt, die zuerst isoliert ist und dann wärmeleitend wird. Die beiden Systeme werden also in thermischen Kontakt gebracht und tauschen Wärme aus.

\begin{figure}[htbp]
    \centering
    \tfigDoppelsystemUSfesteWaermeleitendeWand
    \caption{Abgeschlossenes System aus zwei Untersystemen, die durch eine feste und materieundurchlässige Wand getrennt sind. Die Wand ist zunächst isolierend und wird dann wärmeleitend.}
    \label{fig:DoppelsystemUSfesteWaermeleitendeWand}
\end{figure}

Wir würden dabei erwarten, dass sich die Temperaturen angleichen, $T^{(1)}=T^{(2)}$.

Nach den Postulaten muss für ein abgeschlossenes System $U^{(1)}+U^{(2)} = \mathrm{const}$ bzw. $\diff U^{(1)} = -\diff U^{(2)}$ sein. Das Postulat \ref{post:entropie_maximierung} besagt jetzt, dass sich $U^{(1)}$ und $U^{(2)}$ so einstellen, dass $S$ ein Maximum annimmt, $\diff S=0$. Damit auch das Postulat \ref{post:eigenschaften_entropie} erfüllt ist, muss gelten, dass
\begin{align*}
    S=S^{(1)}\left(U^{(1)},V^{(1)},\sm^{(1)}_k\right) + S^{(2)}\left(U^{(2)},V^{(2)},\sm^{(2)}_k\right).
\end{align*}
Es ist also
\begin{align*}
    \diff S = \frac{\partial S^{(1)}}{\partial U^{(1)}}\diff U^{(1)}+ \frac{\partial S^{(2)}}{\partial U^{(2)}}\diff U^{(2)} = \frac{1}{T^{(1)}}\diff U^{(1)} + \frac{1}{T^{(2)}}\diff U^{(2)},
\end{align*}
da die Volumina und Stoffmengen konstant sind und wegen $\diff U^{(1)} = -\diff U^{(2)}$ ist
\begin{align*}
    \diff S = \left(\frac{1}{T^{(1)}}-\frac{1}{T^{(2)}}\right)\diff U^{(1)} \overset{!}{=} 0.
\end{align*}
Im thermischen Gleichgewicht gilt folglich
\begin{align}
    \label{eq:thermisches_gg_temperatur}
    T^{(1)} = T^{(2)},
\end{align}
wie erwartet. Aus dem Postulat \ref{post:eigenschaften_entropie} folgt auch, dass die Temperatur positiv ist, denn $S$ soll eine monoton ansteigende Funktion von $U$ sein, sodass $\partial S/\partial U > 0$.

Diese Definition der Temperatur als Inverse der Ableitung der Entropie nach der inneren Energie mag zwar ein wenig abstrakt erscheinen, doch gibt es auch andere gleichbedeutende Definitionen der Temperatur, die aber nicht weniger abstrakt sind\footnote{Ein anderer Ansatz wäre, als nullten Hauptsatz die Transitivität der Temperatur zu postulieren \cite{lit:nolting1},
    \begin{align*}
        T^{(1)} = T^{(2)} \quad \text{und}\quad T^{(1)} = T^{(3)} \implication T^{(2)} = T^{(3)}.
    \end{align*}
    Eine weitere Formulierung, in der $1/T$ als integrierender Faktor festgelegt wird (sodass $\diff S=\udiff Q/T$), wurde von Kelvin und Caradathory vorgeschlagen.
    Beide Varianten sind in dem hier gewählten Zugang bereits in den Postulaten \ref{post:entropie_maximierung} und \ref{post:eigenschaften_entropie} enthalten.
}.

\begin{formal}
    Es existiert also eine absolute Temperaturskala. Eine solche ist die Kelvin-Skala, die so definiert ist, dass der Tripelpunkt, also die Koexistenz von Eis, flüssigem Wasser und Wasserdampf, bei \SI{271,16}{\kelvin} liegt.
\end{formal}

Wir haben gesehen, dass die Entropie ein Maximum annimmt. Daraus kann man schließen, dass die zweite Ableitung der Entropie dort kleiner als $0$ ist.

\paragraph*{Wärmefluss}

Wir wissen intuitiv, dass die Wärme von Bereichen hoher Temperatur zu Bereichen niedrigerer Temperatur fließt. Startet man bei einem Anfangszustand mit $T^{(2)}> T^{(1)}$ und hebt dann die Zwangsbedingung auf, kommt es zu einem (quasistatischen) Wärmefluss.
Wegen des Postulats \ref{post:entropie_maximierung} ist
\begin{align*}
    \Delta S= \left(\frac{1}{T^{(1)}}-\frac{1}{T^{(2)}}\right)\Delta U^{(1)} > 0.
\end{align*}
Da aber
\begin{align*}
    T^{(2)}> T^{(1)} \equivalence \frac{1}{T^{(1)}}-\frac{1}{T^{(2)}} < 0
\end{align*}
ist, muss $\Delta U^{(1)} <0$ sein.
\begin{formal}
    Der Wärmefluss findet erwartungsgemäß vom System höherer zum System tieferer Temperatur statt, bis sich beide Temperaturen angeglichen haben.
\end{formal}
Dann ist das Maximum der Entropie erreicht (siehe \Abbref{fig:FunktionEntropieMaximum}).

\begin{figure}[htbp]
    \centering
    \tfigFunktionEntropieMaximum
    \caption{Entropie über innere Energie: Beim thermodynamischen Gleichgewicht nimmt die Entropie ihr Maximum an. Die Änderung $\Delta U^{(1)}$ ist negativ für $\Delta S>0$. }
    \label{fig:FunktionEntropieMaximum}
\end{figure}




\subsection{Mechanisches Gleichgewicht}


\begin{figure}[htbp]
    \centering
    \tfigDoppelsystemUVNbeweglicheWaermeleitendeWand
    \caption{Abgeschlossenes System aus zwei Untersystemen, die durch eine materieundurchlässige Wand getrennt sind. Die Wand ist zunächst fest und isolierend und wird dann beweglich und wärmeleitend.}
    \label{fig:DoppelsystemUVNbeweglicheWaermeleitendeWand}
\end{figure}


Als Nächstes soll ein insgesamt abgeschlossenes System aus zwei Untersystemen behandelt werden, bei dem ein materieundurchlässiger Kolben zuerst fest und isolierend, dann aber beweglich und wärmeleitend ist (siehe \Abbref{fig:DoppelsystemUVNbeweglicheWaermeleitendeWand}). Da es sich um ein abgeschlossenes System konstanten Gesamtvolumens handelt, ist
\begin{align*}
    U^{(1)} + U^{(2)} & = \mathrm{const}  \\
    V^{(1)} + V^{(2)} & = \mathrm{const}.
\end{align*}
Nach den Postulaten \ref{post:entropie_maximierung} und \ref{post:eigenschaften_entropie} ist
\begin{align*}
    \diff S & = \frac{\partial S^{(1)}}{\partial U^{(1)}}\diff U^{(1)} + \frac{\partial S^{(1)}}{\partial V^{(1)}}\diff V^{(1)}+\frac{\partial S^{(2)}}{\partial U^{(2)}}\diff U^{(2)} + \frac{\partial S^{(2)}}{\partial V^{(2)}}\diff V^{(2)} \\
            & = \left(\frac{1}{T^{(1)}}-\frac{1}{T^{(2)}}\right)\diff U^{(1)} + \left(\frac{P^{(1)}}{T^{(1)}}-\frac{P^{(2)}}{T^{(2)}}\right) \diff V^{(1)} = 0.
\end{align*}
Im mechanischen Gleichgewicht gilt also
\begin{align*}
    \frac{1}{T^{(1)}} = \frac{1}{T^{(2)}} , \quad \frac{P^{(1)}}{T^{(1)}}=\frac{P^{(2)}}{T^{(2)}},
\end{align*}
bzw.
\begin{align*}
    T^{(1)} = T^{(2)}, \quad P^{(1)} = P^{(2)}.
\end{align*}
Die hier diskutierten Gleichgewichtsbedingungen mögen trivial erscheinen, doch geht es hier vorrangig um das Testen des Formalismus und dann die anschließende Anwendung auf komplexere Systeme.


\subsection{Gleichgewicht bei Materiefluss\label{sec:gleichgewicht_bei_materiefluss}}

\paragraph*{Chemisches Potential}

Analog zu den vorigen Beispielen wird ein abgeschlossenes System mit zwei Untersystemen betrachtet (siehe \Abbref{fig:DoppelsystemUVNbeweglicheIsolierendeWand}). Diesmal ist die Wand zwar fest, aber wärmeleitend und durchlässig für eine Molekülsorte.

\begin{figure}[htbp]
    \centering
    \tfigDoppelsystemUVNbeweglicheIsolierendeWand
    \caption{Abgeschlossenes System aus zwei Untersystemen, die durch eine feste Wand getrennt sind. Die Wand ist zunächst isolierend und undurchlässig und wird dann wärmeleitend und durchlässig für die Molekülsorte 1.}
    \label{fig:DoppelsystemUVNbeweglicheIsolierendeWand}
\end{figure}

Es gilt
\begin{align*}
    U^{(1)} + U^{(2)}         & = \mathrm{const} \\
    \sm_1^{(1)} + \sm_1^{(2)} & = \mathrm{const}
\end{align*}
und damit
\begin{align*}
    \diff S & = \frac{1}{T^{(1)}}\diff U^{(1)} - \frac{\mu_1^{(1)}}{T^{(1)}}\diff \sm^{(1)}+\frac{1}{T^{(2)}}\diff U^{(2)} - \frac{\mu_1^{(2)}}{T^{(2)}}\diff \sm^{(2)}   \\
            & = \left(\frac{1}{T^{(1)}}-\frac{1}{T^{(2)}}\right)\diff U^{(1)} + \left(\frac{\mu_1^{(1)}}{T^{(1)}}-\frac{\mu_1^{(2)}}{T^{(2)}}\right) \diff \sm^{(1)} = 0.
\end{align*}
Im Gleichgewicht gleichen sich also neben den Temperaturen die chemischen Potential durch Teilchenaustausch an,
\begin{align*}
    T^{(1)} = T^{(2)}, \quad \mu_1^{(1)} = \mu_1^{(2)}.
\end{align*}
Es findet jedoch kein Teilchenfluss statt, wenn bereits $\mu_1^{(1)} = \mu_1^{(2)}$.

\paragraph*{Materiefluss}

Beginnt man bei einem Anfangszustand mit $\mu_1^{(1)} > \mu_1^{(2)}$ und $T^{(1)} = T^{(2)}$ und hebt dann die Zwangsbedingung auf (Wand wird materiedurchlässig), so kommt es zum quasistatischen Materiefluss,
\begin{align*}
    \Delta S = \frac{\mu_1^{(2)} - \mu_1^{(1)}}{T} \Delta \sm_1^{(1)}.
\end{align*}
Da nach dem Postulat \ref{post:entropie_maximierung} die Änderung der Entropie nur positiv sein kann und $(\mu_1^{(2)} - \mu_1^{(1)})/T$ nach unserer Festlegung negativ ist, so ist auch $\Delta \sm_1^{(1)}<0$.

\begin{formal}
    Ein Materiefluss findet von Gebieten hohen zu Gebieten tiefen chemischen Potentials statt, bis $\mu_1^{(1)} = \mu_1^{(2)}$.
\end{formal}

Das chemische Potential $\mu$ ist zentral bei Phasenumwandlungen und chemischen Reaktionen (siehe später) und spielt damit eine führende Rolle in der theoretischen Chemie.



\subsection{Folgerungen aus der Homogenität der Fundamentalbeziehung}

Allein aus der Forderung, dass die Fundamentalbeziehung homogen ist, lassen sich einige formale Schlüsse folgern, die in diesem Kapitel erläutert werden sollen. Die erste Schlussfolgerung ist die Euler-Gleichung.

\paragraph*{Die Euler-Gleichung}

Die innere Energie $U$ ist eine extensive Größe und damit eine homogene Funktion ersten Grades (Größen wie Volumen und Stoffmengen werden verallgemeinert als $X_k$ geschrieben, um eine kompaktere Notation zu ermöglichen),
\begin{align*}
    U(\lambda S,\lambda X_1,\ldots,\lambda X_t) = \lambda U(S,X_1,\ldots,X_t).
\end{align*}
Ableiten nach $\lambda$ liefert
\begin{align*}
    \frac{\partial U}{\partial\lambda} & = \frac{\partial U}{\partial\lambda S}(\lambda S,\lambda X_1,\ldots,\lambda X_t) \frac{\partial\lambda S}{\partial\lambda} + \frac{\partial U}{\partial\lambda X_1}(\lambda S,\lambda X_1,\ldots,\lambda X_t) \frac{\partial\lambda X_1}{\partial\lambda} + \ldots \\
                                       & = T(\lambda S,\lambda X_1,\ldots,\lambda X_t)S + P_1 (\lambda S,\lambda X_1,\ldots,\lambda X_t) X_1 + \ldots                                                                                                                                                       \\
                                       & = \lambda TS + \lambda P_1X_1 + \ldots
\end{align*}
Lässt man nun $\lambda$ gegen 1 gehen, so erhält man die Euler-Gleichung in Energiedarstellung:
\begin{align}
    \label{eq:euler_gleichung_energiedarstellung}
    \boxed{U = TS + \sum_{j=1}^t P_j X_j}
\end{align}
Analog lässt sich die Entropiedarstellung der Euler-Gleichung herleiten:
\begin{align}
    \label{eq:euler_gleichung_entropiedarstellung}
    \boxed{S = \sum_{j=0}^t F_j X_j}
\end{align}
Für ein einfaches System nimmt sie zum Beispiel die Form (Energiedarstellung)
\begin{align*}
    U=TS-PV + \mu_1 \sm_1+\ldots + \mu_r \sm_r
\end{align*}
bzw. (Entropiedarstellung)
\begin{align*}
    S=\frac{1}{T}U + \frac{P}{T}V - \sum_{k=1}^r \frac{\mu_k}{T}\sm_k
\end{align*}
an.


\paragraph*{Gibbs-Duhem-Beziehung}

Bis jetzt haben wir globale Betrachtungen gemacht, die hauptsächlich extensive Variablen behandeln. Nun soll eine differentielle Behandlung folgen, die auch die intensiven Größen berücksichtigt.

Aus der energetischen Fundamentalbeziehung erhält man durch Differenzieren nach den Parametern $(t+1)$ Gleichungen in $(t+1)$ Variablen.
\begin{align*}
    \frac{\partial U}{\partial S} = T = T(S,X_1,\ldots ,X_t), \quad
    \frac{\partial U}{\partial X_k} =P_k = P_k(S,X_1,\ldots ,X_t)
\end{align*}
Die durch diesen Prozess gewonnenen Größen $T,P_1,\ldots,P_t$ sind intensiv, also homogen vom Grad $0$, z.B.
\begin{align*}
    T(S,X_1,\ldots ,X_t)=T(\lambda S,\lambda X_1,\ldots ,\lambda X_t).
\end{align*}
Für $\lambda=1/X_t$ ist dann
\begin{align*}
    T = T\left(\frac{S}{X_t},\frac{X_1}{X_t},\ldots ,1\right), \quad P_k=P_k\left(\frac{S}{X_t},\frac{X_1}{X_t},\ldots ,1\right).
\end{align*}
Dieses Gleichungssystem aus $(t+1)$ Gleichungen enthält jetzt nur noch $t$ Variablen, sodass eine Zustandsgleichung bei der Beziehung zwischen den insgesamt $(t+1)$ intensiven Variablen redundant ist.

Betrachte als Beispiel ein Einkomponentensystem mit $X_t=\sm$, molarer Entropie $s=S/\sm$ und molarem Volumen $v=V/\sm$. Temperatur, Druck und chemisches Potential hängen jeweils nur von der molaren Entropie und dem molaren Volumen ab,
\begin{align*}
    T=T(s,v), \quad P=P(s,v), \quad \mu=\mu(s,v).
\end{align*}
Man kann jetzt zwei dieser Zusammenhänge invertieren (z.B. $s=s(T,P)$ und $v=v(T,P)$) und in den dritten einsetzen ($\mu=\mu(T,P)$). Diese letzte Gleichung ist damit redundant für die Beschreibung des Systems.

Durch Differenzieren der Euler-Gleichung erhalten wir
\begin{align}
    U                    & =TS+ \sum_j P_jX_j                                                 \nonumber \\
    \label{eq:differential_innere_energie1}
    \implication \diff U & = T\diff S + S\diff T + \sum_j P_j\diff X_j + \sum_j X_j\diff P_j.
\end{align}
Andererseits kennen wir aus der Energiedarstellung bereits das Differential der inneren Energie (siehe Gleichung ??) als
\begin{align}
    \label{eq:differential_innere_energie2}
    \diff U = T\diff S + \sum_j P_j \diff X_j.
\end{align}
Durch Vergleich von \eqref{eq:differential_innere_energie1} und \eqref{eq:differential_innere_energie2} folgt eine differentielle Beziehung zwischen den intensiven Variablen,
\begin{align}
    \label{eq:gibbs_duhem}
    \boxed{
        S\diff T + \sum X_j \diff P_j = 0,
    }
\end{align}
was als Gibbs-Duhem-Beziehung bekannt ist.

Im Einkomponentensystem heißt das, dass
\begin{align*}
    S\diff T - V\diff P + \sm \diff \mu = 0.
\end{align*}
Teilen durch $\sm$ liefert
\begin{align*}
    \diff \mu = -s \diff T + v\diff P
\end{align*}
und Integrieren führt auf das chemische Potential $\mu=\mu(T,P)$.

Wir haben also gesehen, dass stets eine intensive Variable von den anderen abhängt. Das gibt Anlass zu der nachfolgenden Definition:
\begin{formal}
    Die Zahl der unabhängigen intensiven Variablen wird als Zahl der thermodynamischen Freiheitsgrade definiert.
\end{formal}

Die Gibbs-Duhem-Beziehung kann auch in der Entropiedarstellung formuliert werden,
\begin{align*}
    \boxed{\sum_{j=0}^t X_j \diff F_j = 0}
\end{align*}
und für ein einfaches System ist dann
\begin{align*}
    U\diff\left(\frac{1}{T}\right) + V\diff\left(\frac{P}{T}\right)-\sum_{k=1}^r \sm_k\diff\left(\frac{\mu_k}{T}\right)
\end{align*}