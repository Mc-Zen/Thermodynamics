%
% -------------------------------------------------------------------
% Thermodynamics lecture at TU Berlin, read by Prof. Holger Stark
% -------------------------------------------------------------------
%

\documentclass[hidelinks, 11pt]{scrbook} 

\usepackage[left=2cm, right=2cm, bottom=1.5cm, top=1.5cm, includeheadfoot]{geometry} 
\usepackage[T1]{fontenc} 
\usepackage[utf8]{inputenc} 
\usepackage[ngerman,english]{babel}
\usepackage[font=footnotesize]{caption}
\usepackage{graphicx} 
\usepackage{subcaption} 
\usepackage{multirow} 
\usepackage{tabularx} 
\usepackage{xcolor} 
\usepackage{amsmath} 
\usepackage{amssymb} 
\usepackage{amsfonts} 
\usepackage{amsxtra} 
\usepackage{mathtools} 
\usepackage{tensor}
\usepackage{enumerate} 
\usepackage{float}
\usepackage{siunitx}
\usepackage{longtable}
\usepackage{colortbl}
\usepackage{cancel}
\usepackage{isotope}
\usepackage{lmodern}
\usepackage{scrhack} 
\sisetup{locale = DE, separate-uncertainty}  
\usepackage[immediate]{silence}
\WarningFilter[temp]{latex}{Command} % filter underline/underbar command warning from sectsty 

% --------------------------
% spacing

%\renewcommand*{\familydefault}{\sfdefault} sans serif font
\raggedbottom
\setlength{\parindent}{0em}
\setlength{\parskip}{1em}

\usepackage[hang]{footmisc}
\setlength{\footnotemargin}{3mm} % space between number and content of footnote
\setlength{\skip\footins}{.5cm} % space between body and footnote section
\setlength{\footnotesep}{0.5cm}  % space between footnotes

% \renewcommand{\arraystretch}{1}
% \makeatletter
% \renewcommand\@pnumwidth{2em} % fix toc overfull hbox
% \makeatother


% --------------------------
% section styles

\newcommand{\example}[1]{\paragraph{#1:}}
% change title font to serif
\addtokomafont{title}{\rmfamily}


% --------------------------
% math

\DeclareMathOperator{\grad}{grad} 
\DeclareMathOperator{\divg}{div} 
\DeclareMathOperator{\rot}{rot} 
\DeclareMathOperator{\real}{\mathfrak{R}} 
\DeclareMathOperator{\imag}{\mathfrak{I}} 

\newcommand{\upupharpoons}{\upharpoonleft\!\upharpoonright}
\newcommand{\updownharpoons}{\upharpoonleft\!\downharpoonright}
% pm sign with minus in paretheses
\newcommand{\varpm}{\mathbin{\vcenter{\hbox{
  \oalign{\hfil$\scriptstyle+$\hfil\cr\noalign{\kern-.3ex}$\scriptscriptstyle({-})$\cr}
}}}}
% mp sign with plus in parentheses
\newcommand{\varmp}{\mathbin{\vcenter{\hbox{
  \oalign{$\scriptstyle({+})$\cr\noalign{\kern-.3ex}\hfil$\scriptscriptstyle-$\hfil\cr}
}}}}
% d with bar through
%\newcommand{\dbar}{d\hspace*{-0.08em}\bar{}\hspace*{0.1em}}
\newcommand{\dbar}{\mkern3mu\mathchar '26 \mkern-11mu \mathrm{d}}
% differential operator have a non-italic d in German equation typesetting 
\newcommand*{\diff}{\mathop{}\!\mathrm{d}} 
\newcommand*{\udiff}{\mathop{}\!\delta} 
\newcommand*{\diffa}[2][]{\mathop{\mathrm{d}^{#1}#2}}
%\newcommand{\diff}{\text{d}}	 
\newcommand{\Angstroem}{\text{\normalfont\AA}}   
\newcommand{\Abbref}[1]{Abb.~\ref{#1}} 
% Vector: bold non-cursive symbols for vectors instead of arrows 
\renewcommand{\vec}[1]{\mathbf{#1}} 

\newcommand{\equivalence}{\;\Leftrightarrow\;}
\newcommand{\implication}{\;\Rightarrow\;}


\newcommand{\formalemph}[1]{\textbf{#1}}
\newcommand{\sm}{n}
\newcommand{\avogadro}{N_\mathrm{A}}

\DeclareSIUnit{\bar}{bar}



\newcommand{\anf}[1]{\glqq #1\grqq{}}

% --------------------------
% other configurations

% set figure description
\addto\captionsngerman{
  \renewcommand{\figurename}{Abb.}
  \renewcommand{\tablename}{Tab.}
}
\graphicspath{{images/}} 
\usepackage[strict]{changepage}

% for formal definitions
\usepackage{framed}

% environment derived from framed.sty: see leftbar environment definition
\definecolor{formalbar}{rgb}{0.1,0.1,.2}
\definecolor{formalshade}{rgb}{0.95,0.95,1}
\definecolor{postulatebar}{rgb}{0.3,0.07,.07}
\definecolor{postulateshade}{rgb}{1,0.9,0.9}
\definecolor{summarybar}{rgb}{0.03,0.15,.03}
\definecolor{summaryshade}{rgb}{.92,.95,0.9}

\newcounter{PostulateCounter}
\renewcommand{\thePostulateCounter}{\Roman{PostulateCounter}}
% Quote box for important statements
\newenvironment{formal}{
  \def\FrameCommand{
    \hspace{1pt}%
    {\color{formalbar}\vrule width 2pt}%
    {\color{formalshade}\vrule width 4pt}%
    \colorbox{formalshade}%
  }
  \MakeFramed{\advance\hsize-\width\FrameRestore}%
  \noindent\hspace{-4.55pt}% disable indenting first paragraph
  \begin{adjustwidth}{}{7pt}%
  \vspace{-10pt}\vspace{2pt}% 
}{\vspace{2pt}\end{adjustwidth}\endMakeFramed}


% Quote box for important statements
\newenvironment{summary}{
  \def\FrameCommand{
    \hspace{1pt}%
    {\color{summarybar}\vrule width 2pt}%
    {\color{summaryshade}\vrule width 4pt}%
    \colorbox{summaryshade}%
  }
  \MakeFramed{\advance\hsize-\width\FrameRestore}%
  \noindent\hspace{-4.55pt}% disable indenting first paragraph
  \begin{adjustwidth}{}{7pt}%
  \vspace{-10pt}\vspace{2pt}% 
  \textbf{Zusammenfassung}

}{\vspace{2pt}\end{adjustwidth}\endMakeFramed}


% Quote box for postulates
\newenvironment{postulate}[1][\unskip]{
  \def\FrameCommand{
    \hspace{1pt}%
    {\color{postulatebar}\vrule width 2pt}%
    {\color{postulateshade}\vrule width 4pt}%
    \colorbox{postulateshade}%
  }
  \MakeFramed{\advance\hsize-\width\FrameRestore}%
  \noindent\hspace{-4.55pt}% disable indenting first paragraph
  \begin{adjustwidth}{}{7pt}%
  \vspace{-10pt}\vspace{2pt}% 
  \refstepcounter{PostulateCounter}
  \textbf{Postulat \Roman{PostulateCounter} \ifx#1\unskip #1\else (#1)\fi:}

}{\vspace{2pt}\end{adjustwidth}\endMakeFramed}


% !TeX root = Theo_IV.tex

\usepackage{tikz}
%\usepackage{pgfplots}
%\pgfplotsset{compat=1.18}

\usetikzlibrary{
    arrows.meta,
    bending,
    positioning,
    decorations.markings,
    intersections,
    calc,
    decorations.pathreplacing,
    decorations.pathmorphing,
    patterns
}
%\tikzexternalize[prefix=figures/,shell escape=-enable-write18] % activate

\tikzset{
    % Colors
    object color/.style={blue!40!black!80!white},
    object style/.style={object color,thick},
    nice green/.style={green!50!black},
    nice orange/.style={red!60!yellow!70!black!90!white},
    nice dark blue/.style={blue!50!black},
    nice light blue/.style={blue!60!white!70!black},
    nice turquoise/.style={blue!50!green},
    polarisation color/.style={purple},
    charge color/.style={blue!50!white!70!black},
    red laser/.style={red!70!black},
    moving system color/.style={blue!60!black!70!white},
    %
    % Coordinate system
    coordsystem/.style={very thin, color=#1!50},
    xlabel/.style={anchor=north west},
    ylabel/.style={anchor=south east},
    %
    % Nodes and points
    invisible point/.style={circle,inner sep=0pt,outer sep=0pt,minimum size=0pt},
    point/.style={invisible point,fill=black,minimum size=4pt},
    %
    % Arrows
    arrow tip/.tip={Stealth},
    arr/.style={->,>={arrow tip}},
    rarr/.style={<-,>={arrow tip}},
    midarrow/.style={postaction=decorate,decoration={markings, mark=at position #1 with {\arrow[xshift=2.5pt]{arrow tip}}} },
    midarrow/.default=.5,
    rmidarrow/.style={postaction=decorate,decoration={markings, mark=at position #1 with {\arrowreversed{arrow tip}}} },
    rmidarrow/.default=.5,
    distance marker/.style={|<->|,>={arrow tip}},
    %
    % Thermodynamics stuff
    piston bar/.style={line width=2pt},
    piston/.style={line width=8pt}
}
\tikzstyle{every node}=[font=\footnotesize]


\newcommand{\tfigTitel}{
    \begin{tikzpicture}
        \pgfmathsetmacro{\shadowangle}{132}
        \newlength{\shadowdistance}
        \pgfmathsetlength{\shadowdistance}{0.1ex}
        \pgfmathsetmacro{\shadowopacity}{1}
        \pgfmathsetmacro{\shadowspread}{0.003}
        \pgfmathsetmacro{\shadowsize}{5}
        \pgfmathtruncatemacro{\totshadow}{100}
        \path[nice dark blue,opacity={\shadowopacity/\totshadow},shift={({132-180}:\shadowdistance)},scale={1+\shadowsize}] 
        foreach \nshadow [evaluate=\nshadow as \angshadow using \nshadow/\totshadow*360] in {1,...,\totshadow}{
            node[align=center] at (\angshadow:\shadowspread) {\huge Theoretische Physik 4\\ \\ \\
            \Large Thermodynamik/Statistische Physik}
            };
        \node[align=center] at (0,0) {\huge Theoretische Physik 4\\ \\ \\ \Large Thermodynamik/Statistische Physik};
    \end{tikzpicture}
}

% 100 particles in a rectangular box
\newcommand{\tfigSystemWithManyParticles}{
    \begin{tikzpicture}[scale=1.5]
        \draw (-1pt,-1pt) rectangle ($(3,2)+(1pt,1pt)$);
        \pgfmathsetseed{1}
        \foreach \i in {1,2,...,100}{
            \draw[fill=black] (rnd*3,rnd*2) circle[radius=.5pt];
        };
    \end{tikzpicture}
}

% Two chambers separated by a piston
\newcommand{\tfigTwoChambersSeparatedByPiston}{
    \begin{tikzpicture}[scale=1.6]
        \draw (0,0) rectangle (2,1);
        \node at (.5,.5) {1};
        \node at (1.5,.5) {2};
        \draw[pattern=north east lines] (.95,0) rectangle (1.05,1);
        \node at (0,-.5) {Wand} edge[arr,bend right=10] (.4,0);
        \node at (1.4,-.5) {Kolben} edge[arr,bend left=10] (1,0);
    \end{tikzpicture}
}

% Rectangular box with piston
\newcommand{\tfigRectangularBoxWithPiston}{
    \begin{tikzpicture}[scale=1.6]
        \draw[line width=2.4pt] (1,0) -- (1,1);
        \draw[line width=2pt] (1,.5) -- +(.8,0);
        \draw[rarr] (1.9,.5) -- +(.7,0) node[midway, above] {$\Delta W$};
        \node at (.5,.5) {$U\uparrow$};
        \draw[draw=black!70!white] (1.2,0) -- (0,0) -- (0,1) -- (1.2,1);
    \end{tikzpicture}
}

% Water with ice cubes. Stiring heats up the water and ice cubes vanish
\newcommand{\tfigWaterStiringIceCubes}{
    \begin{tikzpicture}[water color/.style={nice light blue},wave deco/.style={decoration={bumps,amplitude=-3,segment length=17}},scale=2]
        \begin{scope}
            \fill[blue!20!white!90!black] decorate[wave deco] {(0,.8) -- (1.2,.8)} -- (1.2,0) -- (0,0) -- (0,.8);
            \draw[thick] (0,0) rectangle (1.2,1); 
            \pgfmathsetseed{5}
            \foreach \i in {1,2,...,10} {
                \coordinate (A) at (rnd+.1,rnd*.5+.1);
                \fill[color=white,rotate around={rnd*360:(A)}] (A) rectangle +(2pt,2pt);
            };
            \node[water color] at (.6,.15) {$\mathrm{H}_2\mathrm{O}$};
            \node at (.6,1.2) {A};
        \end{scope}
        \draw[arr] (1.4,.5) -- +(.4,0); 
        \begin{scope}[xshift=2cm]
            \fill[blue!20!white!90!black] decorate[wave deco] {(0,.8) -- (1.2,.8)} -- (1.2,0) -- (0,0) -- (0,.8);
            \draw[thick] (0,0) rectangle (1.2,1); 
            \foreach \i in {1,2,...,4} {
                \coordinate (A) at (rnd+.1,rnd*.5+.1);
                \fill[color=white,rotate around={rnd*360:(A)}] (A) rectangle +(2pt,2pt);
            };
            \node[water color] at (.6,.15) {$\mathrm{H}_2\mathrm{O}$};
            
            \draw (.5,.3) rectangle +(.2,.3) (.55,.3) rectangle +(.1,.3);
            \draw[thick] (.6,.6)-- +(0,.6);
            \draw[arr] (.6,1.3) arc[x radius=.2, y radius=.12,start angle=-270, end angle=90];
        \end{scope}
        \draw[arr] (3.4,.5) -- +(.4,0); 
        \begin{scope}[xshift=4cm]
            \fill[blue!20!white!90!black] decorate[wave deco] {(0,.8) -- (1.2,.8)} -- (1.2,0) -- (0,0) -- (0,.8);
            \draw[thick] (0,0) rectangle (1.2,1);
            \node[water color] at (.6,.15) {$\mathrm{H}_2\mathrm{O}$};
            \node at (.6,1.2) {B};
        \end{scope}
    \end{tikzpicture}
}

% Visualize that W,Q are no state functions. Adding different ratios of W and Q to a system may lead to the same inner energy U. 
\newcommand{\tfigWQAreNoStateFunctions}{
    \begin{tikzpicture}[scale=1.7]
        \node[point,label={north east}:$U(B)$] (Point B) at (2.2,1.4) {};
        \node[point,label={south west}:$U(A)$] (Point A) at (0,0) {} 
        edge[arr,bend left=50] node[midway, anchor=south east] {$\Delta W_1,\Delta Q_1$} node[midway, anchor=north west] {1} (Point B)
        edge[arr,bend right=60] node[midway, anchor=north west] {$\Delta W_2,\Delta Q_2$} node[midway, anchor=south east] {2} (Point B);
    \end{tikzpicture}
}

% Sketch of fundamental relations between inner energy and entropy.
\newcommand{\tfigSchemaFundamentalbeziehung}{    
    \begin{tikzpicture}[scale=1.5]
        \draw[arr] (0,0) node[anchor=north] {$0$} -- (0,2) node[ylabel] {$U$};
        \draw[arr] (0,0) -- (3,0) node[xlabel] {$S$};
        \draw[nice light blue] (0,1) .. controls +(0:1.2) and +(-135:.7) .. (2.5,1.8);
        \begin{scope}[xshift=4.5cm]
            \draw[arr] (0,0) node[anchor=north] {$0$} -- (0,2) node[ylabel] {$S$};
            \draw[arr] (0,0) -- (3,0) node[xlabel] {$U$};
            \draw[nice light blue] (1,0) .. controls +(90:.8) and +(-165:.7) .. (2.8,1.8);
        \end{scope}
    \end{tikzpicture}
}

% System with two subsystems separated by an impermeable and fixed but thermally conductive wall. 
\newcommand{\tfigDoppelsystemUSfesteWaermeleitendeWand}{
    \begin{tikzpicture}[scale=1.6]
        \draw (0,0) rectangle (2,1);
        \node[align=center] at (.5,.5) {$U^{(1)}$\\$S^{(1)}$};
        \node[align=center] at (1.5,.5) {$U^{(2)}$\\$S^{(2)}$};
        \draw[pattern=north east lines] (.95,0) rectangle (1.05,1);
        \node at (1.4,-.5) {fest, undurchlässig, isolierend $\rightarrow$ wärmeleitend} edge[arr,bend left=10] (1,0);
    \end{tikzpicture}
}

% System with two subsystems separated by an impermeable but movable and thermally conductive wall. 
\newcommand{\tfigDoppelsystemUVNbeweglicheWaermeleitendeWand}{
    \begin{tikzpicture}[scale=1.6]
        \draw (0,0) rectangle (2,1);
        \node[align=center] at (.5,.5) {$U^{(1)}$\\$V^{(1)}$\\$\sm^{(1)}$};
        \node[align=center] at (1.5,.5) {$U^{(2)}$\\$V^{(2)}$\\$\sm^{(2)}$};
        \draw[pattern=north east lines] (.95,0) rectangle (1.05,1);
        \node at (1.4,-.5) {undurchlässig, fest $\rightarrow$ beweglich, isolierend $\rightarrow$ wärmeleitend} edge[arr,bend left=10] (1,0);
    \end{tikzpicture}
}

% Function U(S) with maximum. 
\newcommand{\tfigFunktionEntropieMaximum}{
    \begin{tikzpicture}[scale=1.5]
        \draw[arr] (0,0) -- (0,2) node[ylabel] {$S$};
        \draw[arr] (0,0) -- (4,0) node[xlabel] {$U^{(1)}$};
        
        \path let \n{thermisches GG}={2}, \n{x1}={3},\n{x2}={3.5},\n{y1}={-.3*(\n{x1}-\n{thermisches GG})^2+1.5},\n{y2}={-.3*(\n{x2}-\n{thermisches GG})^2+1.5} in        
        coordinate (P1) at (\n{x1},\n{y1})
        coordinate (P2) at (\n{x2},\n{y2});
        \node[point] at (P1) {};
        \node[point] at (P2) {};
        \draw let \n{thermisches GG}={2} in plot[domain=.2:3.8] (\x,{-.3*(\x-\n{thermisches GG})^2+1.5});
        \draw[dashed] let \n{thermisches GG}={2},\p1=(P1),\p2=(P2) in 
        (\n{thermisches GG},1.5) -- +(0,-1.5) node[below,align=center] {thermisches\\Gleichgewicht}
        (0,\y1) -- (P1) -- (\x1,0) (0,\y2) -- (P2) -- (\x2,0);
        \draw[decorate, decoration = {brace}] let \p1=(P1),\p2=(P2) in
        (0,\y2)  --(0,\y1) node[midway,left]{$\Delta S$};
        \draw[decorate, decoration = {brace}] let \p1=(P1),\p2=(P2) in
        (\x2,0)  --(\x1,0) node[midway,below]{$\Delta U^{(1)}$};
        
        \draw[arr] ([shift={(.1,.1)}]P2)to[bend right=3] ([shift={(.1,.1)}]P1);
    \end{tikzpicture}
}


% System with two subsystems separated by a fixed but semipermeable and thermally conductive wall. 
\newcommand{\tfigDoppelsystemUVNbeweglicheIsolierendeWand}{
    \begin{tikzpicture}[scale=1.6]
        \draw (0,0) rectangle (2,1);
        \node[align=center] at (.5,.5) {$U^{(1)}$\\$\sm_1^{(1)}$\\$S^{(1)}$};
        \node[align=center] at (1.5,.5) {$U^{(2)}$\\$\sm_1^{(2)}$\\$S^{(2)}$};
        \draw[pattern=north east lines] (.95,0) rectangle (1.05,1);
        \node at (1.4,-.5) {fest, materieundurchlässig $\rightarrow$ durchlässig für Molekülsorte 1, isolierend $\rightarrow$ wärmeleitend} edge[arr,bend left=10] (1,0);
    \end{tikzpicture}
}


% Degeneracy function g(n,N) for N >> 1 (gaussian approximation)
\newcommand{\tfigDegeneracyFunctionGauss}{
    \begin{tikzpicture}[scale=4]
        \draw[arr] (-.7,0) -- (.7,0) node[xlabel] {$\frac{n}{N}$};
        %\draw[arr] (0,0) -- (0,1) node[ylabel] {$g(n,N)$};
        
        \draw let \n{N}={10} in plot[domain=-.5:.5,smooth,yscale=.003] (\x,{sqrt(2/(3.14159*\n{N}))*2^\n{N}*exp(-2*(\x*\n{N})^2/\n{N})}) 
        coordinate (P) at ({sqrt(1/(2*\n{N}))},{.003*sqrt(2/(3.14159*\n{N}))*2^\n{N}/exp(1)});
        \draw (-.5,.02) -- +(0,-.04) node[below] {$-\frac{1}{2}$}(.5,.02) -- +(0,-.04) node[below] {$\frac{1}{2}$};
        \draw[dashed] let \p1=(P) in (-\x1,0) -- (-\x1,\y1) -- (\x1,\y1) -- (\x1,0); 
        
        \draw[decoration={brace},decorate] let \p1=(P) in (\x1,-.02) -- +(-\x1,0) node[midway, below,yshift=-.04cm] {$\frac{n_n}{N}$};
    \end{tikzpicture}
}


% (1) System with two subsystems and wall, one filled with particles
% (2) barrier opened
\newcommand{\tfigTwoSubsystemsParticlesRemoveWall}{
    \begin{tikzpicture}[scale=1.5]
        \draw (0,0) rectangle (2, 1.3);
        \node at (1,1.5) {(1)};
        \draw (1,0) -- (1,1.3);
        \pgfmathsetseed{2}
        \foreach \n in {1,...,30}{
            \fill (.02+rnd*0.96,.02+rnd*1.26) circle[radius=.5pt];
        }
        
        \draw[arr] (2.3,1.3*2/3) -- +(1.3,0) node[midway, above] {$S$ wächst};
        \draw[rarr] (2.3,1.3/3) -- +(1.3,0) node[midway, below] {?};
        
        \begin{scope}[xshift=3.9cm]
            \node at (1,1.5) {(2)};
            \draw (0,0) rectangle (2, 1.3);
            \pgfmathsetseed{3}
            \foreach \n in {1,...,30}{
                \fill (.02+rnd*1.96,.02+rnd*1.26) circle[radius=.5pt];
            }
        \end{scope}
    \end{tikzpicture}
}


% Degeneracy function for large systems -> delta function
\newcommand{\tfigDegeneracyFunctionLargeSystemsDelta}{
    \begin{tikzpicture}[scale=4]
        \draw[arr] (-.7,0) -- (.7,0) node[xlabel] {$\frac{n_1}{N_1}$};
        \draw[arr] (0,0) -- (0,.7) node[ylabel] {$(g_1g_2)(n_1)$};
        \draw (-.5,.5pt) -- +(0,-1pt) node[below] {$-\frac{1}{2}$};
        \draw (.5,.5pt) -- +(0,-1pt) node[below] {$\frac{1}{2}$};
        \draw[nice green] (.2,.5) -- (.2,0) node[near start, right] {$\frac{\delta_n}{N_1}=10^{-11}$}
        node[below,black] {$\frac{\hat{n}_1}{N_1}$};
        \draw (.5pt,.5) -- (-.5pt,.5) node[left] {$(g_1g_2)_\mathrm{max}$};
    \end{tikzpicture}
}


\newcommand{\tfigPlaceholder}{
    \begin{tikzpicture}
        \filldraw[fill=formalshade] (0,0) rectangle (3,3) ;
        \node[black] at (1.5,1.5){\textbf{?}};
    \end{tikzpicture}
}

% Degeneracy function for large systems -> delta function
\newcommand{\tfigProcessReversibleQuasistationary}{
    \begin{tikzpicture}[scale=1.7]
        \draw[piston] (0,0) -- (0,1);
        \draw[piston bar] (0,.5) -- (1.2,.5);
        \draw[arr] (.1,.7) -- +(.4,0);
        \draw[pattern=north east lines,even odd rule] (-1.4,1.2) rectangle (1.4,-.2) node[pos=0,anchor=south west] {$\Delta S=0$} (-1.2,0) rectangle (1.2,1);
        \pgfmathsetseed{16}
        \foreach \n in {1,...,20}{
            \fill (.02-1.2+rnd*.96,.02+rnd*.96) circle[radius=.5pt];
        }
    \end{tikzpicture}
}
            
\newcommand{\tfigProcessIrreversibleQuasistationary}{
    \begin{tikzpicture}[scale=1.7]
        \foreach \x in {0,.1,...,.6}{
            \draw[line width=3pt] (\x,0) -- +(0,1);
            \draw[arr] (\x,1.2) -- +(0,.4);
        }
        \node at (.8,.5) {\ldots};
        \draw[pattern=north east lines,even odd rule] (-1.4,1.2) rectangle (1.4,-.2) node[pos=0,anchor=south west] {$\Delta S>0$} (-1.2,0) rectangle (1.2,1);
        \pgfmathsetseed{30}
        \foreach \n in {1,...,20}{
            \fill (.02-1.2+rnd*.96,.04+rnd*.90) circle[radius=.5pt];
        }
    \end{tikzpicture}
}

\newcommand{\tfigProcessIrreversibleNonquasistationary}{
    \begin{tikzpicture}[scale=1.7]
        \draw[line width=3pt] (0,0) -- +(0,1);
        \draw[arr] (0,1.2) -- +(0,.4);
        
        \draw[pattern=north east lines,even odd rule] (-1.4,1.2) rectangle (1.4,-.2) node[pos=0,anchor=south west] {$\Delta S>0$} (-1.2,0) rectangle (1.2,1);
        \pgfmathsetseed{29}
        \foreach \n in {1,...,20}{
            \fill (.02-1.2+rnd*.96,.02+rnd*.96) circle[radius=.5pt];
        }
    \end{tikzpicture}
}
\newcommand{\tfigTheoremMaximizedWork}{
    \begin{tikzpicture}
        [
            system node/.style={rectangle,draw,inner sep=10pt}
        ]
        \node[system node, fill=red!20!white] (RWQ) at (3,1) {RWQ};
        \node[system node, fill=black!10!white] (RAQ) at (3,-1) {RAQ};
        \node[point] (P) at (0,0) {};
        \draw[arr] (P) .. controls +(0:.7) and +(180:.7) .. (RAQ.west) node[pos=.7,anchor=north east] {$\Delta W^\mathrm{RAQ}$};
        \draw[arr] (P) .. controls +(0:.7) and +(180:.7) .. (RWQ.west) node[pos=.7,anchor=south east] {$\Delta Q^\mathrm{RWQ}$};
        edge[arr] (RWQ) 
        edge[arr] (RAQ);
        \node[system node,align=center] at (-3,0) {\textbf{TS}\\\\$\Delta S_{AB}^\mathrm{TS}$} edge[arr] node[midway, above] {$-\Delta U_{AB}^\mathrm{TS}$} (P);
    \end{tikzpicture}
}
\usepackage{hyperref} % include hyperref as last package!


\hyphenation{Gleich-ge-wichts-zu-stän-de}

\title{\tfigTitel}
\subtitle{Sommersemester 2022}
\author{von L.K. und K.L.\\bei Professor Holger Stark}
\date{}

% -----------------------------------
% -----------------------------------
% -----------------------------------

\begin{document}

\frontmatter
\maketitle
\tableofcontents

% -----------------------------------
% -----------------------------------

\mainmatter
% !TeX root = Theo_IV.tex

\chapter{Einleitung\label{einleitung}}



\section{Inhalt}

Der Inhalt dieser Vorlesung gliedert sich in zwei Teile, die Thermodynamik und die statistische Physik:

Bei der Thermodynamik geht es im Allgemeinen um Vielteilchensysteme, für die genauen mikroskopischen Positionen interessiert man sich allerdings nicht, sondern vielmehr um makroskopische Größen und Verteilungen.
Die Vielteilchensysteme werden durch den Minimalsatz von makroskopischen Variablen beschrieben, z.B. durch Energie, Volumen und Entropie.
Man spricht tatsächlich bei der Thermodynamik auch von der Lehre der Entropie.

Das Ziel ist es, allgemeine und modellunabhängige Aussagen und Prinzipien zu formulieren und häufig geht es nicht um die Berechnung von Systemparametern, sondern um die Relationen zwischen ihnen.
Damit hat die Thermodynamik einen sehr breiten Anwendungsbereich und findet Verwendung in vielen 
Wissenschaften\footnote{Beispiele: Gase, Magnetismus, Supraleitung, chemische Reaktionen, Phasenübergänge, Schwarzkörperstrahlung, Neutronensterne, schwarze Löcher, Biologie, soziologische Systeme, Ottomotor, Klima, ...}.

Die Thermodynamik stellt auch die Basis für einige Weiterentwicklungen dar, wie z.B. die Nichtgleich\-gewichts-Thermodynamik (in dieser Vorlesung wird eine reine thermostatische Beschreibung behandelt).

Die statistische Physik fasst Parameter von Vielteilchensysteme mithilfe der mikroskopischen Bewegungsgleichungen und statistischen Methoden mit makroskopischen Größen zusammen.



\begin{itemize}
	\item Herbert B. Callen: Thermodynamics and an Introduction to Thermostatistics, John Wiley \& Sons, New York 1985
	\item Gerhard Adam, Otto Hittmaier: Wärmetheorie, Vieweg, Wiesbaden 1992
	\item Charles Kittel, Herbert Kroemer: Thermal Physics, W.H. Freeman and Company, 1980
\end{itemize}

Folgende weiterführende Literatur kann zurate gezogen werden:
\begin{itemize}
	\item Arnold Sommerfeld: Theoretische Physik V: Thermodynamik und Statistik, Harri Deutsch, 1977
	\item Franz Schwabl: Statistische Mechanik, Springer, Berlin 2000
	\item Torsten Fließbach: Statistische Physik - Lehrbuch zur Theoretischen Physik IV, Springer Spektrum, Berlin 2010
	\item Wolfgang Nolting: Grundkurs Theoretische Physik 4/2: Thermodynamik, Springer Spektrum, Berlin 2016
	\item Wolfgang Nolting: Theoretische Physik 6: Statistische Physik, Springer Spektrum, Berlin 2013
	\item Mehran Kardar: Statistical Physics of Particles, Cambridge University Press, Cambridge 2007
\end{itemize}



\section{Grundlegende Konstanten der Thermodynamik}

Für Konstanten, deren Wert per Definition festgelegt wurde, die also exakt sind, wird ein $\equiv $-Zeichen verwendet.


\begin{table}[H]
	\centering
	\begin{tabular}{|l|l|} \hline
		\textbf{Konstante}       & \textbf{Wert}                                                                            \\
		\hline

		Boltzmannkonstante       & \centering\arraybackslash{} $k_\mathrm{B} \equiv \SI{1,38064852e-23}{\joule\per\kelvin}$ \\
		Universelle Gaskonstante & \centering\arraybackslash{} $R \equiv \SI{8,31446261815324}{\joule\per\kelvin\per\mole}$ \\
		Avogadro-Konstante        & \centering\arraybackslash{} $N_\mathrm{A} \equiv \SI{6,02214076e23}{\per\mole}$          \\
		Atomare Masseneinheit    & \centering\arraybackslash{} $u= \SI{1,6605390666050e-27}{\kg}$                           \\
		\hline
	\end{tabular}
\end{table}




\section{Grundlegende Formeln der Thermodynamik}

% !TeX root = Theo_IV.tex

\part{Thermodynamik}

\chapter{Grundlagen und Postulate\label{sec:grundlagen und postulate}}
\section{Zugang zur Thermodynamik}

Üblicherweise wird die Thermodynamik induktiv entwickelt. Aus Erfahrungstatsachen wie Wärme, Temperatur und Abläufen von thermodynamischen Maschinen werden Konzepte und Gesetze wie die Energieerhaltung und die Entropie abgeleitet.

In dieser Vorlesung wird stattdessen der axiomatische Zugang gewählt.
Aus Postulaten zur Energie und insbesondere zur Entropie wird die Thermodynamik aufgebaut und hergeleitet.
Anschließend werden die Konsequenzen dann mit den Erfahrungstatsachen abgeglichen.
Diese Postulate sind die Essenz der Entwicklung der Theorie und sie helfen, die Struktur der Thermodynamik sichtbar zu machen.

Ähnlich lassen sich auch andere Gebiete der Physik behandeln.
So können z.~B. aus dem Hamiltonschen Prinzip die mechanischen Bewegungsgleichungen und aus den Maxwell-Gleichungen die elektrischen und magnetischen Gesetze hergeleitet werden.


\section{Was ist Thermodynamik?\label{sec:was ist thermodynamik}}

Die meisten realen physikalischen Systeme bestehen aus einer sehr großen Zahl von Teilchen. Deren Behandlung mithilfe von mechanischen Bewegungsgleichungen ist allerdings mehr als unhandlich. 
Statt der mikroskopischen Beschreibung mit \num[print-unity-mantissa=false]{1e24} Koordinaten (Ort, Impuls, Molekülfreiheitsgrade) werden nur wenige makroskopische thermodynamische Variablen herangezogen, um ein System handhabbar zu beschreiben.

Dieser Ansatz ist auch physikalisch rechtfertigbar, denn bei realen Messungen findet automatisch eine intrinsische Mittelung statt.
Zum einen findet eine zeitliche Mittelung statt, denn die mikroskopische Bewegung findet auf Zeitskalen von \qty[print-unity-mantissa=false]{1e-15}{\s} bis \qty[print-unity-mantissa=false]{1e-12}{\s}statt, während makroskopische Messungen im Allgemeinen nicht kürzer als \qty[print-unity-mantissa=false]{1e-7}{\s} sind.
Es findet also eine Messung in gewissen Maßstäben zeitunabhängiger Kombinationen der über \num[print-unity-mantissa=false]{1e24} Koordinaten statt.

Zum anderen kommt es zu einer räumlichen Mittelung \textendash{} zum Vergleich: mikroskopische Abmessungen liegen bei $\approx\qty{0.1}{\nm}$ (Atomradius, Gitterkonstante), während makroskopische Messungen in der Regel bei über \qty{100}{\nm} liegen (Größenordnung der Wellenlänge von sichtbarem Licht).
Also wird meist über weit mehr als \num[print-unity-mantissa=false]{1e9} Atome oder Moleküle gemittelt.

\begin{figure}[htbp]
    \centering
    \tfigSystemWithManyParticles
    \caption{System mit vielen Teilchen, z.~B. ein Gas.}
    \label{fig:SystemWithManyParticles}
\end{figure}

Es verbleiben nur wenige Kenngrößen. Mechanische Größen sind zum Beispiel
\begin{itemize}
    \item Volumen $V$,
    \item Druck $P$,
    \item Oberfläche $F$,
    \item Oberflächenspannung $\sigma$ und
    \item hydrodynamische Flussfelder.
\end{itemize}

In der Elektrodynamik misst man in der Regel unter anderem
\begin{itemize}
    \item Ladung $Q$,
    \item Strom $I$,
    \item Magnetisierung $\vec M$,
    \item Magnetfeld $\vec H$,
    \item Polarisation $\vec P$ und
    \item das elektrische Feld $\vec E$.
\end{itemize}

Neu ist jetzt folgendes:
\begin{formal}
    Die Thermodynamik behandelt die makroskopischen Folgen (z.~B. die Wärme) derjenigen Koordinaten, die sich herausmitteln (z.~B. die einzelnen Teilchengeschwindigkeiten).% $\leftrightarrow$ Wärme.
\end{formal}

Die Zufuhr von Wärme in ein System führt z.~B. zur Anregung von atomarer Bewegung und damit einer Temperatur $T$.

In der Mechanik wird das Energie- bzw. Arbeitsdifferential als
\begin{align*}
    \diffa{W} = \vec F\cdot\diffa{\vec r}
\end{align*}
definiert. Hier wird diese Definition nun zu einem Produkt einer \emph{intensiven} Größe und dem Differential einer \emph{extensiven} Größe verallgemeinert:
\begin{align*}
    \diffa{E} = \underbrace{\text{verallgemeinerte Kraft}\: J}_{\text{intensiv}} \times \diffa{\underbrace{(\text{verallgemeinerter Weg}\: X)}_{\text{extensiv}}}.
\end{align*}
Dabei sind $(X,J)$ sogenannte zueinander \emph{konjugierte Variablen}. Die Einheit des Produkts $X\times J$ muss stets eine Energieeinheit sein. Wir werden im Verlauf des Kapitels den Unterschied zwischen intensiven und extensiven Größen erläutern. 

Bereits bekannte Beispiele sind
\begin{itemize}
    \item Druckarbeit $-P\diffa{V}$,
    \item Oberflächenarbeit $\sigma\diffa{F}$,
    \item Magnetisierungsarbeit $\mu_0\vec H\cdot\diffa{\vec M}$ und
    \item Polarisierungsarbeit: $\vec E\cdot\diffa{\vec P}$.
\end{itemize}
Im Verlaufe der Vorlesung wird eine neue Arbeit eingeführt, die den Wärmeübertrag und damit Energietransfer auf verborgene atomare Freiheitsgrade oder Moden beschreibt:
\begin{align*}
    \text{Energietransfer}=\text{Wärmeübertrag}=T\diffa{S}
\end{align*}
mit Temperatur $T$ und Entropie $S$.



\section{Modellsystem, Parameter und Begriffe}

Um ein Grundkonzept zu entwickeln, wird zunächst ein einfaches, idealisiertes System vorausgesetzt, das makroskopisch homogen und isotrop sowie elektrisch neutral ist, in dem keine chemischen Reaktionen ablaufen und das keine elektrischen, magnetischen oder gravitativen Felder besitzt. Auch Randeffekte werden zunächst vernachlässigt, indem angenommen wird, dass das System unendlich groß ist und damit keine Oberfläche hat.

Für dieses System werden dann Parameter wie das Volumen $V$ und die Stoffmengen der beteiligten chemischen Substanzen bestimmt.

Zunächst werden einige Definitionen erläutert:
\begin{enumerate}
    \item Als \emph{Normalbedingungen} bzw. \emph{Standardbedingungen} wird ein Zustand bei \qty{0}{\degreeCelsius} und \qty{1013}{\milli\bar} bezeichnet.
    \item \emph{Stoffmenge} (veraltet auch Molzahl) $\sm$: \qty{1}{\mole} einer Substanz entspricht einer Anzahl von Atomen oder Molekülen, die der Avogadro-Konstante/Loschmidt-Zahl $\avogadro \equiv \qty{6,02214076e23}{\per\mole}$ entspricht\footnote{
          Die Avogadro-Konstante besitzt die Einheit \si{\per\mole} während die Avogadro-Zahl \num{6,02214076e23} dimensionslos ist.
          Bei der Definition der Loschmidt-Zahl kann es aufgrund der historischen Entwicklung zu einiger Verwirrung kommen. Ursprünglich definierte Josef Loschmidt in seiner Arbeit \anf{Zur Grösse der Luftmoleküle} eine Zahl von in einer Volumeneinheit enthaltenen Luftmolekülen, war aber damals noch nicht sehr präzise mit seiner Definition. Als sogenannte Loschmidt-Konstante $N_\mathrm{L}$ wird heute der Wert \qty{2.686780111e25}{\per\cubic\m} definiert, welcher über das molare Volumen eines idealen Gases $V_{m0} = \qty{22,414}{\liter\per\mole}$ mit der Avogadro-Konstante $\avogadro=$ zusammenhängt: $N_\mathrm{L}=\avogadro/V_{m0}$. \cite{lit:loschmidt_constant,lit:loschmidt}
          Im deutschsprachigen Raum wird gelegentlich der Begriff Loschmidt-Zahl aber auch synonym mit der (dimensionslosen) Avogadro-Zahl verwendet. Um Verwirrungen zu vermeiden, wird im Folgenden ausschließlich von der Avogadro-Konstante gesprochen.}.

          Historisch wurde diese Definition gewählt, weil sie der Zahl der Atome in \qty{12}{\g} des Isotops \isotope[12]{C} von Kohlenstoff entspricht.
          Zur Referenz: \qty{1}{\mole} ist auch die Zahl der Moleküle eines idealen Gases unter Normalbedingungen in einem Volumen von $V=\qty{22,413}{\liter}$.

          Die Stoffmenge wird aus einer Teilchenzahl $N_k$ der Molekülsorte $k$ folgendermaßen berechnet:
          \begin{align*}
              \sm_k = \frac{N_k}{\avogadro}.
          \end{align*}
    \item Wir definieren ferner den \emph{Stoffmengenanteil} (früher Molenanteil genannt) als
          \begin{align*}
              x_k  = \frac{\sm_k}{\sum_i \sm_i}
          \end{align*}
          und das \emph{molare Volumen} bzw. \emph{Molvolumen} als
          \begin{align*}
              V_m = \frac{V}{\sum_i \sm_i}.
          \end{align*}
          Beide beschreiben Anteile am Gesamtsystem.
    \item Es wird unterschieden zwischen extensiven Parametern, wie dem Volumen $V$ oder Stoffmengen $\sm_1,\ldots,\sm_r$, die sich beim Zusammenführen mehrerer Teilsysteme additiv verhalten, also mit dem Volumen wachsen,
          \begin{align*}
              2\times(V,\sm_1,\ldots,\sm_r) \rightarrow (2V,2\sm_1,\ldots,2\sm_r)
          \end{align*}
          und intensiven Parametern wie $x_1,\dots,x_r,V_m$, die keine Änderung bei wachsendem Volumen erfahren. Dazu gehören auch die Temperatur $T$ und der Druck $P$.
    \item Als \emph{Zustandsgrößen} werden Größen bezeichnet, die unabhängig von der Vorgeschichte des Systems sind und einfach seinen Zustand beschreiben. Dazu gehören unter anderem das Volumen, die Stoffmenge, die Temperatur und der Druck.
\end{enumerate}




\section{Postulate zur inneren Energie und 1. Hauptsatz der Thermodynamik}

%Die Geschichte des Energiebegriffs beginnt mit Gottfried W. Leibniz im Jahr 1686. Er erkannte, dass die Wirkung einer mechanischen Bewegung 

\paragraph*{Innere Energie}

Die \emph{innere Energie} ist zugleich eine Zustandsgröße und eine extensive Größe.
\begin{formal}
    Makroskopische Systeme besitzen eine genau definierte \formalemph{innere Energie} $U$ (bezogen auf einen willkürlichen Grundzustand), die erhalten bleibt.
\end{formal}

Je nach Temperaturbereich wird ein Energienullpunkt festgelegt.
Für Energien $k_\mathrm{B}T$ im Bereich der Bindungsenergien von Molekülen können z.~B. ruhende Moleküle als Referenz gelten.


\paragraph*{Thermodynamisches Gleichgewicht}

Wir machen die Erfahrung, dass Systeme einfache Endzustände mit einer kleinstmöglichen Zahl von makroskopischen Variablen anstreben. Beispielsweise bewirkt die Reibung in einer Flüssigkeit in einem Glas, dass eine turbulente Strömung zur laminaren Strömung wird und allmählich ganz zur Ruhe kommt.

Diese Beobachtung führt uns zum ersten Postulat:
\begin{postulate}
    \label{post:gleichgewichtszustaende}
    Es gibt spezielle Zustände eines Systems, sogenannte \emph{Gleichgewichtszustände}, die makroskopisch vollkommen durch die Angabe weniger Zustandsgrößen beschrieben sind. 
    Solche Zustandsgrößen sind beispielsweise die innere Energie $U$, das Volumen $V$ und die Stoffmengen $\sm_1,\sm_2,\ldots$ der chemischen Komponenten.
\end{postulate}

In komplexeren Systemen muss z.~B. noch die Polarisation, die Magnetisierung und die Oberfläche berücksichtigt werden. Analog zum Volumen werden diese Größen über ein Arbeitsdifferential $\diffa{W} = J\diffa{X}$ in das System aufgenommen.

Im makroskopischen Gleichgewichtszustand werden viele mikroskopische Zustände im Messzeitraum angenommen, die mit dem makroskopischen Zustand $(U,V,\sm_1,...,\sm_r)$ vereinbar sind. So können z.~B. sehr viele verschiedene Kombinationen aus Teilchenenergien zur gleichen mittleren Energie führen. Normalerweise hat ein System kein \anf{Gedächtnis}, es verhält sich im Prinzip zufällig. Dies wird durch die Ergodenhypothese beschrieben und später präzisiert.

Es gibt aber auch Zustände, für die diese Annahme nicht gilt. Beim \emph{metastabilen Gleichgewicht} sind nicht alle Zustände in der Messzeit erreichbar und die Vorgeschichte ist relevant. Trotzdem ist der Formalismus der Thermodynamik in Teilbereichen davon oder für kurze Zeitspannen anwendbar.


Ein Beispiel dafür %wäre der Übergang von Glas (ungeordnetes System) zu Kristall (geordnet). 
wäre Glas als ungeordnetes, metastabiles System, das eigentlich zur Kristallisation tendiert, es aber nicht tut. Andere Beispiele sind Gedächtnislegierungen mit eingefrorenen Spannungen oder radioaktive Substanzen, die ihre Zusammensetzung durch spontanen Zerfall der Atome verändern.


\paragraph*{Wände}

Wände, Grenzen oder Ränder isolieren ein System und kontrollieren damit die Werte der Zustandsgrößen sowie den Energiefluss, wie am folgenden Beispiel ersichtlich.

Betrachte ein System aus zwei Kammern (siehe \Abbref{fig:TwoChambersSeparatedByPiston}), die durch einen beweglichen Kolben voneinander getrennt werden. Das gesamte System ist von einer festen Wand umgeben. Die Beschaffenheit des Kolbens oder der Wand kann verschiedener Art sein:
\begin{itemize}
    \item Wände können beweglich oder fest sein und kontrollieren so die innere Energie über mechanische Arbeit.
    \item (Semi-) permeable Wände kontrollieren die Stoffmengen $\sm_k$, indem sie beispielsweise nur bestimmte Stoffe hindurchlassen.
    \item Wärmeleitende Wände erlauben einen Wärmefluss, während thermisch isolierende Grenzen ein Angleichen der Temperatur verhindern. Durch Wände werden also thermisch abgeschlossene Systeme ermöglicht, was eine Definition des Wärmebegriffs erlaubt.
\end{itemize}

\begin{figure}[b]
    \centering
    \tfigTwoChambersSeparatedByPiston
    \caption{Zwei Kammern 1 und 2 sind von einer Wand umgeben und durch einen Kolben voneinander getrennt. }
    \label{fig:TwoChambersSeparatedByPiston}
\end{figure}




\paragraph*{Energiemessung}

In einem thermisch isolierten System ändert sich die innere Energie genau mit der mechanischen Arbeit $\Delta W$, die an dem System verrichtet wird,
\begin{align*}
    U(B)-U(A) =\Delta W(A\rightarrow B).
\end{align*}
Auf diese Weise kann die Änderung der inneren Energie durch die mechanische Energie definiert werden, welche wir bereits kennen und messen bzw. berechnen können.
Zum Beispiel erhöht sich die innere Energie, wenn das System mechanisch komprimiert wird (siehe \Abbref{fig:RectangularBoxWithPiston}).

\begin{figure}[t]
    \centering
    \tfigRectangularBoxWithPiston
    \caption{Ein System kann mechanisch komprimiert werden, sodass sich die innere Energie erhöht. Es wird Arbeit an dem System verrichtet. Der Prozess ist in diesem Fall umkehrbar. }
    \label{fig:RectangularBoxWithPiston}
\end{figure}

Allerdings könnte einem System, das eine Flüssigkeit enthält, auch Energie durch Quirlen zugeführt werden, wie in \Abbref{fig:WaterStiringIceCubes} dargestellt, wodurch sich die Temperatur erhöht. Dieser Prozess ist jedoch nicht umkehrbar, da die Temperatur durch Quirlen natürlich nicht wieder verringert werden kann.

\begin{figure}[htbp]
    \centering
    \tfigWaterStiringIceCubes
    \caption{Ein Gefäß mit Wasser einer bestimmten Temperatur (Zustand A) kann durch Quirlen erwärmt werden (Zustand B). Allerdings ist dieser Vorgang nicht auf mechanische Weise umkehrbar, denn durch Quirlen kann das Wasser nicht abgekühlt werden. }
    \label{fig:WaterStiringIceCubes}
\end{figure}

\begin{formal}
    Die innere Energie wird nicht direkt gemessen. Stattdessen kann die Änderung der inneren Energie eines Systems beim Übergang vom Zustand $A$ zum Zustand $B$ bestimmt werden, indem der Prozess $A\rightarrow B$ oder $B\rightarrow A$ rein mechanisch bewirkt wird.
\end{formal}



\paragraph*{Wärme und Wärmeübertrag}

Ein Prozess $A\rightarrow B$ muss nicht rein mechanisch ablaufen. Dabei nimmt das System sogenannte \emph{Wärme} auf oder gibt sie ab. Diese Wärme können wir definieren als Differenz der gesamten Änderung der inneren Energie und jenes Teils, der durch mechanische Arbeit verrichtet wird,
\begin{align*}
    \Delta Q(A\rightarrow B) = [U(B)-U(A)] - \Delta W(A\rightarrow B).
\end{align*}
Dieser Zusammenhang wird durch den 1. Hauptsatz der Thermodynamik zusammengefasst:
\begin{formal}
    \textbf{1. Hauptsatz der Thermodynamik (Energieerhaltungssatz):}
    \label{hs:erster}
    Die innere Energie eines Systems ändert sich mit der zugeführten Wärme und der am System verrichteten Arbeit\footnotemark,
    \begin{align*}
        \diff U = \udiff Q + \udiff W.
    \end{align*}
\end{formal}

\footnotetext{Die hier verwendeteten Notationen mit vollständigem Differential $\diff$ und unvolllständigem Differential $\udiff$ werden im Anschluss erläutert.}

Betrachte als Beispiel eine Volumenänderung, die aufgrund der resultierenden Druckänderung eine Arbeit am System darstellt\footnote{Diese Änderung muss ausreichend langsam erfolgen, damit es sich um eine quasistatische Prozessführung handelt. Bei einer schnellen Änderung entstehen Turbulenzen und das System befindet nicht über den ganzen Zeitraum im thermodynamischen Gleichgewicht.},
\begin{align*}
    \udiff W = -P\diff V.
\end{align*}
Anders als die innere Energie sind Wärme und Arbeit allerdings keine Zustandsgrößen, denn verschiedene Kombinationen von Wärme- und Arbeitszufuhr können auf den gleichen Zustand eines Systems führen (\Abbref{fig:WQAreNoStateFunctions}). Sie dienen beide nur der Änderung der inneren Energie.

\begin{figure}[htb]
    \centering
    \tfigWQAreNoStateFunctions
    \caption{Die mechanische Arbeit $\Delta W$ und die Wärme $\Delta Q$ sind keine Zustandsfunktionen. Einem System können verschiedene Verhältnisse $\Delta W_1,\Delta Q_1$ und $\Delta W_2,\Delta Q_2$ zugeführt werden, die zur selben inneren Energie $U$ führen, sodass sich das System im gleichen Zustand befindet. }
    \label{fig:WQAreNoStateFunctions}
\end{figure}


Bemerkungen:
\begin{itemize}
    \item Der Prozess kann \emph{quasistatisch} erfolgen, also so langsam, dass immer ein Gleichgewicht vorliegt und damit der Druck $P$ homogen ist.
    \item Der Prozess kann \emph{nicht-quasistatisch} erfolgen, wodurch es zu Turbulenzen kommt und $P=P(\vec r,t)$ ortsabhängig wird. Es wird eine Überschussarbeit verrichtet, die in Wärme dissipiert wird.
    \item Zusätzlich zum totalen Differential $\diff$ muss auch das \emph{unvollständige Differential} $\udiff$ eingeführt werden, welches für Größen (wie $Q$ und $W$) angewendet wird, die keine Zustandsfunktionen sind, sondern sogenannte wegabhängige \emph{Prozessgrößen}.
    \item Mechanische Arbeit stellt einen Energieübertrag dar.
    \item Ein quasistatischer Wärmeübertrag stellt nach $\udiff Q=\diff U+P\diff V$ ebenfalls einen Energieübertrag dar.
\end{itemize}


\paragraph*{Wärmeäquivalent}

Das sogenannte \emph{Wärmeäquivalent} besagt, dass Wärme als Energie quantifizierbar ist. Es gilt
\begin{align*}
    \Delta Q = mc\Delta T
\end{align*}
mit spezifischer Wärmekapazität $c$.
Die häufig verwendete Einheit Kalorie entspricht der Energie, um \qty{1}{\g} Wasser bei \qty{1013}{\milli\bar} von \qty{14,5}{\degreeCelsius} auf \qty{15,5}{\degreeCelsius} zu erwärmen. Es ist dabei $\qty{1}{cal} = \qty{4,1855}{\joule}$.



\section{Postulate zur Entropie}

Aus der Experimentalphysik erinnern wir uns, dass die Entropie $S$ mit der Irreversibilität eines Prozesses zusammenhängt. Später werden wir sehen, dass die Entropie durch $S=k_\mathrm{B}\ln{\Omega}$ mit dem Phasenraumvolumen $\Omega$ bestimmt ist.

Zuerst betrachten wir ein abgeschlossenes System mit festgelegten Zwangsbedingungen, beispielsweise zwei Kammern, die durch einen anfangs undurchlässigen, festen und wärmeisolierenden Kolben getrennt sind. Diese Eigenschaften des Kolbens stellen die Zwangsbedingungen dar. Es stellt sich die Frage, welchen Gleichgewichtszustand das System beim Entfernen einer oder mehrerer dieser Zwangsbedingungen einnimmt. Erfahrungsgemäß wissen wir z.~B., dass sich die Temperaturen beider Kammern angleichen, wenn ein Wärmeaustausch erlaubt wird, der Kolben also wärmedurchlässig ist.

Eine allgemeine Antwort auf diese Frage kann mithilfe des Extremalprinzips gewonnen werden, was uns auf das zweite Postulat führt:

% Das bringt uns zum zweiten Postulat, dem Extremalprinzip zur Entropie:
\begin{postulate}
    \label{post:entropie_maximierung}
    Gegeben sei ein isoliertes System, das durch Zwangsbedingungen unterteilt ist. Dann existiert eine Funktion der extensiven Parameter ($U^{(1)},V^{(1)},\sm_k^{(1)};U^{(2)},V^{(2)},\sm_k^{(2)}; \ldots$), genannt \emph{Entropie} $S$, die für alle Gleichgewichtszustände wohldefiniert ist und folgende Eigenschaften besitzt: Lässt man die Zwangsbedingungen fallen, so nehmen die extensiven Parameter Werte an, welche die Entropie maximieren. Der dann erreichte Endzustand heißt \emph{stabiles Gleichgewicht}.

    $S=S(\{U^{(\alpha)},V^{(\alpha)},\sm_k^{(\alpha)}\})$ heißt \emph{entropische Fundamentalbeziehung}. Sie enthält die gesamte Information über das System.
\end{postulate}


Es sei bemerkt, dass dafür die Entropie $S$ von den extensiven Variablen abhängen muss. Dieses Postulat erscheint auf den ersten Blick recht willkürlich, wird aber durch seine Konsequenzen gerechtfertigt, welche sich mit den Erfahrungstatsachen decken\footnote{Außerdem hat sich die Anwendung des Extremalprinzips in vielen Bereichen der Physik erfolgreich bewährt (siehe das Hamiltonsche Prinzip der kleinsten Wirkung sowie die Minimierung der Lagrange-Dichte bei der Quantenfeldtheorie).}. Diese werden in den folgenden Kapiteln erläutert.

Es ist außerdem sinnvoll, einige Eigenschaften für die Entropie zu fordern, die im nächsten Postulat zusammengefasst sind:
\begin{postulate}
    \label{post:eigenschaften_entropie}
    Die Entropie eines zusammengesetzten Systems ist gleich der Summe der Entropien der Teilsysteme,
    \begin{align*}
        S =\sum_\alpha S^{(\alpha)}, \quad S^{(\alpha)} = S^{(\alpha)}(U^{(\alpha)},V^{(\alpha)},\sm_1^{(\alpha)},\ldots, \sm_r^{(\alpha)}).
    \end{align*}
    $S$ ist stetig, differenzierbar und eine streng monoton ansteigende Funktion der inneren Energie $U$.
\end{postulate}

%Wir werden später sehen, dass aus dem zweiten Punkt die Temperatur $T$ hergeleitet werden kann. 
Hieraus folgt, dass $S$ eine extensive Größe ist, denn
\begin{align*}
    S(\lambda U,\lambda V,\lambda\sm_1,\ldots, \lambda\sm_r) = \lambda^1 S(U,V,\sm_1, \ldots,\sm_r).
\end{align*}
Genauer: die Entropie ist eine  homogene Funktion ersten Grades der extensiven Parameter.

Beispielsweise lässt sich für ein allgemeines System mit vielen Komponenten und einer Gesamtstoffmenge $\lambda=\sm=\sum_{k=1}^r\sm_k$,
\begin{align*}
    S(U,V,\sm_1, \dots , \sm_r) = N\cdot S\left(\frac{U}{N},\frac{V}{U},\frac{\sm_1}{N}, \dots , \frac{\sm_r}{N}\right)
\end{align*}
ein Einkomponentensystem ableiten,
\begin{align*}
    S(U,V,N) = N\cdot S(u,v,1) = N\cdot s(u,v),
\end{align*}
mit Energie pro Mol $u=U/N$, Molvolumen $v=V/N$ und Entropie pro Mol $s$. Diese Darstellung wird häufig verwendet, da Angaben in Mol ausreichen und die Rechnungen vereinfachen.

Aus dem Postulat \ref{post:eigenschaften_entropie} folgt ferner, dass für
\begin{align*}
    \left(\frac{\partial S}{\partial U}\right)_{V,\sm_1, \dots ,\sm_r} > 0
\end{align*}
die Beziehung $S(U)$ invertierbar ist,
\begin{align}
    \label{eq:energetische fundamentalbeziehung}
    U = U(S,V,\sm_1, \dots,\sm_r).
\end{align}
Diese neue Beziehung ist als \emph{energetische Fundamentalbeziehung} bekannt und wir werden sehen, dass aus dem Postulat \ref{post:entropie_maximierung} auch eine Minimierung der Energie folgt.



\section{Das Nernst-Postulat (3. Hauptsatz)\label{sec:Nernst}}



\begin{postulate}[Nernst-Postulat]
    \label{post:nernst}
    Für jeden Variablensatz $V,\sm_1,...,\sm_k$ gibt es einen Punkt, an dem gilt:
    \begin{align}
        \label{eq:nernst_postulat}
        S = 0 \quad \text{bei} \quad T =\left(\frac{\partial U}{\partial S}\right)_{V,\sm,...}=0
    \end{align}
    (siehe \Abbref{fig:SchemaFundamentalbeziehung}). Dieser Nullpunkt wird in der Realität jedoch nicht erreicht. 
\end{postulate}

\begin{figure}[H]
    \centering
    \tfigSchemaFundamentalbeziehung
    \caption{Schema zur energetischen (links) und entropischen (rechts) Fundamentalbeziehung. Die Entropie $S$ besitzt im Gegensatz zur inneren Energie einen eindeutigen Nullpunkt.  }
    \label{fig:SchemaFundamentalbeziehung}
\end{figure}

Bemerkungen:
\begin{itemize}
    \item Die Entropie $S$ besitzt einen eindeutigen Nullpunkt, im Gegensatz zu der inneren Energie $U$ (zu der immer eine Konstante addiert werden kann).
    \item Die Größe $T$ wird im Folgenden als Temperatur interpretiert.
    \item Aus dem Postulat \ref{post:eigenschaften_entropie} folgt, dass die Temperatur immer größer oder gleich null ist. Zudem lässt Postulat \ref{post:nernst} schließen, dass es eine absolute Temperaturskala (die Kelvinskala) gibt.
    In der Formulierung nach Planck von 1907 heißt es, dass der Temperaturnullpunkt $T=0$ nicht erreicht wird.
\end{itemize}




\begin{summary}
    \formalemph{Was zeichnet die Thermodynamik aus?}
    Statt unzähliger, einzelner mikroskopischen Größen (wie Koordinaten und Impulse), werden in der Thermodynamik makroskopische Kenngrößen betrachtet. Dies entspricht sowohl einer räumlichen als auch einer zeitlichen Mittelung über die mikroskopischen Größen. Zu den makroskopischen Kenngrößen gehören beispielsweise Volumen, Druck, Temperatur und Energie.
    
    Mit der \emph{Standard- oder Normalbedingung} wird die Prozessbedingung von $1013$ \si{\milli \bar} und $0$\si{\celsius} bezeichnet.

    \formalemph{Wesentliche Größen und Begriffe}
    Wir definieren die 
    \begin{itemize}
        \item \emph{Stoffmenge} als $\sm = N/N_A$, also die Teilchenzahl $N$ geteilt durch die \emph{Avogadro-Konstante}, 
        \item den \emph{Stoffmengenanteil} als $x_k=\sm_k/\sum_i \sm_i$,
        \item das \emph{Molvolumen} als $V_m = V/\sum_i \sm_i$ (alternativ auch mit $v$ gekennzeichnet) und
        \item das Wärmeequivalent $\Delta Q= mc\Delta T$.
    \end{itemize}

    \emph{Zustandsgrößen} beschreiben den Zustand eines Systems, unabhängig von dessen Vorgeschichte (z.~B. $V$, $\sm$, $T$, $P$). 
    Einige Größen sind keine Zustandsgrößen, wie z.~B. die Wärme $Q$ und die Arbeit $W$, denn sie beschreiben keinen eindeutigen Zustand des Systems. Für diese sogenannten \emph{Prozessgrößen} wir das \emph{unvollständige Differential} $\delta$ verwendet. 

    \emph{Extensive} Größen verhalten sich additiv mit dem Volumen des Systems. Dazu gehört neben dem Volumen z.~B. auch die Stoffmenge $\sm$.\\ 
    \emph{Intensive} Größen erfahren keine Änderung mit dem Volumen, z.~B. $T$, $P$, $x$ und $V_m$.
    Eine weitere extensive Kenngröße ist die innere Energie $U$, welche durch die entropische Fundamentalbeziehung gegeben ist; sie ist ebenfalls eine Zustandsgröße. 

    Die \emph{quasistatische} Prozessführung beschreibt eine langsame Prozessführung, welche dazu führt, dass das System sich zu jedem Zeitpunkt im Gleichgewichtszustand befindet. \\
    \emph{Nicht-quasistatische} Prozesse, sind Prozesse, welche bedingt durch eine schnelle Prozessführung Störungen (wie z.~B. Turbulenzen) erfahren und damit Wärme dissipieren. Dies führt dazu, dass sich das System nicht durchgehend in Gleichgewichtszuständen befindet. 

    \formalemph{Postulate und Hauptsätze der Thermodynamik}
    Wir haben ferner eine Reihe von Postulaten aufgestellt, die grob durch Erfahrungstatsachen motiviert sind. Nach dem \formalemph{Postulat~\ref{post:gleichgewichtszustaende}} hat ein System sogenannte Gleichgewichtszustände, welche durch wenige Zustandsgrößen ~($U$, $V$, $\sm$, \ldots) beschrieben werden können. 
    Innerhalb dieser makroskopischen Beschreibung verhält sich das System mikroskopisch zufällig. Im Allgemeinen führen viele mikroskopische Zustände zum gleichen makroskopischen Zustand. 

    Nach dem \formalemph{Postulat~\ref{post:entropie_maximierung}} existiert eine weitere Zustandsfunktion $S$ (welche von den extensiven Größen abhängt, die gesamte Information des Systems enthält und durch die entropische Fundamentalbeziehung beschrieben wird), die wir Entropie nennen. Entfernt man Zwangsbedingungen eines Systems, so nehmen dessen extensiven Größen in Folge die Werte an, welche die Entropie maximieren. 

    Die Entropie ist nach \formalemph{Postulat~\ref{post:eigenschaften_entropie}} eine additive Größe und soll stetig, differenzierbar und eine streng monoton ansteigende Funktion der inneren Energie $U$ sein. Damit gilt auch, dass $S$ eine extensive Größe ist. 

    Zuletzt beschreibt das \formalemph{Nernst-Postulat}, der \formalemph{dritte Hauptsatz der Thermodynamik} (Postulat~\ref{post:nernst}), einen Entropie- und Temperaturnullpunkt $S=0$ bei $T=0$, der aber praktisch nicht erreicht werden kann. 

    Außerdem haben wir den \formalemph{ersten Hauptsatz der Thermodynamik} kennengelernt: den \emph{Energieerhaltungssatz}. Er besagt, dass sich die innere Energie eines abgeschlossenen Systems nach der Gleichung 
    \begin{align*}
        \diff U = \udiff Q + \udiff W
    \end{align*}
    mit der zu- oder abgeführten Wärme und der am System verrichteten Arbeit ändert. 
\end{summary}
% !TeX root = Theo_IV.tex

\chapter{Folgerungen und Gleichgewichtsbedingungen}

In diesem Kapitel soll die Entwicklung der thermodynamischen Theorie sowie die Auswertung des zweiten Postulates erfolgen.

\section{Konjugierte Variablen (Energiedarstellung)}

Zunächst wollen wir die innere Energie $U$ in der differentiellen Form aufschreiben:
\begin{align*}
    \diff U = \left(\frac{\partial U}{\partial S}\right)_{V,\sm_1,\ldots,\sm_r} \diff S + \left(\frac{\partial U}{\partial V}\right)_{S,\sm_1,\ldots,\sm_r} \diff V + \sum_{j = 1}^r \left(\frac{\partial U}{\partial \sm_j}\right)_{S,V,\sm_1,\ldots,\sm_{j - 1},\sm_{j + 1},\ldots,\sm_r} \diff \sm_j.
\end{align*}
Die Faktoren vor den einzelnen Differentialen deuten wir nun, indem wir sie als Größen definieren, welche jeweils zur Differentialgröße konjugiert sind.

Wir erhalten somit die Temperatur $T$
\begin{align}
    \label{eq:def_temperatur}
    T \equiv \left(\frac{\partial U}{\partial S}\right)_{V,\sm_1,\ldots,\sm_r} \geq 0
\end{align}
als konjugierte Größe zur Entropie $S$, den Druck $P$
\begin{align}
    \label{eq:def_druck}
    P \equiv -\left(\frac{\partial U}{\partial V}\right)_{S,\sm_1,\ldots,\sm_r}
\end{align}
als konjugierte Größe zum Volumen $V$ und die chemischen Potentiale $\mu_j$
\begin{align}
    \label{eq:def_chemische_potentiale}
    \mu_j \equiv \sum_{j = 1}^r \left(\frac{\partial U}{\partial \sm_j}\right)_{S,V,\sm_1,\ldots,\sm_{j - 1},\sm_{j + 1},\ldots,\sm_r},
\end{align}
welche jeweils zu $\sm_j$ konjugiert sind.

Die innere Energie lässt sich also in differentieller Form folgendermaßen schreiben:
\begin{align}
    \label{eq:innere_energie_differentielle_form}
    \diff U = T\diff S - P \diff V + \mu_1 \diff \sm_1 + \cdots + \mu_r \diff \sm_r.
\end{align}

Einige Bemerkungen:
\begin{itemize}
    \item Die Größen $U, S,V,\sm_k$ sind extensiv, während die dazu konjugierten Variablen $T,P,\mu_k$ intensive Größen sind.
    \item $J\diff X$ mit intensiver Variable $J$ und extensiver Variable $X$ muss die Einheit einer Energie haben. Folglich ist $[\mu_j] = [U]$ und $[S]=[U/T]$.
\end{itemize}

Es soll nun eine erste Interpretation der Terme in \eqref{eq:innere_energie_differentielle_form} gemacht werden.
Zur Vereinfachung setzen wir zunächst alle Terme $\diff \sm_k=0$.
Der erste Term ist
\begin{align*}
    T\diff S = \diff U+ P\diff V = \udiff Q,
\end{align*}
entspricht also der quasistatischen Wärmezufuhr $\udiff Q$.

Der zweite Term $-P\diff V$ entspricht der bereits eingeführten quasistatischen mechanischen Arbeit $\udiff W_\mathrm{mech}=-P\diff V$.


Die Deutung als Temperatur wird später weiter ausgeführt (siehe Kapitel \ref{sec:thermisches_gleichgewicht})

Zuletzt können wir die Terme
\begin{align*}
    \udiff W_ \mathrm{C} = \sum_{i = 1}^r \mu_i\diff \sm_i
\end{align*}
als quasistatische chemische Arbeit festlegen. Sie beschreibt die Energiezunahme bei Hinzufügen von Materie zu einem System (siehe auch Kapitel \ref{sec:gleichgewicht_bei_materiefluss}).

Insgesamt kann die innere Energie also geschrieben werden als
\begin{align}
    \label{eq:erster_HS_TD}
    \diff U = \udiff Q + \udiff W_\mathrm{mech} + \udiff W_\mathrm{C}.
\end{align}
Dies entspricht dem ersten Hauptsatz der Thermodynamik.
Dabei ist zu beachten, dass die Prozessgrößen ($Q$,$W_{mech}$ und $W_c$) unvollständige Differentiale bilden. Letztere sollen im nächsten Kapitel näher erläutert werden.

Zuletzt wollen wir noch einige Aussagen zur Wärmezufuhr $\udiff Q$ zusammenfassen:
\begin{itemize}
    \item Für quasistatische Prozesse gilt $\udiff Q=T\diff S$ und damit insbesondere $\udiff Q>0\implication T\diff S>0$. Eine Zufuhr von Wärme führt also zu einem Zuwachs der Entropie.
    \item Durch Umstellen erhält man
          \begin{align*}
              \diff S = \frac{1}{T}\udiff Q.
          \end{align*}
          Da das unvollständige Differential $\udiff Q$ mittels des Faktors $1/T$ das vollständige Differential $\diff S$ charakterisiert, nennt man diesen Faktor einen integrierenden Faktor.
    \item Betrachten wir die Wärmezufuhr für nicht quasistatische Prozesse, so heißt das, dass das System keine Gleichgewichtszustände durchläuft, sondern sich im Nichtgleichgewicht entwickelt. Für ein abgeschlossenes System liegt keine Wärmezufuhr vor. Mit $\udiff Q=0$ folgt nach Postulat \ref{post:entropie_maximierung} $\diff S >0$. Die Entropie nimmt für sogenannte irreversible Prozesse zu.\footnote{Irreversibel heißt hier, dass das System nicht mehr spontan in den Ausgangszustand übergeht, also die Entropie nicht abnimmt.} In nicht abgeschlossenen Systemen, also in Systemen, in denen eine Wärmezufuhr stattfinden kann, nimmt die Entropie für irreversible Prozesse entsprechend
          \begin{align*}
              \diff S\geq \frac{\udiff Q}{T}
          \end{align*}
          zu.
          Gleichheit gilt dabei ausschließlich für reversible Prozesse. Es sei angemerkt, dass damit ferner folgt, dass die Entropiezunahme auch auf Anteile zurückzuführen ist, die nicht mit der Wärmezufuhr $\udiff Q$ zusammenhängen.
\end{itemize}

\section{Unvollständige Differentiale}

Wir wollen nun näher auf die mehrfach erwähnten \emph{unvollständigen Differentiale} zurückkommen und eine mathematische Definition nachliefern. Ausgehend von einer Funktion $f(x_1,\ldots,x_r)$ bilden wir das Differential
\begin{align*}
    \diff f = \sum_{i=1}^r \frac{\partial f}{\partial x_i} \diff x_i.
\end{align*}
Für die zweiten Ableitungen gilt für die gemischten Differentiale mit dem \emph{Satz von Schwarz} Vertauschbarkeit der Form:
\begin{align*}
    \frac{\partial ^2 f}{\partial x_i \partial x_j} = \frac{\partial ^2 f}{\partial x_j \partial x_i}.
\end{align*}
Wir konstruieren eine Darstellung $\phi$, auch \emph{Pfaffsche} oder \emph{1-Form} genannt
\begin{align*}
    \phi (x_1,\ldots,x_r)=\sum _{i=1}^r \phi_i(x_1,\ldots,x_r)\diff x_i
\end{align*}
mit dem Differential
\begin{align*}
    \diff f = \phi = \sum_i \phi_i \diff x_i \quad\text{ mit }\quad \phi_i = \frac{\partial f}{\partial x_i}.
\end{align*}
Für einfach zusammenhängende Definitionsbereiche\footnote{Es sei angemerkt, dass für nicht zusammenhängende Gebiete die Integrabilitätsbedingung nur eine notwendige, keine hinreichende Bedingung ist.} ist die Erfüllung des Satzes von Schwarz in Form folgender \emph{Integrabilitätsbedingung}
\begin{align*}
    \frac{\partial \phi_i}{\partial x_j}=\left[\frac{\partial^2f}{\partial x_j\partial x_i}=\frac{\partial^2f}{\partial x_i\partial x_j}\right]=\frac{\partial \phi_j}{\partial x_i}
\end{align*}
hinreichend dafür, dass dieses Differential vollständig ist.
Wir wollen daran erinnern, dass uns diese Integrabilitätsbedingung bereits aus der Mechanik bekannt ist. Dort ist ein betrachtetes Kraftintegral genau dann wegunabhängig, wenn die Kraft als negativer Gradient eines Potentials geschrieben werden kann. In einfach zusammenhängenden Gebieten ist dies gleichbedeutend mit der Aussage, dass die Rotation der Kraft gleich null ist,
\begin{align*}
    \vec{K}=-\nabla U \equivalence \rot\vec{K}=0 \equivalence \epsilon_{ijk}\partial_jK_k.
\end{align*}
Der letzte Ausdruck ist äquivalent zu unserer Integrabilitätsbedingung.
Erfüllt das Differential diese Bedingung nicht, ist es unvollständig und wird mittels der Notation "$\udiff$" gekennzeichnet.
Wir wollen dies am Beispiel der Variablen $S$ und $V$ illustrieren:
\begin{align*}
    \phi = \udiff Q = T\diff S+0 \diff V\equiv \phi_1\diff S+\phi_2 \diff V.
\end{align*}
Wir identifizieren die zwei Funktionen $\phi_1$ und $\phi_2$, prüfen die Integrabilitätsbedingung
\begin{align*}
    \frac{\partial \phi_1}{\partial V} = \frac{\partial T}{\partial V}\neq \frac{\partial \phi_2}{\partial S}=\frac{\partial0}{\partial S}=0
\end{align*}
und stellen fest, dass diese nicht erfüllt ist. Die Wärmezustandsfunktion $Q(S,V)$ existiert nicht, stattdessen definiert $Q$ als Prozessgröße ein unvollständiges Differential.

\section{Zustandsgleichungen (in Energiedarstellung)}
Wir möchten nun weitere Definitionen einführen und den Begriff der \emph{Zustandsgleichung} erläutern. Dafür erinnern wir an das Energiedifferential der Form
\begin{align*}
    \diff U = T\diff S -P\diff V + \sum\mu_j\diff \sm_j,
\end{align*}
welches von den intensiven Variablen $T$,$P$ und $\mu_j$ mit ihren respektiven Zustandsgleichungen
\begin{align*}
    T     & =T(S,V,\sm_1,\ldots,\sm_r)     \\
    P     & =P(S,V,\sm_1,\ldots,\sm_r)     \\
    \mu_j & =\mu_j(S,V,\sm_1,\ldots,\sm_r)
\end{align*}
abhängt. Die Kenntnis dieser Zustandsgleichungen ist äquivalent zur Kenntnis der einzelnen Zustandsfunktion $U(S,V,\sm_1,\ldots,\sm_r)$, da wir die innere Energie über Aufintegration des Differentials erhalten.

Kommen wir erneut auf die Definition der \emph{extensiven} und \emph{intensiven} Größen zurück. Eine extensive Variable, wie z.B. die innere Energie, ist eine homogene Funktion ersten Grades, d.h. es gilt folgender Zusammenhang:
\begin{align*}
    U(\lambda S,\lambda V,\lambda \sm_1,\ldots,\lambda \sm_r) = \lambda U(S,V,\sm_1,\ldots,\sm_r).
\end{align*}
In anderen Worten, "die Variable skaliert mit der Größe des Systems".
Für intensive Variablen, wie z.B. die Temperatur, gilt jedoch
\begin{align*}
    T(\lambda S,\lambda V,\lambda \sm_1,\ldots,\lambda \sm_r)=T(S,V,\sm_1,\ldots,\sm_r),
\end{align*}
denn mit der Extensivität von $U$ folgt
\begin{align*}
    T(\lambda S,\ldots) = \frac{\partial U(\lambda S,\ldots)}{\partial(\lambda S)}=\frac{\partial U(S,\ldots)}{\partial S}=T(S,\ldots).
\end{align*}
Es handelt sich dabei, wie auch beim Druck und beim chemischen Potential, um eine homogene Funktion nullten Grades, die sich bei Skalierung des Systems nicht ändert.

Wir wollen nun zum nächsten thematischen Abschnitt übergehen und uns zwei Verallgemeinerungen anschauen, in entropischer und energetischer Darstellung.
Die Energiedarstellung kann wie folgt geschrieben werden:
\begin{align*}
    U(S,V,\sm_1,\ldots,\sm_r) \rightarrow U(S,X_1,X_2,\ldots,X_t).
\end{align*}
Wir benutzen also verallgemeinerte energetisch extensive Parameter, um die Fundamentalgleichung zu schreiben. Dabei erhalten wir die verallgemeinerten Ableitungen 
\begin{align*}
    \left(\frac{\partial U}{\partial S}\right)_{X_1,\ldots,X_t}                & \equiv T = T(S,X_1,\ldots,X_t)             \\
    \left(\frac{\partial U}{\partial X_j}\right)_{S,\ldots,X_{k\neq j},\ldots} & \equiv P_j = P_j(S,X_1,\ldots,X_t)         \\
\end{align*}
und das Energiedifferential
\begin{align}
    \label{eq:energetische_fundamentalgleichung}
    \diff U  & = T\diff S + \sum_{j=1}^tP_j\diff X_j.
\end{align}
Diese Abstraktion ermöglicht die Beschreibung viel allgemeinerer Systeme, für welche weiterhin das eingeführte Energie- bzw. Arbeitsdifferential $\diff U = P_j\diff X_j$ gilt.
Dabei bezeichnen $X_j$ die verallgemeinerten Volumina oder Wege und $P_j$ die Drücke oder Kräfte.
Es sei angemerkt, dass wir den Druck folglich nur über eine Energiefunktion, welche von S abhängt, erhalten können.

Betrachtet man ein einkomponentiges System, so liegt die Einführung neuer Größen, wie der \emph{molaren inneren Energie} $u$, nahe. Wir führen diese im Folgenden ein:
\begin{align*}
    u=\frac{U}{\sm}=u\left(\frac{S}{\sm},\frac{V}{\sm},\frac{\sm}{\sm}\right)\equiv u(s,v).
\end{align*}
Sie hängt folglich von der molaren Entropie $s$ und dem Molvolumen $v$ ab.
Wir wollen nun das Differential dieser Größe berechnen. Formal schreiben wir
\begin{align*}
    \diff u = \left(\frac{\partial u}{\partial s}\right)_v \diff s + \left(\frac{\partial u}{\partial v}\right)_s\diff v
\end{align*}
mit
\begin{align*}
    \left( \frac{\partial u}{\partial s} \right)_v=\left( \frac{\partial u}{\partial s} \right)_{V,\sm}&=\left( \frac{\partial U}{\partial S} \right)_{V,N} =T                                                             
\end{align*}
und 
\begin{align*}
    \left(\frac{\partial u}{\partial v}\right)_s   & = -P.       
\end{align*}
Das Konstanthalten des Molvolumens ist dabei äquivalent zum Konstanthalten des Volumens und der Molzahl (Stoffmenge). Die partielle Ableitung nach $s$ kann ferner nach $S$ umgeschrieben werden, womit wir die Temperatur $T$ erhalten.
Analog betrachten wir die partielle Ableitung nach $v$ und erhalten den negativen Druck $P$.
Insgesamt folgt das Differential
\begin{align*}
    \diff u = T\diff s -P\diff v.
\end{align*}
Bei einem einkomponentigem System reicht also die Kenntnis über die molaren Größen, um das gesamte System zu beschreiben.

\section{Entropiedarstellung}
Analog zur Energiedarstellung betrachten wir nun eine verallgemeinerte Entropiedarstellung,
\begin{align*}
    S(U,V,\sm_1,\ldots,N_t) \rightarrow S(X_0,X_1,\ldots,X_t).
\end{align*}
Dabei benutzen wir wieder verallgemeinerte, diesmal entropische, extensive Parameter, um die entropische Fundamentalgleichung aufzuschreiben. Analog zur vorigen energetischen Beschreibung erhalten wir das Differential der Form
\begin{align*}
    \diff S = \sum_{j=0}^tF_j\diff X_j ,\quad\text{ mit } \quad F_j\equiv\left( \frac{\partial S}{\partial X_j} \right)_{\ldots X_{k\neq j}}
\end{align*}
mit den entropischen, intensiven Parametern
\begin{align*}
    F_0 & =\frac{1}{T}(U,X_1,\ldots,X_t)                                    \\
    F_1 & =\frac{P}{T}(U,X_1,\ldots,X_t) \quad\text{ (für $X_1=V$)}         \\
    F_k & =-\frac{P_k}{T}(U,X_1,\ldots,X_t)                                 \\
    F_r & =-\frac{\mu_r}{T}(U,X_1,\ldots,X_t) \quad\text{ (für $X_r=\sm_r$)}.
\end{align*}
Die energetische und entropische Beschreibung der Thermodynamik sind äquivalent, jedoch werden wir bei Gleichgewichtsbetrachtungen, die aus dem zweiten Postulat folgen, mit der entropischen Fundamentalbeziehung arbeiten.

\section{Thermisches Gleichgewicht\label{sec:thermisches_gleichgewicht}}

Nun soll es darum gehen, den Inhalt der Postulate \ref{post:entropie_maximierung} und \ref{post:eigenschaften_entropie} auszuwerten.
Es wird folgen, dass sich $T$ so verhält, wie man es von einer Temperatur erwartet.

\paragraph*{Temperatur}

Wir starten wieder mit einem Modellsystem, das insgesamt abgeschlossen ist und aus zwei Untersystemen $(1)$ und $(2)$ besteht, wie in \Abbref{fig:DoppelsystemUSfesteWaermeleitendeWand} darstellt.
Die beiden Untersysteme sind durch eine feste Wand getrennt, die zuerst isoliert ist und dann wärmeleitend wird. Die beiden Systeme werden also in thermischen Kontakt gebracht und tauschen Wärme aus.

\begin{figure}[htbp]
    \centering
    \tfigDoppelsystemUSfesteWaermeleitendeWand
    \caption{Abgeschlossenes System aus zwei Untersystemen, die durch eine feste und materieundurchlässige Wand getrennt sind. Die Wand ist zunächst isolierend und wird dann wärmeleitend.}
    \label{fig:DoppelsystemUSfesteWaermeleitendeWand}
\end{figure}

Wir würden dabei erwarten, dass sich die Temperaturen angleichen, $T^{(1)}=T^{(2)}$.

Nach den Postulaten muss für ein abgeschlossenes System $U^{(1)}+U^{(2)} = \mathrm{const}$ bzw. $\diff U^{(1)} = -\diff U^{(2)}$ sein. Das Postulat \ref{post:entropie_maximierung} besagt jetzt, dass sich $U^{(1)}$ und $U^{(2)}$ so einstellen, dass $S$ ein Maximum annimmt, $\diff S=0$. Damit auch das Postulat \ref{post:eigenschaften_entropie} erfüllt ist, muss gelten, dass
\begin{align*}
    S=S^{(1)}\left(U^{(1)},V^{(1)},\sm^{(1)}_k\right) + S^{(2)}\left(U^{(2)},V^{(2)},\sm^{(2)}_k\right).
\end{align*}
Es ist also
\begin{align*}
    \diff S = \frac{\partial S^{(1)}}{\partial U^{(1)}}\diff U^{(1)}+ \frac{\partial S^{(2)}}{\partial U^{(2)}}\diff U^{(2)} = \frac{1}{T^{(1)}}\diff U^{(1)} + \frac{1}{T^{(2)}}\diff U^{(2)},
\end{align*}
da die Volumina und Stoffmengen konstant sind und wegen $\diff U^{(1)} = -\diff U^{(2)}$ ist
\begin{align*}
    \diff S = \left(\frac{1}{T^{(1)}}-\frac{1}{T^{(2)}}\right)\diff U^{(1)} \overset{!}{=} 0.
\end{align*}
Im thermischen Gleichgewicht gilt folglich
\begin{align}
    \label{eq:thermisches_gg_temperatur}
    T^{(1)} = T^{(2)},
\end{align}
wie erwartet. Aus dem Postulat \ref{post:eigenschaften_entropie} folgt auch, dass die Temperatur positiv ist, denn $S$ soll eine monoton ansteigende Funktion von $U$ sein, sodass $\partial S/\partial U > 0$.

Diese Definition der Temperatur als Inverse der Ableitung der Entropie nach der inneren Energie mag zwar ein wenig abstrakt erscheinen, doch gibt es auch andere gleichbedeutende Definitionen der Temperatur, die aber nicht weniger abstrakt sind\footnote{Ein anderer Ansatz wäre, als nullten Hauptsatz die Transitivität der Temperatur zu postulieren \cite{lit:nolting1},
    \begin{align*}
        T^{(1)} = T^{(2)} \quad \text{und}\quad T^{(1)} = T^{(3)} \implication T^{(2)} = T^{(3)}.
    \end{align*}
    Eine weitere Formulierung, in der $1/T$ als integrierender Faktor festgelegt wird (sodass $\diff S=\udiff Q/T$), wurde von Kelvin und Caradathory vorgeschlagen.
    Beide Varianten sind in dem hier gewählten Zugang bereits in den Postulaten \ref{post:entropie_maximierung} und \ref{post:eigenschaften_entropie} enthalten.
}.

\begin{formal}
    Es existiert also eine absolute Temperaturskala. Eine solche ist die Kelvin-Skala, die so definiert ist, dass der Tripelpunkt, also die Koexistenz von Eis, flüssigem Wasser und Wasserdampf, bei \SI{271,16}{\kelvin} liegt.
\end{formal}

Wir haben gesehen, dass die Entropie ein Maximum annimmt. Daraus kann man schließen, dass die zweite Ableitung der Entropie dort kleiner als $0$ ist.

\paragraph*{Wärmefluss}

Wir wissen intuitiv, dass die Wärme von Bereichen hoher Temperatur zu Bereichen niedrigerer Temperatur fließt. Startet man bei einem Anfangszustand mit $T^{(2)}> T^{(1)}$ und hebt dann die Zwangsbedingung auf, kommt es zu einem (quasistatischen) Wärmefluss.
Wegen des Postulats \ref{post:entropie_maximierung} ist
\begin{align*}
    \Delta S= \left(\frac{1}{T^{(1)}}-\frac{1}{T^{(2)}}\right)\Delta U^{(1)} > 0.
\end{align*}
Da aber
\begin{align*}
    T^{(2)}> T^{(1)} \equivalence \frac{1}{T^{(1)}}-\frac{1}{T^{(2)}} < 0
\end{align*}
ist, muss $\Delta U^{(1)} <0$ sein.
\begin{formal}
    Der Wärmefluss findet erwartungsgemäß vom System höherer zum System tieferer Temperatur statt, bis sich beide Temperaturen angeglichen haben.
\end{formal}
Dann ist das Maximum der Entropie erreicht (siehe \Abbref{fig:FunktionEntropieMaximum}).

\begin{figure}[htbp]
    \centering
    \tfigFunktionEntropieMaximum
    \caption{Entropie über innere Energie: Beim thermodynamischen Gleichgewicht nimmt die Entropie ihr Maximum an. Die Änderung $\Delta U^{(1)}$ ist negativ für $\Delta S>0$. }
    \label{fig:FunktionEntropieMaximum}
\end{figure}




\section{Mechanisches Gleichgewicht}


\begin{figure}[htbp]
    \centering
    \tfigDoppelsystemUVNbeweglicheWaermeleitendeWand
    \caption{Abgeschlossenes System aus zwei Untersystemen, die durch eine materieundurchlässige Wand getrennt sind. Die Wand ist zunächst fest und isolierend und wird dann beweglich und wärmeleitend.}
    \label{fig:DoppelsystemUVNbeweglicheWaermeleitendeWand}
\end{figure}


Als Nächstes soll ein insgesamt abgeschlossenes System aus zwei Untersystemen behandelt werden, bei dem ein materieundurchlässiger Kolben zuerst fest und isolierend, dann aber beweglich und wärmeleitend ist (siehe \Abbref{fig:DoppelsystemUVNbeweglicheWaermeleitendeWand}). Da es sich um ein abgeschlossenes System konstanten Gesamtvolumens handelt, ist
\begin{align*}
    U^{(1)} + U^{(2)} & = \mathrm{const}  \\
    V^{(1)} + V^{(2)} & = \mathrm{const}.
\end{align*}
Nach den Postulaten \ref{post:entropie_maximierung} und \ref{post:eigenschaften_entropie} ist
\begin{align*}
    \diff S & = \frac{\partial S^{(1)}}{\partial U^{(1)}}\diff U^{(1)} + \frac{\partial S^{(1)}}{\partial V^{(1)}}\diff V^{(1)}+\frac{\partial S^{(2)}}{\partial U^{(2)}}\diff U^{(2)} + \frac{\partial S^{(2)}}{\partial V^{(2)}}\diff V^{(2)} \\
            & = \left(\frac{1}{T^{(1)}}-\frac{1}{T^{(2)}}\right)\diff U^{(1)} + \left(\frac{P^{(1)}}{T^{(1)}}-\frac{P^{(2)}}{T^{(2)}}\right) \diff V^{(1)} = 0.
\end{align*}
Im mechanischen Gleichgewicht gilt also
\begin{align*}
    \frac{1}{T^{(1)}} = \frac{1}{T^{(2)}} , \quad \frac{P^{(1)}}{T^{(1)}}=\frac{P^{(2)}}{T^{(2)}},
\end{align*}
bzw.
\begin{align*}
    T^{(1)} = T^{(2)}, \quad P^{(1)} = P^{(2)}.
\end{align*}
Die hier diskutierten Gleichgewichtsbedingungen mögen trivial erscheinen, doch geht es hier vorrangig um das Testen des Formalismus und dann die anschließende Anwendung auf komplexere Systeme.


\section{Gleichgewicht bei Materiefluss\label{sec:gleichgewicht_bei_materiefluss}}

\paragraph*{Chemisches Potential}

Analog zu den vorigen Beispielen wird ein abgeschlossenes System mit zwei Untersystemen betrachtet (siehe \Abbref{fig:DoppelsystemUVNbeweglicheIsolierendeWand}). Diesmal ist die Wand zwar fest, aber wärmeleitend und durchlässig für eine Molekülsorte.

\begin{figure}[htbp]
    \centering
    \tfigDoppelsystemUVNbeweglicheIsolierendeWand
    \caption{Abgeschlossenes System aus zwei Untersystemen, die durch eine feste Wand getrennt sind. Die Wand ist zunächst isolierend und undurchlässig und wird dann wärmeleitend und durchlässig für die Molekülsorte 1.}
    \label{fig:DoppelsystemUVNbeweglicheIsolierendeWand}
\end{figure}

Es gilt
\begin{align*}
    U^{(1)} + U^{(2)}         & = \mathrm{const} \\
    \sm_1^{(1)} + \sm_1^{(2)} & = \mathrm{const}
\end{align*}
und damit
\begin{align*}
    \diff S & = \frac{1}{T^{(1)}}\diff U^{(1)} - \frac{\mu_1^{(1)}}{T^{(1)}}\diff \sm^{(1)}+\frac{1}{T^{(2)}}\diff U^{(2)} - \frac{\mu_1^{(2)}}{T^{(2)}}\diff \sm^{(2)}   \\
            & = \left(\frac{1}{T^{(1)}}-\frac{1}{T^{(2)}}\right)\diff U^{(1)} + \left(\frac{\mu_1^{(1)}}{T^{(1)}}-\frac{\mu_1^{(2)}}{T^{(2)}}\right) \diff \sm^{(1)} = 0.
\end{align*}
Im Gleichgewicht gleichen sich also neben den Temperaturen die chemischen Potential durch Teilchenaustausch an,
\begin{align*}
    T^{(1)} = T^{(2)}, \quad \mu_1^{(1)} = \mu_1^{(2)}.
\end{align*}
Es findet jedoch kein Teilchenfluss statt, wenn bereits $\mu_1^{(1)} = \mu_1^{(2)}$.

\paragraph*{Materiefluss}

Beginnt man bei einem Anfangszustand mit $\mu_1^{(1)} > \mu_1^{(2)}$ und $T^{(1)} = T^{(2)}$ und hebt dann die Zwangsbedingung auf (Wand wird materiedurchlässig), so kommt es zum quasistatischen Materiefluss,
\begin{align*}
    \Delta S = \frac{\mu_1^{(2)} - \mu_1^{(1)}}{T} \Delta \sm_1^{(1)}.
\end{align*}
Da nach dem Postulat \ref{post:entropie_maximierung} die Änderung der Entropie nur positiv sein kann und $(\mu_1^{(2)} - \mu_1^{(1)})/T$ nach unserer Festlegung negativ ist, so ist auch $\Delta \sm_1^{(1)}<0$.

\begin{formal}
    Ein Materiefluss findet von Gebieten hohen zu Gebieten tiefen chemischen Potentials statt, bis $\mu_1^{(1)} = \mu_1^{(2)}$.
\end{formal}

Das chemische Potential $\mu$ ist zentral bei Phasenumwandlungen und chemischen Reaktionen (siehe später) und spielt damit eine führende Rolle in der theoretischen Chemie.



\section{Folgerungen aus der Homogenität der Fundamentalbeziehung}

Allein aus der Forderung, dass die Fundamentalbeziehung homogen ist, lassen sich einige formale Schlüsse folgern, die in diesem Kapitel erläutert werden sollen. Die erste Schlussfolgerung ist die Euler-Gleichung.

\paragraph*{Die Euler-Gleichung}

Die innere Energie $U$ ist eine extensive Größe und damit eine homogene Funktion ersten Grades (Größen wie Volumen und Stoffmengen werden verallgemeinert als $X_k$ geschrieben, um eine kompaktere Notation zu ermöglichen),
\begin{align*}
    U(\lambda S,\lambda X_1,\ldots,\lambda X_t) = \lambda U(S,X_1,\ldots,X_t).
\end{align*}
Ableiten nach $\lambda$ liefert
\begin{align*}
    \frac{\partial U}{\partial\lambda} & = \frac{\partial U}{\partial\lambda S}(\lambda S,\lambda X_1,\ldots,\lambda X_t) \frac{\partial\lambda S}{\partial\lambda} + \frac{\partial U}{\partial\lambda X_1}(\lambda S,\lambda X_1,\ldots,\lambda X_t) \frac{\partial\lambda X_1}{\partial\lambda} + \ldots \\
                                       & = T(\lambda S,\lambda X_1,\ldots,\lambda X_t)S + P_1 (\lambda S,\lambda X_1,\ldots,\lambda X_t) X_1 + \ldots                                                                                                                                                       \\
                                       & = \lambda TS + \lambda P_1X_1 + \ldots
\end{align*}
Lässt man nun $\lambda$ gegen 1 gehen, so erhält man die Euler-Gleichung in Energiedarstellung:
\begin{align}
    \label{eq:euler_gleichung_energiedarstellung}
    \boxed{U = TS + \sum_{j=1}^t P_j X_j.}
\end{align}
Analog lässt sich die Entropiedarstellung der Euler-Gleichung herleiten:
\begin{align}
    \label{eq:euler_gleichung_entropiedarstellung}
    \boxed{S = \sum_{j=0}^t F_j X_j.}
\end{align}
Für ein einfaches System nimmt sie zum Beispiel die Form (Energiedarstellung)
\begin{align*}
    U=TS-PV + \mu_1 \sm_1+\ldots + \mu_r \sm_r
\end{align*}
bzw. (Entropiedarstellung)
\begin{align*}
    S=\frac{1}{T}U + \frac{P}{T}V - \sum_{k=1}^r \frac{\mu_k}{T}\sm_k
\end{align*}
an.


\paragraph*{Gibbs-Duhem-Beziehung}

Bis jetzt haben wir globale Betrachtungen gemacht, die hauptsächlich extensive Variablen behandeln. Nun soll eine differentielle Behandlung folgen, die auch die intensiven Größen berücksichtigt.

Aus der energetischen Fundamentalbeziehung erhält man durch Differenzieren nach den Parametern $(t+1)$ Gleichungen in $(t+1)$ Variablen.
\begin{align*}
    \frac{\partial U}{\partial S} = T = T(S,X_1,\ldots ,X_t), \quad
    \frac{\partial U}{\partial X_k} =P_k = P_k(S,X_1,\ldots ,X_t)
\end{align*}
Die durch diesen Prozess gewonnenen Größen $T,P_1,\ldots,P_t$ sind intensiv, also homogen vom Grad $0$, z.B.
\begin{align*}
    T(S,X_1,\ldots ,X_t)=T(\lambda S,\lambda X_1,\ldots ,\lambda X_t).
\end{align*}
Für $\lambda=1/X_t$ ist dann
\begin{align*}
    T = T\left(\frac{S}{X_t},\frac{X_1}{X_t},\ldots ,1\right), \quad P_k=P_k\left(\frac{S}{X_t},\frac{X_1}{X_t},\ldots ,1\right).
\end{align*}
Dieses Gleichungssystem aus $(t+1)$ Gleichungen enthält jetzt nur noch $t$ Variablen, sodass eine Zustandsgleichung bei der Beziehung zwischen den insgesamt $(t+1)$ intensiven Variablen redundant ist.

Betrachte als Beispiel ein Einkomponentensystem mit $X_t=\sm$, molarer Entropie $s=S/\sm$ und molarem Volumen $v=V/\sm$. Temperatur, Druck und chemisches Potential hängen jeweils nur von der molaren Entropie und dem molaren Volumen ab,
\begin{align*}
    T=T(s,v), \quad P=P(s,v), \quad \mu=\mu(s,v).
\end{align*}
Man kann jetzt zwei dieser Zusammenhänge invertieren (z.B. $s=s(T,P)$ und $v=v(T,P)$) und in den dritten einsetzen ($\mu=\mu(T,P)$). Diese letzte Gleichung ist damit redundant für die Beschreibung des Systems.

Durch Differenzieren der Euler-Gleichung erhalten wir
\begin{align}
    U                    & =TS+ \sum_j P_jX_j                                                 \nonumber \\
    \label{eq:differential_innere_energie1}
    \implication \diff U & = T\diff S + S\diff T + \sum_j P_j\diff X_j + \sum_j X_j\diff P_j.
\end{align}
Andererseits kennen wir aus der Energiedarstellung bereits das Differential der inneren Energie (siehe Gleichung \eqref{eq:energetische_fundamentalgleichung}) als
\begin{align}
    \label{eq:differential_innere_energie2}
    \diff U = T\diff S + \sum_j P_j \diff X_j.
\end{align}
Durch Vergleich von \eqref{eq:differential_innere_energie1} und \eqref{eq:differential_innere_energie2} folgt eine differentielle Beziehung zwischen den intensiven Variablen,
\begin{align}
    \label{eq:gibbs_duhem}
    \boxed{
        S\diff T + \sum X_j \diff P_j = 0,
    }
\end{align}
was als Gibbs-Duhem-Beziehung bekannt ist.

Im Einkomponentensystem heißt das, dass
\begin{align*}
    S\diff T - V\diff P + \sm \diff \mu = 0.
\end{align*}
Teilen durch $\sm$ liefert
\begin{align*}
    \diff \mu = -s \diff T + v\diff P
\end{align*}
und Integrieren führt auf das chemische Potential $\mu=\mu(T,P)$.

Wir haben also gesehen, dass stets eine intensive Variable von den anderen abhängt. Das gibt Anlass zu der nachfolgenden Definition:
\begin{formal}
    Die Zahl der unabhängigen intensiven Variablen wird als Zahl der thermodynamischen Freiheitsgrade definiert.
\end{formal}

Die Gibbs-Duhem-Beziehung kann auch in der Entropiedarstellung formuliert werden,
\begin{align*}
    \boxed{\sum_{j=0}^t X_j \diff F_j = 0}
\end{align*}
und für ein einfaches System ist dann
\begin{align*}
    U\diff\left(\frac{1}{T}\right) + V\diff\left(\frac{P}{T}\right)-\sum_{k=1}^r \sm_k\diff\left(\frac{\mu_k}{T}\right)=0.
\end{align*}

% !TeX root = Theo_IV.tex
\chapter{Antwortkoeffizienten und Beispielsysteme}\label{sec:Antwortkoeffizienten und Beispielsysteme}
Im Folgenden wollen wir uns mit der Anwendung des bislang hergeleiteten Formalismus der Wärmelehre beschäftigen. Dazu erarbeiten wir uns die Herleitungen verschiedener Antwortkoeffizienten wie der spezifischen Wärme, beleuchten die sogenannten Maxwellbeziehungen und beschreiben das einkomponentige, ideale Gas.
\section{Antwortkoeffizienten: spezifische Wärme und andere Ableitungen}
Die zugrunde liegende Frage, welche die Beschäftigung mit sogenannten \emph{Antwortkoeffizienten} motiviert, ist die Frage nach der Reaktivität eines Systems auf äußere Einflüsse. Dem entstammt auch die Benennung dieser \textendash{} im Folgenden genauer erörterten \textendash{} Materialkonstanten als Antwortkoeffizienten.
Ein prominentes Beispiel einer solchen Materialkonstante ist die spezifische Wärme, die wir uns ebenfalls im Laufe dieses Unterkapitels näher anschauen wollen.


Ausgangspunkt unserer Betrachtungen bilden die Temperatur $T$ und der Druck $P$ in Energiedarstellung (also als Funktionen von Entropie $S$ und Volumen $V$), wobei wir annehmen, dass wir mit konstanten Stoffmengen (Molzahlen) arbeiten.  Die möglichen Wege im Zustandsraum können wir mit den genannten Größen wie folgt charakterisieren:


\paragraph*{Isochore}
Der Begriff \emph{Isochore} bezeichnet Zustandsänderungen bei gleichbleibenden Volumina. Damit folgt:
\begin{align*}
    V=V_0=\mathrm{const}.
\end{align*}
Die Größen $T$ und $P$ sind damit nur noch Funktionen von $S$ und so ist $T=T(P)_{V_0}$ bzw. $P=P(T)_{V_0}$ (dazu wird die Entropie $S$ durch Umkehren einer der Zustandsfunktionen $T$ und $P$ und Einsetzen in die andere eliminiert).


\paragraph*{Isentrope}
Analog gehen wir bei der Betrachtung der \emph{Isentrope} vor, welche Zustandsänderungen bezeichnen, bei denen die Entropie gleich bleibt:
\begin{align*}
    S=S_0=\mathrm{const}.%\quad\rightarrow\quad T=T(S_0,V) \quad\text{und}\quad P=(S_0,V) \quad\rightarrow\quad T=T(P)|_{S_0}.
\end{align*}
Wir motivieren diese Betrachtung durch den Verweis darauf, dass sie bei der Beschreibung der Schallausbreitung essentiell ist. Dies ist dadurch bedingt, dass die dort ablaufenden Kompressionsprozesse derartig schnell verlaufen, dass keine Wärmezufuhr oder -abfuhr stattfinden kann und die Prozesse desweiteren isentropisch sind. Unter den \emph{adiabatischen} Zustandsänderungen versteht man Prozesse, welche ohne Wärmeaustausch mit der Umgebung ablaufen, $\udiff Q=0$. Isentrope Prozesse sind adiabatisch, der Umkehrschluss gilt jedoch nicht. 


Wir wollen uns auch weiterhin auf die Zustandsvariablen Temperatur und Druck konzentrieren, wobei wir im Allgemeinen kontrollierbare (bzw. im Experiment kontrollierte) Zustandsvariablen wie diese beiden nun auch als \emph{Koordinaten} bezeichnen werden. Es sei darauf verwiesen, dass die Entropie $S$ keine solche direkt kontrollierbare Größe ist.

Durch Invertierung der energetischen Zustandsfunktionen erhalten wir die neuen Zustandsfunktionen
\begin{align*}
    \boxed{S=S(T,P) \quad\mathrm{und}\quad V=V(T,P)}\:.
\end{align*}
Erneut wollen wir mögliche Wege im Zustandsraum bei gleichbleibenden Parametern betrachten.


\paragraph*{Isotherme}
Zum einen können wir die Temperatur festhalten \textendash{} wir sprechen dann von \emph{Isothermen}, für welche
\begin{align*}
    T=T_0=\mathrm{const}%\quad\rightarrow\quad S=S(T_0,P) \quad\text{und}\quad V=(T_0,P) \quad\rightarrow\quad S=S(V)|_{T_0}
\end{align*}
gilt.



\paragraph*{Isobare}
Zum anderen können wir den Druck konstant halten \textendash{} hier sprechen wir von \emph{Isobaren}, für welche wiederum
\begin{align*}
    P=P_0=\mathrm{const}%\quad\rightarrow\quad S=S(T,P_0) \quad\text{und}\quad V=(T,P_0) \quad\rightarrow\quad S=S(V)|_{P_0}
\end{align*}
folgt.
Wir fragen uns nun, wie Änderungen der Größen $T$ und $P$ auf das Volumen $V$ wirken. Dazu schreiben wir
\begin{align*}
    \diff V (T,P)= \left( \frac{\partial V}{\partial T}\right)_P \diff T+\left( \frac{\partial V}{\partial P}\right)_T \diff P
\end{align*}
und erhalten die \emph{relative Volumenänderung}
\begin{align*}
    \frac{\diff V}{V}=\frac{1}{V}\left( \frac{\partial V}{\partial T}\right)_P\diff T+\frac{1}{V}\left( \frac{\partial V}{\partial P}\right)_T\diff P \equiv \alpha(T,P)\diff T-\kappa_T(T,P)\diff P.
\end{align*}
Zum einen haben wir den neuen Koeffizienten
\begin{align*}
    \boxed{\alpha=\frac{1}{V}\left( \frac{\partial V}{\partial T}\right)_P}\:,
\end{align*}
den \emph{thermischen Ausdehnungskoeffizienten} identifiziert, welcher das Maß der Systemausdehnung mit der Temperatur bei konstantem Druck beschreibt. Zum anderen identifizieren wir die \emph{isotherme Kompressibilität}
\begin{align*}
    \boxed{\kappa_T=-\frac{1}{V}\left( \frac{\partial V}{\partial P}\right)_T}\:,
\end{align*}
welche die Reaktion des Systems auf entsprechende Drücke und bei konstanter Entropie\footnote{Hier betrachten wir die Entropie und nicht die Temperatur, da sich letztere bei den schnell ablaufenden Prozessen unvermeidlich verändert. Nichtsdestotrotz bleibt ein Wärmeaustausch aus, die Entropie ist also konstant.} charakterisiert.


Nun wollen wir die Materialkonstanten herleiten, welche bei Prozessen mit Entropieänderung relevant sind.
Dazu betrachten wir das entropische Differential
\begin{align*}
    \diff S (T,P)= \left( \frac{\partial S}{\partial T}\right)_P \diff T+\left( \frac{\partial S}{\partial P}\right)_T \diff P
\end{align*}
und erhalten darüber den \emph{quasistatischen Wärmefluss pro Mol}
\begin{align*}
    \frac{T}{\sm}\diff S=\frac{\udiff Q}{\sm}=\frac{T}{\sm}\left( \frac{\partial S}{\partial T}\right)_P \diff T+\frac{T}{\sm}\left( \frac{\partial S}{\partial P}\right)_T \diff P\equiv c_p(T,P)\diff T+\ldots %.
\end{align*}
Wieder identifizieren wir einen neuen Koeffizienten (der nicht ausgeschriebene zweite Term führt hier auf keinen besonderen Koeffizienten): Die \emph{molare spezifische Wärme} bei konstantem Druck
\begin{align*}
    \boxed{c_p=\frac{T}{\sm} \left( \frac{\partial S}{\partial T} \right)_P= \frac{1}{\sm} \left( \frac{\udiff Q}{\diff T} \right)_P }\:.
\end{align*}
Die Größe $\udiff Q$ ist natürlich eigentlich ein unvollständiges Differential, sodass der letzte Term nur beschränkt gültig ist. 
Die insgesamt zugeführte Wärme lässt sich darüber nun leicht durch eine Aufintegration der Form
\begin{align*}
    Q(T)=\int_{T_0}^{T} \udiff Q = \int_{T_0}^{T}\Tilde{T} \diff S = \int_{T_0}^{T}\sm c_p\diff \Tilde{T}
\end{align*}
bestimmen.

Es existiert eine weitere Herleitung der spezifischen Wärme, welche kurz skizziert werden soll. Ausgangspunkt bildet das uns inzwischen gut bekannte, unvollständige Differential
\begin{align*}
    \udiff Q=\diff U +P\diff V= \diff(U+PV)|_{P=\mathrm{const}}.
\end{align*}
Den letzten Ausdruck erhalten wir unter der Voraussetzung, dass der Druck $P$ konstant ist. Schreiben wir die innere Energie als Funktion der altbekannten extensiven Größen, so wird ersichtlich, dass die innere Energie hier letztlich von den intensiven Größen $T$ und $P$ abhängt:
\begin{align*}
    U=U(S(T,P),V(T,P))=U(T,P).
\end{align*}
Es sei darauf hingewiesen, dass es sich bei letzterer Formulierung nicht um eine energetische Fundamentalbeziehung handelt, denn die Abhängigkeiten von $S$ und $V$ fehlen.


Wir führen nun das neue \emph{thermodynamische Potential} der \emph{Enthalpie}
\begin{align*}
    \boxed{H=U+PV=H(T,P)}
\end{align*}
ein \textendash{} wir werden zu einem späteren Zeitpunkt genauer darauf zurückkommen \textendash{} und schreiben die spezifische Wärme mittels der neuen Größe als
\begin{align*}
    c_p=\frac{1}{\sm}\left( \frac{\partial H}{\partial T}\right)_P.
\end{align*}
Eine kleine Randnotiz für den Leser: Die Inhalte, die wir hier erarbeiten, sind in der Regel mathematisch einfach. Entscheidend sind die Überlegungen zur Festlegung und Betrachtung konstanter Größen und der daraus folgenden Zusammenhänge.


Wir ändern nun die zu betrachtenden Koordinaten und konzentrieren uns auf Systeme mit kontrollierbarer Temperatur $T$ und kontrollierbarem Volumen $V$.
Analog zu vorigem Vorgehen schreiben wir die Zustandsfunktionen
\begin{align*}
    S=S(T,V)\quad\mathrm{und}\quad P=P(T,V)
\end{align*}
in Abhängigkeit der gewählten Koordinaten und schreiben den sich daraus ergebenden Ausdruck für den molaren quasistatischen Wärmefluss
\begin{align*}
    \frac{T}{\sm}\diff S= \frac{\udiff Q}{\sm}=\frac{T}{\sm}\left(\frac{\partial S}{\partial T}\right)_V \diff T +\frac{T}{\sm}\left(\frac{\partial S}{\partial V}\right)_T \diff V\equiv c_v(T,V)\diff T+\ldots
\end{align*}
Erneut erhalten wir damit einen Ausdruck für die molare spezifische Wärme, diesmal bei konstantem Volumen:
\begin{align*}
    \boxed{c_v=\frac{T}{\sm}\left(\frac{\partial S}{\partial T}\right)_V=\frac{1}{\sm}\left(\frac{\udiff Q}{\diff T}\right)_V}\:,
\end{align*}
obwohl $\udiff Q$ eigentlich wieder ein unvollständiges Differential ist. 
Dank der Konstanz des Volumens können wir die Relation $\udiff Q=\diff U$ nutzen und für die spezifische Wärme alternativ
\begin{align*}
    \boxed{c_v=\frac{1}{\sm}\left(\frac{\partial U}{\partial T}\right)_V}
\end{align*}
schreiben.

Wir wollen zum Abschluss dieses Unterkapitels die beiden definierten Größen $c_p$ und $c_v$ für die spezifische Wärme noch in Relation setzen. Dazu machen wir folgende Überlegung: Betrachten wir eine Erwärmung bei konstantem Druck $P$ mit $c_p$,
so wird zusätzlich zur Temperaturerhöhung auch mechanische Arbeit zur Ausdehnung des Volumens verrichtet. Mit unseren Definitionen der spezifischen Wärme folgt daraus \textbf{$c_p >c_v$}.


Wir greifen nun auf die bereits eingeführte Integrabilitätsbedingung zurück und erörtern am Beispiel der inneren Energie die Relationen zwischen den zweiten Ableitungen, welche wir auch als \emph{Maxwellbeziehungen} bezeichnen.
Fangen wir wieder beim bekannten, vollständigen Energiedifferential
\begin{align*}
    \diff U = T\diff S-P\diff V+\mu \diff \sm 
\end{align*}
an. Mithilfe der Integrabilitätsbedingung
\begin{align*}
    \frac{\partial^2U}{\partial V \partial S}=\frac{\partial^2U}{\partial S \partial V}
\end{align*}
gelangen wir wegen $T=\partial U/\partial S$ und $P=-\partial U/\partial V$ zu folgenden Maxwell-Relationen:
\begin{align*}
    \boxed{\left(\frac{\partial T}{\partial V}\right)_{S,\sm }=-\left(\frac{\partial P}{\partial S}\right)_{V,\sm }=-T\left(\frac{\partial P}{\udiff Q}\right)_{V,\sm }}\;
\end{align*}
(natürlich ist $\udiff Q$ erneut ein unvollständiges Differential). 
Dabei charakterisiert der erste Term die Temperaturänderung bei isentropischer Volumenänderung. Der letzte Term definiert die Druckänderung bei isochorem Wärmezufluss. An dieser Stelle wollen wir bereits als Ausblick erwähnen, dass alle bildbaren zweiten Ableitungen der Größen eines betrachteten Systems mittels eines Minimalsatzes dreier Antwortkoeffizienten dargestellt werden können.

Für einfache Einkomponentensysteme gilt folgende Relation für die bisher eingeführten Antwortkoeffizienten (ohne Beweis):
\begin{align*}
    \boxed{c_p=c_v+\frac{TV\alpha^2}{\sm \kappa_T}}\:.
\end{align*}
Besonders nützlich  ist, dass die Koeffizienten experimentell sehr gut zugänglich sind. Die genaue Herleitung der Relation soll später erfolgen.

\section{Beispiel: Das einkomponentige ideale Gas}
Wir wollen das bisher Erarbeitete nun im Rahmen des Modells der \emph{idealen Gase} anwenden. Ideal heißt in diesem Kontext, dass wir die Annahme machen, Moleküle verhalten sich wie Punktteilchen, ohne \textendash{} mit Ausnahme von Stößen \textendash{} miteinander zu wechselwirken. Die Realisierung eines solchen idealen Gases erfolgt durch ausreichende Verdünnung eines tatsächlichen Gases.
Welche Zustandsgleichungen gelten nun für dieses ideale Gas?
Die prominenteste Gleichung dürfte die \emph{ideale Gasgleichung} sein:
\begin{align}
    \label{eq:ideale gasgleichung}
    \boxed{PV=\sm RT}\:.
\end{align}
Dabei bezeichnet $R$ die \emph{universelle Gaskonstante} und ist das Produkt der \emph{Avogadrokonstante} (welche die Anzahl der Teilchen in einem Mol angibt) und der \emph{Boltzmann-Konstante} (deren Bedeutung wir später noch erörtern werden):
\begin{align*}
    R=\avogadro k_\mathrm{B}=\qty{ 8,314 462 618 153 24}{\joule\per\mole\per\kelvin}
\end{align*}
mit $\avogadro=\qty{6,02214076e23}{\per\mole}$ und $k_\mathrm{B}=\qty{1,38064852e23}{\joule\per\kelvin}$. 

Im Allgemeinen gilt unter konstantgehaltener Temperatur das \emph{Boyle-Mariottsche Gesetz},
\begin{align*}
    PV=\mathrm{const}. 
\end{align*}
und unter konstantgehaltenem Druck das \emph{Gay-Lussacsche Gesetz},
\begin{align*}
    \frac{V}{T}=\mathrm{const},
\end{align*}
welche beide bekannte Ergebnisse experimenteller Untersuchungen sind.
Ferner gibt es die \emph{kalorische Zustandsgleichung}:
\begin{align}
    \label{eq:kalorische Zustandsgleichung}
    \boxed{U=\frac{f}{2}\sm RT}\:.
\end{align}
$f$ bezeichnet dabei die Zahl der Freiheitsgrade. Die innere Energie ist also für einfache\footnote{Die Kennzeichnung als \anf{einfaches} Gas wird im Laufe der nächsten Kapitel näher erläutert. Wir wollen bereits vorgreifen und darauf verweisen, dass damit Zustände mit höheren Quantenzahlen beschrieben werden, welche sich entsprechend \anf{klassisch} verhalten.} ideale Gase proportional zur Temperatur des Systems.
Die Gleichung ist ein direktes Ergebnis der statistischen Mechanik und ferner des dort genutzten \emph{Gleichverteilungssatzes}.
\begin{formal}
    Der \formalemph{Gleichverteilungssatz} der statistischen Mechanik besagt, dass ein System pro Freiheitsgrad $f$ eine molare innere Energie $u=RT/2$ annimmt.
\end{formal}
Wir wollen dies am Beispiel eines Atoms illustrieren, welches drei Freiheitsgrade für die Translation im dreidimensionalen Raum besitzt. Für diatomare Moleküle kommen noch weitere Freiheitsgrade hinzu: Zwei für Rotationen entlang der Achsen senkrecht zur Molekülachse und zwei für Schwingungen, bzw. für die gespeicherte potentielle Energie und die vorliegende kinetische Energie des Moleküls.
Die das Molekül beschreibende Hamiltonfunktion sieht wie folgt aus:
\begin{align*}
    H=\frac{p_x^2+p_y^2+p_z^2}{2M}+\frac{p_{\phi}^2+p_{\psi}^2}{2\Theta}+\frac{\mu^2}{2m}+\frac{1}{2}m\omega^2d^2.
\end{align*}
Der erste Summand beschreibt die Translation, der zweite die Rotation und die letzten beiden die Schwingung.
Der Freiheitsgrad $f$ entspricht hier auch der Zahl der quadratischen Einzelterme der Hamiltonfunktion.

An dieser Stelle machen wir einen kleinen Exkurs in die Quantenmechanik:
Nicht alle Freiheitsgrade sind zu einem beliebigen Zeitpunkt in einem System angeregt. Begründen können wir dies mithilfe der Quantelung der Energieeigenwerte in der Quantenmechanik.
Für Oszillatoren nehmen die Energieeigenwerte beispielsweise die Werte $E=(n+1/2)\hbar\omega$ mit $n=0,1,2,\ldots$ an, für Rotatoren $E=J(J+1)\hbar^2/(2\Theta)$ mit $J=0,1,2,\ldots$.
Bedingt durch diese Quantelungen ist das klassische, kontinuierliche Verhalten von physikalischen Systemen erst bei höheren Temperaturen beobachtbar, nämlich dann, wenn hohe Quantenzahlen vorliegen. Für Oszillatoren und Rotatoren liegen die Größenordnungen bei $k_\mathrm{B}T\gg\hbar\omega $, respektive $k_\mathrm{B}T\gg\hbar^2/\Theta$.
Das dadurch eintretende \anf{Auftauen} der Freiheitsgrade kann ab entsprechenden Schwellenwerten festgestellt werden. Für unsere zwei Beispiele liegen die Auftaubereiche für Wasserstoff bei über \qty{1000}{\kelvin} für Oszillationen und bei \qty{200}{\kelvin} für Rotationen.
Anfangs dominieren also translatorische Prozesse, während mit Zunahme der Temperatur langsam auch der Rotationsbereich und noch später der Schwingungsbereich \anf{auftaut} (siehe auch \Abbref{fig:ThawingOfDegreesOfFreedom}).
Diese Freiheitsgradverteilungen in Abhängigkeit der Temperatur werden wir später mittels der Beschreibung des wärmeabhängigen Verlaufs der spezifischen Wärme illustrieren.

Wir wollen zunächst das Gelernte nutzen, um sowohl das chemische Potential, als auch die Entropie aufzuschreiben. Aus den zwei eingeführten Zustandsgleichungen \eqref{eq:ideale gasgleichung} und \eqref{eq:kalorische Zustandsgleichung} folgt
\begin{align}
    \label{eq:entropischeZustandsgleichungenIdealeGase}
    \boxed{\frac{P}{T}=\frac{R}{v} \quad\text{und}\quad \frac{1}{T}=\frac{f}{2}\frac{R}{u}}
\end{align}
in der Entropiedarstellung. 
Das chemische Potential bestimmen wir wie folgt über die \emph{Gibbs-Duhem-Gleichung} \eqref{eq:gibbs duhem einfaches system} für ein einfaches System
\begin{align*}
    \diff \left(\frac{\mu}{T}\right) & =u\diff\left(\frac{1}{T}\right)+v\diff \left(\frac{P}{T}\right)                                    \\
                                     & =-\frac{f}{2}\frac{R}{u}\diff u-\frac{R}{v}\diff v                                                 \\
                                     & =\frac{\partial \frac{\mu}{T}}{\partial u}\diff u+\frac{\partial \frac{\mu}{T}}{\partial v}\diff v
\end{align*}
un erhalten mit anschließender Integration
\begin{align*}
    \frac{\mu}{T}-\left(\frac{\mu}{T}\right)_0=-\frac{f}{2}R\ln\frac{u}{u_0}-R\ln\frac{v}{v_0}.
\end{align*}
Wir können dabei $\mu_0$ und $v_0$ als dem Referenzzustand zugehörige Größen auffassen.
Die Integrationskonstante $(\mu/T)_0$ bildet dabei das chemische Potential des Referenzzustandes.
Mithilfe der Eulergleichung lässt sich die Entropie $S$ nun als Funktion von innerer Energie $U$, Volumen $V$ und Stoffmenge $\sm$ bestimmen. Wir wollen jedoch einen alternativen Weg einschlagen und die molare Entropie mittels des Differentials
\begin{align*}
    \diff s & =\frac{1}{T}\diff u+\frac{P}{T}\diff v            \\
            & =\frac{f}{2}\frac{R}{u}\diff u+\frac{R}{v}\diff v
\end{align*}
bestimmen.
Dabei haben wir im letzten Schritt die kürzlich hergeleiteten Beziehungen \eqref{eq:entropischeZustandsgleichungenIdealeGase} genutzt.
Wir integrieren und erhalten analog zu voriger Vorgehensweise für die beiden nicht gemischten Terme
\begin{align}
    \label{eq:EntropieEinkomponentigesIdealesGas}
    \boxed{s=s_0+\frac{f}{2}R\ln\frac{u}{u_0}+R\ln\frac{v}{v_0}}\:.
\end{align}
Noch ist die Entropie jedoch nicht vollständig, da die unbestimmte Integrationskonstante $s_0$ vorliegt. Wir wissen jedoch, dass die Entropie \textendash{} bedingt durch den absoluten Temperaturnullpunkt \textendash{} ebenfalls einen absoluten Nullpunkt bei $T=0$ haben muss.
Die statistische Mechanik berücksichtigt diesen Zusammenhang und liefert eine Vervollständigung des Entropieausdrucks für monoatomare ideale Gase in Form der \emph{Sackur-Tetrode-Formel}, welche jedoch hier nicht weiter ausgeführt werden soll.

Wir wenden uns den Antwortkoeffizienten für das einkomponentige ideale Gas zu. Bereits kennengelernt haben wir die isotherme Kompressibilität
\begin{align*}
    \kappa_T=-\frac{1}{V}\left(\frac{\partial V}{\partial P}\right)_T=\frac{1}{P}
\end{align*}
und den thermischen Ausdehnungskoeffizienten
\begin{align*}
    \alpha=\frac{1}{V}\left(\frac{\partial V}{\partial T}\right)_P=\frac{1}{T},
\end{align*}
wobei wir uns die ideale Gasgleichung $PV=\sm RT$ zunutze gemacht haben, um die partiellen Ableitungen auszuschreiben.
Aus der Maxwell-Relation folgt für die spezifischen Wärmekoeffizienten
\begin{align*}
    c_p=c_v+\frac{TV\alpha^2}{\sm \kappa_T}.
\end{align*}
Setzen wir die eben formulierten Koeffizienten ein, erhalten wir für das ideale Gas die Relation
\begin{align*}
    \boxed{c_p=c_v+R}\:.
\end{align*}
Dabei bezeichnet der Koeffizient
\begin{align*}
    c_v=\frac{1}{\sm }\left(\frac{\partial U}{\partial T}\right)_V=\frac{f}{2}R
\end{align*}
den spezifischen Wärme-Antwortkoeffizienten des idealen Gases bei konstantem Volumen und
\begin{align*}
    c_p=\frac{f+2}{2}R
\end{align*}
den Antwortkoeffizienten bei konstantem Druck. Die Ausdrücke erhalten wir jeweils mithilfe der kalorischen Zustandsgleichung.
Das Verhältnis
\begin{align*}
    \frac{c_p}{c_v}=\frac{f+2}{f}=\gamma
\end{align*}
der beiden Koeffizienten bezeichnet man als \emph{Adiabatenexponent}.
Je größer $f$, die Zahl der Freiheitsgrade, desto weniger weicht der Exponent von eins ab.
Er spielt vor allem für die Beschreibung der Wege im Zustandsraum eine wichtige Rolle, welche wir nun näher beleuchten wollen.

Wir betrachten vorerst isentrope ($s=\mathrm{const}$) Prozesse:
Aus der Umstellung von \eqref{eq:EntropieEinkomponentigesIdealesGas} nach $e^{(s-s_0)/R}$ folgt für diese Prozesse die Konstanz des Terms $u^{(f/2)}v$.  Mit den Relationen
\begin{align*}
    u \propto T \quad\text{und}\quad \frac{f}{2}=\frac{1}{\gamma -1}\quad\rightarrow\quad TV^{\gamma-1}=\mathrm{const}
\end{align*}
und der Beziehung $T\propto PV$ der idealen Gasgleichung folgt ferner die Konstanz
\begin{align*}
    \boxed{PV^{\gamma}=\mathrm{const}}\:.
\end{align*}
Alternativ kann diese Gleichung mit $V\propto T/P$ auch als
\begin{align*}
    P^{1-\gamma}T^\gamma=\mathrm{const}
\end{align*}
geschrieben werden.
Die drei vorgestellten Ausdrücke charakterisieren alle eine äquivalente Quantifizierung isentroper Prozesse.
Für $\gamma = 1$, also für Systeme großer Freiheitsgrade bzw. Dimensionen, erhalten wir den Grenzfall der bekannten idealen Gasgleichung.
Analog können die Quantifizierungen
\begin{itemize}
    \item isochorer Prozesse: ${P}/{T}=\mathrm{const}$,
    \item isobarer Prozesse: ${V}/{T}=\mathrm{const}$ und
    \item isothermer Prozesse $PV=\mathrm{const}$
\end{itemize}
hergeleitet werden.

Wir wollen uns nun überlegen, ob wir auch einen alternativen Zugang zur Beschreibung der inneren Energie finden können, welcher nicht der statistischen Mechanik und ihrer Beziehung
\begin{align*}
    U=\frac{f}{2}\sm RT
\end{align*}
entspringt, sondern phänomenologischer Natur ist.
Dabei steht uns die messbare Relation $PV=\sm RT$ der idealen Gasgleichung zur Verfügung.
In der Experimentalphysik begegnet uns der \emph{Gay-Lussacsche Überstromversuch}. Dieser isoliert ein zu untersuchendes Gas in einer Hälfte eines zweigeteilten Kastens. Im nächsten Schritt öffnen wir einen begrenzten Bereich der Trennwand und ermöglichen dadurch die freie Expansion des Gases in die zweite Hälfte. Nun beobachten wir, dass die Temperatur des Systems konstant bleibt.
Aufgrund der Isolation des Systems bleibt dessen innere Energie erhalten und es gilt:
\begin{align*}
    \diff u(T,V)=0=\left(\frac{\partial u}{\partial T}\right)_v\diff T+\left(\frac{\partial u}{\partial v}\right)_T\diff v.
\end{align*}
Ferner folgt mit unserem Experimentablauf und der beobachteten Konstanz der Temperatur
\begin{align*}
    \diff v \neq 0,\quad\diff T=0\quad\rightarrow\quad \left(\frac{\partial u}{\partial v}\right)_T=0
\end{align*}
und führt uns somit zu der Einsicht
\begin{align*}
    \boxed{u=u(T)}\:.
\end{align*}
\begin{formal}
    Die innere Energie eines abgeschlossenen Systems eines allgemeinen idealen Gases ist ausschließlich von dessen Temperatur abhängig.
\end{formal}
Damit folgt auch, dass die spezifische Wärme $c_v$ \textendash{} wie erwartet \textendash{} ausschließlich temperaturabhängig ist\footnote{Wir erinnern an dieser Stelle an den gemachten Vorgriff bezüglich der \anf{Auftaubereiche} der Freiheitsgrade, welche im quantenmechanischen Formalismus verankert sind},
\begin{align*}
    c_v=\left(\frac{\partial u}{\partial T}\right)_v=c_v(T),
\end{align*}
und wir erhalten zum Abschluss unserer Betrachtungen die molare innere Energie über Aufintegration der inneren Wärme der Form:
\begin{align*}
    u=u_0+\int_{T_0}^Tc_v(\Tilde{T})\diff \Tilde{T}.
\end{align*}
Invertieren wir die gefundene Darstellung der inneren Energie, können wir mittels der Entropiedarstellung
\begin{align*}
    \frac{1}{T}=f(u)=\left(\frac{\partial s}{\partial u}\right)_v
\end{align*}
schreiben. Die ideale Gasgleichung führt uns bekanntermaßen auf
\begin{align*}
    \frac{P}{T}=\frac{R}{v}=\left(\frac{\partial s}{\partial v}\right)_u.
\end{align*}
Damit haben wir zwei Zustandsgleichungen gefunden, welche von unseren extensiven Größen abhängen.
Wir können diese nun nutzen, um die molare Entropie mittels Integration aufzuschreiben:
\begin{align*}
    s=s_0+\int_{u_0}^u\frac{1}{\Tilde{T}}(u')\diff u'+R\ln\frac{v}{v_0}\quad\text{mit}\quad\diff u'=c_v(\Tilde{T})\diff \Tilde{T},
\end{align*}
wobei mit $\diff u=c_v(T)\diff T$ abschließend
\begin{align*}
    \boxed{s=s_0+\int_{u_0}^u\frac{c_v(\Tilde{T})}{\Tilde{T}}\diff \Tilde{T}+R\ln\frac{v}{v_0}}
\end{align*}
folgt.
Dabei handelt es sich um eine parametrische Darstellung der entropischen Fundamentalgleichung.

Wir wollen nun ein konkretes Beispiel für den Verlauf der spezifischen Wärme $c_v$ geben und betrachten dazu ein aus zweiatomigen Molekülen bestehendes ideales Gas. Für dieses gilt, wie wir bereits erläutert haben,
\begin{align*}
    u=\frac{f}{2}RT\quad\text{mit}\quad f=7\quad\rightarrow\quad c_v=\frac{f}{2}R=\frac{7}{2}R,
\end{align*}
sodass der spezifische Wärmeverlauf in Abhängigkeit der Temperatur $T$ wie in Abbildung~\Abbref{fig:ThawingOfDegreesOfFreedom} aussieht. 
\begin{figure}[htbp]
    \centering
    \tfigThawingOfDegreesOfFreedom
    \caption{Temperaturabhängigkeit der spezifischen Wärme eines idealen Gases zweiatomiger Moleküle.}
    \label{fig:ThawingOfDegreesOfFreedom}
\end{figure}

Die angekündigten, von der Besetzbarkeit der Zustände abhängigen, Auftaubereiche der Freiheitsgrade schlagen sich wie erwartet hier in der spezifischen Wärme nieder.
Wir beobachten ferner, dass für Festkörper ein zu $T^3$ proportionaler Verlauf vorliegt, welcher durch die Anschauung von Phononen und ihren Schwingungen plausibilisiert werden kann \textendash{} darauf wollen wir im Rahmen der statistischen Mechanik erneut zu sprechen kommen.

Wir wollen uns nun erweiterte Systeme vorstellen, in denen eine Mischung einfacher idealer Gase vorliegt. Es stellt sich uns die Frage, wie die Fundamentalgleichungen aussehen, die dieses neue System beschreiben. Für die innere Energie,
\begin{align*}
    \boxed{U=\left(\sum_j\frac{f_j}{2}\sm _j\right)RT}\:,
\end{align*}
ist die Herleitung naheliegend, da wir die neuen, unterschiedlichen Freiheitsgrade und Stoffmengen in die altbekannte Fundamentalgleichung einsetzen können und über die hinzukommenden Terme nur summieren müssen (dies ist uns deshalb erlaubt, da wir für ideale Gase davon ausgehen, dass keine Wechselwirkungen zwischen den Gasmolekülen vorliegen). Auch die entropische Fundamentalgleichung ergibt sich als Summation über die Einzelentropien der unterschiedlichen Gase:
\begin{align*}
    \boxed{S=\sum_j\sm _js_j=\sum_j\sm _j\left(s_{j0}+\frac{f_j}{2}R\ln\frac{T}{T_0}+R\ln\frac{V}{\sm _jv_0}\right)}
\end{align*}
(wieder charakterisieren die Größen $T_0,v_0,s_{j0}$ den Referenzzustand).
Dies ist eine Konsequenz des additiven Charakters der Entropie und wird durch das \emph{Gibbs'sche Theorem} beschrieben:
\begin{align*}
    \boxed{S(\text{Mischung idealer Gase})=\sum_jS_j(\text{Einzelgase bei } V,T)}.
\end{align*}
\begin{formal}
    Das \formalemph{Gibb'sche Theorem} besagt, dass sich die Gesamtentropie eines Systems gemischter idealer Gase als die Summe der Einzelentropien der Gase schreiben lässt.
\end{formal}
Veranschaulichen können wir es, indem wir uns einen sogenannten \emph{Mischapparat} vorstellen, welcher zwei unterschiedliche Gase bei gleichem Druck $P=P_1=P_2$ und gleicher Temperatur $T=T_1=T_2$ und zwei semipermeable Membranen beinhaltet, die für je eine Gassorte durchlässig sind. 

\begin{figure}[htbp]
    \centering
    \begin{subfigure}[b]{.45\textwidth}
        \centering
        \tfigMixingMachineOne        
        \caption{}
        \label{fig:MixingMachineOne}
    \end{subfigure}
    \begin{subfigure}[b]{.45\textwidth}
        \centering
        \tfigMixingMachineTwo           
        \caption{}
        \label{fig:MixingMachineTwo}
    \end{subfigure}
    \caption{Dartstellung eines Mischapparates}
    \label{fig:MixingMaschine}
\end{figure}
Wir stellen uns vor, dass die Ausgangssituation der \Abbref{fig:MixingMachineOne} entspricht \textendash{} Membran (1) ist nur für das erste Gas durchlässig, Membran (2) nur für das zweite, sodass beide Gase noch in getrennten Kammern vorliegen. Nun kann der Kolben gemeinsam mit
Membran (1) quasistatisch%(??) 
bewegt werden, bis die Gase \textendash{} immer noch je im selben Volumen enthalten \textendash{} vollständig miteinander vermischt sind (\Abbref{fig:MixingMachineTwo}). Es ist ersichtlich, dass dieser Vorgang ohne weitere Prozesse abläuft, Volumina $V_1,V_2$, Stoffmengen $\sm _1,\sm _2$, Druck $P_1,P_2$und Temperatur $T_1,T_2$ also durchweg konstant bleiben und letztlich auch die innere Energie $U$ erhalten wird. Folglich ist auch die Gesamtentropie $S$ vor und nach der Mischung hier gleich groß.

Wir wollen darauf verweisen, dass \emph{Mischentropien} sehr wohl existieren, d.~h., dass Mischprozesse einen Zuwachs der Entropie verursachen können. Dies setzt jedoch einen anderen Versuchsaufbau voraus, für welchen beispielsweise keine Erhaltung der Volumina $V_1,V_2$ gilt. 
\begin{figure}[htbp]
    \centering
    \tfigMixTwoGases
    \caption{Versuchsaufbau mit Mischentropie}
    \label{fig:MixTwoGases}
\end{figure}

Das simpelste Beispiel wird durch eine Kammer beschrieben, welche durch eine Trennwand geteilt wird. Links und rechts der Trennwand befinden sich bei gleicher Temperatur $T$ und Druck,
\begin{align*}
    P\propto\frac{\sm _1}{V_1}=\frac{\sm _2}{V_2}=\frac{\sm }{V}=\frac{\sm _1+\sm _2}{V_1+V_2},
\end{align*}
zwei unterschiedliche Gase. Wird die Trennwand nun gemäß \Abbref{fig:MixTwoGases} entfernt, so unterscheiden sich die anfangs eingenommenen Volumina der Gase, $V_1$, respektive $V_2$, von dem zuletzt eingenommenen Gesamtvolumen $V_1+V_2$.
Wir wollen die Entropie des Endzustandes beschreiben und mit der Entropie vor Mischung in Relation setzen. Dazu schreiben wir zuerst die Zustandsgleichung
\begin{align*}
    S=\sum_j\sm _js_j=\sum_j\sm _j\left(s_{j0}+\frac{f_j}{2}R\ln\frac{T}{T_0}+R\ln\frac{V}{\sm _jv_0}\right)
\end{align*}
auf. Wir wollen nun den letzten Term wie folgt umschreiben:
\begin{align*}
    \ln\frac{V}{\sm _jv_0}=\ln\frac{V}{\sm v_0}-\ln\frac{\sm _j}{\sm }\quad\text{mit}\quad\frac{\sm _j}{\sm }=x_j\quad\text{und}\quad \sm =\sum_j\sm _j.
\end{align*}
Nach Einsetzen erhalten wir
\begin{align*}
    \boxed{
        \begin{aligned}
            S & =\sum_j\sm _j\left(s_{j0}+\frac{f_j}{2}R\ln\frac{T}{T_0}+R\ln\frac{V}{\sm v_0}\right)-\sm R\sum_jx_j\ln x_j                 \\
              & =\sum_jS_j+S_m
        \end{aligned}
    }\:.
\end{align*}
Dies entspricht der Entropie des Systems vor Mischung (die Einzelgase bei Temperatur $T$ und Druck $P \propto n_j/v_j=n/V$) und einem zusätzlichen positiven Term für die Mischungsentropie $S_m$.
\paragraph*{Gibbs'sches Paradoxon}
Wir haben bislang jedoch noch eine Anwendung des Theorems übersehen, die zum Widerspruch führt und auch als \emph{Gibbs'sches Paradoxon} bekannt ist.
Betrachten wir einmal den Grenzfall gleicher Gaskomponenten \textendash{} wir werden feststellen müssen, dass dies beim beschriebenen Mischprozess zu einer unsinnigen Entropiezunahme führt. Schließlich können wir uns vorstellen, die Trennwand zu einem beliebigen Zeitpunkt wieder einzusetzen und damit die Anfangsbedingungen wieder hergestellt zu haben \textendash{} für die natürlich die anfängliche Entropie vorherrscht. Vertrauen wir der Herleitung des positiven Mischterms der Entropie bedingungslos, so haben wir gerade einen simplen Prozess der Entropieverringerung beschrieben, welcher sich äußerst unphysikalisch anfühlt. 

Die Quantenmechanik hilft uns hier glücklicherweise das Paradoxon begründet aufzulösen: Die Systeme vor und nach der Trennung unterscheiden sich physikalisch nicht (natürlich setzen wir erneut gleiche Anfangstemperaturen und Drücke voraus). Alle Gasmoleküle werden als zueinander identisch angesehen, sodass keine zusätzlichen Systemkonfigurationen bei Mischung vorliegen, weshalb auch der Entropie-Mischterm plausiblerweise in diesem Fall null bleibt.

\section{Das Van-der-Waals-Gas\label{sec:das van der waals gas}}
Wir wollen uns jetzt ausgehend von den idealen Gasen den realen Gasen und der Van-der-Waals-Gleichung zuwenden.

Die Wechselwirkungen der Atome und Moleküle idealer Gase sind aufgrund ihrer großen mittleren Abstände verschwindend gering und werden deshalb nicht betrachtet.
Anders verhält es sich bei realen Gasen. Tragen wir das Wechselwirkungspotential in Abhängigkeit der Abstände wie in [Abb??] auf, so erkennen wir in unmittelbarer Radiusgrößenordnung starke Abstoßungen; bei größeren Abständen hingegen Anziehung (auch Kohäsion genannt). 

Für reale Gase beobachten wir zwei wesentliche Effekte: Zum einen die sogenannte Volumenreduktion, welche das effektiv vom Gas einnehmbare Volumen, $v_\mathrm{eff}=v-b$, beschreibt ($v$ bezeichnet das Behältervolumen und $b$ das Eigenvolumen eines Moles der Gasmoleküle).
Zum anderen liegt eine Druckkorrektur, $P=P_\mathrm{kin}-a/v^2$, vor. Der erste Term beschreibt den kinetischen Druck, der der thermischen Bewegung entspringt, der zweite die intermolekulare Anziehung, welche den Druck auf die Behälterwand reduziert. 
Da letztere Anziehung von den Molekülwechselwirkungen abhängt, ist sie proportional zur Molekülpaaranzahl ($\propto N^2/V^2$).

Wir setzen die für das reale Gas diskutierten Größen in die ideale Gasgleichung ein:
\begin{align*}
    P_\mathrm{kin} v_\mathrm{eff}=RT\\
    (P+\frac{a}{v^2})(v-b)=RT
\end{align*}
und erhalten die sogenannte \emph{Van-der-Waals-Gleichung},
\begin{align*}
    \boxed{ P=\frac{RT}{v-b}-\frac{a}{v^2}}\:
\end{align*}
(in der statistischen Mechanik lässt sich diese Gleichung ebenfalls herleiten).
Da $a$ und $b$ empirische Konstanten sind, liefert die Gleichung keine strenge quantitative Beschreibung realer Gase.
Sie erlaubt jedoch eine qualitative Beschreibung ihrer Eigenschaften und der gas-flüssig Phasenübergänge.

Bevor wir zum nächsten Unterkapitel übergehen, wollen wir (ohne Beweis) noch zwei interessante Zusammenhänge für reale Gase vorstellen.
Zum einen gilt für die Ableitung der inneren molaren Energie nach dem molaren Volumen unter konstanter Temperatur: 
\begin{align*}
    \left( \frac{\partial u}{\partial v} \right)_T =\frac{a}{v^2}>0.
\end{align*}
Wird das Volumen verringert, so nehmen die Wechselwirkungen zwischen den Molekülen und damit die innere Energie zu.
Ferner folgt aus dieser Beziehung überraschenderweise, dass die spezifische Wärme volumenunabhängig ist, $c_v=c_v(T)$, denn es gilt
\begin{align*}
    \frac{\partial c_v}{\partial v}=\frac{\partial}{\partial v}\frac{\partial u}{\partial T}=\frac{\partial}{\partial T}\frac{a}{v^2}=0.
\end{align*} 

\begin{summary}
    In diesem Kapitel haben wir sogenannte Antwortkoeffizienten, das einkomponentige und mehrkomponentige ideale Gas und das Van-der-Waals-Gas kennengelernt und untersucht. 
    
    Die \emph{Antwortkoeffizienten} beschreiben die Reaktion eines Systems unter festgelegten Zustandsänderungen.  Dabei unterscheidet man die \emph{isochoren} ($V=const$, $P/T=const$), \emph{isothermen} ($T=const$, $PV=const$), \emph{isobaren} ($P=const$, $V/T=const$) und \emph{isentropen} ($S=const$, $PV^{\gamma}=const$) Zustandsänderungen \textendash{} letztere umfassen auch adiabatische (ohne  Wärmeaustausch stattfindende) reversible ($\Delta S=0$) Prozesse. Nicht jeder adiabatischer Prozess ist jedoch isentrop.

    Die Beschreibung der relativen Volumenänderung eines Systems mit Kontrolle über Temperatur und Druck führt auf den \emph{thermischen Ausdehnungskoeffizienten} $\alpha$ und die \emph{isotherme Kompressibilität} $\kappa_T$:
    \begin{align*}
        \alpha=\frac{1}{V}\left( \frac{\partial V}{\partial T}\right)_P, \; \kappa_T=-\frac{1}{V}\left( \frac{\partial V}{\partial P}\right)_T.
    \end{align*} 
    Die Beschreibung der Entropieänderung bzw. des molaren quasistatischen Wärmeflusses eines Systems lässt auf die isobare molare spezifische Wärme $c_p$,
    \begin{align*}
    c_p=\frac{T}{\sm} \left( \frac{\partial S}{\partial T} \right)_P= \frac{1}{\sm} \left( \frac{\udiff Q}{\diff T} \right)_P =\frac{1}{\sm}\left( \frac{\partial H}{\partial T}\right)_P,
    \end{align*}
    schließen. Analog lässt sich unter Kontrolle von Temperatur und Volumen die isochore molare spezifische Wärme $c_V$ definieren:
    \begin{align*}
    c_v=\frac{T}{\sm}\left(\frac{\partial S}{\partial T}\right)_V=\frac{1}{\sm}\left(\frac{\udiff Q}{\diff T}\right)_V.
    \end{align*}
    Dabei wurde die Enthalpie als thermodynamisches Potential wie folgt eingeführt:
    \begin{align*}
        H=U+PV=H(T,P).
    \end{align*}
    Die Antwortkoeffizienten werden über
    \begin{align*}
    c_p=c_v+\frac{TV\alpha^2}{\sm \kappa_T}
    \end{align*}
    verknüpft.
    Der Adiabatenexponent $\gamma$ bildet das Verhältnis der spezifischen Wärmekapazitätskoeffizienten ab und geht für große Systeme mit entsprechend großen Freiheitsgraden $f$ gegen $1$:
    \begin{align*}
        \gamma=\frac{c_p}{c_v}=\frac{f+2}{f}.
    \end{align*}
    Die \emph{Maxwell-Beziehungen} beschreiben die Relationen zwischen den zweiten Ableitungen der inneren Energie (unter Nutzung der Integrabilitätsbedingung) mit:
    \begin{align*}
    \left(\frac{\partial T}{\partial V}\right)_{S,\sm }=-\left(\frac{\partial P}{\partial S}\right)_{V,\sm }=-T\left(\frac{\partial P}{\udiff Q}\right)_{V,\sm }.
    \end{align*}
    Wir haben ferner das \emph{ideale Gas} eingeführt. Dessen Idealisierung besteht dabei in der Vernachlässigung der räumlichen Ausdehnung der \textendash{} und Wechselwirkungen zwischen den \textendash{} Gasteilchen.
    Das einkomponentige ideale Gas lässt sich mittels der \emph{idealen Gasgleichung} 
    \begin{align*}
        PV=\sm RT
    \end{align*}
    beschreiben. $R=8,3 mol^{-1}K^{-1}$ bezeichnet dabei die universelle Gaskonstante.

    Aus diesem Gesetz kann für konstante Temperatur das Boyle-Mariottsche-Gesetz ($PV=const$), respektive das Gay-Lussacsche-Gesetz ($VT^{-1}=const$) für konstanten Druck, abgeleitet werden.
    Ebenso grundlegend ist die \emph{kalorische Zustandsgleichung},
    \begin{align*}
        U=\frac{f}{2}\sm RT,
    \end{align*}
    welche sich aus dem Gleichverteilungssatz der statistischen Mechanik ergibt:
    \begin{align*}
        u=\frac{1}{2}RT.
    \end{align*}
    Die innere Energie eines Gases ist nur von dessen Temperatur abhängig. Dies kann mittels des Gay-Lussacschen-Überstromversuchs demonstriert werden, welcher in einem abgeschlossenen System einen Gasstrom unter konstantbleibender Temperatur erzeugt.
    
    Beim Übergang zu einem \emph{mehrkomponentigen} idealen Gas greift das \emph{Gibbs'sche Theorem}, welches besagt, dass die Gesamtentropie des Gasgemisches der Summe der Einzelgasentropien entspricht (Voraussetzung ist natürlich die Konstanz der Zustandsggrößen Temperatur, Volumen und Druck). Bleiben eine oder mehrere Größen über den Mischprozess hinweg nicht erhalten (ändert sich z.B. das Volumen der Einzelgase), so ergibt sich eine positive \emph{Mischentropie} $\Delta S_m>0$. Das \emph{Gibbs'sche Paradoxon} beschreibt den scheinbaren Widerspruch, der sich aus der Anwendung des Theorems auf gleiche Gaskomponenten ergibt. Hier entsteht jedoch \textendash{} entsprechend der physikalischen Intuition \textendash{} keine Mischentropie, da die Gasteilchen im quantenmechanischen Sinne als identisch betrachtet werden und somit keine neuen Systemkonfigurationen unter Mischung eingeführt werden.
    
    Das \emph{Van-der-Waals-Gas} ist eine Erweiterung des idealen Gases, da es die Wechselwirkungen zwischen den Gasteilchen (Kohäsion und Abstoßung) und das effektive Volumen (unter Einberechnung der räumlichen Ausdehnung der Gasteilchen), als auch den effektiven Druck (mittels Einberechnung der Kohäsion) berücksichtigt.
    Es verhält sich entsprechend 
    \begin{align*}
        P=\frac{RT}{v-b}-\frac{a}{v^2},
    \end{align*}
    der sogenannten \emph{Van-der-Waals-Gleichung}.
\end{summary}
\input{4_Mikroskopische_Erklärung_der_Entropie}
% !TeX root = Theo_IV.tex

\chapter{Thermodynamische Prozesse und Maschinen}
Im Laufe dieses Kapitels wollen wir uns mit unterschiedlichen thermodynamischen Prozessen und Maschinen auseinandersetzen. Ein Beispiel für letztere sind Wärmekraft- und Carnot-Maschinen, welche deshalb für uns interessant sind, weil die Entwicklung des Entropiebegriffes historisch durch ihre Beschreibung begründet wurde.
Die Realisierung thermodynamischer Maschinen baut auf zwei wesentlichen Prinzipien auf: der Energieerhaltung (und den damit verbundenen physikalischen Gesetzen der Mechanik) und der Monotonie der Entropiezunahme ($\Delta S \geq 0$) in abgeschlossenen Systemen.
Wir wollen zunächst mit einigen Charakterisierungen thermodynamischer Prozesse anfangen.


\section{Quasistatische Prozesse}
Die quasistatischen Prozesse sind uns mittlerweile sehr gut bekannt. Sie bilden Prozessführungen, welche zu jedem Zeitpunkt Gleichgewichtszustände des Systems beschreiben.
\paragraph*{Der thermodynamische Konfigurationsraum}
Wir wollen zur Veranschaulichung und Charakterisierung allgemeiner Prozesse den sogenannten thermodynamischen Konfigurationsraum einführen. Dieser wird durch die extensiven Variablen eines betrachteten Systems aufgespannt und ist exemplarisch in Abb. [Abb][Ref] dargestellt.
Die Menge aller Gleichgewichtszustände eines quasistatischen Prozesses liegen im Konfigurationsraum auf einer Hyperfläche $S=S(U,V,X_j$), die der Beziehung 
\begin{align*}
    \left(\frac{\partial S}{\partial U}\right)_{V,X_j}=\frac{1}{T}>0
\end{align*}
(welche aus Postulat \ref{post:eigenschaften_entropie} folgt) unterliegt.  
Wir können auch zusammengesetzte, abgeschlossene Systeme leicht im Konfigurationsraum darstellen, da sich die extensiven Größen eines Teilsystemes aus der Additivität und der Kenntnis der Größen des anderen Teilsystems ergeben. Formal ist die Beschreibung der Hyperfläche durch $S=S(U^{(1)},X_j^{(1)},U^{(2)},X_j^{(2)})$ in diesem Fall also zu 
\begin{align*}
    S&=S(U^{(1)},X_j^{(1)},U=U^{(1)}+U^{(2)},X_j=X_j^{(1)}+X_j^{(2)})\\
    &=S(U^{(1)},X_j^{(1)},U,X_j)
\end{align*}
äquivalent.
Nichtgleichgewichtszustände hingegen haben, bedingt durch ihre Dynamik, viel mehr Dimensionen und spannen einen bedeutend größeren Raum auf. Dieser beinhaltet z.~B. Inhomogenitäten, Turbulenzen, Flussfelder und viele mehr. 

\paragraph*{Reale und quasistatische Prozesse}
Wir wissen, dass es sich bei der quasistatischen Prozessführung um eine Idealisierung handelt, welche reale Elemente - wie Geschwindigkeiten, Flüsse, Raten und weitere - vernachlässigt. 
Nichtsdestotrotz ist diese Idealisierung für uns nützlich, da die Thermodynamik sehr genaue (Vor-)Aussagen über nur eben solche Prozessführungen erlaubt.
Für die Abbildung realer Prozesse müssen wir die Diskrepanz zur Idealisierung jedoch berücksichtigen. (I.~d.~R. liegen für reale Prozesse wie in Abb. [Abb](??) [Ref] illustriert ein Anfangszustand $A$ und ein Endzustand $Z$ auf der Hyperfläche vor, welche über einen beliebigen Weg im Konfigurationsraum (über Nichtgleichgewichtszustände) miteinander verbunden sind).  
Eine mögliche Annäherung, welche den Übergang von Idealisierung zu Realität überbrückt, ist die Darstellung realer Prozesse als dichte Abfolge von Gleichgewichtszuständen über Nichtgleichgewichtszustände. Anschaulich gesprochen entspricht dies einer Abfolge von Zuständen auf einer Hyperfläche $S$ im Konfigurationsraum, die über Wege - welche nicht auf der Hyperfläche selbst liegen - miteinander verbunden sind. Auch dies haben wir in Abb. [Abb2 neben Abb1][Ref](??) illustriert.    
\begin{formal}
     \formalemph{Reale Prozesse} entwickeln sich i.~d.~R. ausgehend von einem Punkt $A$ auf der Hyperfläche $S$ über Nichtgleichgewichtszustände des Konfigurationsraumes (außerhalb von $S$) zu einem Endzustand $Z$, welcher wiederum als Gleichgewichtszustand in $S$ liegt.

     \formalemph{Quasistatische (auch quasistationäre) Prozesse} entwickeln sich über eine dichte Abfolge von Gleichgewichtszuständen in $S$, welche über Nichtgleichgewichtszustände verbunden sind.
 \end{formal}

\paragraph*{Zeitkonstanten}
Nach Erläuterung dieser Näherung stellt sich uns die Frage, durch welche quantitative Größe quasistatische Prozesse von nicht-quasistatischen Prozessen unterschieden werden können.
Wir wollen dies anhand eines Beispiels erörtern.

\begin{figure}[htbp]
    \centering
    \tfigAdiabaticGasCompression
    \caption{Adiabatische Kompression eines Gases}
    \label{fig:AdiabaticGasCompression}
\end{figure}

In \Abbref{fig:AdiabaticGasCompression} haben wir eine adiabatische Gaskompression mittels eines Kolbens in einer Kammer abgebildet. Drücken wir den Kolben schnell genug ein, so entstehen Verwirbelungen des Gases, für welche wir die folgende Vereinfachung treffen: Sie breiten sich vom Kolben über die gesamte Kammerlänge $l$ bis zur gegenüberliegenden Kammerwand mit Schallgeschwindigkeit $c$ aus. Damit folgt für die \anf{Lebensdauer} dieser Störung $\tau =l/c$. (Um ein Gefühl für die Größenordnung zu gewinnen, können wir eine Kammerlänge $l=\qty{1}{\m}$ einsetzen und erhalten $\tau=\qty{0.003}{\s}$.) Wollen wir den Kolben also quasistatisch eindrücken, so muss für die Laufzeit $\Delta t \gg\tau$ gelten.
\begin{formal}
    \formalemph{Quasistatische Prozesse} werden über die Beziehung $\Delta t \gg\tau$ charakterisiert: Ihre Prozesszeiten $\Delta t$, welche zwischen Gleichgewichtszuständen liegen, müssen wesentlich größer als die charakteristischen Relaxationszeiten $\tau$ der Systemstörungen sein.
    Es gilt $\Delta S\geq\Delta Q_{\mathrm{mess}}/T$.
\end{formal}  

\section{Reversible und irreversible Prozesse}

\paragraph*{Definitionen und Bemerkungen }
Wir wollen im Folgenden eine genaue Unterscheidung reversibler und irreversibler Prozesse geben. 
Dazu betrachten wir ein abgeschlossenes, zusammengesetztes System, welches durch die Lockerung einer Zwangsbedingung von Zustand $A$
in Zustand $B$ übergeht. Dem Postulat \ref{post:eigenschaften_entropie} zufolge ist die Entropie des Endzustandes größer als die Entropie des Anfangszustandes.
Folglich ist der umgekehrte, spontane Übergang von $B$ nach $A$ \glqq verboten\grqq{} und wir bezeichnen den Zustandsübergang von $A$ nach $B$ als irreversibel. Nicht-quasistatische Prozesse sind im Folgeschluss immer irreversibel, der Umkehrschluss gilt nicht.
Reversible Prozesse hingegen sind umkehrbare Prozesse, d.~h. die mit ihnen verknüpfte Entropieänderung ist gleich null.
Sie sind quasistatisch, wobei der Umkehrschluss im Gegenzug wieder nicht immer gilt. %[Grafik??]
\begin{formal}
    Ein \formalemph{reversibler Prozess} ist ein quasistatischer Prozess, welcher keine Entropieänderung verursacht ($\Delta S=0$) und damit auch umkehrbar ist. 
    Ein \formalemph{irreversibler Prozess} ist ein Prozess, welcher bedingt durch eine Entropiezunahme ($\Delta S>0$) in abgeschlossenen Systemen nicht umkehrbar ist. Nicht-quasistatische Prozesse sind immer irreversibel, aber auch quasistatische Prozesse können irreversibel sein.
\end{formal}
Wir wollen die eingeführten Begriffe anhand einiger Beispielprozesse veranschaulichen:
Ausgangspunkt bildet in allen Fällen ein \emph{thermisch isolierter} Kasten, in dessen linken Hälfte sich ein Gas befindet. Damit sind alle Ausgangszustände identisch.
\begin{figure}[htbp]
    \centering
    \begin{subfigure}[b]{.32\textwidth}
        \centering
        \tfigProcessReversibleQuasistationary        
        \caption{}
        \label{fig:ProcessReversibleQuasistationary}
    \end{subfigure}
    \begin{subfigure}[b]{.32\textwidth}
        \centering
        \tfigProcessIrreversibleQuasistationary           
        \caption{}
        \label{fig:ProcessIrreversibleQuasistationary}
    \end{subfigure}
    \begin{subfigure}[b]{.32\textwidth}
        \centering     
        \tfigProcessIrreversibleNonquasistationary
        \caption{}
        \label{fig:ProcessIrreversibleNonquasistationary}
    \end{subfigure}
    \caption{Vergleich unterschiedlicher Prozessführungen}
    \label{fig:ProcessReversibleIrreversibleQuasistationary}
\end{figure}
\paragraph*{Reversibel und quasistatisch}
Im ersten Aufbau, \Abbref{fig:ProcessReversibleQuasistationary}, wird der Kasten durch einen Kolben geteilt, welcher frei beweglich ist. Es läuft eine quasistatische, adiabatische Gasexpansion ab, welche den Kolben verschiebt. Die Verrichtung dieser mechanischen Arbeit führt zur Verringerung der inneren Energie und damit auch der Temperatur des Systems. Die Entropie hat sich insgesamt jedoch nicht verändert.
\paragraph*{Irreversibel und quasistatisch}
Im zweiten Aufbau, \Abbref{fig:ProcessIrreversibleQuasistationary}, wird der Kasten durch viele dicht beieinander liegende, entfernbare Trennwände geteilt. Wir können durch sukzessives Entfernen der Wände eine quasistatische Prozessführung anleiten, jedoch ist diese mit einer Entropiezunahme verbunden - der Prozess ist irreversibel.
\paragraph*{Irreversibel und nicht-quasistatisch}
Im dritten Aufbau, \Abbref{fig:ProcessIrreversibleNonquasistationary}, wird der Kasten durch eine einzige entfernbare Trennwand geteilt. Ziehen wir diese heraus, so findet eine nicht-quasistatische Zustandsänderung statt. Zwar gelangen wir zum selben Endzustand wie beim vorigen Experiment, jedoch laufen wir aufgrund der entstehenden Strömungen keinen Weg innerhalb der definierten Hyperfläche $S$ im Konfigurationsraum ab. Auch hier handelt es sich folglich um einen irreversiblen Prozess, welcher mit einer zu \ref{fig:ProcessIrreversibleQuasistationary} identischen Entropiezunahme verknüpft ist.

Es sei angemerkt, dass wir irreversible Prozesse durch die Ankopplung eines weiteren Systemes (also in offenen Systemen) realisieren können. Dazu muss die Entropie in Form eines Wärmeflusses vom Teilsystem abgegeben und vom angekoppelten System aufgenommen werden. 
Ein Beispiel einer derartigen offenen Systemkopplung ist die Erde. Sie beherbergt Prozesse, welche mit einer Entropieabnahme (geordneten Lebensformen) verbunden sind. Jedoch wird zugleich Entropie in Form von Wärme an ihre Umgebung abgegeben.

\paragraph*{Reversible Quellen und Reservoire}
Wir wollen einige weitere nützliche idealisierte Bausteine einführen, die uns später bei der Beschreibung kontrollierter Systemparameter, der Charakterisierung neuer thermodynamischer Potentiale sowie der Erarbeitung der Ensembletheorie der statistischen Mechanik helfen werden.
%ggf Tabelle anlegen (links bild, rechts text, vice versa, vice versa) 
\begin{itemize}
    \item \textbf{Reversible Arbeitsquelle (RAQ):} Ein idealisiertes System, mit welchem (mechanische) Arbeit reversibel ($\Delta S=\Delta Q/T=0$) ausgetauscht werden kann, nennt man eine \emph{reversible Arbeitsquelle}, kurz \emph{RAQ}. Eine derartige Arbeitsquelle/-senke ist also vollständig wärmeisoliert und hat eine konstante Entropie $S^{\mathrm{RAQ}}=\mathrm{const}$. Ein Beispiel für eine RAQ ist ein mechanisches System ohne Reibung.
    \item \textbf{Reversible Wärmequelle (RWQ):} Ein idealisiertes System, mit welchem Wärme reversibel ausgetauscht werden kann, nennt man eine \emph{reversible Wärmequelle}, kurz \emph{RWQ}. Eine derartige Wärmequelle/-senke ist von starren Wänden umgeben (keine mechanische Arbeit kann daran verrichtet werden). Für die Änderung ihrer inneren Energie gilt: $\diff U^{\mathrm{RWQ}}=\udiff Q^{\mathrm{RWQ}}=T\diff S=c(T)\diff T$.
    \item \textbf{Volumenreservoir:} Eine sehr große RAQ, deren Druck $P$ unabhängig von Volumen- und innerer Energieänderung konstant bleibt, bezeichnen wir als \emph{Volumenreservoir}.\footnote{Mathematisch beschreiben wir mit $(\partial P/\partial U)_{V,N_i}=(\partial P/\partial V)_{U,N_i}=0$ eine homogene Funktion $-1$-ten Grades ($\partial P/\partial \lambda U=\lambda ^{-1}\partial P/\partial U$), welche für unendlich groß-dimensionierte Systeme ($\lambda\rightarrow \infty$) somit gegen 0 strebt.}
    \item \textbf{Wärmereservoir:} Eine sehr große RWQ, deren Temperatur $T$ unabhängig von Änderungen der inneren Energie konstant bleibt, bezeichnen wir als \emph{Wärmereservoir}.\footnote{Analog zum Volumenreservoir beschreibt $(\partial T/\partial U)_{V,N_i}=0$ eine homogene Funktion $-1$-ten Grades.}
\end{itemize}
Ein Beispiel für ein Wärme- und Volumenreservoir ist die Atmosphäre (in bestimmten änderungsfreien Zeitabschnitten). Bedingt durch ihre Größenordnung wirken sich lokale Temperatur- und Druckänderungen auf der Erde nicht messbar auf sie aus.
\section{Prozesse maximaler Arbeit}
Welche Arbeitsleistung kann nun maximal von einem System verrichtet werden, wenn es von einem Zustand in einen anderen übergeht? 
Um diese Frage zu beantworten nutzen wir die eingeführten, reversiblen Quellen und betrachten Arbeit- und Wärmefluss separat. 

\paragraph*{Das Theorem maximaler Arbeit}
\begin{figure}[htbp]
    \centering
    \tfigTheoremMaximizedWork
    \caption{Kopplung eines Teilsystems mit Wärme- und Arbeitsquelle}
    \label{fig:MaximaleArbeit}
\end{figure}
In \Abbref{fig:MaximaleArbeit} veranschaulichen wir die (insgesamt abgeschlossene) Kopplung des Teilsystems TS mit einer RWQ und einer RAQ. Die Änderung der inneren Energie des Teilsystems beim Zustandsübergang ($\Delta U^{\mathrm{TS}}_{AB}=U^{\mathrm{TS}}_{B}-U^{\mathrm{TS}}_{A}$) wird in Form von Wärme ($\Delta Q^{\mathrm{RWQ}}$) an die RWQ und in Form von Arbeit ($\Delta W^{\mathrm{RAQ}}$) reversibel an die RAQ abgegeben. (Positive Änderungsterme kennzeichnen hier Größenzunahmen.) Wir fassen das Ganze entsprechend dem ersten Hauptsatz der Thermodynamik (\ref{hs:erster}) zusammen:
\begin{align}
    \label{eq:maximaleArbeit}
    -\Delta U^{\mathrm{TS}}_{AB}=\Delta Q^{\mathrm{RWQ}}+\Delta W^{\mathrm{RAQ}}
\end{align}
Für reversible und irreversible Prozesse des Systems ergeben sich folgende Größenbilanzen:
\renewcommand*{\arraystretch}{1.3}
\arrayrulecolor{formalshade!90!black}
\begin{table}[b]
    \centering
    \caption{Entropiebilanz eines abgeschlossenen gekoppelten Systems}
    \begin{tabularx}{.5\textwidth}{|l|X|l|}
        \hline
        \rowcolor{formalshade!98!blue}
        &Reversibel &Irreversibel\\
        \hline
        \rowcolor{formalshade}
        Gesamtsystem&$\Delta S^\mathrm{Ges}=0$&$\Delta S^\mathrm{Ges}>0$\\
        \rowcolor{formalshade!80!white}
        Teilsystem TS&$\Delta S^{\mathrm{TS}}_{\mathrm{AB}}$&$\Delta S^{\mathrm{TS}}_{\mathrm{AB}}$\\
        \rowcolor{formalshade}
        RAQ&$0$&$0$\\
        \rowcolor{formalshade!80!white}
        RWQ&$-\Delta S^{\mathrm{TS}}_\mathrm{AB}$&$-\Delta S^{\mathrm{TS}}_\mathrm{AB}+\Delta S^\mathrm{Ges}$\\
        \hline
    \end{tabularx} 
    \label{tab:Entropiebilanz}
\end{table}
\begin{itemize}
    \item \textbf{Entropie (der RWQ):} Bei reversiblen Prozessen innerhalb eines abgeschlossenen Systems ist die Gesamtentropie erhalten ($\Delta S^\mathrm{Ges}=0$). Die Entropieänderung des Teilsystems entspricht $\Delta S^{\mathrm{TS}}_{\mathrm{AB}}$. Da die RAQ per Definition keine Entropieänderung erfährt, schließen wir daraus, dass $\Delta S^\mathrm{RWQ}=-\Delta S^{\mathrm{TS}}_\mathrm{AB}$ ist.
    
    Für irreversible Prozesse verhalten sich das Teilsystem und die RAQ gleich. (Wir gehen hier von irreversiblen Prozessen aus, die den selben Anfangs- und Endzustand, $A$ und $B$, wie die dazu verglichenen reversiblen Prozesse haben.)
    Die Gesamtentropie nimmt in diesem Fall jedoch mit $\Delta S^\mathrm{Ges}>0$ zu, sodass für die RWQ $\Delta S^\mathrm{RWQ}=-\Delta S^{\mathrm{TS}}_\mathrm{AB}+\Delta S^\mathrm{Ges}$ folgt.
    (Der Vergleich ist in Tabelle \ref{tab:Entropiebilanz} noch einmal zusammengefasst.) 

    Die Entropiezunahme der RWQ ist folglich für reversible Prozesse kleiner als für irreversible:
    \begin{align*}
        \boxed{\Delta S^\mathrm{RWQ}_\mathrm{rev}<\Delta S^\mathrm{RWQ}_\mathrm{irr}}\:.
    \end{align*}
    
    \item \textbf{Wärme (der RWQ):} Bekanntermaßen gilt für die RWQ 
    \begin{align*}
        \Delta U^\mathrm{RWQ}=\Delta Q^\mathrm{RWQ}=\int_{\Delta S^\mathrm{RWQ}}T^\mathrm{RWQ}\diff S^\mathrm{RWQ},
    \end{align*}
    womit für die Wärmeaufnahmen 
    \begin{align*}
        \boxed{\Delta Q_\mathrm{rev}^\mathrm{RWQ}<\Delta Q_\mathrm{irr}^\mathrm{RWQ}}
    \end{align*}
    folgt. Die Wärmezunahme der RWQ ist (analog zur Entropiezunahme) für reversible Prozesse kleiner als für irreversible Prozesse.
    \item \textbf{Arbeit (der RAQ):} Über Umstellen von \ref{eq:maximaleArbeit},
    \begin{align*}
        \Delta W^\mathrm{RAQ}=-\Delta U_\mathrm{AB}^\mathrm{TS}-\Delta Q^\mathrm{RWQ},
    \end{align*}
    schließen wir somit auch darauf, dass die Arbeitsaufnahme für reversible Prozesse größer ist als für irreversible:
    \begin{align*}
        \boxed{\Delta W^\mathrm{RAQ}_\mathrm{rev}>\Delta W^\mathrm{RAQ}_\mathrm{irr}}\:.
    \end{align*}
\end{itemize} 
\begin{formal}
    \formalemph{Theorem maximaler Arbeit:} Der Arbeitsübertrag auf die RAQ ist für reversible Zustandsänderungen maximal, der Wärmefluss in die RWQ zugleich minimal. 
\end{formal}
Die mechanische Energieausbeute eines Prozesses lässt sich wie erwartet über die Minimierung von Wärmeverlusten maximieren. Idealisierte, reversible Prozesse gleicher Anfangs- und Endzustände realer Prozesse liefern dabei deren gültige Schranken. Bei irreversiblen Prozessen dissipiert ein Teil der möglichen Arbeitsleistung in Form von Wärme (beispielsweise über Reibung).

Eine \emph{Wärmepumpe} ist eine Maschine, welche Arbeit verrichtet ($\Delta W^\mathrm{RAQ}<0$) um Wärme der RWQ ins System zu pumpen ($\Delta Q^\mathrm{RWQ}<0$).
Für die Wärme- und Entropiebilanz folgt damit:
\begin{align*}
    \boxed{
    \begin{aligned}
            -\Delta Q^\mathrm{RWQ}_\mathrm{rev}>-\Delta Q^\mathrm{RWQ}_\mathrm{irr}\\
            -\Delta W^\mathrm{RAQ}_\mathrm{rev}<-\Delta W^\mathrm{RAQ}_\mathrm{irr}
    \end{aligned}
    }\:.
\end{align*}
Für reversible Prozesse ist die Wärmeaufnahme aus der RWQ also maximal und die Arbeit der RAQ minimal.






% \section{Prozesse maximaler Arbeit}

% Wir wollen nun untersuchen, welche Prozesse die maximale Arbeit verrichten und wie sich diese verhalten. 

% \paragraph*{Theorem maximaler Arbeit}

% \begin{formal}
%     Von allen Prozessen $A\rightarrow B$ ist der Arbeitsübertrag auf die reversible Arbeitsquelle maximal (und der Wärmefluss in die reversible Wärmequelle minimal), wenn der Ablauf reversibel ist. 
% \end{formal}

% Das entspricht einer Optimierung der mechanischen Energieausbeute durch Minimierung der Wärmeverluste. 


\paragraph*{Spezialfall}

Für ein Teilsystem TS, das die Wärme $\Delta Q^{\mathrm{RWQ}}$ aus der reversiblen Wärmequelle zugeführt bekommt und an das keine reversible Arbeitsquelle gekoppelt ist, lässt sich der zweite Hauptsatz der Thermodynamik ableiten. 
Im reversiblen Ablauf ändert sich die Entropie nämlich mit 
\begin{align*}
    \Delta S^{\mathrm{TS}} = \int \diff S^\mathrm{TS} = -\int \diff S^\mathrm{RWQ} = - \int \frac{\udiff Q^\mathrm{RWQ}}{T} = \int \frac{\udiff Q^\mathrm{TS}_\mathrm{rev}}{T} .
\end{align*}
Ohne das Teilsystem TS ist dann 
\begin{align}
    \label{eq:2.HS_teil1}
    \diff S = \frac{\udiff Q_\mathrm{rev}}{T},
\end{align}
was dem ersten Teil des zweiten Hauptsatzes der Thermodynamik entspricht (Gleichheit). 

Im irreversiblen Ablauf ist dagegen 
\begin{align*}
    \Delta S^{\mathrm{TS}} = \int \diff S^\mathrm{TS} = -\int \diff S^\mathrm{RWQ} +\Delta S = - \int \frac{\udiff Q^\mathrm{RWQ}}{T^\mathrm{RWQ}} + \Delta S = \int \frac{\udiff Q^\mathrm{TS}_\mathrm{irr}}{T^\mathrm{RWQ}} +\Delta S
\end{align*}
mit $\Delta S>0$. Insgesamt ist dann 
\begin{align}
    \label{eq:2.HS_teil2}
    \diff S > \frac{\udiff Q_\mathrm{irr}}{T},
\end{align}
was dem zweiten Teil des zweiten Hauptsatzes der Thermodynamik entspricht (strenge Ungleichheit). 


Kombiniert man die Gleichungen \eqref{eq:2.HS_teil1} und \eqref{eq:2.HS_teil2} und verwendet, dass $\udiff Q=\diff U-\udiff W_\mathrm{mech}-\udiff W_\mathrm{chem}$, so erhält man die Grundrelation der Thermodynamik:
\begin{align}
    \label{eq:grundrelation_der_TD}
    \boxed{T\diff S\geq \diff U-\udiff W_\mathrm{mech}-\udiff W_\mathrm{chem}}\:.
\end{align}



\section{Wirkungsgrad von Maschinen}

Eine thermodynamische Maschine besteht i.d.R. aus einem Teilsystem TS (später auch Hilfssystem genannt), das an zwei reversible Wärmereservoire angekoppelt wird \textendash{} ein heißes (h) und ein kaltes (k) \textendash{} sowie an eine reversible Arbeitsquelle RAQ. TS könnte z.B. ein Kolben sein.  

Insgesamt ist das System idealerweise nach außen hin abgeschlossen, $\diff U=0$. Ferner soll sich der Zustand des Teilsystems TS nach einem vollständigen Arbeitszyklus nicht geändert haben. 

Nach dem ersten Hauptsatz gilt ($\diff U=0$)
\begin{align}
    \label{eq:maschine_HS1}
    \udiff Q_\mathrm{h} +\udiff Q_\mathrm{k} + \udiff W^\mathrm{RAQ} = 0
\end{align}
und nach dem zweiten für einen reversiblen Ablauf 
\begin{align}
    \label{eq:maschine_HS2}
    \diff S_\mathrm{h} + \diff S_\mathrm{k} = 0 \equivalence \frac{\udiff Q_\mathrm{h}}{T_\mathrm{h}} + \frac{\udiff Q_\mathrm{k}}{T_\mathrm{k}} = 0. 
\end{align}
Dabei ist natürlich $\udiff Q_\mathrm{h}=-\udiff Q_\mathrm{TS}$ u.s.w. 

\paragraph*{Thermodynamische Maschine}

Bei einer thermodynamischen Maschine wird allgemein Wärme $\udiff Q_\mathrm{h}$ aus einer Energiequelle (z.B. ein Ofen oder Dampfkessel) mit hoher Temperatur $T_\mathrm{h}$ über das Teilsystem (z.B. ein Kolben) in Form von Arbeit $\udiff W^\mathrm{RAQ}$ an die Maschinerie und in Form von Wärme $\udiff Q_\mathrm{k}$ an ein Kühlsystem mit niedrigerer Temperatur abgegeben (siehe \Abbref{fig:ThermodynamicMaschine}). 

\begin{figure}[htbp]
    \centering
    \tfigThermodynamicMaschine
    \caption{Allgemeine thermodynamische Maschine: Eine Energiequelle (z.B. ein Ofen oder Dampfkessel) mit Temperatur $T_\mathrm{h}$ gibt Wärme an das Hilfssystem TS ab, welches die aufgenommene Energie in Form von Wärme an ein Kühlsystem mit der Temperatur $T_\mathrm{k}$ und in Form von mechanischer Arbeit an eine Maschinerie verteilt. }
    \label{fig:ThermodynamicMaschine}
\end{figure}

Nach Gleichung \eqref{eq:maschine_HS2} gilt für den reversiblen Ablauf
\begin{align*}
    \udiff Q_\mathrm{k} = \frac{T_\mathrm{k}}{T_\mathrm{h}} (-\udiff Q_\mathrm{h})
\end{align*}
und nach Gleichung \eqref{eq:maschine_HS1}
\begin{align*}
    \udiff W^\mathrm{RAQ} = \left( 1-\frac{T_\mathrm{k}}{T_\mathrm{h}} \right) (-\udiff Q_\mathrm{h}).
\end{align*}
Der Wirkungsgrad $\eta$ beschreibt das Verhältnis der Energie, die in Arbeit (i.~d.~R. mechanische) umgesetzt wird:
\begin{align*}
    \eta = \frac{\udiff W^\mathrm{RAQ}}{-\udiff Q_\mathrm{h}} = 1-\frac{T_\mathrm{k}}{T_\mathrm{h}}=\frac{T_\mathrm{h}-T_\mathrm{k}}{T_\mathrm{h}}. 
\end{align*}
Dieser Wirkungsgrad ist unabhängig von dem genauen Ablauf des Prozess und daher universell gültig. 

Der Wirkungsgrad wird verbessert, wenn die Temperatur $T_\mathrm{h}$ des heißen Wärmereservoirs hoch und die Temperatur $T_\mathrm{k}$ des kalten Reservoirs niedrig ist. Für $T_\mathrm{k}\rightarrow 0$ wird der Wirkungsgrad $1$. 
Dann ist $\udiff W^\mathrm{RAQ}=-\udiff Q_\mathrm{h}$. 
Betrachtet man die Gleichung \eqref{eq:maschine_HS2}, so sieht man, dass für Temperaturen nahe dem absoluten Nullpunkt nur wenig Wärme in das kalte Wärmereservoir fließen muss, damit die Entropieänderung des heißen Wärmereservoirs kompensiert wird (die gesamte Entropieänderung muss durch das kalte Wärmereservoir aufgenommen werden, da die Maschinerie als reversible Arbeitsquelle keine Wärme und damit auch keine Entropie aufnimmt). 

Sind die Temperaturen $T_\mathrm{h}$ und $T_\mathrm{k}$ gleich, so ist der Wirkungsgrad $0$. 
\begin{formal}
    Ein Prozess, dessen einziges Resultat ist, dass Wärme aus einer Wärmequelle vollständig in mechanische Arbeit umgewandelt wird, ist nicht möglich. Es gibt folglich kein Perpetuum mobile zweiter Art. 
\end{formal}


Im irreversiblen Fall ist der Umsatz in mechanische Arbeit sogar noch schlechter, 
\begin{align*}
    \Delta W_\mathrm{rev}^\mathrm{RAQ} > \Delta W_\mathrm{irr}^\mathrm{RAQ}
\end{align*}
und damit der Wirkungsgrad kleiner als im reversiblen Fall,
\begin{align*}
    \eta_\mathrm{rev}>\eta_\mathrm{irr}. 
\end{align*}



\paragraph*{Kühlschrank}

Beim Kühlschrank wird das oben beschriebene Prinzip der thermodynamischen Maschine umgekehrt. Mithilfe eines Motors wird über das Hilfssystem TS der Umgebung Wärme zugeführt. Dabei erhöht sich aber die Entropie der Umgebung. Für einen reversiblen Ablauf muss sich daher die Entropie und somit die Temperatur des ebenfalls an das Hilfssystem gekoppelten Kühlschranks verringern. 


\begin{figure}[htbp]
    \centering
    \tfigFridge
    \caption{Prinzip eines Kühlschranks: Einem Kühlschrank wird Wärme entzogen, indem ein Motor (reversible Arbeitsquelle) über ein Hilfssystem TS der Umgebung Wärme zuführt, deren Entropie sich dabei erhöht, sodass sich die Entropie des Kühlschranks verringern muss. }
    \label{fig:fridge}
\end{figure}

Idealerweise wird $-\udiff Q_\mathrm{k}$ maximal und $-\udiff W^\mathrm{RAQ}$ minimal (wegen der Stromrechnung). Der Wirkungsgrad der Kühlleistung ist gegeben durch 
\begin{align*}
    \eta = \frac{-\udiff Q_\mathrm{k}}{-\udiff W^\mathrm{RAQ}} = \frac{T_\mathrm{k}}{T_\mathrm{h}-T_\mathrm{k}}. 
\end{align*}
Ist die Umgebungstemperatur gleich der Kühlschranktemperatur, so geht die Effizienz gegen unendlich, da überhaupt keine Kühlleistung nötig ist. Für $T_\mathrm{k}\rightarrow 0$ geht $\eta$ gegen $0$, sodass der absolute Nullpunkt $T_\mathrm{k}$ nicht erreicht werden kann. 

Beim irreversiblen Ablauf ist wiederum
\begin{align*}
    \eta_\mathrm{rev}>\eta_\mathrm{irr}. 
\end{align*}


\paragraph*{Wärmepumpe}

Eine Wärmepumpe hat genau die gleiche Funktionsweise wie ein Kühlschrank, nur dass hier statt dem kalten das heiße System betrachtet wird. Der Wärmeleistungskoeffizient ist analog zum vorigen Abschnitt durch 
\begin{align*}
    \eta = \frac{T_\mathrm{h}}{T_\mathrm{h}-T_\mathrm{k}}
\end{align*}
gegeben, wobei jetzt im Zähler aber $T_\mathrm{h}$ anstelle von $T_\mathrm{k}$ steht. 



\section{Der Carnot-Zyklus}

Wir haben gesehen, dass eine thermodynamische Maschine Energie in Form von Wärme und Arbeit zwischen verschiedenen Wärmereservoirs und Arbeitsquellen transportiert. Dazu ist i.~d.~R. ein Hilfssystem nötig, dessen Aufgabe es ist, eine reversible Verteilung von $\Delta Q_\mathrm{h}$ auf $\Delta W^\mathrm{RAQ}$ und $\Delta Q_\mathrm{k}$ zu verwirklichen. Dabei darf sich das Hilfssystem selbst aber netto nicht verändern. 


Dies wird durch Kreisprozesse wie den Carnot-Zyklus realisiert. In \Abbref{fig:CarnotCycleIndicatorDiagram} ist für diesen das Indikatordiagramm als Temperatur über die Entropie dargestellt. Im Schritt AB findet eine isotherme Expansion statt, in der sich die Wärme um $\Delta Q_{AB}=(S_B-S_A)T_\mathrm{h} = \Delta S T_\mathrm{h}$ ändert . Im Schritt BC wird isentrop expandiert. Es folgt eine isotherme Kompression von C nach D, bei der die Wärme $\Delta Q_{CD}=-\Delta S T_\mathrm{k}$ abgegeben wird und zuletzt eine isentrope Kompression von D zurück nach A. Dabei wird ständig an die reversible Arbeitsquelle gekoppelt (Volumenarbeit durch Expansion und Kompression). 

Die verrichtete Arbeit ist nach dem ersten Hauptsatz der Thermodynamik durch 
\begin{align*}
    \Delta W^\mathrm{RAQ}=\Delta Q_{AB}+\Delta Q_{CD} = \Delta S(T_\mathrm{h}-T_\mathrm{k})
\end{align*}
und der Wirkungsgrad wie im vorigen Abschnitt beschrieben durch 
\begin{align}
    \eta_\mathrm{M} = \frac{\Delta W^\mathrm{RAQ}}{\Delta Q_{AB}} = \frac{T_\mathrm{h}-T_\mathrm{k}}{T_\mathrm{h}}
\end{align}
gegeben. In der Realität wird aufgrund von mechanischer Reibung und durch Abweichung von quasistationären Zuständen jedoch meist nur ein Wirkungsgrad erzielt, der etwa der Hälfte des idealen Wirkungsgrades entspricht. 

\begin{figure}[htbp]
    \centering
    \tfigCarnotCycleIndicatorDiagram
    \caption{Der Carnot-Zyklus}
    \label{fig:CarnotCycleIndicatorDiagram}
\end{figure}

Die Carnot-Maschine kann mit Hilfe eines idealen Gases realisiert werden.
Wir beschreiben den geschlossenen Prozess unter Zuhilfenahme eines PV-Diagrammes ([Abb][ref]):
Im ersten Schritt (von A nach B) findet die isotherme Gasexpansion statt. Der Druck nimmt entsprechend der kennengelernten Gasgleichung mit $V^{-1}$ ab. Für die isentrope Expansion (von B nach C) folgt eine stärkere Druckabnahme, welche der Relation $P \propto V^{-\gamma}$ folgt. \footnote{$Es sei an \gamma=(f+2)/f>1$, den eingeführten Adiabatenexponenten erinnert.} Im Anschluss erfolgen wieder eine isotherme (unter Ankopplung der Wärmequelle) und isentrope (unter Abkopplung der Wärmequelle) Kompression. Die eingeschlossene Fläche entspricht erneut der Arbeit, die der Prozess leistet. 

Zum Abschluss dieses Kapitels wollen wir einen kurzen Exkurs in die Geschichte machen und zwei interessante Nebenbemerkungen hinzufügen: 

\begin{itemize}
    \item Clausius erkannte, dass für geschilderte ideale Kreisprozesse das Kreisintegral über die Größe $\udiff Q/T$ null sein musste. Er entdeckte damit das totale Differential einer neuen Zustandsfunktion, einer bislang unbekannten Größe, die das System charakterisiert: die Entropie $S$.  
    \item Interessant ist die Beobachtung, dass eine thermodynamische Maschine der Messung von Temperatur dienen kann. 
    Über den Wirkungsgrad der Maschine kann nämlich das Verhältnis der absoluten Temperaturen des Systems bestimmt werden. Durch Festlegung eines Referenzpunktes (des Tripelpunktes von Wasser) kann die damit erhaltene Temperaturskala in die absolute Kelvinskala überführt werden.
    \item Da die Entropieänderung dem Produkt aus Wärmeänderung und Temperatur entspricht, kann nach Messung der Temperatur auch die Entropie ermittelt werden. Der Referenzpunkt der Entropie liegt nach dem Nernst-Postulat beim absoluten Temperaturnullpunkt.
\end{itemize}










\raggedright
\bibliography{literatur}  
\bibliographystyle{bib}

\end{document}
