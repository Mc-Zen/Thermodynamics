%
% -------------------------------------------------------------------
% Thermodynamics lecture at TU Berlin, read by Prof. Holger Stark
% -------------------------------------------------------------------
%

\documentclass[hidelinks, 11pt]{scrbook} 

\usepackage[left=2cm, right=2cm, bottom=1.5cm, top=1.5cm, includeheadfoot]{geometry} 
\usepackage[T1]{fontenc} 
\usepackage[utf8]{inputenc} 
\usepackage[english,ngerman]{babel}
\usepackage[font=footnotesize]{caption}
\usepackage{graphicx} 
\usepackage{multirow} 
\usepackage{tabularx} 
\usepackage{xcolor} 
\usepackage{amsmath} 
\usepackage{amssymb} 
\usepackage{amsfonts} 
\usepackage{amsxtra} 
\usepackage{mathtools} 
\usepackage{tensor}
\usepackage{enumerate} 
\usepackage{float}
\usepackage{siunitx}
\usepackage{longtable}
\usepackage{cancel}
\usepackage{isotope}
\usepackage{lmodern}
\usepackage{scrhack} 
\sisetup{locale = DE, separate-uncertainty}  
\usepackage[immediate]{silence}
\WarningFilter[temp]{latex}{Command} % filter underline/underbar command warning from sectsty 

% --------------------------
% spacing

%\renewcommand*{\familydefault}{\sfdefault} sans serif font
\raggedbottom
\setlength{\parindent}{0em}
\setlength{\parskip}{1em}

\usepackage[hang]{footmisc}
\setlength{\footnotemargin}{3mm} % space between number and content of footnote
\setlength{\skip\footins}{.5cm} % space between body and footnote section
\setlength{\footnotesep}{0.5cm}  % space between footnotes

% \renewcommand{\arraystretch}{1}
% \makeatletter
% \renewcommand\@pnumwidth{2em} % fix toc overfull hbox
% \makeatother


% --------------------------
% section styles

\newcommand{\example}[1]{\paragraph{#1:}}
% change title font to serif
\addtokomafont{title}{\rmfamily}


% --------------------------
% math

\DeclareMathOperator{\grad}{grad} 
\DeclareMathOperator{\divg}{div} 
\DeclareMathOperator{\rot}{rot} 
\DeclareMathOperator{\real}{\mathfrak{R}} 
\DeclareMathOperator{\imag}{\mathfrak{I}} 

\newcommand{\upupharpoons}{\upharpoonleft\!\upharpoonright}
\newcommand{\updownharpoons}{\upharpoonleft\!\downharpoonright}
% pm sign with minus in paretheses
\newcommand{\varpm}{\mathbin{\vcenter{\hbox{
  \oalign{\hfil$\scriptstyle+$\hfil\cr\noalign{\kern-.3ex}$\scriptscriptstyle({-})$\cr}
}}}}
% mp sign with plus in parentheses
\newcommand{\varmp}{\mathbin{\vcenter{\hbox{
  \oalign{$\scriptstyle({+})$\cr\noalign{\kern-.3ex}\hfil$\scriptscriptstyle-$\hfil\cr}
}}}}
% d with bar through
%\newcommand{\dbar}{d\hspace*{-0.08em}\bar{}\hspace*{0.1em}}
\newcommand{\dbar}{\mkern3mu\mathchar '26 \mkern-11mu d}
% differential operator have a non-italic d in German equation typesetting 
\newcommand*{\diff}{\mathop{}\!\mathrm{d}} 
\newcommand*{\udiff}{\mathop{}\!\dbar} 
\newcommand*{\diffa}[2][]{\mathop{\mathrm{d}^{#1}#2}}
%\newcommand{\diff}{\text{d}}	 
\newcommand{\Angstroem}{\text{\normalfont\AA}}   
\newcommand{\Abbref}[1]{Abb.~\ref{#1}} 
% Vector: bold non-cursive symbols for vectors instead of arrows 
\renewcommand{\vec}[1]{\mathbf{#1}} 

\newcommand{\equivalence}{\;\Leftrightarrow\;}
\newcommand{\implication}{\;\Rightarrow\;}


\newcommand{\sm}{n}
\newcommand{\avogadro}{N_\mathrm{A}}

\DeclareSIUnit{\bar}{bar}



\newcommand{\anf}[1]{\glqq #1\grqq{}}

% --------------------------
% other configurations

% set figure description
\addto\captionsngerman{
  \renewcommand{\figurename}{Abb.}
  \renewcommand{\tablename}{Tab.}
}
\graphicspath{{images/}} 
\usepackage[strict]{changepage}

% for formal definitions
\usepackage{framed}

% environment derived from framed.sty: see leftbar environment definition
\definecolor{formalbar}{rgb}{0.1,0.1,.2}
\definecolor{formalshade}{rgb}{0.95,0.95,1}
\definecolor{postulatebar}{rgb}{0.3,0.07,.07}
\definecolor{postulateshade}{rgb}{1,0.9,0.9}

\newcounter{PostulateCounter}
\renewcommand{\thePostulateCounter}{\Roman{PostulateCounter}}
% Quote box for important statements
\newenvironment{formal}{
  \def\FrameCommand{
    \hspace{1pt}%
    {\color{formalbar}\vrule width 2pt}%
    {\color{formalshade}\vrule width 4pt}%
    \colorbox{formalshade}%
  }
  \MakeFramed{\advance\hsize-\width\FrameRestore}%
  \noindent\hspace{-4.55pt}% disable indenting first paragraph
  \begin{adjustwidth}{}{7pt}%
  \vspace{-10pt}\vspace{2pt}% 
}{\vspace{2pt}\end{adjustwidth}\endMakeFramed}

% Quote box for postulates
\newenvironment{postulate}{
  \def\FrameCommand{
    \hspace{1pt}%
    {\color{postulatebar}\vrule width 2pt}%
    {\color{postulateshade}\vrule width 4pt}%
    \colorbox{postulateshade}%
  }
  \MakeFramed{\advance\hsize-\width\FrameRestore}%
  \noindent\hspace{-4.55pt}% disable indenting first paragraph
  \begin{adjustwidth}{}{7pt}%
  \vspace{-10pt}\vspace{2pt}% 
  \refstepcounter{PostulateCounter}
  \textbf{Postulat \Roman{PostulateCounter}:}

}{\vspace{2pt}\end{adjustwidth}\endMakeFramed}


% !TeX root = Theo_IV.tex

\usepackage{tikz}
%\usepackage{pgfplots}
%\pgfplotsset{compat=1.18}

\usetikzlibrary{
    arrows.meta,
    bending,
    positioning,
    decorations.markings,
    intersections,
    calc,
    decorations.pathreplacing,
    decorations.pathmorphing,
    patterns
}
%\tikzexternalize[prefix=figures/,shell escape=-enable-write18] % activate

\tikzset{
    % Colors
    object color/.style={blue!40!black!80!white},
    object style/.style={object color,thick},
    nice green/.style={green!50!black},
    nice orange/.style={red!60!yellow!70!black!90!white},
    nice dark blue/.style={blue!50!black},
    nice light blue/.style={blue!60!white!70!black},
    nice turquoise/.style={blue!50!green},
    polarisation color/.style={purple},
    charge color/.style={blue!50!white!70!black},
    red laser/.style={red!70!black},
    moving system color/.style={blue!60!black!70!white},
    %
    % Coordinate system
    coordsystem/.style={very thin, color=#1!50},
    xlabel/.style={anchor=north west},
    ylabel/.style={anchor=south east},
    %
    % Nodes and points
    invisible point/.style={circle,inner sep=0pt,outer sep=0pt,minimum size=0pt},
    point/.style={invisible point,fill=black,minimum size=4pt},
    %
    % Arrows
    arrow tip/.tip={Stealth},
    arr/.style={->,>={arrow tip}},
    rarr/.style={<-,>={arrow tip}},
    midarrow/.style={postaction=decorate,decoration={markings, mark=at position #1 with {\arrow[xshift=2.5pt]{arrow tip}}} },
    midarrow/.default=.5,
    rmidarrow/.style={postaction=decorate,decoration={markings, mark=at position #1 with {\arrowreversed{arrow tip}}} },
    rmidarrow/.default=.5,
    distance marker/.style={|<->|,>={arrow tip}},
    %
    % Thermodynamics stuff
    piston bar/.style={line width=2pt},
    piston/.style={line width=8pt}
}
\tikzstyle{every node}=[font=\footnotesize]


\newcommand{\tfigTitel}{
    \begin{tikzpicture}
        \pgfmathsetmacro{\shadowangle}{132}
        \newlength{\shadowdistance}
        \pgfmathsetlength{\shadowdistance}{0.1ex}
        \pgfmathsetmacro{\shadowopacity}{1}
        \pgfmathsetmacro{\shadowspread}{0.003}
        \pgfmathsetmacro{\shadowsize}{5}
        \pgfmathtruncatemacro{\totshadow}{100}
        \path[nice dark blue,opacity={\shadowopacity/\totshadow},shift={({132-180}:\shadowdistance)},scale={1+\shadowsize}] 
        foreach \nshadow [evaluate=\nshadow as \angshadow using \nshadow/\totshadow*360] in {1,...,\totshadow}{
            node[align=center] at (\angshadow:\shadowspread) {\huge Theoretische Physik 4\\ \\ \\
            \Large Thermodynamik/Statistische Physik}
            };
        \node[align=center] at (0,0) {\huge Theoretische Physik 4\\ \\ \\ \Large Thermodynamik/Statistische Physik};
    \end{tikzpicture}
}

% 100 particles in a rectangular box
\newcommand{\tfigSystemWithManyParticles}{
    \begin{tikzpicture}[scale=1.5]
        \draw (-1pt,-1pt) rectangle ($(3,2)+(1pt,1pt)$);
        \pgfmathsetseed{1}
        \foreach \i in {1,2,...,100}{
            \draw[fill=black] (rnd*3,rnd*2) circle[radius=.5pt];
        };
    \end{tikzpicture}
}

% Two chambers separated by a piston
\newcommand{\tfigTwoChambersSeparatedByPiston}{
    \begin{tikzpicture}[scale=1.6]
        \draw (0,0) rectangle (2,1);
        \node at (.5,.5) {1};
        \node at (1.5,.5) {2};
        \draw[pattern=north east lines] (.95,0) rectangle (1.05,1);
        \node at (0,-.5) {Wand} edge[arr,bend right=10] (.4,0);
        \node at (1.4,-.5) {Kolben} edge[arr,bend left=10] (1,0);
    \end{tikzpicture}
}

% Rectangular box with piston
\newcommand{\tfigRectangularBoxWithPiston}{
    \begin{tikzpicture}[scale=1.6]
        \draw[line width=2.4pt] (1,0) -- (1,1);
        \draw[line width=2pt] (1,.5) -- +(.8,0);
        \draw[rarr] (1.9,.5) -- +(.7,0) node[midway, above] {$\Delta W$};
        \node at (.5,.5) {$U\uparrow$};
        \draw[draw=black!70!white] (1.2,0) -- (0,0) -- (0,1) -- (1.2,1);
    \end{tikzpicture}
}

% Water with ice cubes. Stiring heats up the water and ice cubes vanish
\newcommand{\tfigWaterStiringIceCubes}{
    \begin{tikzpicture}[water color/.style={nice light blue},wave deco/.style={decoration={bumps,amplitude=-3,segment length=17}},scale=2]
        \begin{scope}
            \fill[blue!20!white!90!black] decorate[wave deco] {(0,.8) -- (1.2,.8)} -- (1.2,0) -- (0,0) -- (0,.8);
            \draw[thick] (0,0) rectangle (1.2,1); 
            \pgfmathsetseed{5}
            \foreach \i in {1,2,...,10} {
                \coordinate (A) at (rnd+.1,rnd*.5+.1);
                \fill[color=white,rotate around={rnd*360:(A)}] (A) rectangle +(2pt,2pt);
            };
            \node[water color] at (.6,.15) {$\mathrm{H}_2\mathrm{O}$};
            \node at (.6,1.2) {A};
        \end{scope}
        \draw[arr] (1.4,.5) -- +(.4,0); 
        \begin{scope}[xshift=2cm]
            \fill[blue!20!white!90!black] decorate[wave deco] {(0,.8) -- (1.2,.8)} -- (1.2,0) -- (0,0) -- (0,.8);
            \draw[thick] (0,0) rectangle (1.2,1); 
            \foreach \i in {1,2,...,4} {
                \coordinate (A) at (rnd+.1,rnd*.5+.1);
                \fill[color=white,rotate around={rnd*360:(A)}] (A) rectangle +(2pt,2pt);
            };
            \node[water color] at (.6,.15) {$\mathrm{H}_2\mathrm{O}$};
            
            \draw (.5,.3) rectangle +(.2,.3) (.55,.3) rectangle +(.1,.3);
            \draw[thick] (.6,.6)-- +(0,.6);
            \draw[arr] (.6,1.3) arc[x radius=.2, y radius=.12,start angle=-270, end angle=90];
        \end{scope}
        \draw[arr] (3.4,.5) -- +(.4,0); 
        \begin{scope}[xshift=4cm]
            \fill[blue!20!white!90!black] decorate[wave deco] {(0,.8) -- (1.2,.8)} -- (1.2,0) -- (0,0) -- (0,.8);
            \draw[thick] (0,0) rectangle (1.2,1);
            \node[water color] at (.6,.15) {$\mathrm{H}_2\mathrm{O}$};
            \node at (.6,1.2) {B};
        \end{scope}
    \end{tikzpicture}
}

% Visualize that W,Q are no state functions. Adding different ratios of W and Q to a system may lead to the same inner energy U. 
\newcommand{\tfigWQAreNoStateFunctions}{
    \begin{tikzpicture}[scale=1.7]
        \node[point,label={north east}:$U(B)$] (Point B) at (2.2,1.4) {};
        \node[point,label={south west}:$U(A)$] (Point A) at (0,0) {} 
        edge[arr,bend left=50] node[midway, anchor=south east] {$\Delta W_1,\Delta Q_1$} node[midway, anchor=north west] {1} (Point B)
        edge[arr,bend right=60] node[midway, anchor=north west] {$\Delta W_2,\Delta Q_2$} node[midway, anchor=south east] {2} (Point B);
    \end{tikzpicture}
}

% Sketch of fundamental relations between inner energy and entropy.
\newcommand{\tfigSchemaFundamentalbeziehung}{    
    \begin{tikzpicture}[scale=1.5]
        \draw[arr] (0,0) node[anchor=north] {$0$} -- (0,2) node[ylabel] {$U$};
        \draw[arr] (0,0) -- (3,0) node[xlabel] {$S$};
        \draw[nice light blue] (0,1) .. controls +(0:1.2) and +(-135:.7) .. (2.5,1.8);
        \begin{scope}[xshift=4.5cm]
            \draw[arr] (0,0) node[anchor=north] {$0$} -- (0,2) node[ylabel] {$S$};
            \draw[arr] (0,0) -- (3,0) node[xlabel] {$U$};
            \draw[nice light blue] (1,0) .. controls +(90:.8) and +(-165:.7) .. (2.8,1.8);
        \end{scope}
    \end{tikzpicture}
}

% System with two subsystems separated by an impermeable and fixed but thermally conductive wall. 
\newcommand{\tfigDoppelsystemUSfesteWaermeleitendeWand}{
    \begin{tikzpicture}[scale=1.6]
        \draw (0,0) rectangle (2,1);
        \node[align=center] at (.5,.5) {$U^{(1)}$\\$S^{(1)}$};
        \node[align=center] at (1.5,.5) {$U^{(2)}$\\$S^{(2)}$};
        \draw[pattern=north east lines] (.95,0) rectangle (1.05,1);
        \node at (1.4,-.5) {fest, undurchlässig, isolierend $\rightarrow$ wärmeleitend} edge[arr,bend left=10] (1,0);
    \end{tikzpicture}
}

% System with two subsystems separated by an impermeable but movable and thermally conductive wall. 
\newcommand{\tfigDoppelsystemUVNbeweglicheWaermeleitendeWand}{
    \begin{tikzpicture}[scale=1.6]
        \draw (0,0) rectangle (2,1);
        \node[align=center] at (.5,.5) {$U^{(1)}$\\$V^{(1)}$\\$\sm^{(1)}$};
        \node[align=center] at (1.5,.5) {$U^{(2)}$\\$V^{(2)}$\\$\sm^{(2)}$};
        \draw[pattern=north east lines] (.95,0) rectangle (1.05,1);
        \node at (1.4,-.5) {undurchlässig, fest $\rightarrow$ beweglich, isolierend $\rightarrow$ wärmeleitend} edge[arr,bend left=10] (1,0);
    \end{tikzpicture}
}

% Function U(S) with maximum. 
\newcommand{\tfigFunktionEntropieMaximum}{
    \begin{tikzpicture}[scale=1.5]
        \draw[arr] (0,0) -- (0,2) node[ylabel] {$S$};
        \draw[arr] (0,0) -- (4,0) node[xlabel] {$U^{(1)}$};
        
        \path let \n{thermisches GG}={2}, \n{x1}={3},\n{x2}={3.5},\n{y1}={-.3*(\n{x1}-\n{thermisches GG})^2+1.5},\n{y2}={-.3*(\n{x2}-\n{thermisches GG})^2+1.5} in        
        coordinate (P1) at (\n{x1},\n{y1})
        coordinate (P2) at (\n{x2},\n{y2});
        \node[point] at (P1) {};
        \node[point] at (P2) {};
        \draw let \n{thermisches GG}={2} in plot[domain=.2:3.8] (\x,{-.3*(\x-\n{thermisches GG})^2+1.5});
        \draw[dashed] let \n{thermisches GG}={2},\p1=(P1),\p2=(P2) in 
        (\n{thermisches GG},1.5) -- +(0,-1.5) node[below,align=center] {thermisches\\Gleichgewicht}
        (0,\y1) -- (P1) -- (\x1,0) (0,\y2) -- (P2) -- (\x2,0);
        \draw[decorate, decoration = {brace}] let \p1=(P1),\p2=(P2) in
        (0,\y2)  --(0,\y1) node[midway,left]{$\Delta S$};
        \draw[decorate, decoration = {brace}] let \p1=(P1),\p2=(P2) in
        (\x2,0)  --(\x1,0) node[midway,below]{$\Delta U^{(1)}$};
        
        \draw[arr] ([shift={(.1,.1)}]P2)to[bend right=3] ([shift={(.1,.1)}]P1);
    \end{tikzpicture}
}


% System with two subsystems separated by a fixed but semipermeable and thermally conductive wall. 
\newcommand{\tfigDoppelsystemUVNbeweglicheIsolierendeWand}{
    \begin{tikzpicture}[scale=1.6]
        \draw (0,0) rectangle (2,1);
        \node[align=center] at (.5,.5) {$U^{(1)}$\\$\sm_1^{(1)}$\\$S^{(1)}$};
        \node[align=center] at (1.5,.5) {$U^{(2)}$\\$\sm_1^{(2)}$\\$S^{(2)}$};
        \draw[pattern=north east lines] (.95,0) rectangle (1.05,1);
        \node at (1.4,-.5) {fest, materieundurchlässig $\rightarrow$ durchlässig für Molekülsorte 1, isolierend $\rightarrow$ wärmeleitend} edge[arr,bend left=10] (1,0);
    \end{tikzpicture}
}


% Degeneracy function g(n,N) for N >> 1 (gaussian approximation)
\newcommand{\tfigDegeneracyFunctionGauss}{
    \begin{tikzpicture}[scale=4]
        \draw[arr] (-.7,0) -- (.7,0) node[xlabel] {$\frac{n}{N}$};
        %\draw[arr] (0,0) -- (0,1) node[ylabel] {$g(n,N)$};
        
        \draw let \n{N}={10} in plot[domain=-.5:.5,smooth,yscale=.003] (\x,{sqrt(2/(3.14159*\n{N}))*2^\n{N}*exp(-2*(\x*\n{N})^2/\n{N})}) 
        coordinate (P) at ({sqrt(1/(2*\n{N}))},{.003*sqrt(2/(3.14159*\n{N}))*2^\n{N}/exp(1)});
        \draw (-.5,.02) -- +(0,-.04) node[below] {$-\frac{1}{2}$}(.5,.02) -- +(0,-.04) node[below] {$\frac{1}{2}$};
        \draw[dashed] let \p1=(P) in (-\x1,0) -- (-\x1,\y1) -- (\x1,\y1) -- (\x1,0); 
        
        \draw[decoration={brace},decorate] let \p1=(P) in (\x1,-.02) -- +(-\x1,0) node[midway, below,yshift=-.04cm] {$\frac{n_n}{N}$};
    \end{tikzpicture}
}


% (1) System with two subsystems and wall, one filled with particles
% (2) barrier opened
\newcommand{\tfigTwoSubsystemsParticlesRemoveWall}{
    \begin{tikzpicture}[scale=1.5]
        \draw (0,0) rectangle (2, 1.3);
        \node at (1,1.5) {(1)};
        \draw (1,0) -- (1,1.3);
        \pgfmathsetseed{2}
        \foreach \n in {1,...,30}{
            \fill (.02+rnd*0.96,.02+rnd*1.26) circle[radius=.5pt];
        }
        
        \draw[arr] (2.3,1.3*2/3) -- +(1.3,0) node[midway, above] {$S$ wächst};
        \draw[rarr] (2.3,1.3/3) -- +(1.3,0) node[midway, below] {?};
        
        \begin{scope}[xshift=3.9cm]
            \node at (1,1.5) {(2)};
            \draw (0,0) rectangle (2, 1.3);
            \pgfmathsetseed{3}
            \foreach \n in {1,...,30}{
                \fill (.02+rnd*1.96,.02+rnd*1.26) circle[radius=.5pt];
            }
        \end{scope}
    \end{tikzpicture}
}


% Degeneracy function for large systems -> delta function
\newcommand{\tfigDegeneracyFunctionLargeSystemsDelta}{
    \begin{tikzpicture}[scale=4]
        \draw[arr] (-.7,0) -- (.7,0) node[xlabel] {$\frac{n_1}{N_1}$};
        \draw[arr] (0,0) -- (0,.7) node[ylabel] {$(g_1g_2)(n_1)$};
        \draw (-.5,.5pt) -- +(0,-1pt) node[below] {$-\frac{1}{2}$};
        \draw (.5,.5pt) -- +(0,-1pt) node[below] {$\frac{1}{2}$};
        \draw[nice green] (.2,.5) -- (.2,0) node[near start, right] {$\frac{\delta_n}{N_1}=10^{-11}$}
        node[below,black] {$\frac{\hat{n}_1}{N_1}$};
        \draw (.5pt,.5) -- (-.5pt,.5) node[left] {$(g_1g_2)_\mathrm{max}$};
    \end{tikzpicture}
}


\newcommand{\tfigPlaceholder}{
    \begin{tikzpicture}
        \filldraw[fill=formalshade] (0,0) rectangle (3,3) ;
        \node[black] at (1.5,1.5){\textbf{?}};
    \end{tikzpicture}
}

% Degeneracy function for large systems -> delta function
\newcommand{\tfigProcessReversibleQuasistationary}{
    \begin{tikzpicture}[scale=1.7]
        \draw[piston] (0,0) -- (0,1);
        \draw[piston bar] (0,.5) -- (1.2,.5);
        \draw[arr] (.1,.7) -- +(.4,0);
        \draw[pattern=north east lines,even odd rule] (-1.4,1.2) rectangle (1.4,-.2) node[pos=0,anchor=south west] {$\Delta S=0$} (-1.2,0) rectangle (1.2,1);
        \pgfmathsetseed{16}
        \foreach \n in {1,...,20}{
            \fill (.02-1.2+rnd*.96,.02+rnd*.96) circle[radius=.5pt];
        }
    \end{tikzpicture}
}
            
\newcommand{\tfigProcessIrreversibleQuasistationary}{
    \begin{tikzpicture}[scale=1.7]
        \foreach \x in {0,.1,...,.6}{
            \draw[line width=3pt] (\x,0) -- +(0,1);
            \draw[arr] (\x,1.2) -- +(0,.4);
        }
        \node at (.8,.5) {\ldots};
        \draw[pattern=north east lines,even odd rule] (-1.4,1.2) rectangle (1.4,-.2) node[pos=0,anchor=south west] {$\Delta S>0$} (-1.2,0) rectangle (1.2,1);
        \pgfmathsetseed{30}
        \foreach \n in {1,...,20}{
            \fill (.02-1.2+rnd*.96,.04+rnd*.90) circle[radius=.5pt];
        }
    \end{tikzpicture}
}

\newcommand{\tfigProcessIrreversibleNonquasistationary}{
    \begin{tikzpicture}[scale=1.7]
        \draw[line width=3pt] (0,0) -- +(0,1);
        \draw[arr] (0,1.2) -- +(0,.4);
        
        \draw[pattern=north east lines,even odd rule] (-1.4,1.2) rectangle (1.4,-.2) node[pos=0,anchor=south west] {$\Delta S>0$} (-1.2,0) rectangle (1.2,1);
        \pgfmathsetseed{29}
        \foreach \n in {1,...,20}{
            \fill (.02-1.2+rnd*.96,.02+rnd*.96) circle[radius=.5pt];
        }
    \end{tikzpicture}
}
\newcommand{\tfigTheoremMaximizedWork}{
    \begin{tikzpicture}
        [
            system node/.style={rectangle,draw,inner sep=10pt}
        ]
        \node[system node, fill=red!20!white] (RWQ) at (3,1) {RWQ};
        \node[system node, fill=black!10!white] (RAQ) at (3,-1) {RAQ};
        \node[point] (P) at (0,0) {};
        \draw[arr] (P) .. controls +(0:.7) and +(180:.7) .. (RAQ.west) node[pos=.7,anchor=north east] {$\Delta W^\mathrm{RAQ}$};
        \draw[arr] (P) .. controls +(0:.7) and +(180:.7) .. (RWQ.west) node[pos=.7,anchor=south east] {$\Delta Q^\mathrm{RWQ}$};
        edge[arr] (RWQ) 
        edge[arr] (RAQ);
        \node[system node,align=center] at (-3,0) {\textbf{TS}\\\\$\Delta S_{AB}^\mathrm{TS}$} edge[arr] node[midway, above] {$-\Delta U_{AB}^\mathrm{TS}$} (P);
    \end{tikzpicture}
}
\usepackage{hyperref} % include hyperref as last package!



\title{\tfigTitel}
\subtitle{Sommersemester 2022}
\author{von Kyano Levi\\bei Professor Holger Stark}
\date{}

% -----------------------------------
% -----------------------------------
% -----------------------------------

\begin{document}

\frontmatter
\maketitle
\tableofcontents

% -----------------------------------
% -----------------------------------

\mainmatter
% !TeX root = Theo_IV.tex

\chapter{Einleitung\label{einleitung}}



\section{Inhalt}

Der Inhalt dieser Vorlesung gliedert sich in zwei Teile, die Thermodynamik und die statistische Physik:

Bei der Thermodynamik geht es im Allgemeinen um Vielteilchensysteme, für die genauen mikroskopischen Positionen interessiert man sich allerdings nicht, sondern vielmehr um makroskopische Größen und Verteilungen.
Die Vielteilchensysteme werden durch den Minimalsatz von makroskopischen Variablen beschrieben, z.B. durch Energie, Volumen und Entropie.
Man spricht tatsächlich bei der Thermodynamik auch von der Lehre der Entropie.

Das Ziel ist es, allgemeine und modellunabhängige Aussagen und Prinzipien zu formulieren und häufig geht es nicht um die Berechnung von Systemparametern, sondern um die Relationen zwischen ihnen.
Damit hat die Thermodynamik einen sehr breiten Anwendungsbereich und findet Verwendung in vielen 
Wissenschaften\footnote{Beispiele: Gase, Magnetismus, Supraleitung, chemische Reaktionen, Phasenübergänge, Schwarzkörperstrahlung, Neutronensterne, schwarze Löcher, Biologie, soziologische Systeme, Ottomotor, Klima, ...}.

Die Thermodynamik stellt auch die Basis für einige Weiterentwicklungen dar, wie z.B. die Nichtgleich\-gewichts-Thermodynamik (in dieser Vorlesung wird eine reine thermostatische Beschreibung behandelt).

Die statistische Physik fasst Parameter von Vielteilchensysteme mithilfe der mikroskopischen Bewegungsgleichungen und statistischen Methoden mit makroskopischen Größen zusammen.



\begin{itemize}
	\item Herbert B. Callen: Thermodynamics and an Introduction to Thermostatistics, John Wiley \& Sons, New York 1985
	\item Gerhard Adam, Otto Hittmaier: Wärmetheorie, Vieweg, Wiesbaden 1992
	\item Charles Kittel, Herbert Kroemer: Thermal Physics, W.H. Freeman and Company, 1980
\end{itemize}

Folgende weiterführende Literatur kann zurate gezogen werden:
\begin{itemize}
	\item Arnold Sommerfeld: Theoretische Physik V: Thermodynamik und Statistik, Harri Deutsch, 1977
	\item Franz Schwabl: Statistische Mechanik, Springer, Berlin 2000
	\item Torsten Fließbach: Statistische Physik - Lehrbuch zur Theoretischen Physik IV, Springer Spektrum, Berlin 2010
	\item Wolfgang Nolting: Grundkurs Theoretische Physik 4/2: Thermodynamik, Springer Spektrum, Berlin 2016
	\item Wolfgang Nolting: Theoretische Physik 6: Statistische Physik, Springer Spektrum, Berlin 2013
	\item Mehran Kardar: Statistical Physics of Particles, Cambridge University Press, Cambridge 2007
\end{itemize}



\section{Grundlegende Konstanten der Thermodynamik}

Für Konstanten, deren Wert per Definition festgelegt wurde, die also exakt sind, wird ein $\equiv $-Zeichen verwendet.


\begin{table}[H]
	\centering
	\begin{tabular}{|l|l|} \hline
		\textbf{Konstante}       & \textbf{Wert}                                                                            \\
		\hline

		Boltzmannkonstante       & \centering\arraybackslash{} $k_\mathrm{B} \equiv \SI{1,38064852e-23}{\joule\per\kelvin}$ \\
		Universelle Gaskonstante & \centering\arraybackslash{} $R \equiv \SI{8,31446261815324}{\joule\per\kelvin\per\mole}$ \\
		Avogadro-Konstante        & \centering\arraybackslash{} $N_\mathrm{A} \equiv \SI{6,02214076e23}{\per\mole}$          \\
		Atomare Masseneinheit    & \centering\arraybackslash{} $u= \SI{1,6605390666050e-27}{\kg}$                           \\
		\hline
	\end{tabular}
\end{table}




\section{Grundlegende Formeln der Thermodynamik}

% !TeX root = Theo_IV.tex

\part{Thermodynamik}

\chapter{Grundlagen und Postulate\label{sec:grundlagen und postulate}}
\section{Zugang zur Thermodynamik}

Üblicherweise wird die Thermodynamik induktiv entwickelt. Aus Erfahrungstatsachen wie Wärme, Temperatur und Abläufen von thermodynamischen Maschinen werden Konzepte und Gesetze wie die Energieerhaltung und die Entropie abgeleitet.

In dieser Vorlesung wird stattdessen der axiomatische Zugang gewählt.
Aus Postulaten zur Energie und insbesondere zur Entropie wird die Thermodynamik aufgebaut und hergeleitet.
Anschließend werden die Konsequenzen dann mit den Erfahrungstatsachen abgeglichen.
Diese Postulate sind die Essenz der Entwicklung der Theorie und sie helfen, die Struktur der Thermodynamik sichtbar zu machen.

Ähnlich lassen sich auch andere Gebiete der Physik behandeln.
So können z.~B. aus dem Hamiltonschen Prinzip die mechanischen Bewegungsgleichungen und aus den Maxwell-Gleichungen die elektrischen und magnetischen Gesetze hergeleitet werden.


\section{Was ist Thermodynamik?\label{sec:was ist thermodynamik}}

Die meisten realen physikalischen Systeme bestehen aus einer sehr großen Zahl von Teilchen. Deren Behandlung mithilfe von mechanischen Bewegungsgleichungen ist allerdings mehr als unhandlich. 
Statt der mikroskopischen Beschreibung mit \num[print-unity-mantissa=false]{1e24} Koordinaten (Ort, Impuls, Molekülfreiheitsgrade) werden nur wenige makroskopische thermodynamische Variablen herangezogen, um ein System handhabbar zu beschreiben.

Dieser Ansatz ist auch physikalisch rechtfertigbar, denn bei realen Messungen findet automatisch eine intrinsische Mittelung statt.
Zum einen findet eine zeitliche Mittelung statt, denn die mikroskopische Bewegung findet auf Zeitskalen von \qty[print-unity-mantissa=false]{1e-15}{\s} bis \qty[print-unity-mantissa=false]{1e-12}{\s}statt, während makroskopische Messungen im Allgemeinen nicht kürzer als \qty[print-unity-mantissa=false]{1e-7}{\s} sind.
Es findet also eine Messung in gewissen Maßstäben zeitunabhängiger Kombinationen der über \num[print-unity-mantissa=false]{1e24} Koordinaten statt.

Zum anderen kommt es zu einer räumlichen Mittelung \textendash{} zum Vergleich: mikroskopische Abmessungen liegen bei $\approx\qty{0.1}{\nm}$ (Atomradius, Gitterkonstante), während makroskopische Messungen in der Regel bei über \qty{100}{\nm} liegen (Größenordnung der Wellenlänge von sichtbarem Licht).
Also wird meist über weit mehr als \num[print-unity-mantissa=false]{1e9} Atome oder Moleküle gemittelt.

\begin{figure}[htbp]
    \centering
    \tfigSystemWithManyParticles
    \caption{System mit vielen Teilchen, z.~B. ein Gas.}
    \label{fig:SystemWithManyParticles}
\end{figure}

Es verbleiben nur wenige Kenngrößen. Mechanische Größen sind zum Beispiel
\begin{itemize}
    \item Volumen $V$,
    \item Druck $P$,
    \item Oberfläche $F$,
    \item Oberflächenspannung $\sigma$ und
    \item hydrodynamische Flussfelder.
\end{itemize}

In der Elektrodynamik misst man in der Regel unter anderem
\begin{itemize}
    \item Ladung $Q$,
    \item Strom $I$,
    \item Magnetisierung $\vec M$,
    \item Magnetfeld $\vec H$,
    \item Polarisation $\vec P$ und
    \item das elektrische Feld $\vec E$.
\end{itemize}

Neu ist jetzt folgendes:
\begin{formal}
    Die Thermodynamik behandelt die makroskopischen Folgen (z.~B. die Wärme) derjenigen Koordinaten, die sich herausmitteln (z.~B. die einzelnen Teilchengeschwindigkeiten).% $\leftrightarrow$ Wärme.
\end{formal}

Die Zufuhr von Wärme in ein System führt z.~B. zur Anregung von atomarer Bewegung und damit einer Temperatur $T$.

In der Mechanik wird das Energie- bzw. Arbeitsdifferential als
\begin{align*}
    \diffa{W} = \vec F\cdot\diffa{\vec r}
\end{align*}
definiert. Hier wird diese Definition nun zu einem Produkt einer \emph{intensiven} Größe und dem Differential einer \emph{extensiven} Größe verallgemeinert:
\begin{align*}
    \diffa{E} = \underbrace{\text{verallgemeinerte Kraft}\: J}_{\text{intensiv}} \times \diffa{\underbrace{(\text{verallgemeinerter Weg}\: X)}_{\text{extensiv}}}.
\end{align*}
Dabei sind $(X,J)$ sogenannte zueinander \emph{konjugierte Variablen}. Die Einheit des Produkts $X\times J$ muss stets eine Energieeinheit sein. Wir werden im Verlauf des Kapitels den Unterschied zwischen intensiven und extensiven Größen erläutern. 

Bereits bekannte Beispiele sind
\begin{itemize}
    \item Druckarbeit $-P\diffa{V}$,
    \item Oberflächenarbeit $\sigma\diffa{F}$,
    \item Magnetisierungsarbeit $\mu_0\vec H\cdot\diffa{\vec M}$ und
    \item Polarisierungsarbeit: $\vec E\cdot\diffa{\vec P}$.
\end{itemize}
Im Verlaufe der Vorlesung wird eine neue Arbeit eingeführt, die den Wärmeübertrag und damit Energietransfer auf verborgene atomare Freiheitsgrade oder Moden beschreibt:
\begin{align*}
    \text{Energietransfer}=\text{Wärmeübertrag}=T\diffa{S}
\end{align*}
mit Temperatur $T$ und Entropie $S$.



\section{Modellsystem, Parameter und Begriffe}

Um ein Grundkonzept zu entwickeln, wird zunächst ein einfaches, idealisiertes System vorausgesetzt, das makroskopisch homogen und isotrop sowie elektrisch neutral ist, in dem keine chemischen Reaktionen ablaufen und das keine elektrischen, magnetischen oder gravitativen Felder besitzt. Auch Randeffekte werden zunächst vernachlässigt, indem angenommen wird, dass das System unendlich groß ist und damit keine Oberfläche hat.

Für dieses System werden dann Parameter wie das Volumen $V$ und die Stoffmengen der beteiligten chemischen Substanzen bestimmt.

Zunächst werden einige Definitionen erläutert:
\begin{enumerate}
    \item Als \emph{Normalbedingungen} bzw. \emph{Standardbedingungen} wird ein Zustand bei \qty{0}{\degreeCelsius} und \qty{1013}{\milli\bar} bezeichnet.
    \item \emph{Stoffmenge} (veraltet auch Molzahl) $\sm$: \qty{1}{\mole} einer Substanz entspricht einer Anzahl von Atomen oder Molekülen, die der Avogadro-Konstante/Loschmidt-Zahl $\avogadro \equiv \qty{6,02214076e23}{\per\mole}$ entspricht\footnote{
          Die Avogadro-Konstante besitzt die Einheit \si{\per\mole} während die Avogadro-Zahl \num{6,02214076e23} dimensionslos ist.
          Bei der Definition der Loschmidt-Zahl kann es aufgrund der historischen Entwicklung zu einiger Verwirrung kommen. Ursprünglich definierte Josef Loschmidt in seiner Arbeit \anf{Zur Grösse der Luftmoleküle} eine Zahl von in einer Volumeneinheit enthaltenen Luftmolekülen, war aber damals noch nicht sehr präzise mit seiner Definition. Als sogenannte Loschmidt-Konstante $N_\mathrm{L}$ wird heute der Wert \qty{2.686780111e25}{\per\cubic\m} definiert, welcher über das molare Volumen eines idealen Gases $V_{m0} = \qty{22,414}{\liter\per\mole}$ mit der Avogadro-Konstante $\avogadro=$ zusammenhängt: $N_\mathrm{L}=\avogadro/V_{m0}$. \cite{lit:loschmidt_constant,lit:loschmidt}
          Im deutschsprachigen Raum wird gelegentlich der Begriff Loschmidt-Zahl aber auch synonym mit der (dimensionslosen) Avogadro-Zahl verwendet. Um Verwirrungen zu vermeiden, wird im Folgenden ausschließlich von der Avogadro-Konstante gesprochen.}.

          Historisch wurde diese Definition gewählt, weil sie der Zahl der Atome in \qty{12}{\g} des Isotops \isotope[12]{C} von Kohlenstoff entspricht.
          Zur Referenz: \qty{1}{\mole} ist auch die Zahl der Moleküle eines idealen Gases unter Normalbedingungen in einem Volumen von $V=\qty{22,413}{\liter}$.

          Die Stoffmenge wird aus einer Teilchenzahl $N_k$ der Molekülsorte $k$ folgendermaßen berechnet:
          \begin{align*}
              \sm_k = \frac{N_k}{\avogadro}.
          \end{align*}
    \item Wir definieren ferner den \emph{Stoffmengenanteil} (früher Molenanteil genannt) als
          \begin{align*}
              x_k  = \frac{\sm_k}{\sum_i \sm_i}
          \end{align*}
          und das \emph{molare Volumen} bzw. \emph{Molvolumen} als
          \begin{align*}
              V_m = \frac{V}{\sum_i \sm_i}.
          \end{align*}
          Beide beschreiben Anteile am Gesamtsystem.
    \item Es wird unterschieden zwischen extensiven Parametern, wie dem Volumen $V$ oder Stoffmengen $\sm_1,\ldots,\sm_r$, die sich beim Zusammenführen mehrerer Teilsysteme additiv verhalten, also mit dem Volumen wachsen,
          \begin{align*}
              2\times(V,\sm_1,\ldots,\sm_r) \rightarrow (2V,2\sm_1,\ldots,2\sm_r)
          \end{align*}
          und intensiven Parametern wie $x_1,\dots,x_r,V_m$, die keine Änderung bei wachsendem Volumen erfahren. Dazu gehören auch die Temperatur $T$ und der Druck $P$.
    \item Als \emph{Zustandsgrößen} werden Größen bezeichnet, die unabhängig von der Vorgeschichte des Systems sind und einfach seinen Zustand beschreiben. Dazu gehören unter anderem das Volumen, die Stoffmenge, die Temperatur und der Druck.
\end{enumerate}




\section{Postulate zur inneren Energie und 1. Hauptsatz der Thermodynamik}

%Die Geschichte des Energiebegriffs beginnt mit Gottfried W. Leibniz im Jahr 1686. Er erkannte, dass die Wirkung einer mechanischen Bewegung 

\paragraph*{Innere Energie}

Die \emph{innere Energie} ist zugleich eine Zustandsgröße und eine extensive Größe.
\begin{formal}
    Makroskopische Systeme besitzen eine genau definierte \formalemph{innere Energie} $U$ (bezogen auf einen willkürlichen Grundzustand), die erhalten bleibt.
\end{formal}

Je nach Temperaturbereich wird ein Energienullpunkt festgelegt.
Für Energien $k_\mathrm{B}T$ im Bereich der Bindungsenergien von Molekülen können z.~B. ruhende Moleküle als Referenz gelten.


\paragraph*{Thermodynamisches Gleichgewicht}

Wir machen die Erfahrung, dass Systeme einfache Endzustände mit einer kleinstmöglichen Zahl von makroskopischen Variablen anstreben. Beispielsweise bewirkt die Reibung in einer Flüssigkeit in einem Glas, dass eine turbulente Strömung zur laminaren Strömung wird und allmählich ganz zur Ruhe kommt.

Diese Beobachtung führt uns zum ersten Postulat:
\begin{postulate}
    \label{post:gleichgewichtszustaende}
    Es gibt spezielle Zustände eines Systems, sogenannte \emph{Gleichgewichtszustände}, die makroskopisch vollkommen durch die Angabe weniger Zustandsgrößen beschrieben sind. 
    Solche Zustandsgrößen sind beispielsweise die innere Energie $U$, das Volumen $V$ und die Stoffmengen $\sm_1,\sm_2,\ldots$ der chemischen Komponenten.
\end{postulate}

In komplexeren Systemen muss z.~B. noch die Polarisation, die Magnetisierung und die Oberfläche berücksichtigt werden. Analog zum Volumen werden diese Größen über ein Arbeitsdifferential $\diffa{W} = J\diffa{X}$ in das System aufgenommen.

Im makroskopischen Gleichgewichtszustand werden viele mikroskopische Zustände im Messzeitraum angenommen, die mit dem makroskopischen Zustand $(U,V,\sm_1,...,\sm_r)$ vereinbar sind. So können z.~B. sehr viele verschiedene Kombinationen aus Teilchenenergien zur gleichen mittleren Energie führen. Normalerweise hat ein System kein \anf{Gedächtnis}, es verhält sich im Prinzip zufällig. Dies wird durch die Ergodenhypothese beschrieben und später präzisiert.

Es gibt aber auch Zustände, für die diese Annahme nicht gilt. Beim \emph{metastabilen Gleichgewicht} sind nicht alle Zustände in der Messzeit erreichbar und die Vorgeschichte ist relevant. Trotzdem ist der Formalismus der Thermodynamik in Teilbereichen davon oder für kurze Zeitspannen anwendbar.


Ein Beispiel dafür %wäre der Übergang von Glas (ungeordnetes System) zu Kristall (geordnet). 
wäre Glas als ungeordnetes, metastabiles System, das eigentlich zur Kristallisation tendiert, es aber nicht tut. Andere Beispiele sind Gedächtnislegierungen mit eingefrorenen Spannungen oder radioaktive Substanzen, die ihre Zusammensetzung durch spontanen Zerfall der Atome verändern.


\paragraph*{Wände}

Wände, Grenzen oder Ränder isolieren ein System und kontrollieren damit die Werte der Zustandsgrößen sowie den Energiefluss, wie am folgenden Beispiel ersichtlich.

Betrachte ein System aus zwei Kammern (siehe \Abbref{fig:TwoChambersSeparatedByPiston}), die durch einen beweglichen Kolben voneinander getrennt werden. Das gesamte System ist von einer festen Wand umgeben. Die Beschaffenheit des Kolbens oder der Wand kann verschiedener Art sein:
\begin{itemize}
    \item Wände können beweglich oder fest sein und kontrollieren so die innere Energie über mechanische Arbeit.
    \item (Semi-) permeable Wände kontrollieren die Stoffmengen $\sm_k$, indem sie beispielsweise nur bestimmte Stoffe hindurchlassen.
    \item Wärmeleitende Wände erlauben einen Wärmefluss, während thermisch isolierende Grenzen ein Angleichen der Temperatur verhindern. Durch Wände werden also thermisch abgeschlossene Systeme ermöglicht, was eine Definition des Wärmebegriffs erlaubt.
\end{itemize}

\begin{figure}[b]
    \centering
    \tfigTwoChambersSeparatedByPiston
    \caption{Zwei Kammern 1 und 2 sind von einer Wand umgeben und durch einen Kolben voneinander getrennt. }
    \label{fig:TwoChambersSeparatedByPiston}
\end{figure}




\paragraph*{Energiemessung}

In einem thermisch isolierten System ändert sich die innere Energie genau mit der mechanischen Arbeit $\Delta W$, die an dem System verrichtet wird,
\begin{align*}
    U(B)-U(A) =\Delta W(A\rightarrow B).
\end{align*}
Auf diese Weise kann die Änderung der inneren Energie durch die mechanische Energie definiert werden, welche wir bereits kennen und messen bzw. berechnen können.
Zum Beispiel erhöht sich die innere Energie, wenn das System mechanisch komprimiert wird (siehe \Abbref{fig:RectangularBoxWithPiston}).

\begin{figure}[t]
    \centering
    \tfigRectangularBoxWithPiston
    \caption{Ein System kann mechanisch komprimiert werden, sodass sich die innere Energie erhöht. Es wird Arbeit an dem System verrichtet. Der Prozess ist in diesem Fall umkehrbar. }
    \label{fig:RectangularBoxWithPiston}
\end{figure}

Allerdings könnte einem System, das eine Flüssigkeit enthält, auch Energie durch Quirlen zugeführt werden, wie in \Abbref{fig:WaterStiringIceCubes} dargestellt, wodurch sich die Temperatur erhöht. Dieser Prozess ist jedoch nicht umkehrbar, da die Temperatur durch Quirlen natürlich nicht wieder verringert werden kann.

\begin{figure}[htbp]
    \centering
    \tfigWaterStiringIceCubes
    \caption{Ein Gefäß mit Wasser einer bestimmten Temperatur (Zustand A) kann durch Quirlen erwärmt werden (Zustand B). Allerdings ist dieser Vorgang nicht auf mechanische Weise umkehrbar, denn durch Quirlen kann das Wasser nicht abgekühlt werden. }
    \label{fig:WaterStiringIceCubes}
\end{figure}

\begin{formal}
    Die innere Energie wird nicht direkt gemessen. Stattdessen kann die Änderung der inneren Energie eines Systems beim Übergang vom Zustand $A$ zum Zustand $B$ bestimmt werden, indem der Prozess $A\rightarrow B$ oder $B\rightarrow A$ rein mechanisch bewirkt wird.
\end{formal}



\paragraph*{Wärme und Wärmeübertrag}

Ein Prozess $A\rightarrow B$ muss nicht rein mechanisch ablaufen. Dabei nimmt das System sogenannte \emph{Wärme} auf oder gibt sie ab. Diese Wärme können wir definieren als Differenz der gesamten Änderung der inneren Energie und jenes Teils, der durch mechanische Arbeit verrichtet wird,
\begin{align*}
    \Delta Q(A\rightarrow B) = [U(B)-U(A)] - \Delta W(A\rightarrow B).
\end{align*}
Dieser Zusammenhang wird durch den 1. Hauptsatz der Thermodynamik zusammengefasst:
\begin{formal}
    \textbf{1. Hauptsatz der Thermodynamik (Energieerhaltungssatz):}
    \label{hs:erster}
    Die innere Energie eines Systems ändert sich mit der zugeführten Wärme und der am System verrichteten Arbeit\footnotemark,
    \begin{align*}
        \diff U = \udiff Q + \udiff W.
    \end{align*}
\end{formal}

\footnotetext{Die hier verwendeteten Notationen mit vollständigem Differential $\diff$ und unvolllständigem Differential $\udiff$ werden im Anschluss erläutert.}

Betrachte als Beispiel eine Volumenänderung, die aufgrund der resultierenden Druckänderung eine Arbeit am System darstellt\footnote{Diese Änderung muss ausreichend langsam erfolgen, damit es sich um eine quasistatische Prozessführung handelt. Bei einer schnellen Änderung entstehen Turbulenzen und das System befindet nicht über den ganzen Zeitraum im thermodynamischen Gleichgewicht.},
\begin{align*}
    \udiff W = -P\diff V.
\end{align*}
Anders als die innere Energie sind Wärme und Arbeit allerdings keine Zustandsgrößen, denn verschiedene Kombinationen von Wärme- und Arbeitszufuhr können auf den gleichen Zustand eines Systems führen (\Abbref{fig:WQAreNoStateFunctions}). Sie dienen beide nur der Änderung der inneren Energie.

\begin{figure}[htb]
    \centering
    \tfigWQAreNoStateFunctions
    \caption{Die mechanische Arbeit $\Delta W$ und die Wärme $\Delta Q$ sind keine Zustandsfunktionen. Einem System können verschiedene Verhältnisse $\Delta W_1,\Delta Q_1$ und $\Delta W_2,\Delta Q_2$ zugeführt werden, die zur selben inneren Energie $U$ führen, sodass sich das System im gleichen Zustand befindet. }
    \label{fig:WQAreNoStateFunctions}
\end{figure}


Bemerkungen:
\begin{itemize}
    \item Der Prozess kann \emph{quasistatisch} erfolgen, also so langsam, dass immer ein Gleichgewicht vorliegt und damit der Druck $P$ homogen ist.
    \item Der Prozess kann \emph{nicht-quasistatisch} erfolgen, wodurch es zu Turbulenzen kommt und $P=P(\vec r,t)$ ortsabhängig wird. Es wird eine Überschussarbeit verrichtet, die in Wärme dissipiert wird.
    \item Zusätzlich zum totalen Differential $\diff$ muss auch das \emph{unvollständige Differential} $\udiff$ eingeführt werden, welches für Größen (wie $Q$ und $W$) angewendet wird, die keine Zustandsfunktionen sind, sondern sogenannte wegabhängige \emph{Prozessgrößen}.
    \item Mechanische Arbeit stellt einen Energieübertrag dar.
    \item Ein quasistatischer Wärmeübertrag stellt nach $\udiff Q=\diff U+P\diff V$ ebenfalls einen Energieübertrag dar.
\end{itemize}


\paragraph*{Wärmeäquivalent}

Das sogenannte \emph{Wärmeäquivalent} besagt, dass Wärme als Energie quantifizierbar ist. Es gilt
\begin{align*}
    \Delta Q = mc\Delta T
\end{align*}
mit spezifischer Wärmekapazität $c$.
Die häufig verwendete Einheit Kalorie entspricht der Energie, um \qty{1}{\g} Wasser bei \qty{1013}{\milli\bar} von \qty{14,5}{\degreeCelsius} auf \qty{15,5}{\degreeCelsius} zu erwärmen. Es ist dabei $\qty{1}{cal} = \qty{4,1855}{\joule}$.



\section{Postulate zur Entropie}

Aus der Experimentalphysik erinnern wir uns, dass die Entropie $S$ mit der Irreversibilität eines Prozesses zusammenhängt. Später werden wir sehen, dass die Entropie durch $S=k_\mathrm{B}\ln{\Omega}$ mit dem Phasenraumvolumen $\Omega$ bestimmt ist.

Zuerst betrachten wir ein abgeschlossenes System mit festgelegten Zwangsbedingungen, beispielsweise zwei Kammern, die durch einen anfangs undurchlässigen, festen und wärmeisolierenden Kolben getrennt sind. Diese Eigenschaften des Kolbens stellen die Zwangsbedingungen dar. Es stellt sich die Frage, welchen Gleichgewichtszustand das System beim Entfernen einer oder mehrerer dieser Zwangsbedingungen einnimmt. Erfahrungsgemäß wissen wir z.~B., dass sich die Temperaturen beider Kammern angleichen, wenn ein Wärmeaustausch erlaubt wird, der Kolben also wärmedurchlässig ist.

Eine allgemeine Antwort auf diese Frage kann mithilfe des Extremalprinzips gewonnen werden, was uns auf das zweite Postulat führt:

% Das bringt uns zum zweiten Postulat, dem Extremalprinzip zur Entropie:
\begin{postulate}
    \label{post:entropie_maximierung}
    Gegeben sei ein isoliertes System, das durch Zwangsbedingungen unterteilt ist. Dann existiert eine Funktion der extensiven Parameter ($U^{(1)},V^{(1)},\sm_k^{(1)};U^{(2)},V^{(2)},\sm_k^{(2)}; \ldots$), genannt \emph{Entropie} $S$, die für alle Gleichgewichtszustände wohldefiniert ist und folgende Eigenschaften besitzt: Lässt man die Zwangsbedingungen fallen, so nehmen die extensiven Parameter Werte an, welche die Entropie maximieren. Der dann erreichte Endzustand heißt \emph{stabiles Gleichgewicht}.

    $S=S(\{U^{(\alpha)},V^{(\alpha)},\sm_k^{(\alpha)}\})$ heißt \emph{entropische Fundamentalbeziehung}. Sie enthält die gesamte Information über das System.
\end{postulate}


Es sei bemerkt, dass dafür die Entropie $S$ von den extensiven Variablen abhängen muss. Dieses Postulat erscheint auf den ersten Blick recht willkürlich, wird aber durch seine Konsequenzen gerechtfertigt, welche sich mit den Erfahrungstatsachen decken\footnote{Außerdem hat sich die Anwendung des Extremalprinzips in vielen Bereichen der Physik erfolgreich bewährt (siehe das Hamiltonsche Prinzip der kleinsten Wirkung sowie die Minimierung der Lagrange-Dichte bei der Quantenfeldtheorie).}. Diese werden in den folgenden Kapiteln erläutert.

Es ist außerdem sinnvoll, einige Eigenschaften für die Entropie zu fordern, die im nächsten Postulat zusammengefasst sind:
\begin{postulate}
    \label{post:eigenschaften_entropie}
    Die Entropie eines zusammengesetzten Systems ist gleich der Summe der Entropien der Teilsysteme,
    \begin{align*}
        S =\sum_\alpha S^{(\alpha)}, \quad S^{(\alpha)} = S^{(\alpha)}(U^{(\alpha)},V^{(\alpha)},\sm_1^{(\alpha)},\ldots, \sm_r^{(\alpha)}).
    \end{align*}
    $S$ ist stetig, differenzierbar und eine streng monoton ansteigende Funktion der inneren Energie $U$.
\end{postulate}

%Wir werden später sehen, dass aus dem zweiten Punkt die Temperatur $T$ hergeleitet werden kann. 
Hieraus folgt, dass $S$ eine extensive Größe ist, denn
\begin{align*}
    S(\lambda U,\lambda V,\lambda\sm_1,\ldots, \lambda\sm_r) = \lambda^1 S(U,V,\sm_1, \ldots,\sm_r).
\end{align*}
Genauer: die Entropie ist eine  homogene Funktion ersten Grades der extensiven Parameter.

Beispielsweise lässt sich für ein allgemeines System mit vielen Komponenten und einer Gesamtstoffmenge $\lambda=\sm=\sum_{k=1}^r\sm_k$,
\begin{align*}
    S(U,V,\sm_1, \dots , \sm_r) = N\cdot S\left(\frac{U}{N},\frac{V}{U},\frac{\sm_1}{N}, \dots , \frac{\sm_r}{N}\right)
\end{align*}
ein Einkomponentensystem ableiten,
\begin{align*}
    S(U,V,N) = N\cdot S(u,v,1) = N\cdot s(u,v),
\end{align*}
mit Energie pro Mol $u=U/N$, Molvolumen $v=V/N$ und Entropie pro Mol $s$. Diese Darstellung wird häufig verwendet, da Angaben in Mol ausreichen und die Rechnungen vereinfachen.

Aus dem Postulat \ref{post:eigenschaften_entropie} folgt ferner, dass für
\begin{align*}
    \left(\frac{\partial S}{\partial U}\right)_{V,\sm_1, \dots ,\sm_r} > 0
\end{align*}
die Beziehung $S(U)$ invertierbar ist,
\begin{align}
    \label{eq:energetische fundamentalbeziehung}
    U = U(S,V,\sm_1, \dots,\sm_r).
\end{align}
Diese neue Beziehung ist als \emph{energetische Fundamentalbeziehung} bekannt und wir werden sehen, dass aus dem Postulat \ref{post:entropie_maximierung} auch eine Minimierung der Energie folgt.



\section{Das Nernst-Postulat (3. Hauptsatz)\label{sec:Nernst}}



\begin{postulate}[Nernst-Postulat]
    \label{post:nernst}
    Für jeden Variablensatz $V,\sm_1,...,\sm_k$ gibt es einen Punkt, an dem gilt:
    \begin{align}
        \label{eq:nernst_postulat}
        S = 0 \quad \text{bei} \quad T =\left(\frac{\partial U}{\partial S}\right)_{V,\sm,...}=0
    \end{align}
    (siehe \Abbref{fig:SchemaFundamentalbeziehung}). Dieser Nullpunkt wird in der Realität jedoch nicht erreicht. 
\end{postulate}

\begin{figure}[H]
    \centering
    \tfigSchemaFundamentalbeziehung
    \caption{Schema zur energetischen (links) und entropischen (rechts) Fundamentalbeziehung. Die Entropie $S$ besitzt im Gegensatz zur inneren Energie einen eindeutigen Nullpunkt.  }
    \label{fig:SchemaFundamentalbeziehung}
\end{figure}

Bemerkungen:
\begin{itemize}
    \item Die Entropie $S$ besitzt einen eindeutigen Nullpunkt, im Gegensatz zu der inneren Energie $U$ (zu der immer eine Konstante addiert werden kann).
    \item Die Größe $T$ wird im Folgenden als Temperatur interpretiert.
    \item Aus dem Postulat \ref{post:eigenschaften_entropie} folgt, dass die Temperatur immer größer oder gleich null ist. Zudem lässt Postulat \ref{post:nernst} schließen, dass es eine absolute Temperaturskala (die Kelvinskala) gibt.
    In der Formulierung nach Planck von 1907 heißt es, dass der Temperaturnullpunkt $T=0$ nicht erreicht wird.
\end{itemize}




\begin{summary}
    \formalemph{Was zeichnet die Thermodynamik aus?}
    Statt unzähliger, einzelner mikroskopischen Größen (wie Koordinaten und Impulse), werden in der Thermodynamik makroskopische Kenngrößen betrachtet. Dies entspricht sowohl einer räumlichen als auch einer zeitlichen Mittelung über die mikroskopischen Größen. Zu den makroskopischen Kenngrößen gehören beispielsweise Volumen, Druck, Temperatur und Energie.
    
    Mit der \emph{Standard- oder Normalbedingung} wird die Prozessbedingung von $1013$ \si{\milli \bar} und $0$\si{\celsius} bezeichnet.

    \formalemph{Wesentliche Größen und Begriffe}
    Wir definieren die 
    \begin{itemize}
        \item \emph{Stoffmenge} als $\sm = N/N_A$, also die Teilchenzahl $N$ geteilt durch die \emph{Avogadro-Konstante}, 
        \item den \emph{Stoffmengenanteil} als $x_k=\sm_k/\sum_i \sm_i$,
        \item das \emph{Molvolumen} als $V_m = V/\sum_i \sm_i$ (alternativ auch mit $v$ gekennzeichnet) und
        \item das Wärmeequivalent $\Delta Q= mc\Delta T$.
    \end{itemize}

    \emph{Zustandsgrößen} beschreiben den Zustand eines Systems, unabhängig von dessen Vorgeschichte (z.~B. $V$, $\sm$, $T$, $P$). 
    Einige Größen sind keine Zustandsgrößen, wie z.~B. die Wärme $Q$ und die Arbeit $W$, denn sie beschreiben keinen eindeutigen Zustand des Systems. Für diese sogenannten \emph{Prozessgrößen} wir das \emph{unvollständige Differential} $\delta$ verwendet. 

    \emph{Extensive} Größen verhalten sich additiv mit dem Volumen des Systems. Dazu gehört neben dem Volumen z.~B. auch die Stoffmenge $\sm$.\\ 
    \emph{Intensive} Größen erfahren keine Änderung mit dem Volumen, z.~B. $T$, $P$, $x$ und $V_m$.
    Eine weitere extensive Kenngröße ist die innere Energie $U$, welche durch die entropische Fundamentalbeziehung gegeben ist; sie ist ebenfalls eine Zustandsgröße. 

    Die \emph{quasistatische} Prozessführung beschreibt eine langsame Prozessführung, welche dazu führt, dass das System sich zu jedem Zeitpunkt im Gleichgewichtszustand befindet. \\
    \emph{Nicht-quasistatische} Prozesse, sind Prozesse, welche bedingt durch eine schnelle Prozessführung Störungen (wie z.~B. Turbulenzen) erfahren und damit Wärme dissipieren. Dies führt dazu, dass sich das System nicht durchgehend in Gleichgewichtszuständen befindet. 

    \formalemph{Postulate und Hauptsätze der Thermodynamik}
    Wir haben ferner eine Reihe von Postulaten aufgestellt, die grob durch Erfahrungstatsachen motiviert sind. Nach dem \formalemph{Postulat~\ref{post:gleichgewichtszustaende}} hat ein System sogenannte Gleichgewichtszustände, welche durch wenige Zustandsgrößen ~($U$, $V$, $\sm$, \ldots) beschrieben werden können. 
    Innerhalb dieser makroskopischen Beschreibung verhält sich das System mikroskopisch zufällig. Im Allgemeinen führen viele mikroskopische Zustände zum gleichen makroskopischen Zustand. 

    Nach dem \formalemph{Postulat~\ref{post:entropie_maximierung}} existiert eine weitere Zustandsfunktion $S$ (welche von den extensiven Größen abhängt, die gesamte Information des Systems enthält und durch die entropische Fundamentalbeziehung beschrieben wird), die wir Entropie nennen. Entfernt man Zwangsbedingungen eines Systems, so nehmen dessen extensiven Größen in Folge die Werte an, welche die Entropie maximieren. 

    Die Entropie ist nach \formalemph{Postulat~\ref{post:eigenschaften_entropie}} eine additive Größe und soll stetig, differenzierbar und eine streng monoton ansteigende Funktion der inneren Energie $U$ sein. Damit gilt auch, dass $S$ eine extensive Größe ist. 

    Zuletzt beschreibt das \formalemph{Nernst-Postulat}, der \formalemph{dritte Hauptsatz der Thermodynamik} (Postulat~\ref{post:nernst}), einen Entropie- und Temperaturnullpunkt $S=0$ bei $T=0$, der aber praktisch nicht erreicht werden kann. 

    Außerdem haben wir den \formalemph{ersten Hauptsatz der Thermodynamik} kennengelernt: den \emph{Energieerhaltungssatz}. Er besagt, dass sich die innere Energie eines abgeschlossenen Systems nach der Gleichung 
    \begin{align*}
        \diff U = \udiff Q + \udiff W
    \end{align*}
    mit der zu- oder abgeführten Wärme und der am System verrichteten Arbeit ändert. 
\end{summary}
% !TeX root = Theo_IV.tex

\chapter{Folgerungen und Gleichgewichtsbedingungen}

In diesem Kapitel soll die Entwicklung der thermodynamischen Theorie sowie die Auswertung des zweiten Postulates erfolgen.

\section{Konjugierte Variablen (Energiedarstellung)}

Zunächst wollen wir die innere Energie $U$ in der differentiellen Form aufschreiben:
\begin{align*}
    \diff U = \left(\frac{\partial U}{\partial S}\right)_{V,\sm_1,\ldots,\sm_r} \diff S + \left(\frac{\partial U}{\partial V}\right)_{S,\sm_1,\ldots,\sm_r} \diff V + \sum_{j = 1}^r \left(\frac{\partial U}{\partial \sm_j}\right)_{S,V,\sm_1,\ldots,\sm_{j - 1},\sm_{j + 1},\ldots,\sm_r} \diff \sm_j.
\end{align*}
Die Faktoren vor den einzelnen Differentialen deuten wir nun, indem wir sie als Größen definieren, welche jeweils zur Differentialgröße konjugiert sind.

Wir erhalten somit die Temperatur $T$
\begin{align}
    \label{eq:def_temperatur}
    T \equiv \left(\frac{\partial U}{\partial S}\right)_{V,\sm_1,\ldots,\sm_r} \geq 0
\end{align}
als konjugierte Größe zur Entropie $S$, den Druck $P$
\begin{align}
    \label{eq:def_druck}
    P \equiv -\left(\frac{\partial U}{\partial V}\right)_{S,\sm_1,\ldots,\sm_r}
\end{align}
als konjugierte Größe zum Volumen $V$ und die chemischen Potentiale $\mu_j$
\begin{align}
    \label{eq:def_chemische_potentiale}
    \mu_j \equiv \sum_{j = 1}^r \left(\frac{\partial U}{\partial \sm_j}\right)_{S,V,\sm_1,\ldots,\sm_{j - 1},\sm_{j + 1},\ldots,\sm_r},
\end{align}
welche jeweils zu $\sm_j$ konjugiert sind.

Die innere Energie lässt sich also in differentieller Form folgendermaßen schreiben:
\begin{align}
    \label{eq:innere_energie_differentielle_form}
    \diff U = T\diff S - P \diff V + \mu_1 \diff \sm_1 + \cdots + \mu_r \diff \sm_r.
\end{align}

Einige Bemerkungen:
\begin{itemize}
    \item Die Größen $U, S,V,\sm_k$ sind extensiv, während die dazu konjugierten Variablen $T,P,\mu_k$ intensive Größen sind.
    \item $J\diff X$ mit intensiver Variable $J$ und extensiver Variable $X$ muss die Einheit einer Energie haben. Folglich ist $[\mu_j] = [U]$ und $[S]=[U/T]$.
\end{itemize}

Es soll nun eine erste Interpretation der Terme in \eqref{eq:innere_energie_differentielle_form} gemacht werden.
Zur Vereinfachung setzen wir zunächst alle Terme $\diff \sm_k=0$.
Der erste Term ist
\begin{align*}
    T\diff S = \diff U+ P\diff V = \udiff Q,
\end{align*}
entspricht also der quasistatischen Wärmezufuhr $\udiff Q$.

Der zweite Term $-P\diff V$ entspricht der bereits eingeführten quasistatischen mechanischen Arbeit $\udiff W_\mathrm{mech}=-P\diff V$.


Die Deutung als Temperatur wird später weiter ausgeführt (siehe Kapitel \ref{sec:thermisches_gleichgewicht})

Zuletzt können wir die Terme
\begin{align*}
    \udiff W_ \mathrm{C} = \sum_{i = 1}^r \mu_i\diff \sm_i
\end{align*}
als quasistatische chemische Arbeit festlegen. Sie beschreibt die Energiezunahme bei Hinzufügen von Materie zu einem System (siehe auch Kapitel \ref{sec:gleichgewicht_bei_materiefluss}).

Insgesamt kann die innere Energie also geschrieben werden als
\begin{align}
    \label{eq:erster_HS_TD}
    \diff U = \udiff Q + \udiff W_\mathrm{mech} + \udiff W_\mathrm{C}.
\end{align}
Dies entspricht dem ersten Hauptsatz der Thermodynamik.
Dabei ist zu beachten, dass die Prozessgrößen ($Q$,$W_{mech}$ und $W_c$) unvollständige Differentiale bilden. Letztere sollen im nächsten Kapitel näher erläutert werden.

Zuletzt wollen wir noch einige Aussagen zur Wärmezufuhr $\udiff Q$ zusammenfassen:
\begin{itemize}
    \item Für quasistatische Prozesse gilt $\udiff Q=T\diff S$ und damit insbesondere $\udiff Q>0\implication T\diff S>0$. Eine Zufuhr von Wärme führt also zu einem Zuwachs der Entropie.
    \item Durch Umstellen erhält man
          \begin{align*}
              \diff S = \frac{1}{T}\udiff Q.
          \end{align*}
          Da das unvollständige Differential $\udiff Q$ mittels des Faktors $1/T$ das vollständige Differential $\diff S$ charakterisiert, nennt man diesen Faktor einen integrierenden Faktor.
    \item Betrachten wir die Wärmezufuhr für nicht quasistatische Prozesse, so heißt das, dass das System keine Gleichgewichtszustände durchläuft, sondern sich im Nichtgleichgewicht entwickelt. Für ein abgeschlossenes System liegt keine Wärmezufuhr vor. Mit $\udiff Q=0$ folgt nach Postulat \ref{post:entropie_maximierung} $\diff S >0$. Die Entropie nimmt für sogenannte irreversible Prozesse zu.\footnote{Irreversibel heißt hier, dass das System nicht mehr spontan in den Ausgangszustand übergeht, also die Entropie nicht abnimmt.} In nicht abgeschlossenen Systemen, also in Systemen, in denen eine Wärmezufuhr stattfinden kann, nimmt die Entropie für irreversible Prozesse entsprechend
          \begin{align*}
              \diff S\geq \frac{\udiff Q}{T}
          \end{align*}
          zu.
          Gleichheit gilt dabei ausschließlich für reversible Prozesse. Es sei angemerkt, dass damit ferner folgt, dass die Entropiezunahme auch auf Anteile zurückzuführen ist, die nicht mit der Wärmezufuhr $\udiff Q$ zusammenhängen.
\end{itemize}

\section{Unvollständige Differentiale}

Wir wollen nun näher auf die mehrfach erwähnten \emph{unvollständigen Differentiale} zurückkommen und eine mathematische Definition nachliefern. Ausgehend von einer Funktion $f(x_1,\ldots,x_r)$ bilden wir das Differential
\begin{align*}
    \diff f = \sum_{i=1}^r \frac{\partial f}{\partial x_i} \diff x_i.
\end{align*}
Für die zweiten Ableitungen gilt für die gemischten Differentiale mit dem \emph{Satz von Schwarz} Vertauschbarkeit der Form:
\begin{align*}
    \frac{\partial ^2 f}{\partial x_i \partial x_j} = \frac{\partial ^2 f}{\partial x_j \partial x_i}.
\end{align*}
Wir konstruieren eine Darstellung $\phi$, auch \emph{Pfaffsche} oder \emph{1-Form} genannt
\begin{align*}
    \phi (x_1,\ldots,x_r)=\sum _{i=1}^r \phi_i(x_1,\ldots,x_r)\diff x_i
\end{align*}
mit dem Differential
\begin{align*}
    \diff f = \phi = \sum_i \phi_i \diff x_i \quad\text{ mit }\quad \phi_i = \frac{\partial f}{\partial x_i}.
\end{align*}
Für einfach zusammenhängende Definitionsbereiche\footnote{Es sei angemerkt, dass für nicht zusammenhängende Gebiete die Integrabilitätsbedingung nur eine notwendige, keine hinreichende Bedingung ist.} ist die Erfüllung des Satzes von Schwarz in Form folgender \emph{Integrabilitätsbedingung}
\begin{align*}
    \frac{\partial \phi_i}{\partial x_j}=\left[\frac{\partial^2f}{\partial x_j\partial x_i}=\frac{\partial^2f}{\partial x_i\partial x_j}\right]=\frac{\partial \phi_j}{\partial x_i}
\end{align*}
hinreichend dafür, dass dieses Differential vollständig ist.
Wir wollen daran erinnern, dass uns diese Integrabilitätsbedingung bereits aus der Mechanik bekannt ist. Dort ist ein betrachtetes Kraftintegral genau dann wegunabhängig, wenn die Kraft als negativer Gradient eines Potentials geschrieben werden kann. In einfach zusammenhängenden Gebieten ist dies gleichbedeutend mit der Aussage, dass die Rotation der Kraft gleich null ist,
\begin{align*}
    \vec{K}=-\nabla U \equivalence \rot\vec{K}=0 \equivalence \epsilon_{ijk}\partial_jK_k.
\end{align*}
Der letzte Ausdruck ist äquivalent zu unserer Integrabilitätsbedingung.
Erfüllt das Differential diese Bedingung nicht, ist es unvollständig und wird mittels der Notation "$\udiff$" gekennzeichnet.
Wir wollen dies am Beispiel der Variablen $S$ und $V$ illustrieren:
\begin{align*}
    \phi = \udiff Q = T\diff S+0 \diff V\equiv \phi_1\diff S+\phi_2 \diff V.
\end{align*}
Wir identifizieren die zwei Funktionen $\phi_1$ und $\phi_2$, prüfen die Integrabilitätsbedingung
\begin{align*}
    \frac{\partial \phi_1}{\partial V} = \frac{\partial T}{\partial V}\neq \frac{\partial \phi_2}{\partial S}=\frac{\partial0}{\partial S}=0
\end{align*}
und stellen fest, dass diese nicht erfüllt ist. Die Wärmezustandsfunktion $Q(S,V)$ existiert nicht, stattdessen definiert $Q$ als Prozessgröße ein unvollständiges Differential.

\section{Zustandsgleichungen (in Energiedarstellung)}
Wir möchten nun weitere Definitionen einführen und den Begriff der \emph{Zustandsgleichung} erläutern. Dafür erinnern wir an das Energiedifferential der Form
\begin{align*}
    \diff U = T\diff S -P\diff V + \sum\mu_j\diff \sm_j,
\end{align*}
welches von den intensiven Variablen $T$,$P$ und $\mu_j$ mit ihren respektiven Zustandsgleichungen
\begin{align*}
    T     & =T(S,V,\sm_1,\ldots,\sm_r)     \\
    P     & =P(S,V,\sm_1,\ldots,\sm_r)     \\
    \mu_j & =\mu_j(S,V,\sm_1,\ldots,\sm_r)
\end{align*}
abhängt. Die Kenntnis dieser Zustandsgleichungen ist äquivalent zur Kenntnis der einzelnen Zustandsfunktion $U(S,V,\sm_1,\ldots,\sm_r)$, da wir die innere Energie über Aufintegration des Differentials erhalten.

Kommen wir erneut auf die Definition der \emph{extensiven} und \emph{intensiven} Größen zurück. Eine extensive Variable, wie z.B. die innere Energie, ist eine homogene Funktion ersten Grades, d.h. es gilt folgender Zusammenhang:
\begin{align*}
    U(\lambda S,\lambda V,\lambda \sm_1,\ldots,\lambda \sm_r) = \lambda U(S,V,\sm_1,\ldots,\sm_r).
\end{align*}
In anderen Worten, "die Variable skaliert mit der Größe des Systems".
Für intensive Variablen, wie z.B. die Temperatur, gilt jedoch
\begin{align*}
    T(\lambda S,\lambda V,\lambda \sm_1,\ldots,\lambda \sm_r)=T(S,V,\sm_1,\ldots,\sm_r),
\end{align*}
denn mit der Extensivität von $U$ folgt
\begin{align*}
    T(\lambda S,\ldots) = \frac{\partial U(\lambda S,\ldots)}{\partial(\lambda S)}=\frac{\partial U(S,\ldots)}{\partial S}=T(S,\ldots).
\end{align*}
Es handelt sich dabei, wie auch beim Druck und beim chemischen Potential, um eine homogene Funktion nullten Grades, die sich bei Skalierung des Systems nicht ändert.

Wir wollen nun zum nächsten thematischen Abschnitt übergehen und uns zwei Verallgemeinerungen anschauen, in entropischer und energetischer Darstellung.
Die Energiedarstellung kann wie folgt geschrieben werden:
\begin{align*}
    U(S,V,\sm_1,\ldots,\sm_r) \rightarrow U(S,X_1,X_2,\ldots,X_t).
\end{align*}
Wir benutzen also verallgemeinerte energetisch extensive Parameter, um die Fundamentalgleichung zu schreiben. Dabei erhalten wir die verallgemeinerten Ableitungen 
\begin{align*}
    \left(\frac{\partial U}{\partial S}\right)_{X_1,\ldots,X_t}                & \equiv T = T(S,X_1,\ldots,X_t)             \\
    \left(\frac{\partial U}{\partial X_j}\right)_{S,\ldots,X_{k\neq j},\ldots} & \equiv P_j = P_j(S,X_1,\ldots,X_t)         \\
\end{align*}
und das Energiedifferential
\begin{align}
    \label{eq:energetische_fundamentalgleichung}
    \diff U  & = T\diff S + \sum_{j=1}^tP_j\diff X_j.
\end{align}
Diese Abstraktion ermöglicht die Beschreibung viel allgemeinerer Systeme, für welche weiterhin das eingeführte Energie- bzw. Arbeitsdifferential $\diff U = P_j\diff X_j$ gilt.
Dabei bezeichnen $X_j$ die verallgemeinerten Volumina oder Wege und $P_j$ die Drücke oder Kräfte.
Es sei angemerkt, dass wir den Druck folglich nur über eine Energiefunktion, welche von S abhängt, erhalten können.

Betrachtet man ein einkomponentiges System, so liegt die Einführung neuer Größen, wie der \emph{molaren inneren Energie} $u$, nahe. Wir führen diese im Folgenden ein:
\begin{align*}
    u=\frac{U}{\sm}=u\left(\frac{S}{\sm},\frac{V}{\sm},\frac{\sm}{\sm}\right)\equiv u(s,v).
\end{align*}
Sie hängt folglich von der molaren Entropie $s$ und dem Molvolumen $v$ ab.
Wir wollen nun das Differential dieser Größe berechnen. Formal schreiben wir
\begin{align*}
    \diff u = \left(\frac{\partial u}{\partial s}\right)_v \diff s + \left(\frac{\partial u}{\partial v}\right)_s\diff v
\end{align*}
mit
\begin{align*}
    \left( \frac{\partial u}{\partial s} \right)_v=\left( \frac{\partial u}{\partial s} \right)_{V,\sm}&=\left( \frac{\partial U}{\partial S} \right)_{V,N} =T                                                             
\end{align*}
und 
\begin{align*}
    \left(\frac{\partial u}{\partial v}\right)_s   & = -P.       
\end{align*}
Das Konstanthalten des Molvolumens ist dabei äquivalent zum Konstanthalten des Volumens und der Molzahl (Stoffmenge). Die partielle Ableitung nach $s$ kann ferner nach $S$ umgeschrieben werden, womit wir die Temperatur $T$ erhalten.
Analog betrachten wir die partielle Ableitung nach $v$ und erhalten den negativen Druck $P$.
Insgesamt folgt das Differential
\begin{align*}
    \diff u = T\diff s -P\diff v.
\end{align*}
Bei einem einkomponentigem System reicht also die Kenntnis über die molaren Größen, um das gesamte System zu beschreiben.

\section{Entropiedarstellung}
Analog zur Energiedarstellung betrachten wir nun eine verallgemeinerte Entropiedarstellung,
\begin{align*}
    S(U,V,\sm_1,\ldots,N_t) \rightarrow S(X_0,X_1,\ldots,X_t).
\end{align*}
Dabei benutzen wir wieder verallgemeinerte, diesmal entropische, extensive Parameter, um die entropische Fundamentalgleichung aufzuschreiben. Analog zur vorigen energetischen Beschreibung erhalten wir das Differential der Form
\begin{align*}
    \diff S = \sum_{j=0}^tF_j\diff X_j ,\quad\text{ mit } \quad F_j\equiv\left( \frac{\partial S}{\partial X_j} \right)_{\ldots X_{k\neq j}}
\end{align*}
mit den entropischen, intensiven Parametern
\begin{align*}
    F_0 & =\frac{1}{T}(U,X_1,\ldots,X_t)                                    \\
    F_1 & =\frac{P}{T}(U,X_1,\ldots,X_t) \quad\text{ (für $X_1=V$)}         \\
    F_k & =-\frac{P_k}{T}(U,X_1,\ldots,X_t)                                 \\
    F_r & =-\frac{\mu_r}{T}(U,X_1,\ldots,X_t) \quad\text{ (für $X_r=\sm_r$)}.
\end{align*}
Die energetische und entropische Beschreibung der Thermodynamik sind äquivalent, jedoch werden wir bei Gleichgewichtsbetrachtungen, die aus dem zweiten Postulat folgen, mit der entropischen Fundamentalbeziehung arbeiten.

\section{Thermisches Gleichgewicht\label{sec:thermisches_gleichgewicht}}

Nun soll es darum gehen, den Inhalt der Postulate \ref{post:entropie_maximierung} und \ref{post:eigenschaften_entropie} auszuwerten.
Es wird folgen, dass sich $T$ so verhält, wie man es von einer Temperatur erwartet.

\paragraph*{Temperatur}

Wir starten wieder mit einem Modellsystem, das insgesamt abgeschlossen ist und aus zwei Untersystemen $(1)$ und $(2)$ besteht, wie in \Abbref{fig:DoppelsystemUSfesteWaermeleitendeWand} darstellt.
Die beiden Untersysteme sind durch eine feste Wand getrennt, die zuerst isoliert ist und dann wärmeleitend wird. Die beiden Systeme werden also in thermischen Kontakt gebracht und tauschen Wärme aus.

\begin{figure}[htbp]
    \centering
    \tfigDoppelsystemUSfesteWaermeleitendeWand
    \caption{Abgeschlossenes System aus zwei Untersystemen, die durch eine feste und materieundurchlässige Wand getrennt sind. Die Wand ist zunächst isolierend und wird dann wärmeleitend.}
    \label{fig:DoppelsystemUSfesteWaermeleitendeWand}
\end{figure}

Wir würden dabei erwarten, dass sich die Temperaturen angleichen, $T^{(1)}=T^{(2)}$.

Nach den Postulaten muss für ein abgeschlossenes System $U^{(1)}+U^{(2)} = \mathrm{const}$ bzw. $\diff U^{(1)} = -\diff U^{(2)}$ sein. Das Postulat \ref{post:entropie_maximierung} besagt jetzt, dass sich $U^{(1)}$ und $U^{(2)}$ so einstellen, dass $S$ ein Maximum annimmt, $\diff S=0$. Damit auch das Postulat \ref{post:eigenschaften_entropie} erfüllt ist, muss gelten, dass
\begin{align*}
    S=S^{(1)}\left(U^{(1)},V^{(1)},\sm^{(1)}_k\right) + S^{(2)}\left(U^{(2)},V^{(2)},\sm^{(2)}_k\right).
\end{align*}
Es ist also
\begin{align*}
    \diff S = \frac{\partial S^{(1)}}{\partial U^{(1)}}\diff U^{(1)}+ \frac{\partial S^{(2)}}{\partial U^{(2)}}\diff U^{(2)} = \frac{1}{T^{(1)}}\diff U^{(1)} + \frac{1}{T^{(2)}}\diff U^{(2)},
\end{align*}
da die Volumina und Stoffmengen konstant sind und wegen $\diff U^{(1)} = -\diff U^{(2)}$ ist
\begin{align*}
    \diff S = \left(\frac{1}{T^{(1)}}-\frac{1}{T^{(2)}}\right)\diff U^{(1)} \overset{!}{=} 0.
\end{align*}
Im thermischen Gleichgewicht gilt folglich
\begin{align}
    \label{eq:thermisches_gg_temperatur}
    T^{(1)} = T^{(2)},
\end{align}
wie erwartet. Aus dem Postulat \ref{post:eigenschaften_entropie} folgt auch, dass die Temperatur positiv ist, denn $S$ soll eine monoton ansteigende Funktion von $U$ sein, sodass $\partial S/\partial U > 0$.

Diese Definition der Temperatur als Inverse der Ableitung der Entropie nach der inneren Energie mag zwar ein wenig abstrakt erscheinen, doch gibt es auch andere gleichbedeutende Definitionen der Temperatur, die aber nicht weniger abstrakt sind\footnote{Ein anderer Ansatz wäre, als nullten Hauptsatz die Transitivität der Temperatur zu postulieren \cite{lit:nolting1},
    \begin{align*}
        T^{(1)} = T^{(2)} \quad \text{und}\quad T^{(1)} = T^{(3)} \implication T^{(2)} = T^{(3)}.
    \end{align*}
    Eine weitere Formulierung, in der $1/T$ als integrierender Faktor festgelegt wird (sodass $\diff S=\udiff Q/T$), wurde von Kelvin und Caradathory vorgeschlagen.
    Beide Varianten sind in dem hier gewählten Zugang bereits in den Postulaten \ref{post:entropie_maximierung} und \ref{post:eigenschaften_entropie} enthalten.
}.

\begin{formal}
    Es existiert also eine absolute Temperaturskala. Eine solche ist die Kelvin-Skala, die so definiert ist, dass der Tripelpunkt, also die Koexistenz von Eis, flüssigem Wasser und Wasserdampf, bei \SI{271,16}{\kelvin} liegt.
\end{formal}

Wir haben gesehen, dass die Entropie ein Maximum annimmt. Daraus kann man schließen, dass die zweite Ableitung der Entropie dort kleiner als $0$ ist.

\paragraph*{Wärmefluss}

Wir wissen intuitiv, dass die Wärme von Bereichen hoher Temperatur zu Bereichen niedrigerer Temperatur fließt. Startet man bei einem Anfangszustand mit $T^{(2)}> T^{(1)}$ und hebt dann die Zwangsbedingung auf, kommt es zu einem (quasistatischen) Wärmefluss.
Wegen des Postulats \ref{post:entropie_maximierung} ist
\begin{align*}
    \Delta S= \left(\frac{1}{T^{(1)}}-\frac{1}{T^{(2)}}\right)\Delta U^{(1)} > 0.
\end{align*}
Da aber
\begin{align*}
    T^{(2)}> T^{(1)} \equivalence \frac{1}{T^{(1)}}-\frac{1}{T^{(2)}} < 0
\end{align*}
ist, muss $\Delta U^{(1)} <0$ sein.
\begin{formal}
    Der Wärmefluss findet erwartungsgemäß vom System höherer zum System tieferer Temperatur statt, bis sich beide Temperaturen angeglichen haben.
\end{formal}
Dann ist das Maximum der Entropie erreicht (siehe \Abbref{fig:FunktionEntropieMaximum}).

\begin{figure}[htbp]
    \centering
    \tfigFunktionEntropieMaximum
    \caption{Entropie über innere Energie: Beim thermodynamischen Gleichgewicht nimmt die Entropie ihr Maximum an. Die Änderung $\Delta U^{(1)}$ ist negativ für $\Delta S>0$. }
    \label{fig:FunktionEntropieMaximum}
\end{figure}




\section{Mechanisches Gleichgewicht}


\begin{figure}[htbp]
    \centering
    \tfigDoppelsystemUVNbeweglicheWaermeleitendeWand
    \caption{Abgeschlossenes System aus zwei Untersystemen, die durch eine materieundurchlässige Wand getrennt sind. Die Wand ist zunächst fest und isolierend und wird dann beweglich und wärmeleitend.}
    \label{fig:DoppelsystemUVNbeweglicheWaermeleitendeWand}
\end{figure}


Als Nächstes soll ein insgesamt abgeschlossenes System aus zwei Untersystemen behandelt werden, bei dem ein materieundurchlässiger Kolben zuerst fest und isolierend, dann aber beweglich und wärmeleitend ist (siehe \Abbref{fig:DoppelsystemUVNbeweglicheWaermeleitendeWand}). Da es sich um ein abgeschlossenes System konstanten Gesamtvolumens handelt, ist
\begin{align*}
    U^{(1)} + U^{(2)} & = \mathrm{const}  \\
    V^{(1)} + V^{(2)} & = \mathrm{const}.
\end{align*}
Nach den Postulaten \ref{post:entropie_maximierung} und \ref{post:eigenschaften_entropie} ist
\begin{align*}
    \diff S & = \frac{\partial S^{(1)}}{\partial U^{(1)}}\diff U^{(1)} + \frac{\partial S^{(1)}}{\partial V^{(1)}}\diff V^{(1)}+\frac{\partial S^{(2)}}{\partial U^{(2)}}\diff U^{(2)} + \frac{\partial S^{(2)}}{\partial V^{(2)}}\diff V^{(2)} \\
            & = \left(\frac{1}{T^{(1)}}-\frac{1}{T^{(2)}}\right)\diff U^{(1)} + \left(\frac{P^{(1)}}{T^{(1)}}-\frac{P^{(2)}}{T^{(2)}}\right) \diff V^{(1)} = 0.
\end{align*}
Im mechanischen Gleichgewicht gilt also
\begin{align*}
    \frac{1}{T^{(1)}} = \frac{1}{T^{(2)}} , \quad \frac{P^{(1)}}{T^{(1)}}=\frac{P^{(2)}}{T^{(2)}},
\end{align*}
bzw.
\begin{align*}
    T^{(1)} = T^{(2)}, \quad P^{(1)} = P^{(2)}.
\end{align*}
Die hier diskutierten Gleichgewichtsbedingungen mögen trivial erscheinen, doch geht es hier vorrangig um das Testen des Formalismus und dann die anschließende Anwendung auf komplexere Systeme.


\section{Gleichgewicht bei Materiefluss\label{sec:gleichgewicht_bei_materiefluss}}

\paragraph*{Chemisches Potential}

Analog zu den vorigen Beispielen wird ein abgeschlossenes System mit zwei Untersystemen betrachtet (siehe \Abbref{fig:DoppelsystemUVNbeweglicheIsolierendeWand}). Diesmal ist die Wand zwar fest, aber wärmeleitend und durchlässig für eine Molekülsorte.

\begin{figure}[htbp]
    \centering
    \tfigDoppelsystemUVNbeweglicheIsolierendeWand
    \caption{Abgeschlossenes System aus zwei Untersystemen, die durch eine feste Wand getrennt sind. Die Wand ist zunächst isolierend und undurchlässig und wird dann wärmeleitend und durchlässig für die Molekülsorte 1.}
    \label{fig:DoppelsystemUVNbeweglicheIsolierendeWand}
\end{figure}

Es gilt
\begin{align*}
    U^{(1)} + U^{(2)}         & = \mathrm{const} \\
    \sm_1^{(1)} + \sm_1^{(2)} & = \mathrm{const}
\end{align*}
und damit
\begin{align*}
    \diff S & = \frac{1}{T^{(1)}}\diff U^{(1)} - \frac{\mu_1^{(1)}}{T^{(1)}}\diff \sm^{(1)}+\frac{1}{T^{(2)}}\diff U^{(2)} - \frac{\mu_1^{(2)}}{T^{(2)}}\diff \sm^{(2)}   \\
            & = \left(\frac{1}{T^{(1)}}-\frac{1}{T^{(2)}}\right)\diff U^{(1)} + \left(\frac{\mu_1^{(1)}}{T^{(1)}}-\frac{\mu_1^{(2)}}{T^{(2)}}\right) \diff \sm^{(1)} = 0.
\end{align*}
Im Gleichgewicht gleichen sich also neben den Temperaturen die chemischen Potential durch Teilchenaustausch an,
\begin{align*}
    T^{(1)} = T^{(2)}, \quad \mu_1^{(1)} = \mu_1^{(2)}.
\end{align*}
Es findet jedoch kein Teilchenfluss statt, wenn bereits $\mu_1^{(1)} = \mu_1^{(2)}$.

\paragraph*{Materiefluss}

Beginnt man bei einem Anfangszustand mit $\mu_1^{(1)} > \mu_1^{(2)}$ und $T^{(1)} = T^{(2)}$ und hebt dann die Zwangsbedingung auf (Wand wird materiedurchlässig), so kommt es zum quasistatischen Materiefluss,
\begin{align*}
    \Delta S = \frac{\mu_1^{(2)} - \mu_1^{(1)}}{T} \Delta \sm_1^{(1)}.
\end{align*}
Da nach dem Postulat \ref{post:entropie_maximierung} die Änderung der Entropie nur positiv sein kann und $(\mu_1^{(2)} - \mu_1^{(1)})/T$ nach unserer Festlegung negativ ist, so ist auch $\Delta \sm_1^{(1)}<0$.

\begin{formal}
    Ein Materiefluss findet von Gebieten hohen zu Gebieten tiefen chemischen Potentials statt, bis $\mu_1^{(1)} = \mu_1^{(2)}$.
\end{formal}

Das chemische Potential $\mu$ ist zentral bei Phasenumwandlungen und chemischen Reaktionen (siehe später) und spielt damit eine führende Rolle in der theoretischen Chemie.



\section{Folgerungen aus der Homogenität der Fundamentalbeziehung}

Allein aus der Forderung, dass die Fundamentalbeziehung homogen ist, lassen sich einige formale Schlüsse folgern, die in diesem Kapitel erläutert werden sollen. Die erste Schlussfolgerung ist die Euler-Gleichung.

\paragraph*{Die Euler-Gleichung}

Die innere Energie $U$ ist eine extensive Größe und damit eine homogene Funktion ersten Grades (Größen wie Volumen und Stoffmengen werden verallgemeinert als $X_k$ geschrieben, um eine kompaktere Notation zu ermöglichen),
\begin{align*}
    U(\lambda S,\lambda X_1,\ldots,\lambda X_t) = \lambda U(S,X_1,\ldots,X_t).
\end{align*}
Ableiten nach $\lambda$ liefert
\begin{align*}
    \frac{\partial U}{\partial\lambda} & = \frac{\partial U}{\partial\lambda S}(\lambda S,\lambda X_1,\ldots,\lambda X_t) \frac{\partial\lambda S}{\partial\lambda} + \frac{\partial U}{\partial\lambda X_1}(\lambda S,\lambda X_1,\ldots,\lambda X_t) \frac{\partial\lambda X_1}{\partial\lambda} + \ldots \\
                                       & = T(\lambda S,\lambda X_1,\ldots,\lambda X_t)S + P_1 (\lambda S,\lambda X_1,\ldots,\lambda X_t) X_1 + \ldots                                                                                                                                                       \\
                                       & = \lambda TS + \lambda P_1X_1 + \ldots
\end{align*}
Lässt man nun $\lambda$ gegen 1 gehen, so erhält man die Euler-Gleichung in Energiedarstellung:
\begin{align}
    \label{eq:euler_gleichung_energiedarstellung}
    \boxed{U = TS + \sum_{j=1}^t P_j X_j.}
\end{align}
Analog lässt sich die Entropiedarstellung der Euler-Gleichung herleiten:
\begin{align}
    \label{eq:euler_gleichung_entropiedarstellung}
    \boxed{S = \sum_{j=0}^t F_j X_j.}
\end{align}
Für ein einfaches System nimmt sie zum Beispiel die Form (Energiedarstellung)
\begin{align*}
    U=TS-PV + \mu_1 \sm_1+\ldots + \mu_r \sm_r
\end{align*}
bzw. (Entropiedarstellung)
\begin{align*}
    S=\frac{1}{T}U + \frac{P}{T}V - \sum_{k=1}^r \frac{\mu_k}{T}\sm_k
\end{align*}
an.


\paragraph*{Gibbs-Duhem-Beziehung}

Bis jetzt haben wir globale Betrachtungen gemacht, die hauptsächlich extensive Variablen behandeln. Nun soll eine differentielle Behandlung folgen, die auch die intensiven Größen berücksichtigt.

Aus der energetischen Fundamentalbeziehung erhält man durch Differenzieren nach den Parametern $(t+1)$ Gleichungen in $(t+1)$ Variablen.
\begin{align*}
    \frac{\partial U}{\partial S} = T = T(S,X_1,\ldots ,X_t), \quad
    \frac{\partial U}{\partial X_k} =P_k = P_k(S,X_1,\ldots ,X_t)
\end{align*}
Die durch diesen Prozess gewonnenen Größen $T,P_1,\ldots,P_t$ sind intensiv, also homogen vom Grad $0$, z.B.
\begin{align*}
    T(S,X_1,\ldots ,X_t)=T(\lambda S,\lambda X_1,\ldots ,\lambda X_t).
\end{align*}
Für $\lambda=1/X_t$ ist dann
\begin{align*}
    T = T\left(\frac{S}{X_t},\frac{X_1}{X_t},\ldots ,1\right), \quad P_k=P_k\left(\frac{S}{X_t},\frac{X_1}{X_t},\ldots ,1\right).
\end{align*}
Dieses Gleichungssystem aus $(t+1)$ Gleichungen enthält jetzt nur noch $t$ Variablen, sodass eine Zustandsgleichung bei der Beziehung zwischen den insgesamt $(t+1)$ intensiven Variablen redundant ist.

Betrachte als Beispiel ein Einkomponentensystem mit $X_t=\sm$, molarer Entropie $s=S/\sm$ und molarem Volumen $v=V/\sm$. Temperatur, Druck und chemisches Potential hängen jeweils nur von der molaren Entropie und dem molaren Volumen ab,
\begin{align*}
    T=T(s,v), \quad P=P(s,v), \quad \mu=\mu(s,v).
\end{align*}
Man kann jetzt zwei dieser Zusammenhänge invertieren (z.B. $s=s(T,P)$ und $v=v(T,P)$) und in den dritten einsetzen ($\mu=\mu(T,P)$). Diese letzte Gleichung ist damit redundant für die Beschreibung des Systems.

Durch Differenzieren der Euler-Gleichung erhalten wir
\begin{align}
    U                    & =TS+ \sum_j P_jX_j                                                 \nonumber \\
    \label{eq:differential_innere_energie1}
    \implication \diff U & = T\diff S + S\diff T + \sum_j P_j\diff X_j + \sum_j X_j\diff P_j.
\end{align}
Andererseits kennen wir aus der Energiedarstellung bereits das Differential der inneren Energie (siehe Gleichung \eqref{eq:energetische_fundamentalgleichung}) als
\begin{align}
    \label{eq:differential_innere_energie2}
    \diff U = T\diff S + \sum_j P_j \diff X_j.
\end{align}
Durch Vergleich von \eqref{eq:differential_innere_energie1} und \eqref{eq:differential_innere_energie2} folgt eine differentielle Beziehung zwischen den intensiven Variablen,
\begin{align}
    \label{eq:gibbs_duhem}
    \boxed{
        S\diff T + \sum X_j \diff P_j = 0,
    }
\end{align}
was als Gibbs-Duhem-Beziehung bekannt ist.

Im Einkomponentensystem heißt das, dass
\begin{align*}
    S\diff T - V\diff P + \sm \diff \mu = 0.
\end{align*}
Teilen durch $\sm$ liefert
\begin{align*}
    \diff \mu = -s \diff T + v\diff P
\end{align*}
und Integrieren führt auf das chemische Potential $\mu=\mu(T,P)$.

Wir haben also gesehen, dass stets eine intensive Variable von den anderen abhängt. Das gibt Anlass zu der nachfolgenden Definition:
\begin{formal}
    Die Zahl der unabhängigen intensiven Variablen wird als Zahl der thermodynamischen Freiheitsgrade definiert.
\end{formal}

Die Gibbs-Duhem-Beziehung kann auch in der Entropiedarstellung formuliert werden,
\begin{align*}
    \boxed{\sum_{j=0}^t X_j \diff F_j = 0}
\end{align*}
und für ein einfaches System ist dann
\begin{align*}
    U\diff\left(\frac{1}{T}\right) + V\diff\left(\frac{P}{T}\right)-\sum_{k=1}^r \sm_k\diff\left(\frac{\mu_k}{T}\right)=0.
\end{align*}



\raggedright
\bibliography{literatur}  
\bibliographystyle{bib}

\end{document}
