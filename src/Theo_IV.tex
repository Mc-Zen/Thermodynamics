%
% -------------------------------------------------------------------
% Thermodynamics lecture at TU Berlin, read by Prof. Holger Stark
% -------------------------------------------------------------------
%

\documentclass[hidelinks, 11pt]{scrbook} 

\usepackage[left=2cm, right=2cm, bottom=1.5cm, top=1.5cm, includeheadfoot]{geometry} 
\usepackage[T1]{fontenc} 
\usepackage[utf8]{inputenc} 
\usepackage[english,ngerman]{babel}
\usepackage[font=footnotesize]{caption}
\usepackage{graphicx} 
\usepackage{multirow} 
\usepackage{tabularx} 
\usepackage{xcolor} 
\usepackage{amsmath} 
\usepackage{amssymb} 
\usepackage{amsfonts} 
\usepackage{amsxtra} 
\usepackage{mathtools} 
\usepackage{tensor}
\usepackage{enumerate} 
\usepackage{float}
\usepackage{siunitx}
\usepackage{longtable}
\usepackage{cancel}
\usepackage{isotope}
\usepackage{lmodern}
\usepackage{scrhack} 
\sisetup{locale = DE, separate-uncertainty}  
\usepackage[immediate]{silence}
\WarningFilter[temp]{latex}{Command} % filter underline/underbar command warning from sectsty 

% --------------------------
% spacing

%\renewcommand*{\familydefault}{\sfdefault} sans serif font
\raggedbottom
\setlength{\parindent}{0em}
\setlength{\parskip}{1em}

\usepackage[hang]{footmisc}
\setlength{\footnotemargin}{3mm} % space between number and content of footnote
\setlength{\skip\footins}{.5cm} % space between body and footnote section
\setlength{\footnotesep}{0.5cm}  % space between footnotes

% \renewcommand{\arraystretch}{1}
% \makeatletter
% \renewcommand\@pnumwidth{2em} % fix toc overfull hbox
% \makeatother


% --------------------------
% section styles

\newcommand{\example}[1]{\paragraph{#1:}}
% change title font to serif
\addtokomafont{title}{\rmfamily}


% --------------------------
% math

\DeclareMathOperator{\grad}{grad} 
\DeclareMathOperator{\divg}{div} 
\DeclareMathOperator{\rot}{rot} 
\DeclareMathOperator{\real}{\mathfrak{R}} 
\DeclareMathOperator{\imag}{\mathfrak{I}} 

\newcommand{\upupharpoons}{\upharpoonleft\!\upharpoonright}
\newcommand{\updownharpoons}{\upharpoonleft\!\downharpoonright}
% pm sign with minus in paretheses
\newcommand{\varpm}{\mathbin{\vcenter{\hbox{
  \oalign{\hfil$\scriptstyle+$\hfil\cr\noalign{\kern-.3ex}$\scriptscriptstyle({-})$\cr}
}}}}
% mp sign with plus in parentheses
\newcommand{\varmp}{\mathbin{\vcenter{\hbox{
  \oalign{$\scriptstyle({+})$\cr\noalign{\kern-.3ex}\hfil$\scriptscriptstyle-$\hfil\cr}
}}}}
% differential operator have a non-italic d in German equation typesetting 
\newcommand*{\diff}{\mathop{}\!\mathrm{d}} 
\newcommand*{\diffa}[2][]{\mathop{\mathrm{d}^{#1}#2}}
%\newcommand{\diff}{\text{d}}	 
\newcommand{\Angstroem}{\text{\normalfont\AA}}   
\newcommand{\Abbref}[1]{Abb.~\ref{#1}} 
% Vector: bold non-cursive symbols for vectors instead of arrows 
\renewcommand{\vec}[1]{\mathbf{#1}} 

\newcommand{\equivalence}{\;\Leftrightarrow\;}
\newcommand{\implication}{\;\Rightarrow\;}


% --------------------------
% other configurations

% set figure description
\addto\captionsngerman{
  \renewcommand{\figurename}{Abb.}
  \renewcommand{\tablename}{Tab.}
}
\graphicspath{{images/}} 
\usepackage[strict]{changepage}

% for formal definitions
\usepackage{framed}

% environment derived from framed.sty: see leftbar environment definition
\definecolor{formalbar}{rgb}{0.1,0.1,.2}
\definecolor{formalshade}{rgb}{0.95,0.95,1}

% Quote box for important statements
\newenvironment{formal}{
  \def\FrameCommand{
    \hspace{1pt}%
    {\color{formalbar}\vrule width 2pt}%
    {\color{formalshade}\vrule width 4pt}%
    \colorbox{formalshade}%
  }
  \MakeFramed{\advance\hsize-\width\FrameRestore}%
  \noindent\hspace{-4.55pt}% disable indenting first paragraph
  \begin{adjustwidth}{}{7pt}%
  \vspace{-10pt}\vspace{2pt}% 
}{\vspace{2pt}\end{adjustwidth}\endMakeFramed}

% !TeX root = Theo_IV.tex

\usepackage{tikz}
%\usepackage{pgfplots}
%\pgfplotsset{compat=1.18}

\usetikzlibrary{
    arrows.meta,
    bending,
    positioning,
    decorations.markings,
    intersections,
    calc,
    decorations.pathreplacing,
    decorations.pathmorphing,
    patterns
}
%\tikzexternalize[prefix=figures/,shell escape=-enable-write18] % activate

\tikzset{
    % Colors
    object color/.style={blue!40!black!80!white},
    object style/.style={object color,thick},
    nice green/.style={green!50!black},
    nice orange/.style={red!60!yellow!70!black!90!white},
    nice dark blue/.style={blue!50!black},
    nice light blue/.style={blue!60!white!70!black},
    nice turquoise/.style={blue!50!green},
    polarisation color/.style={purple},
    charge color/.style={blue!50!white!70!black},
    red laser/.style={red!70!black},
    moving system color/.style={blue!60!black!70!white},
    %
    % Coordinate system
    coordsystem/.style={very thin, color=#1!50},
    xlabel/.style={anchor=north west},
    ylabel/.style={anchor=south east},
    %
    % Nodes and points
    invisible point/.style={circle,inner sep=0pt,outer sep=0pt,minimum size=0pt},
    point/.style={invisible point,fill=black,minimum size=4pt},
    %
    % Arrows
    arrow tip/.tip={Stealth},
    arr/.style={->,>={arrow tip}},
    rarr/.style={<-,>={arrow tip}},
    midarrow/.style={postaction=decorate,decoration={markings, mark=at position #1 with {\arrow[xshift=2.5pt]{arrow tip}}} },
    midarrow/.default=.5,
    rmidarrow/.style={postaction=decorate,decoration={markings, mark=at position #1 with {\arrowreversed{arrow tip}}} },
    rmidarrow/.default=.5,
    distance marker/.style={|<->|,>={arrow tip}},
    %
    % Thermodynamics stuff
    piston bar/.style={line width=2pt},
    piston/.style={line width=8pt}
}
\tikzstyle{every node}=[font=\footnotesize]


\newcommand{\tfigTitel}{
    \begin{tikzpicture}
        \pgfmathsetmacro{\shadowangle}{132}
        \newlength{\shadowdistance}
        \pgfmathsetlength{\shadowdistance}{0.1ex}
        \pgfmathsetmacro{\shadowopacity}{1}
        \pgfmathsetmacro{\shadowspread}{0.003}
        \pgfmathsetmacro{\shadowsize}{5}
        \pgfmathtruncatemacro{\totshadow}{100}
        \path[nice dark blue,opacity={\shadowopacity/\totshadow},shift={({132-180}:\shadowdistance)},scale={1+\shadowsize}] 
        foreach \nshadow [evaluate=\nshadow as \angshadow using \nshadow/\totshadow*360] in {1,...,\totshadow}{
            node[align=center] at (\angshadow:\shadowspread) {\huge Theoretische Physik 4\\ \\ \\
            \Large Thermodynamik/Statistische Physik}
            };
        \node[align=center] at (0,0) {\huge Theoretische Physik 4\\ \\ \\ \Large Thermodynamik/Statistische Physik};
    \end{tikzpicture}
}

% 100 particles in a rectangular box
\newcommand{\tfigSystemWithManyParticles}{
    \begin{tikzpicture}[scale=1.5]
        \draw (-1pt,-1pt) rectangle ($(3,2)+(1pt,1pt)$);
        \pgfmathsetseed{1}
        \foreach \i in {1,2,...,100}{
            \draw[fill=black] (rnd*3,rnd*2) circle[radius=.5pt];
        };
    \end{tikzpicture}
}

% Two chambers separated by a piston
\newcommand{\tfigTwoChambersSeparatedByPiston}{
    \begin{tikzpicture}[scale=1.6]
        \draw (0,0) rectangle (2,1);
        \node at (.5,.5) {1};
        \node at (1.5,.5) {2};
        \draw[pattern=north east lines] (.95,0) rectangle (1.05,1);
        \node at (0,-.5) {Wand} edge[arr,bend right=10] (.4,0);
        \node at (1.4,-.5) {Kolben} edge[arr,bend left=10] (1,0);
    \end{tikzpicture}
}

% Rectangular box with piston
\newcommand{\tfigRectangularBoxWithPiston}{
    \begin{tikzpicture}[scale=1.6]
        \draw[line width=2.4pt] (1,0) -- (1,1);
        \draw[line width=2pt] (1,.5) -- +(.8,0);
        \draw[rarr] (1.9,.5) -- +(.7,0) node[midway, above] {$\Delta W$};
        \node at (.5,.5) {$U\uparrow$};
        \draw[draw=black!70!white] (1.2,0) -- (0,0) -- (0,1) -- (1.2,1);
    \end{tikzpicture}
}

% Water with ice cubes. Stiring heats up the water and ice cubes vanish
\newcommand{\tfigWaterStiringIceCubes}{
    \begin{tikzpicture}[water color/.style={nice light blue},wave deco/.style={decoration={bumps,amplitude=-3,segment length=17}},scale=2]
        \begin{scope}
            \fill[blue!20!white!90!black] decorate[wave deco] {(0,.8) -- (1.2,.8)} -- (1.2,0) -- (0,0) -- (0,.8);
            \draw[thick] (0,0) rectangle (1.2,1); 
            \pgfmathsetseed{5}
            \foreach \i in {1,2,...,10} {
                \coordinate (A) at (rnd+.1,rnd*.5+.1);
                \fill[color=white,rotate around={rnd*360:(A)}] (A) rectangle +(2pt,2pt);
            };
            \node[water color] at (.6,.15) {$\mathrm{H}_2\mathrm{O}$};
            \node at (.6,1.2) {A};
        \end{scope}
        \draw[arr] (1.4,.5) -- +(.4,0); 
        \begin{scope}[xshift=2cm]
            \fill[blue!20!white!90!black] decorate[wave deco] {(0,.8) -- (1.2,.8)} -- (1.2,0) -- (0,0) -- (0,.8);
            \draw[thick] (0,0) rectangle (1.2,1); 
            \foreach \i in {1,2,...,4} {
                \coordinate (A) at (rnd+.1,rnd*.5+.1);
                \fill[color=white,rotate around={rnd*360:(A)}] (A) rectangle +(2pt,2pt);
            };
            \node[water color] at (.6,.15) {$\mathrm{H}_2\mathrm{O}$};
            
            \draw (.5,.3) rectangle +(.2,.3) (.55,.3) rectangle +(.1,.3);
            \draw[thick] (.6,.6)-- +(0,.6);
            \draw[arr] (.6,1.3) arc[x radius=.2, y radius=.12,start angle=-270, end angle=90];
        \end{scope}
        \draw[arr] (3.4,.5) -- +(.4,0); 
        \begin{scope}[xshift=4cm]
            \fill[blue!20!white!90!black] decorate[wave deco] {(0,.8) -- (1.2,.8)} -- (1.2,0) -- (0,0) -- (0,.8);
            \draw[thick] (0,0) rectangle (1.2,1);
            \node[water color] at (.6,.15) {$\mathrm{H}_2\mathrm{O}$};
            \node at (.6,1.2) {B};
        \end{scope}
    \end{tikzpicture}
}

% Visualize that W,Q are no state functions. Adding different ratios of W and Q to a system may lead to the same inner energy U. 
\newcommand{\tfigWQAreNoStateFunctions}{
    \begin{tikzpicture}[scale=1.7]
        \node[point,label={north east}:$U(B)$] (Point B) at (2.2,1.4) {};
        \node[point,label={south west}:$U(A)$] (Point A) at (0,0) {} 
        edge[arr,bend left=50] node[midway, anchor=south east] {$\Delta W_1,\Delta Q_1$} node[midway, anchor=north west] {1} (Point B)
        edge[arr,bend right=60] node[midway, anchor=north west] {$\Delta W_2,\Delta Q_2$} node[midway, anchor=south east] {2} (Point B);
    \end{tikzpicture}
}

% Sketch of fundamental relations between inner energy and entropy.
\newcommand{\tfigSchemaFundamentalbeziehung}{    
    \begin{tikzpicture}[scale=1.5]
        \draw[arr] (0,0) node[anchor=north] {$0$} -- (0,2) node[ylabel] {$U$};
        \draw[arr] (0,0) -- (3,0) node[xlabel] {$S$};
        \draw[nice light blue] (0,1) .. controls +(0:1.2) and +(-135:.7) .. (2.5,1.8);
        \begin{scope}[xshift=4.5cm]
            \draw[arr] (0,0) node[anchor=north] {$0$} -- (0,2) node[ylabel] {$S$};
            \draw[arr] (0,0) -- (3,0) node[xlabel] {$U$};
            \draw[nice light blue] (1,0) .. controls +(90:.8) and +(-165:.7) .. (2.8,1.8);
        \end{scope}
    \end{tikzpicture}
}

% System with two subsystems separated by an impermeable and fixed but thermally conductive wall. 
\newcommand{\tfigDoppelsystemUSfesteWaermeleitendeWand}{
    \begin{tikzpicture}[scale=1.6]
        \draw (0,0) rectangle (2,1);
        \node[align=center] at (.5,.5) {$U^{(1)}$\\$S^{(1)}$};
        \node[align=center] at (1.5,.5) {$U^{(2)}$\\$S^{(2)}$};
        \draw[pattern=north east lines] (.95,0) rectangle (1.05,1);
        \node at (1.4,-.5) {fest, undurchlässig, isolierend $\rightarrow$ wärmeleitend} edge[arr,bend left=10] (1,0);
    \end{tikzpicture}
}

% System with two subsystems separated by an impermeable but movable and thermally conductive wall. 
\newcommand{\tfigDoppelsystemUVNbeweglicheWaermeleitendeWand}{
    \begin{tikzpicture}[scale=1.6]
        \draw (0,0) rectangle (2,1);
        \node[align=center] at (.5,.5) {$U^{(1)}$\\$V^{(1)}$\\$\sm^{(1)}$};
        \node[align=center] at (1.5,.5) {$U^{(2)}$\\$V^{(2)}$\\$\sm^{(2)}$};
        \draw[pattern=north east lines] (.95,0) rectangle (1.05,1);
        \node at (1.4,-.5) {undurchlässig, fest $\rightarrow$ beweglich, isolierend $\rightarrow$ wärmeleitend} edge[arr,bend left=10] (1,0);
    \end{tikzpicture}
}

% Function U(S) with maximum. 
\newcommand{\tfigFunktionEntropieMaximum}{
    \begin{tikzpicture}[scale=1.5]
        \draw[arr] (0,0) -- (0,2) node[ylabel] {$S$};
        \draw[arr] (0,0) -- (4,0) node[xlabel] {$U^{(1)}$};
        
        \path let \n{thermisches GG}={2}, \n{x1}={3},\n{x2}={3.5},\n{y1}={-.3*(\n{x1}-\n{thermisches GG})^2+1.5},\n{y2}={-.3*(\n{x2}-\n{thermisches GG})^2+1.5} in        
        coordinate (P1) at (\n{x1},\n{y1})
        coordinate (P2) at (\n{x2},\n{y2});
        \node[point] at (P1) {};
        \node[point] at (P2) {};
        \draw let \n{thermisches GG}={2} in plot[domain=.2:3.8] (\x,{-.3*(\x-\n{thermisches GG})^2+1.5});
        \draw[dashed] let \n{thermisches GG}={2},\p1=(P1),\p2=(P2) in 
        (\n{thermisches GG},1.5) -- +(0,-1.5) node[below,align=center] {thermisches\\Gleichgewicht}
        (0,\y1) -- (P1) -- (\x1,0) (0,\y2) -- (P2) -- (\x2,0);
        \draw[decorate, decoration = {brace}] let \p1=(P1),\p2=(P2) in
        (0,\y2)  --(0,\y1) node[midway,left]{$\Delta S$};
        \draw[decorate, decoration = {brace}] let \p1=(P1),\p2=(P2) in
        (\x2,0)  --(\x1,0) node[midway,below]{$\Delta U^{(1)}$};
        
        \draw[arr] ([shift={(.1,.1)}]P2)to[bend right=3] ([shift={(.1,.1)}]P1);
    \end{tikzpicture}
}


% System with two subsystems separated by a fixed but semipermeable and thermally conductive wall. 
\newcommand{\tfigDoppelsystemUVNbeweglicheIsolierendeWand}{
    \begin{tikzpicture}[scale=1.6]
        \draw (0,0) rectangle (2,1);
        \node[align=center] at (.5,.5) {$U^{(1)}$\\$\sm_1^{(1)}$\\$S^{(1)}$};
        \node[align=center] at (1.5,.5) {$U^{(2)}$\\$\sm_1^{(2)}$\\$S^{(2)}$};
        \draw[pattern=north east lines] (.95,0) rectangle (1.05,1);
        \node at (1.4,-.5) {fest, materieundurchlässig $\rightarrow$ durchlässig für Molekülsorte 1, isolierend $\rightarrow$ wärmeleitend} edge[arr,bend left=10] (1,0);
    \end{tikzpicture}
}


% Degeneracy function g(n,N) for N >> 1 (gaussian approximation)
\newcommand{\tfigDegeneracyFunctionGauss}{
    \begin{tikzpicture}[scale=4]
        \draw[arr] (-.7,0) -- (.7,0) node[xlabel] {$\frac{n}{N}$};
        %\draw[arr] (0,0) -- (0,1) node[ylabel] {$g(n,N)$};
        
        \draw let \n{N}={10} in plot[domain=-.5:.5,smooth,yscale=.003] (\x,{sqrt(2/(3.14159*\n{N}))*2^\n{N}*exp(-2*(\x*\n{N})^2/\n{N})}) 
        coordinate (P) at ({sqrt(1/(2*\n{N}))},{.003*sqrt(2/(3.14159*\n{N}))*2^\n{N}/exp(1)});
        \draw (-.5,.02) -- +(0,-.04) node[below] {$-\frac{1}{2}$}(.5,.02) -- +(0,-.04) node[below] {$\frac{1}{2}$};
        \draw[dashed] let \p1=(P) in (-\x1,0) -- (-\x1,\y1) -- (\x1,\y1) -- (\x1,0); 
        
        \draw[decoration={brace},decorate] let \p1=(P) in (\x1,-.02) -- +(-\x1,0) node[midway, below,yshift=-.04cm] {$\frac{n_n}{N}$};
    \end{tikzpicture}
}


% (1) System with two subsystems and wall, one filled with particles
% (2) barrier opened
\newcommand{\tfigTwoSubsystemsParticlesRemoveWall}{
    \begin{tikzpicture}[scale=1.5]
        \draw (0,0) rectangle (2, 1.3);
        \node at (1,1.5) {(1)};
        \draw (1,0) -- (1,1.3);
        \pgfmathsetseed{2}
        \foreach \n in {1,...,30}{
            \fill (.02+rnd*0.96,.02+rnd*1.26) circle[radius=.5pt];
        }
        
        \draw[arr] (2.3,1.3*2/3) -- +(1.3,0) node[midway, above] {$S$ wächst};
        \draw[rarr] (2.3,1.3/3) -- +(1.3,0) node[midway, below] {?};
        
        \begin{scope}[xshift=3.9cm]
            \node at (1,1.5) {(2)};
            \draw (0,0) rectangle (2, 1.3);
            \pgfmathsetseed{3}
            \foreach \n in {1,...,30}{
                \fill (.02+rnd*1.96,.02+rnd*1.26) circle[radius=.5pt];
            }
        \end{scope}
    \end{tikzpicture}
}


% Degeneracy function for large systems -> delta function
\newcommand{\tfigDegeneracyFunctionLargeSystemsDelta}{
    \begin{tikzpicture}[scale=4]
        \draw[arr] (-.7,0) -- (.7,0) node[xlabel] {$\frac{n_1}{N_1}$};
        \draw[arr] (0,0) -- (0,.7) node[ylabel] {$(g_1g_2)(n_1)$};
        \draw (-.5,.5pt) -- +(0,-1pt) node[below] {$-\frac{1}{2}$};
        \draw (.5,.5pt) -- +(0,-1pt) node[below] {$\frac{1}{2}$};
        \draw[nice green] (.2,.5) -- (.2,0) node[near start, right] {$\frac{\delta_n}{N_1}=10^{-11}$}
        node[below,black] {$\frac{\hat{n}_1}{N_1}$};
        \draw (.5pt,.5) -- (-.5pt,.5) node[left] {$(g_1g_2)_\mathrm{max}$};
    \end{tikzpicture}
}


\newcommand{\tfigPlaceholder}{
    \begin{tikzpicture}
        \filldraw[fill=formalshade] (0,0) rectangle (3,3) ;
        \node[black] at (1.5,1.5){\textbf{?}};
    \end{tikzpicture}
}

% Degeneracy function for large systems -> delta function
\newcommand{\tfigProcessReversibleQuasistationary}{
    \begin{tikzpicture}[scale=1.7]
        \draw[piston] (0,0) -- (0,1);
        \draw[piston bar] (0,.5) -- (1.2,.5);
        \draw[arr] (.1,.7) -- +(.4,0);
        \draw[pattern=north east lines,even odd rule] (-1.4,1.2) rectangle (1.4,-.2) node[pos=0,anchor=south west] {$\Delta S=0$} (-1.2,0) rectangle (1.2,1);
        \pgfmathsetseed{16}
        \foreach \n in {1,...,20}{
            \fill (.02-1.2+rnd*.96,.02+rnd*.96) circle[radius=.5pt];
        }
    \end{tikzpicture}
}
            
\newcommand{\tfigProcessIrreversibleQuasistationary}{
    \begin{tikzpicture}[scale=1.7]
        \foreach \x in {0,.1,...,.6}{
            \draw[line width=3pt] (\x,0) -- +(0,1);
            \draw[arr] (\x,1.2) -- +(0,.4);
        }
        \node at (.8,.5) {\ldots};
        \draw[pattern=north east lines,even odd rule] (-1.4,1.2) rectangle (1.4,-.2) node[pos=0,anchor=south west] {$\Delta S>0$} (-1.2,0) rectangle (1.2,1);
        \pgfmathsetseed{30}
        \foreach \n in {1,...,20}{
            \fill (.02-1.2+rnd*.96,.04+rnd*.90) circle[radius=.5pt];
        }
    \end{tikzpicture}
}

\newcommand{\tfigProcessIrreversibleNonquasistationary}{
    \begin{tikzpicture}[scale=1.7]
        \draw[line width=3pt] (0,0) -- +(0,1);
        \draw[arr] (0,1.2) -- +(0,.4);
        
        \draw[pattern=north east lines,even odd rule] (-1.4,1.2) rectangle (1.4,-.2) node[pos=0,anchor=south west] {$\Delta S>0$} (-1.2,0) rectangle (1.2,1);
        \pgfmathsetseed{29}
        \foreach \n in {1,...,20}{
            \fill (.02-1.2+rnd*.96,.02+rnd*.96) circle[radius=.5pt];
        }
    \end{tikzpicture}
}
\newcommand{\tfigTheoremMaximizedWork}{
    \begin{tikzpicture}
        [
            system node/.style={rectangle,draw,inner sep=10pt}
        ]
        \node[system node, fill=red!20!white] (RWQ) at (3,1) {RWQ};
        \node[system node, fill=black!10!white] (RAQ) at (3,-1) {RAQ};
        \node[point] (P) at (0,0) {};
        \draw[arr] (P) .. controls +(0:.7) and +(180:.7) .. (RAQ.west) node[pos=.7,anchor=north east] {$\Delta W^\mathrm{RAQ}$};
        \draw[arr] (P) .. controls +(0:.7) and +(180:.7) .. (RWQ.west) node[pos=.7,anchor=south east] {$\Delta Q^\mathrm{RWQ}$};
        edge[arr] (RWQ) 
        edge[arr] (RAQ);
        \node[system node,align=center] at (-3,0) {\textbf{TS}\\\\$\Delta S_{AB}^\mathrm{TS}$} edge[arr] node[midway, above] {$-\Delta U_{AB}^\mathrm{TS}$} (P);
    \end{tikzpicture}
}
\usepackage{hyperref} % include hyperref as last package!



\title{\tfigTitel}
\subtitle{Sommersemester 2022}
\author{von Kyano Levi\\bei Professor Holger Stark}
\date{}

% -----------------------------------
% -----------------------------------
% -----------------------------------

\begin{document}

\frontmatter
\maketitle
\tableofcontents

% -----------------------------------
% -----------------------------------

\mainmatter
% !TeX root = Theo_IV.tex

\chapter{Einleitung\label{einleitung}}



\section{Inhalt}

Der Inhalt dieser Vorlesung gliedert sich in zwei Teile, die Thermodynamik und die statistische Physik:

Bei der Thermodynamik geht es im Allgemeinen um Vielteilchensysteme, für die genauen mikroskopischen Positionen interessiert man sich allerdings nicht, sondern vielmehr um makroskopische Größen und Verteilungen.
Die Vielteilchensysteme werden durch den Minimalsatz von makroskopischen Variablen beschrieben, z.B. durch Energie, Volumen und Entropie.
Man spricht tatsächlich bei der Thermodynamik auch von der Lehre der Entropie.

Das Ziel ist es, allgemeine und modellunabhängige Aussagen und Prinzipien zu formulieren und häufig geht es nicht um die Berechnung von Systemparametern, sondern um die Relationen zwischen ihnen.
Damit hat die Thermodynamik einen sehr breiten Anwendungsbereich und findet Verwendung in vielen 
Wissenschaften\footnote{Beispiele: Gase, Magnetismus, Supraleitung, chemische Reaktionen, Phasenübergänge, Schwarzkörperstrahlung, Neutronensterne, schwarze Löcher, Biologie, soziologische Systeme, Ottomotor, Klima, ...}.

Die Thermodynamik stellt auch die Basis für einige Weiterentwicklungen dar, wie z.B. die Nichtgleich\-gewichts-Thermodynamik (in dieser Vorlesung wird eine reine thermostatische Beschreibung behandelt).

Die statistische Physik fasst Parameter von Vielteilchensysteme mithilfe der mikroskopischen Bewegungsgleichungen und statistischen Methoden mit makroskopischen Größen zusammen.



\begin{itemize}
	\item Herbert B. Callen: Thermodynamics and an Introduction to Thermostatistics, John Wiley \& Sons, New York 1985
	\item Gerhard Adam, Otto Hittmaier: Wärmetheorie, Vieweg, Wiesbaden 1992
	\item Charles Kittel, Herbert Kroemer: Thermal Physics, W.H. Freeman and Company, 1980
\end{itemize}

Folgende weiterführende Literatur kann zurate gezogen werden:
\begin{itemize}
	\item Arnold Sommerfeld: Theoretische Physik V: Thermodynamik und Statistik, Harri Deutsch, 1977
	\item Franz Schwabl: Statistische Mechanik, Springer, Berlin 2000
	\item Torsten Fließbach: Statistische Physik - Lehrbuch zur Theoretischen Physik IV, Springer Spektrum, Berlin 2010
	\item Wolfgang Nolting: Grundkurs Theoretische Physik 4/2: Thermodynamik, Springer Spektrum, Berlin 2016
	\item Wolfgang Nolting: Theoretische Physik 6: Statistische Physik, Springer Spektrum, Berlin 2013
	\item Mehran Kardar: Statistical Physics of Particles, Cambridge University Press, Cambridge 2007
\end{itemize}



\section{Grundlegende Konstanten der Thermodynamik}

Für Konstanten, deren Wert per Definition festgelegt wurde, die also exakt sind, wird ein $\equiv $-Zeichen verwendet.


\begin{table}[H]
	\centering
	\begin{tabular}{|l|l|} \hline
		\textbf{Konstante}       & \textbf{Wert}                                                                            \\
		\hline

		Boltzmannkonstante       & \centering\arraybackslash{} $k_\mathrm{B} \equiv \SI{1,38064852e-23}{\joule\per\kelvin}$ \\
		Universelle Gaskonstante & \centering\arraybackslash{} $R \equiv \SI{8,31446261815324}{\joule\per\kelvin\per\mole}$ \\
		Avogadro-Konstante        & \centering\arraybackslash{} $N_\mathrm{A} \equiv \SI{6,02214076e23}{\per\mole}$          \\
		Atomare Masseneinheit    & \centering\arraybackslash{} $u= \SI{1,6605390666050e-27}{\kg}$                           \\
		\hline
	\end{tabular}
\end{table}




\section{Grundlegende Formeln der Thermodynamik}

% !TeX root = Theo_IV.tex

\chapter{Thermodynamik}
\section{}
\subsection{Zugang zur Thermodynamik}

Üblicherweise wird die Thermodynamik induktiv entwickelt. Aus Erfahrungstatsachen, wie Wärme, Temperatur, thermodynamische Maschinen, usw. werden Konzepte und Gesetze, wie die Energieerhaltung und die Entropie usw. abgeleitet.

In dieser Vorlesung wird stattdessen der axiomatische Zugang gewählt.
Aus Postulaten zur Energie und insbesondere zur Entropie wird die Thermodynamik aufgebaut und hergeleitet.
Anschließend werden die Konsequenzen dann mit den Erfahrungstatsachen abgeglichen.
Diese Postulate sind die Essenz der Entwicklung der Theorie und sie helfen, die Struktur der Thermodynamik sichtbar zu machen.

Ähnlich lassen sich auch andere Gebiete der Physik behandeln.
So können z.B. aus dem Hamiltonschen Prinzip die mechanischen Bewegungsgleichungen und aus den Maxwell-Gleichungen die elektrischen und magnetischen Gesetze hergeleitet werden. 


\subsection{Was ist Thermodynamik?}

Statt der mikroskopischen Beschreibung mit \num{1e24} Koordinaten (Ort, Impuls, Molekülfreiheitsgrade) werden nur wenige makroskopische thermodynamische Variablen quantifiziert.

Dieser Ansatz ist auch physikalisch rechtfertigbar, denn bei realen Messungen findet automatisch eine intrinsische Mittelung statt.
Zum einen findet eine zeitliche Mittelung statt, denn die mikroskopische Bewegung findet auf Zeitskalen von \SIrange{1e-15}{1e-12}{\s} statt, während makroskopische Messungen im Allgemeinen nicht kürzer als \SI{1e-7}{\s} sind.
Es findet also eine Messung in gewissen Maßstäben zeitunabhängiger Kombinationen der über \num{1e24} Koordinaten statt.

Zum anderen kommt es zu einer räumlichen Mittelung, zum Vergleich: mikroskopische Abmessungen liegen bei um die \SI{0.1}{\nm} (Atomradius, Gitterkonstante), während makroskopische Messungen in der Regel bei über \SI{100}{\nm} liegen (Größenordnung der Wellenlänge von sichtbarem Licht).
Also wird meist über weit mehr als \num{1e9} Atome oder Moleküle gemittelt.


Es verbleiben nur wenige Kenngrößen. Mechanische Größen sind zum Beispiel
\begin{itemize}
    \item Volumen $V$,
    \item Druck $P$,
    \item Oberfläche $F$,
    \item Oberflächenspannung $\sigma$ und
    \item hydrodynamische Flussfelder.
\end{itemize}

In der Elektrodynamik misst man in der Regel unter anderem
\begin{itemize}
    \item Ladung $Q$,
    \item Strom $I$,
    \item Magnetisierung $\vec M$,
    \item Magnetfeld $\vec H$,
    \item Polarisation $\vec P$ und 
    \item das elektrische Feld $\vec E$. 
\end{itemize}

Neu ist jetzt folgendes:
\begin{formal}
    Die Thermodynamik behandelt die makroskopischen Folgen derjenigen Koordinaten, die sich herausmitteln $\leftrightarrow$ Wärme. 
\end{formal}

Die Zufuhr von Wärme in ein System führt z.B. zur Anregung von atomarer Bewegung und damit einer Temperatur $T$. 

In der Mechanik wird das Energie- bzw. Arbeitsdifferential als 
\begin{align*}
    \diffa{W} = \vec F\cdot\diffa{\vec r}
\end{align*}
definiert. Hier wird diese Definition nun verallgemeinert:
\begin{align*}
    \diffa{E} = \underbrace{\text{verallgemeinerte Kraft}\: J}_{\text{intensiv}} \times \diffa{\underbrace{(\text{verallgemeinerter Weg}\: X)}_{\text{extensiv}}}
\end{align*}
Dabei sind $(X,J)$ zueinander konjugierte Variablen. Die Einheit des Produkts $X\times J$ muss stets eine Energieeinheit sein. 

Bereits bekannte Beispiele sind 
\begin{itemize}
    \item Druckarbeit $-P\diffa{V}$,
    \item Oberflächenarbeit $\sigma\diffa{F}$,
    \item Magnetisierungsarbeit $\mu_0\vec H\cdot\diffa{\vec M}$ und 
    \item Polarisierungsarbeit: $\vec E\cdot\diffa{\vec P}$.
\end{itemize}
Im Verlaufe der Vorlesung wird eine neue Arbeit eingeführt, die den Wärmeübertrag und damit Energietransfer auf verborgene atomare Freiheitsgraden oder Moden beschreibt:
\begin{align*}
    \text{Energietransfer}=\text{Wärmeübertrag}=\vec T\diffa{S}
\end{align*}
mit Temperatur $T$ und Entropie $S$. 


\subsection{Modellsystem, Parameter und Begriffe}

Um ein Grundkonzept zu entwickeln, wird zunächst ein einfaches, idealisiertes System vorausgesetzt, das makroskopisch, homogen und isotrop, elektrisch neutral ist, in dem keine chemischen Reaktionen ablaufen und das keine elektrische, magnetische oder Gravitationsfelder besitzt. 
Auch Randeffekte werden zunächst vernachlässigt, indem angenommen wird, dass das System unendlich groß ist und damit keine Oberfläche hat. 

Für dieses System werden dann Parameter bestimmt, wie das Volumen $V$ und die Stoffmengen der beteiligten chemischen Substanzen. 

Zunächst werden einige Definitionen erläutert:
\begin{enumerate}
    \item Stoffmenge $n$: \SI{1}{\mole} einer Substanz entspricht einer Menge der Zahl an Atomen oder Molekülen, die der Avogadro-Konstante/Loschmidt-Zahl $N_\mathrm{A} \equiv \SI{6,02214076e23}{\per\mole}$ entspricht. Historisch wurde diese Definition gewählt, weil sie der Zahl der Atome in \SI{12}{\g} vom Isotop \isotope[12]{C} von Kohlenstoff entspricht. 
\end{enumerate}



\end{document}
