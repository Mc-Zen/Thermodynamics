% !TeX root = Theo_IV.tex

\part{Thermodynamik}

\chapter{Grundlagen und Postulate\label{sec:grundlagen und postulate}}
\section{Zugang zur Thermodynamik}

Üblicherweise wird die Thermodynamik induktiv entwickelt. Aus Erfahrungstatsachen wie Wärme, Temperatur und Abläufen von thermodynamischen Maschinen werden Konzepte und Gesetze wie die Energieerhaltung und die Entropie abgeleitet.

In dieser Vorlesung wird stattdessen der axiomatische Zugang gewählt.
Aus Postulaten zur Energie und insbesondere zur Entropie wird die Thermodynamik aufgebaut und hergeleitet.
Anschließend werden die Konsequenzen dann mit den Erfahrungstatsachen abgeglichen.
Diese Postulate sind die Essenz der Entwicklung der Theorie und sie helfen, die Struktur der Thermodynamik sichtbar zu machen.

Ähnlich lassen sich auch andere Gebiete der Physik behandeln.
So können z.~B. aus dem Hamiltonschen Prinzip die mechanischen Bewegungsgleichungen und aus den Maxwell-Gleichungen die elektrischen und magnetischen Gesetze hergeleitet werden.


\section{Was ist Thermodynamik?\label{sec:was ist thermodynamik}}

Die meisten realen physikalischen Systeme bestehen aus einer sehr großen Zahl von Teilchen. Deren Behandlung mithilfe von mechanischen Bewegungsgleichungen ist allerdings mehr als unhandlich. 
Statt der mikroskopischen Beschreibung mit \num[print-unity-mantissa=false]{1e24} Koordinaten (Ort, Impuls, Molekülfreiheitsgrade) werden nur wenige makroskopische thermodynamische Variablen herangezogen, um ein System handhabbar zu beschreiben.

Dieser Ansatz ist auch physikalisch rechtfertigbar, denn bei realen Messungen findet automatisch eine intrinsische Mittelung statt.
Zum einen findet eine zeitliche Mittelung statt, denn die mikroskopische Bewegung findet auf Zeitskalen von \qty[print-unity-mantissa=false]{1e-15}{\s} bis \qty[print-unity-mantissa=false]{1e-12}{\s}statt, während makroskopische Messungen im Allgemeinen nicht kürzer als \qty[print-unity-mantissa=false]{1e-7}{\s} sind.
Es findet also eine Messung in gewissen Maßstäben zeitunabhängiger Kombinationen der über \num[print-unity-mantissa=false]{1e24} Koordinaten statt.

Zum anderen kommt es zu einer räumlichen Mittelung \textendash{} zum Vergleich: mikroskopische Abmessungen liegen bei $\approx\qty{0.1}{\nm}$ (Atomradius, Gitterkonstante), während makroskopische Messungen in der Regel bei über \qty{100}{\nm} liegen (Größenordnung der Wellenlänge von sichtbarem Licht).
Also wird meist über weit mehr als \num[print-unity-mantissa=false]{1e9} Atome oder Moleküle gemittelt.

\begin{figure}[htbp]
    \centering
    \tfigSystemWithManyParticles
    \caption{System mit vielen Teilchen, z.~B. ein Gas.}
    \label{fig:SystemWithManyParticles}
\end{figure}

Es verbleiben nur wenige Kenngrößen. Mechanische Größen sind zum Beispiel
\begin{itemize}
    \item Volumen $V$,
    \item Druck $P$,
    \item Oberfläche $F$,
    \item Oberflächenspannung $\sigma$ und
    \item hydrodynamische Flussfelder.
\end{itemize}

In der Elektrodynamik misst man in der Regel unter anderem
\begin{itemize}
    \item Ladung $Q$,
    \item Strom $I$,
    \item Magnetisierung $\vec M$,
    \item Magnetfeld $\vec H$,
    \item Polarisation $\vec P$ und
    \item das elektrische Feld $\vec E$.
\end{itemize}

Neu ist jetzt folgendes:
\begin{formal}
    Die Thermodynamik behandelt die makroskopischen Folgen (z.~B. die Wärme) derjenigen Koordinaten, die sich herausmitteln (z.~B. die einzelnen Teilchengeschwindigkeiten).% $\leftrightarrow$ Wärme.
\end{formal}

Die Zufuhr von Wärme in ein System führt z.~B. zur Anregung von atomarer Bewegung und damit einer Temperatur $T$.

In der Mechanik wird das Energie- bzw. Arbeitsdifferential als
\begin{align*}
    \diffa{W} = \vec F\cdot\diffa{\vec r}
\end{align*}
definiert. Hier wird diese Definition nun zu einem Produkt einer \emph{intensiven} Größe und dem Differential einer \emph{extensiven} Größe verallgemeinert:
\begin{align*}
    \diffa{E} = \underbrace{\text{verallgemeinerte Kraft}\: J}_{\text{intensiv}} \times \diffa{\underbrace{(\text{verallgemeinerter Weg}\: X)}_{\text{extensiv}}}.
\end{align*}
Dabei sind $(X,J)$ sogenannte zueinander \emph{konjugierte Variablen}. Die Einheit des Produkts $X\times J$ muss stets eine Energieeinheit sein. Wir werden im Verlauf des Kapitels den Unterschied zwischen intensiven und extensiven Größen erläutern. 

Bereits bekannte Beispiele sind
\begin{itemize}
    \item Druckarbeit $-P\diffa{V}$,
    \item Oberflächenarbeit $\sigma\diffa{F}$,
    \item Magnetisierungsarbeit $\mu_0\vec H\cdot\diffa{\vec M}$ und
    \item Polarisierungsarbeit: $\vec E\cdot\diffa{\vec P}$.
\end{itemize}
Im Verlaufe der Vorlesung wird eine neue Arbeit eingeführt, die den Wärmeübertrag und damit Energietransfer auf verborgene atomare Freiheitsgrade oder Moden beschreibt:
\begin{align*}
    \text{Energietransfer}=\text{Wärmeübertrag}=T\diffa{S}
\end{align*}
mit Temperatur $T$ und Entropie $S$.



\section{Modellsystem, Parameter und Begriffe}

Um ein Grundkonzept zu entwickeln, wird zunächst ein einfaches, idealisiertes System vorausgesetzt, das makroskopisch homogen und isotrop sowie elektrisch neutral ist, in dem keine chemischen Reaktionen ablaufen und das keine elektrischen, magnetischen oder gravitativen Felder besitzt. Auch Randeffekte werden zunächst vernachlässigt, indem angenommen wird, dass das System unendlich groß ist und damit keine Oberfläche hat.

Für dieses System werden dann Parameter wie das Volumen $V$ und die Stoffmengen der beteiligten chemischen Substanzen bestimmt.

Zunächst werden einige Definitionen erläutert:
\begin{enumerate}
    \item Als \emph{Normalbedingungen} bzw. \emph{Standardbedingungen} wird ein Zustand bei \qty{0}{\degreeCelsius} und \qty{1013}{\milli\bar} bezeichnet.
    \item \emph{Stoffmenge} (veraltet auch Molzahl) $\sm$: \qty{1}{\mole} einer Substanz entspricht einer Anzahl von Atomen oder Molekülen, die der Avogadro-Konstante/Loschmidt-Zahl $\avogadro \equiv \qty{6,02214076e23}{\per\mole}$ entspricht\footnote{
          Die Avogadro-Konstante besitzt die Einheit \si{\per\mole} während die Avogadro-Zahl \num{6,02214076e23} dimensionslos ist.
          Bei der Definition der Loschmidt-Zahl kann es aufgrund der historischen Entwicklung zu einiger Verwirrung kommen. Ursprünglich definierte Josef Loschmidt in seiner Arbeit \anf{Zur Grösse der Luftmoleküle} eine Zahl von in einer Volumeneinheit enthaltenen Luftmolekülen, war aber damals noch nicht sehr präzise mit seiner Definition. Als sogenannte Loschmidt-Konstante $N_\mathrm{L}$ wird heute der Wert \qty{2.686780111e25}{\per\cubic\m} definiert, welcher über das molare Volumen eines idealen Gases $V_{m0} = \qty{22,414}{\liter\per\mole}$ mit der Avogadro-Konstante $\avogadro=$ zusammenhängt: $N_\mathrm{L}=\avogadro/V_{m0}$. \cite{lit:loschmidt_constant,lit:loschmidt}
          Im deutschsprachigen Raum wird gelegentlich der Begriff Loschmidt-Zahl aber auch synonym mit der (dimensionslosen) Avogadro-Zahl verwendet. Um Verwirrungen zu vermeiden, wird im Folgenden ausschließlich von der Avogadro-Konstante gesprochen.}.

          Historisch wurde diese Definition gewählt, weil sie der Zahl der Atome in \qty{12}{\g} des Isotops \isotope[12]{C} von Kohlenstoff entspricht.
          Zur Referenz: \qty{1}{\mole} ist auch die Zahl der Moleküle eines idealen Gases unter Normalbedingungen in einem Volumen von $V=\qty{22,413}{\liter}$.

          Die Stoffmenge wird aus einer Teilchenzahl $N_k$ der Molekülsorte $k$ folgendermaßen berechnet:
          \begin{align*}
              \sm_k = \frac{N_k}{\avogadro}.
          \end{align*}
    \item Wir definieren ferner den \emph{Stoffmengenanteil} (früher Molenanteil genannt) als
          \begin{align*}
              x_k  = \frac{\sm_k}{\sum_i \sm_i}
          \end{align*}
          und das \emph{molare Volumen} bzw. \emph{Molvolumen} als
          \begin{align*}
              V_m = \frac{V}{\sum_i \sm_i}.
          \end{align*}
          Beide beschreiben Anteile am Gesamtsystem.
    \item Es wird unterschieden zwischen extensiven Parametern, wie dem Volumen $V$ oder Stoffmengen $\sm_1,\ldots,\sm_r$, die sich beim Zusammenführen mehrerer Teilsysteme additiv verhalten, also mit dem Volumen wachsen,
          \begin{align*}
              2\times(V,\sm_1,\ldots,\sm_r) \rightarrow (2V,2\sm_1,\ldots,2\sm_r)
          \end{align*}
          und intensiven Parametern wie $x_1,\dots,x_r,V_m$, die keine Änderung bei wachsendem Volumen erfahren. Dazu gehören auch die Temperatur $T$ und der Druck $P$.
    \item Als \emph{Zustandsgrößen} werden Größen bezeichnet, die unabhängig von der Vorgeschichte des Systems sind und einfach seinen Zustand beschreiben. Dazu gehören unter anderem das Volumen, die Stoffmenge, die Temperatur und der Druck.
\end{enumerate}




\section{Postulate zur inneren Energie und 1. Hauptsatz der Thermodynamik}

%Die Geschichte des Energiebegriffs beginnt mit Gottfried W. Leibniz im Jahr 1686. Er erkannte, dass die Wirkung einer mechanischen Bewegung 

\paragraph*{Innere Energie}

Die \emph{innere Energie} ist zugleich eine Zustandsgröße und eine extensive Größe.
\begin{formal}
    Makroskopische Systeme besitzen eine genau definierte \formalemph{innere Energie} $U$ (bezogen auf einen willkürlichen Grundzustand), die erhalten bleibt.
\end{formal}

Je nach Temperaturbereich wird ein Energienullpunkt festgelegt.
Für Energien $k_\mathrm{B}T$ im Bereich der Bindungsenergien von Molekülen können z.~B. ruhende Moleküle als Referenz gelten.


\paragraph*{Thermodynamisches Gleichgewicht}

Wir machen die Erfahrung, dass Systeme einfache Endzustände mit einer kleinstmöglichen Zahl von makroskopischen Variablen anstreben. Beispielsweise bewirkt die Reibung in einer Flüssigkeit in einem Glas, dass eine turbulente Strömung zur laminaren Strömung wird und allmählich ganz zur Ruhe kommt.

Diese Beobachtung führt uns zum ersten Postulat:
\begin{postulate}
    \label{post:gleichgewichtszustaende}
    Es gibt spezielle Zustände eines Systems, sogenannte \emph{Gleichgewichtszustände}, die makroskopisch vollkommen durch die Angabe weniger Zustandsgrößen beschrieben sind. 
    Solche Zustandsgrößen sind beispielsweise die innere Energie $U$, das Volumen $V$ und die Stoffmengen $\sm_1,\sm_2,\ldots$ der chemischen Komponenten.
\end{postulate}

In komplexeren Systemen muss z.~B. noch die Polarisation, die Magnetisierung und die Oberfläche berücksichtigt werden. Analog zum Volumen werden diese Größen über ein Arbeitsdifferential $\diffa{W} = J\diffa{X}$ in das System aufgenommen.

Im makroskopischen Gleichgewichtszustand werden viele mikroskopische Zustände im Messzeitraum angenommen, die mit dem makroskopischen Zustand $(U,V,\sm_1,...,\sm_r)$ vereinbar sind. So können z.~B. sehr viele verschiedene Kombinationen aus Teilchenenergien zur gleichen mittleren Energie führen. Normalerweise hat ein System kein \anf{Gedächtnis}, es verhält sich im Prinzip zufällig. Dies wird durch die Ergodenhypothese beschrieben und später präzisiert.

Es gibt aber auch Zustände, für die diese Annahme nicht gilt. Beim \emph{metastabilen Gleichgewicht} sind nicht alle Zustände in der Messzeit erreichbar und die Vorgeschichte ist relevant. Trotzdem ist der Formalismus der Thermodynamik in Teilbereichen davon oder für kurze Zeitspannen anwendbar.


Ein Beispiel dafür %wäre der Übergang von Glas (ungeordnetes System) zu Kristall (geordnet). 
wäre Glas als ungeordnetes, metastabiles System, das eigentlich zur Kristallisation tendiert, es aber nicht tut. Andere Beispiele sind Gedächtnislegierungen mit eingefrorenen Spannungen oder radioaktive Substanzen, die ihre Zusammensetzung durch spontanen Zerfall der Atome verändern.


\paragraph*{Wände}

Wände, Grenzen oder Ränder isolieren ein System und kontrollieren damit die Werte der Zustandsgrößen sowie den Energiefluss, wie am folgenden Beispiel ersichtlich.

Betrachte ein System aus zwei Kammern (siehe \Abbref{fig:TwoChambersSeparatedByPiston}), die durch einen beweglichen Kolben voneinander getrennt werden. Das gesamte System ist von einer festen Wand umgeben. Die Beschaffenheit des Kolbens oder der Wand kann verschiedener Art sein:
\begin{itemize}
    \item Wände können beweglich oder fest sein und kontrollieren so die innere Energie über mechanische Arbeit.
    \item (Semi-) permeable Wände kontrollieren die Stoffmengen $\sm_k$, indem sie beispielsweise nur bestimmte Stoffe hindurchlassen.
    \item Wärmeleitende Wände erlauben einen Wärmefluss, während thermisch isolierende Grenzen ein Angleichen der Temperatur verhindern. Durch Wände werden also thermisch abgeschlossene Systeme ermöglicht, was eine Definition des Wärmebegriffs erlaubt.
\end{itemize}

\begin{figure}[b]
    \centering
    \tfigTwoChambersSeparatedByPiston
    \caption{Zwei Kammern 1 und 2 sind von einer Wand umgeben und durch einen Kolben voneinander getrennt. }
    \label{fig:TwoChambersSeparatedByPiston}
\end{figure}




\paragraph*{Energiemessung}

In einem thermisch isolierten System ändert sich die innere Energie genau mit der mechanischen Arbeit $\Delta W$, die an dem System verrichtet wird,
\begin{align*}
    U(B)-U(A) =\Delta W(A\rightarrow B).
\end{align*}
Auf diese Weise kann die Änderung der inneren Energie durch die mechanische Energie definiert werden, welche wir bereits kennen und messen bzw. berechnen können.
Zum Beispiel erhöht sich die innere Energie, wenn das System mechanisch komprimiert wird (siehe \Abbref{fig:RectangularBoxWithPiston}).

\begin{figure}[t]
    \centering
    \tfigRectangularBoxWithPiston
    \caption{Ein System kann mechanisch komprimiert werden, sodass sich die innere Energie erhöht. Es wird Arbeit an dem System verrichtet. Der Prozess ist in diesem Fall umkehrbar. }
    \label{fig:RectangularBoxWithPiston}
\end{figure}

Allerdings könnte einem System, das eine Flüssigkeit enthält, auch Energie durch Quirlen zugeführt werden, wie in \Abbref{fig:WaterStiringIceCubes} dargestellt, wodurch sich die Temperatur erhöht. Dieser Prozess ist jedoch nicht umkehrbar, da die Temperatur durch Quirlen natürlich nicht wieder verringert werden kann.

\begin{figure}[htbp]
    \centering
    \tfigWaterStiringIceCubes
    \caption{Ein Gefäß mit Wasser einer bestimmten Temperatur (Zustand A) kann durch Quirlen erwärmt werden (Zustand B). Allerdings ist dieser Vorgang nicht auf mechanische Weise umkehrbar, denn durch Quirlen kann das Wasser nicht abgekühlt werden. }
    \label{fig:WaterStiringIceCubes}
\end{figure}

\begin{formal}
    Die innere Energie wird nicht direkt gemessen. Stattdessen kann die Änderung der inneren Energie eines Systems beim Übergang vom Zustand $A$ zum Zustand $B$ bestimmt werden, indem der Prozess $A\rightarrow B$ oder $B\rightarrow A$ rein mechanisch bewirkt wird.
\end{formal}



\paragraph*{Wärme und Wärmeübertrag}

Ein Prozess $A\rightarrow B$ muss nicht rein mechanisch ablaufen. Dabei nimmt das System sogenannte \emph{Wärme} auf oder gibt sie ab. Diese Wärme können wir definieren als Differenz der gesamten Änderung der inneren Energie und jenes Teils, der durch mechanische Arbeit verrichtet wird,
\begin{align*}
    \Delta Q(A\rightarrow B) = [U(B)-U(A)] - \Delta W(A\rightarrow B).
\end{align*}
Dieser Zusammenhang wird durch den 1. Hauptsatz der Thermodynamik zusammengefasst:
\begin{formal}
    \textbf{1. Hauptsatz der Thermodynamik (Energieerhaltungssatz):}
    \label{hs:erster}
    Die innere Energie eines Systems ändert sich mit der zugeführten Wärme und der am System verrichteten Arbeit\footnotemark,
    \begin{align*}
        \diff U = \udiff Q + \udiff W.
    \end{align*}
\end{formal}

\footnotetext{Die hier verwendeteten Notationen mit vollständigem Differential $\diff$ und unvolllständigem Differential $\udiff$ werden im Anschluss erläutert.}

Betrachte als Beispiel eine Volumenänderung, die aufgrund der resultierenden Druckänderung eine Arbeit am System darstellt\footnote{Diese Änderung muss ausreichend langsam erfolgen, damit es sich um eine quasistatische Prozessführung handelt. Bei einer schnellen Änderung entstehen Turbulenzen und das System befindet nicht über den ganzen Zeitraum im thermodynamischen Gleichgewicht.},
\begin{align*}
    \udiff W = -P\diff V.
\end{align*}
Anders als die innere Energie sind Wärme und Arbeit allerdings keine Zustandsgrößen, denn verschiedene Kombinationen von Wärme- und Arbeitszufuhr können auf den gleichen Zustand eines Systems führen (\Abbref{fig:WQAreNoStateFunctions}). Sie dienen beide nur der Änderung der inneren Energie.

\begin{figure}[htb]
    \centering
    \tfigWQAreNoStateFunctions
    \caption{Die mechanische Arbeit $\Delta W$ und die Wärme $\Delta Q$ sind keine Zustandsfunktionen. Einem System können verschiedene Verhältnisse $\Delta W_1,\Delta Q_1$ und $\Delta W_2,\Delta Q_2$ zugeführt werden, die zur selben inneren Energie $U$ führen, sodass sich das System im gleichen Zustand befindet. }
    \label{fig:WQAreNoStateFunctions}
\end{figure}


Bemerkungen:
\begin{itemize}
    \item Der Prozess kann \emph{quasistatisch} erfolgen, also so langsam, dass immer ein Gleichgewicht vorliegt und damit der Druck $P$ homogen ist.
    \item Der Prozess kann \emph{nicht-quasistatisch} erfolgen, wodurch es zu Turbulenzen kommt und $P=P(\vec r,t)$ ortsabhängig wird. Es wird eine Überschussarbeit verrichtet, die in Wärme dissipiert wird.
    \item Zusätzlich zum totalen Differential $\diff$ muss auch das \emph{unvollständige Differential} $\udiff$ eingeführt werden, welches für Größen (wie $Q$ und $W$) angewendet wird, die keine Zustandsfunktionen sind, sondern sogenannte wegabhängige \emph{Prozessgrößen}.
    \item Mechanische Arbeit stellt einen Energieübertrag dar.
    \item Ein quasistatischer Wärmeübertrag stellt nach $\udiff Q=\diff U+P\diff V$ ebenfalls einen Energieübertrag dar.
\end{itemize}


\paragraph*{Wärmeäquivalent}

Das sogenannte \emph{Wärmeäquivalent} besagt, dass Wärme als Energie quantifizierbar ist. Es gilt
\begin{align*}
    \Delta Q = mc\Delta T
\end{align*}
mit spezifischer Wärmekapazität $c$.
Die häufig verwendete Einheit Kalorie entspricht der Energie, um \qty{1}{\g} Wasser bei \qty{1013}{\milli\bar} von \qty{14,5}{\degreeCelsius} auf \qty{15,5}{\degreeCelsius} zu erwärmen. Es ist dabei $\qty{1}{cal} = \qty{4,1855}{\joule}$.



\section{Postulate zur Entropie}

Aus der Experimentalphysik erinnern wir uns, dass die Entropie $S$ mit der Irreversibilität eines Prozesses zusammenhängt. Später werden wir sehen, dass die Entropie durch $S=k_\mathrm{B}\ln{\Omega}$ mit dem Phasenraumvolumen $\Omega$ bestimmt ist.

Zuerst betrachten wir ein abgeschlossenes System mit festgelegten Zwangsbedingungen, beispielsweise zwei Kammern, die durch einen anfangs undurchlässigen, festen und wärmeisolierenden Kolben getrennt sind. Diese Eigenschaften des Kolbens stellen die Zwangsbedingungen dar. Es stellt sich die Frage, welchen Gleichgewichtszustand das System beim Entfernen einer oder mehrerer dieser Zwangsbedingungen einnimmt. Erfahrungsgemäß wissen wir z.~B., dass sich die Temperaturen beider Kammern angleichen, wenn ein Wärmeaustausch erlaubt wird, der Kolben also wärmedurchlässig ist.

Eine allgemeine Antwort auf diese Frage kann mithilfe des Extremalprinzips gewonnen werden, was uns auf das zweite Postulat führt:

% Das bringt uns zum zweiten Postulat, dem Extremalprinzip zur Entropie:
\begin{postulate}
    \label{post:entropie_maximierung}
    Gegeben sei ein isoliertes System, das durch Zwangsbedingungen unterteilt ist. Dann existiert eine Funktion der extensiven Parameter ($U^{(1)},V^{(1)},\sm_k^{(1)};U^{(2)},V^{(2)},\sm_k^{(2)}; \ldots$), genannt \emph{Entropie} $S$, die für alle Gleichgewichtszustände wohldefiniert ist und folgende Eigenschaften besitzt: Lässt man die Zwangsbedingungen fallen, so nehmen die extensiven Parameter Werte an, welche die Entropie maximieren. Der dann erreichte Endzustand heißt \emph{stabiles Gleichgewicht}.

    $S=S(\{U^{(\alpha)},V^{(\alpha)},\sm_k^{(\alpha)}\})$ heißt \emph{entropische Fundamentalbeziehung}. Sie enthält die gesamte Information über das System.
\end{postulate}


Es sei bemerkt, dass dafür die Entropie $S$ von den extensiven Variablen abhängen muss. Dieses Postulat erscheint auf den ersten Blick recht willkürlich, wird aber durch seine Konsequenzen gerechtfertigt, welche sich mit den Erfahrungstatsachen decken\footnote{Außerdem hat sich die Anwendung des Extremalprinzips in vielen Bereichen der Physik erfolgreich bewährt (siehe das Hamiltonsche Prinzip der kleinsten Wirkung sowie die Minimierung der Lagrange-Dichte bei der Quantenfeldtheorie).}. Diese werden in den folgenden Kapiteln erläutert.

Es ist außerdem sinnvoll, einige Eigenschaften für die Entropie zu fordern, die im nächsten Postulat zusammengefasst sind:
\begin{postulate}
    \label{post:eigenschaften_entropie}
    Die Entropie eines zusammengesetzten Systems ist gleich der Summe der Entropien der Teilsysteme,
    \begin{align*}
        S =\sum_\alpha S^{(\alpha)}, \quad S^{(\alpha)} = S^{(\alpha)}(U^{(\alpha)},V^{(\alpha)},\sm_1^{(\alpha)},\ldots, \sm_r^{(\alpha)}).
    \end{align*}
    $S$ ist stetig, differenzierbar und eine streng monoton ansteigende Funktion der inneren Energie $U$.
\end{postulate}

%Wir werden später sehen, dass aus dem zweiten Punkt die Temperatur $T$ hergeleitet werden kann. 
Hieraus folgt, dass $S$ eine extensive Größe ist, denn
\begin{align*}
    S(\lambda U,\lambda V,\lambda\sm_1,\ldots, \lambda\sm_r) = \lambda^1 S(U,V,\sm_1, \ldots,\sm_r).
\end{align*}
Genauer: die Entropie ist eine  homogene Funktion ersten Grades der extensiven Parameter.

Beispielsweise lässt sich für ein allgemeines System mit vielen Komponenten und einer Gesamtstoffmenge $\lambda=\sm=\sum_{k=1}^r\sm_k$,
\begin{align*}
    S(U,V,\sm_1, \dots , \sm_r) = N\cdot S\left(\frac{U}{N},\frac{V}{U},\frac{\sm_1}{N}, \dots , \frac{\sm_r}{N}\right)
\end{align*}
ein Einkomponentensystem ableiten,
\begin{align*}
    S(U,V,N) = N\cdot S(u,v,1) = N\cdot s(u,v),
\end{align*}
mit Energie pro Mol $u=U/N$, Molvolumen $v=V/N$ und Entropie pro Mol $s$. Diese Darstellung wird häufig verwendet, da Angaben in Mol ausreichen und die Rechnungen vereinfachen.

Aus dem Postulat \ref{post:eigenschaften_entropie} folgt ferner, dass für
\begin{align*}
    \left(\frac{\partial S}{\partial U}\right)_{V,\sm_1, \dots ,\sm_r} > 0
\end{align*}
die Beziehung $S(U)$ invertierbar ist,
\begin{align}
    \label{eq:energetische fundamentalbeziehung}
    U = U(S,V,\sm_1, \dots,\sm_r).
\end{align}
Diese neue Beziehung ist als \emph{energetische Fundamentalbeziehung} bekannt und wir werden sehen, dass aus dem Postulat \ref{post:entropie_maximierung} auch eine Minimierung der Energie folgt.



\section{Das Nernst-Postulat (3. Hauptsatz)\label{sec:Nernst}}



\begin{postulate}[Nernst-Postulat]
    \label{post:nernst}
    Für jeden Variablensatz $V,\sm_1,...,\sm_k$ gibt es einen Punkt, an dem gilt:
    \begin{align}
        \label{eq:nernst_postulat}
        S = 0 \quad \text{bei} \quad T =\left(\frac{\partial U}{\partial S}\right)_{V,\sm,...}=0
    \end{align}
    (siehe \Abbref{fig:SchemaFundamentalbeziehung}). Dieser Nullpunkt wird in der Realität jedoch nicht erreicht. 
\end{postulate}

\begin{figure}[H]
    \centering
    \tfigSchemaFundamentalbeziehung
    \caption{Schema zur energetischen (links) und entropischen (rechts) Fundamentalbeziehung. Die Entropie $S$ besitzt im Gegensatz zur inneren Energie einen eindeutigen Nullpunkt.  }
    \label{fig:SchemaFundamentalbeziehung}
\end{figure}

Bemerkungen:
\begin{itemize}
    \item Die Entropie $S$ besitzt einen eindeutigen Nullpunkt, im Gegensatz zu der inneren Energie $U$ (zu der immer eine Konstante addiert werden kann).
    \item Die Größe $T$ wird im Folgenden als Temperatur interpretiert.
    \item Aus dem Postulat \ref{post:eigenschaften_entropie} folgt, dass die Temperatur immer größer oder gleich null ist. Zudem lässt Postulat \ref{post:nernst} schließen, dass es eine absolute Temperaturskala (die Kelvinskala) gibt.
    In der Formulierung nach Planck von 1907 heißt es, dass der Temperaturnullpunkt $T=0$ nicht erreicht wird.
\end{itemize}




\begin{summary}
    \formalemph{Was zeichnet die Thermodynamik aus?}
    Statt unzähliger, einzelner mikroskopischen Größen (wie Koordinaten und Impulse), werden in der Thermodynamik makroskopische Kenngrößen betrachtet. Dies entspricht sowohl einer räumlichen als auch einer zeitlichen Mittelung über die mikroskopischen Größen. Zu den makroskopischen Kenngrößen gehören beispielsweise Volumen, Druck, Temperatur und Energie.
    
    Mit der \emph{Standard- oder Normalbedingung} wird die Prozessbedingung von $1013$ \si{\milli \bar} und $0$\si{\celsius} bezeichnet.

    \formalemph{Wesentliche Größen und Begriffe}
    Wir definieren die 
    \begin{itemize}
        \item \emph{Stoffmenge} als $\sm = N/N_A$, also die Teilchenzahl $N$ geteilt durch die \emph{Avogadro-Konstante}, 
        \item den \emph{Stoffmengenanteil} als $x_k=\sm_k/\sum_i \sm_i$,
        \item das \emph{Molvolumen} als $V_m = V/\sum_i \sm_i$ (alternativ auch mit $v$ gekennzeichnet) und
        \item das Wärmeequivalent $\Delta Q= mc\Delta T$.
    \end{itemize}

    \emph{Zustandsgrößen} beschreiben den Zustand eines Systems, unabhängig von dessen Vorgeschichte (z.~B. $V$, $\sm$, $T$, $P$). 
    Einige Größen sind keine Zustandsgrößen, wie z.~B. die Wärme $Q$ und die Arbeit $W$, denn sie beschreiben keinen eindeutigen Zustand des Systems. Für diese sogenannten \emph{Prozessgrößen} wir das \emph{unvollständige Differential} $\delta$ verwendet. 

    \emph{Extensive} Größen verhalten sich additiv mit dem Volumen des Systems. Dazu gehört neben dem Volumen z.~B. auch die Stoffmenge $\sm$.\\ 
    \emph{Intensive} Größen erfahren keine Änderung mit dem Volumen, z.~B. $T$, $P$, $x$ und $V_m$.
    Eine weitere extensive Kenngröße ist die innere Energie $U$, welche durch die entropische Fundamentalbeziehung gegeben ist; sie ist ebenfalls eine Zustandsgröße. 

    Die \emph{quasistatische} Prozessführung beschreibt eine langsame Prozessführung, welche dazu führt, dass das System sich zu jedem Zeitpunkt im Gleichgewichtszustand befindet. \\
    \emph{Nicht-quasistatische} Prozesse, sind Prozesse, welche bedingt durch eine schnelle Prozessführung Störungen (wie z.~B. Turbulenzen) erfahren und damit Wärme dissipieren. Dies führt dazu, dass sich das System nicht durchgehend in Gleichgewichtszuständen befindet. 

    \formalemph{Postulate und Hauptsätze der Thermodynamik}
    Wir haben ferner eine Reihe von Postulaten aufgestellt, die grob durch Erfahrungstatsachen motiviert sind. Nach dem \formalemph{Postulat~\ref{post:gleichgewichtszustaende}} hat ein System sogenannte Gleichgewichtszustände, welche durch wenige Zustandsgrößen ~($U$, $V$, $\sm$, \ldots) beschrieben werden können. 
    Innerhalb dieser makroskopischen Beschreibung verhält sich das System mikroskopisch zufällig. Im Allgemeinen führen viele mikroskopische Zustände zum gleichen makroskopischen Zustand. 

    Nach dem \formalemph{Postulat~\ref{post:entropie_maximierung}} existiert eine weitere Zustandsfunktion $S$ (welche von den extensiven Größen abhängt, die gesamte Information des Systems enthält und durch die entropische Fundamentalbeziehung beschrieben wird), die wir Entropie nennen. Entfernt man Zwangsbedingungen eines Systems, so nehmen dessen extensiven Größen in Folge die Werte an, welche die Entropie maximieren. 

    Die Entropie ist nach \formalemph{Postulat~\ref{post:eigenschaften_entropie}} eine additive Größe und soll stetig, differenzierbar und eine streng monoton ansteigende Funktion der inneren Energie $U$ sein. Damit gilt auch, dass $S$ eine extensive Größe ist. 

    Zuletzt beschreibt das \formalemph{Nernst-Postulat}, der \formalemph{dritte Hauptsatz der Thermodynamik} (Postulat~\ref{post:nernst}), einen Entropie- und Temperaturnullpunkt $S=0$ bei $T=0$, der aber praktisch nicht erreicht werden kann. 

    Außerdem haben wir den \formalemph{ersten Hauptsatz der Thermodynamik} kennengelernt: den \emph{Energieerhaltungssatz}. Er besagt, dass sich die innere Energie eines abgeschlossenen Systems nach der Gleichung 
    \begin{align*}
        \diff U = \udiff Q + \udiff W
    \end{align*}
    mit der zu- oder abgeführten Wärme und der am System verrichteten Arbeit ändert. 
\end{summary}