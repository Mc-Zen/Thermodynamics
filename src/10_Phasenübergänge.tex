% !TeX root = Theo_IV.tex

\chapter{Phasenübergänge\label{sec:phasenuebergaenge}}

Als Phasenübergang wird die Umwandlung einer Phase in eine andere bezeichnet. Ein typisches Beispiel ist der Wechsel zwischen Aggregatzuständen, z.~B. Verdampfung oder Kondensation. In \Abbref{fig:Phasendiagramme} sind zwei Phasendiagramme für die Phasen fest, flüssig und gasförmig dargestellt. Markiert sind jeweils der Tripelpunkt und der kritische Punkt. Kurven stellen Phasenkoexistenzlinien dar, bei der zwei Phasen (beispielsweise die feste und die flüssige) gleichzeitig vorhanden sind. Am Tripelpunkt treffen sich die Phasenkoexistenzlinien und alle drei Phasen liegen zugleich vor. Der kritische Punkt liegt am Ende der Koexistenzlinie von flüssiger und gasförmiger Phase. Hier geht der Dichteunterschied der beiden Phasen gegen null und Phasenübergänge zweiter Ordnung finden statt. Rechts wird das Phasendiagramm von Wasser gezeigt \textendash{} hier offenbart sich eine Anomalie: unter höheren Druck sinkt die Schmelztemperatur.

Phasenübergänge werden in zwei Kategorien eingeteilt: Phasenübergänge erster und zweiter Ordnung.

\begin{figure}[htbp]
    \centering
    \tfigPhaseDiagramNoAnomaly

    \tfigPhaseDiagramWithAnomaly
    \caption{Phasendiagramme ohne (oben) und mit Anomalie (unten). }
    \label{fig:Phasendiagramme}
\end{figure}




\section{Übergänge erster Ordnung in Einkomponentensystemen}

Beispiele für Phasenübergänge erster Ordnung sind Unstetigkeiten in der Dichte oder dem Molvolumen sowie latente Wärme.

Wir wollen zuerst eine exemplarische Diskussion anhand der Van-der-Waals-Zustandsgleichung machen.


\paragraph*{Van-der-Waals-Zustandsgleichung}

Wir haben das Van-der-Waals-Gas bereits in Kapitel~\ref{sec:das van der waals gas}  kennengelernt und die Zustandsgleichung
\begin{align}
    \label{eq:van der waals 1}
    \left( P+\frac{a}{v^2}  \right) (v-b) =RT \\
    \label{eq:van der waals 2}
    \equivalence P=\frac{NRT}{V-Nb}-\frac{N^2a}{V^2},
\end{align}
die es beschreibt. Für konstante Temperatur erhalten wir aus \eqref{eq:van der waals 1}
\begin{align}
    \label{eq:van der waals isotherme}
    v^3-\left( b+\frac{RT}{P} \right)v^2+\frac{a}{P}v-\frac{ab}{p} = 0.
\end{align}
Die Gleichung hat entweder drei Lösungen (für kleinere $T$) oder eine (siehe \Abbref{fig:VanDerWaalsIsotherm}).

Am kritischen Punkt $\{T_\mathrm{k},P_\mathrm{k},v_\mathrm{k}\}$ ist ${\partial P}/{\partial v} = 0$ und ${\partial^2 P}/{\partial v^2}=0$ und somit
\begin{align*}
    T_\mathrm{k}=\frac{8a}{27Rb}, \quad P_\mathrm{k}=\frac{a}{27b^2},\quad v_\mathrm{k}=3b.
\end{align*}
Diese Relationen verwenden wir, um die Van-der-Waals-Gleichung in reduzierten Größen $p=P/P_\mathrm{k}$, $t=T/T_\mathrm{k}$ und $\Tilde{v}=v/v_\mathrm{k}$ zu formulieren:
\begin{align}
    \label{eq:van der waals reduziert}
    \left( p+\frac{3}{\Tilde{v}^2} \right)(3\Tilde{v}-1) = 8t.
\end{align}
Dies ist eine universelle Gleichung, denn sie hängt nicht von den Gasparametern $a$ und $b$ und damit dem Gas selbst ab. Folglich verhält sich der flüssig-gasförmig-Übergang bei allen Gasen ähnlich, was auch als Ähnlichkeitsgesetz bekannt ist und empirisch semi-quantitativ bestätigt wurde.

\begin{figure}[htbp]
    \centering
    \tfigVanDerWaalsIsotherm
    \caption{Verschiedene Isotherme bei unterschiedlichen Temperaturen $T$ im $v/v_\mathrm{k}$-$p/p_\mathrm{k}$-Diagramm nach der (universellen) Van-der-Waals-Gleichung. }
    \label{fig:VanDerWaalsIsotherm}
\end{figure}


\paragraph*{Stabilität und Phasentrennung}

Wir betrachten nun einzelne Isotherme $T<T_\mathrm{k}$, also solche mit drei Nullstellen, wie beispielhaft in \Abbref{fig:PhaseSeparation} (links) dargestellt. Rechts ist der Zusammenhang von $p$ und $v$ invertiert. Man sieht sofort, dass $v(P)$ im Bereich von B bis R nicht eindeutig ist und es stellt sich die Frage, welcher Wert dort nun wirklich angenommen wird.

\begin{figure}[htbp]
    \centering
    \tfigPhaseSeparation
    \tfigPhaseSeparationInverted
    \caption{Links: Schematische Isotherme im $P$-$v$-Diagramm zur Darstellung eines Phasenübergangs unterhalb des kritischen Punktes. Rechts: Gleiche Isotherme im invertierten $v$-$P$-Diagramm.
        Einzelne Punkte auf der Isotherme sind mit Buchstaben benannt. Die Lösungen der Gleichung \eqref{eq:van der waals isotherme} entsprechen den Punkten O, K und D. Die physikalische Isotherme ist in grün gekennzeichnet, während instabile Bereiche in rot und metastabile in blau markiert sind. }
    \label{fig:PhaseSeparation}
\end{figure}

Auch ist bemerkenswert, dass im roten Bereich MLKJF die Stabilitätsbedingung
\begin{align*}
    \kappa_T=-\frac{1}{V}\left(\frac{\partial V}{\partial P}\right)_T>0 \equivalence \left( \frac{\partial P}{\partial v} \right)_T<0
\end{align*}
verletzt ist (positive Steigung). Ein Teil der Isotherme ist also unphysikalisch und das System nicht stabil. Hier findet die Phasentrennung in eine Phase mit kleinem Volumen (flüssige Phase) und großen Volumen (gasförmige Phase) statt.
%??

\paragraph*{Chemisches Potential}

Wir wollen als nächstes die Gibbs-Duhem-Beziehung \eqref{eq:gibbs_duhem}
\begin{align*}
    \diff\mu=-s\diff T+v\diff P=\frac{1}{N}\diff G
\end{align*}
anwenden, um das chemische Potential zu untersuchen. Integration entlang der Isotherme ($\diff T=0$) liefert
\begin{align*}
    \mu(T=\mathrm{konst},P) = \int v\diff P +\phi(T)
\end{align*}
mit unbekannter aber hier nicht relevanter Integrationskonstante $\phi(T)$. Der Verlauf von $\mu$ ist in \Abbref{fig:PhaseSeparationChemPotential} abgebildet.

\begin{figure}[htbp]
    \centering
    \tfigPhaseSeparationChemPotential
    \caption{Schematische Isotherme aus \Abbref{fig:PhaseSeparation} im $p$-$\mu$-Diagramm. Wieder ist in grün die physikalisch realisierte Isotherme markiert. Im Punkt D,O weist diese einen Knick auf, da hier beide Phasen koexistieren und auf den anderen Zweig gewechselt wird. }
    \label{fig:PhaseSeparationChemPotential}
\end{figure}

Das thermische Gleichgewicht wird durch das Minimum von $\mu=G/N$ bestimmt. Das sind dann die stabilen Zustände.  Damit gibt es einen Knick von $\mu(P)$ bei $\mu(D)=\mu(O)$ und auch eine Unstetigkeit in $v=(\partial\mu/\partial P)_T$. In dem fraglichen Bereich koexistieren zwei Phasen (hier Gas und Flüssigkeit) mit unterschiedlicher Dichte und Molvolumen.


\paragraph*{Phasenverhalten}

Koppeln wir nun das System an ein Wärme- und ein Volumenreservoir, halten die Temperatur $T$ konstant und erlauben $P$ sich quasistatisch zu verändern.

Im ersten Bereich $P<P_B$ ist das Molvolumen $v$ eindeutig. Im Bereich $P_B<P<P_D$ ist $v(C)>v(L)>v(N)$. Wir wissen, dass der Zustand L instabil ist, da $\partial v/\partial P>0$. Das Potential $\mu(C)$ ist kleiner als $\mu(N)$ und C ist im thermischen Gleichgewicht der realisierte Zustand. N ist zwar ebenfalls nicht instabil, aber entspricht nicht dem Minimum von $G$ und ist daher metastabil.

Für $P_D=P_K=P_O$ ist $\mu(D) = \mu(O)$. Hier wird der Zweig gewechselt und zwei Phasen koexistieren (jeweils mit $v(D)$ und $v(O)$). Die übrigen Bereiche werden analog zu den ersten beiden behandelt. Die physikalische Isotherme ist folglich die grüne Kurve in den Abbildungen \Abbref{fig:PhaseSeparation} und \Abbref{fig:PhaseSeparationChemPotential}.


\paragraph*{Koexistenzpunkt}

Die genaue Lage des Koexistenzpunktes lässt sich mithilfe der Bedingung $\mu(O)=\mu(D)$ bestimmen:
\begin{align*}
    \mu(O)-\mu(D) & = \int_D^Ov(P)\diff P \overset{!}{=}0                                       \\
                  & = \int_D^F v\diff P+\int_F^K v\diff P+\int_K^M v\diff P+\int_M^O v\diff P=0 \\
    \implication \int_D^F v\diff P-\int_K^F v\diff P=\int_M^K v\diff P-\int_M^O v\diff P.
\end{align*}
\begin{formal}
   \formalemph{Maxwell-Konstruktion:} Die Fläche unterhalb der grünen Gerade in \Abbref{fig:PhaseSeparation} zu der Isothermen von O bis K entspricht der Fläche oberhalb der grünen Geraden zur Isothermen von K bis D. 
\end{formal}



Die Bedingung $\mu(O)=\mu(D)$ ist auch sinnvoll, denn an dem Phasenkoexistenzpunkt werden Teilchen zwischen zwei Phase/Systemen ausgetauscht. Damit dies aber im thermodynamischen Gleichgewicht geschieht, muss das chemische Potential gleich sein.


\paragraph*{Diskussion des Phasenübergangs}

Betrachten wir nun wieder das Molvolumen (siehe \Abbref{fig:PhaseSeparation}). Für $P<P_D=P_O$ ist $v(P)$ groß und für $P>P_O=P_D$ klein. Es kommt zu einem Volumensprung am Koexistenzpunkt, denn
\begin{align*}
    v=\frac{1}{N}\left( \frac{\partial G}{\partial P} \right)_T =\left( \frac{\partial \mu}{\partial P} \right)_T
\end{align*}
ist dort unstetig. Wir sehen, dass ein Volumensprung einer Unstetigkeit der ersten Ableitung von $\mu=G/N$ entspricht. Aus den gewonnenen Erkenntnissen leiten wir folgende Bezeichungen ab:
\begin{itemize}
    \item Ein unstetiger Wechsel von Eigenschaften entspricht dem Übergang zwischen zwei Phasen
    \item Ein Phasenübergang erster Ordnung entspricht einer Unstetigkeit in der ersten Ableitung eines geeigneten thermodynamischen Potentials (z.~B. $G=\mu N$).
\end{itemize}

Die Zweige im $\mu$-$P$-Diagramm stellen Phasen im Phasendiagramm dar. Der Schnittpunkt der Zweige ist ein Phasenübergang.

Auch die isotherme Kompressibilität, welche proportional zu $|\partial v/\partial P|$ ist, ist bei $P<P_D$ groß und bei $P>P_O$ klein (besser in \Abbref{fig:VanDerWaalsIsotherm} zu sehen, die Kurven verlaufen links vom Koexistenzpunkt steil und rechts flach), was unseren Vorstellungen von einem Gas (hohe Kompressibilität) und einer Flüssigkeit (geringe Kompressibilität) entspricht.


Ab dem kritischen Punkt (bei Temperaturen $T>T_\mathrm{K}$ bzw. Drücken $P>P_\mathrm{k}$) liegt allerdings kein Volumensprung mehr vor (nur eine Lösung von Gleichung~\eqref{eq:van der waals isotherme}). Wird der Prozess so gewählt, dass der Übergang bei $T_\mathrm{k}$ stattfindet, so handelt es sich um einen Phasenübergang zweiter Ordnung, der im nächsten Kapitel behandelt wird. Es ist aber auch ein Weg denkbar, bei dem zuerst die Temperatur auf über $T_\mathrm{k}$ erhöht wird, dann das Volumen vergrößert oder verkleinert wird und die Temperatur anschließend wieder auf ihren Ursprungswert abgesenkt wird. In diesem Fall findet kein Phasenübergang statt, denn die Koexistenzlinie endet im kritischen Punkt.


\paragraph*{Entropiesprung}

Auch die Entropie muss bei einem Phasenübergang erster Ordnung einen Sprung aufweisen, da sie ebenfalls eine Ableitung des unstetigen chemischen Potentials ist: $s=-(\partial\mu/\partial T)_P$. Wir wollen diesen Entropiesprung berechnen:
\begin{align*}
    \Delta s = s_D-s_O = \int_D^O \left( \frac{\partial s}{\partial v} \right)_T\diff v = \int_{O\rightarrow D} \left( \frac{\partial P}{\partial T} \right)_T\diff v,
\end{align*}
wobei im ersten Schritt verwendet wurde, dass $\diff T=0$ und im zweiten Schritt die Maxwellbeziehungen angewandt wurden. Die Ableitung lässt sich herausziehen und als 
\begin{align*}
    \Delta s = \lim_{\Delta T\rightarrow 0} \frac{1}{\Delta T} \int_{O\rightarrow D}\left\{ P(T+\Delta T,v) - P(T,v) \right\}\diff v.
\end{align*}
schreiben. Wegen der Maxwellkonstruktion ist das Integral von $P$ entlang der realen Isotherme gleich dem entlang der van-der-Waals-Isotherme:
\begin{align*}
    \int_{O\rightarrow D} P\diff v = \int_{OMKFD} P\diff v.
\end{align*}
% und damit 
% \begin{align*}
%     \lim_{\Delta T\rightarrow 0} \frac{1}{\Delta T}.
% \end{align*}
Unter Zuhilfenahme der van-der-Waals-Zustandsgleichung \eqref{eq:van der waals 2} kann $\Delta s$ direkt berechnet werden:
\begin{align}
    \Delta s = \int_{v_O}^{v_D} \left(\frac{\partial P }{\partial T}\right)_v\diff v = \int_{v_O}^{v_D} \frac{R}{v-b}\diff v = R \ln\left( \frac{v_D-b}{v_O-b} \right) > 0. 
\end{align}
Wegen $\Delta s>0$ nimmt die Entropie beim Übergang von der flüssigen zur gasförmigen Phase sprunghaft zu. 


\paragraph*{Latente Wärme}

Bei dem Übergang von O nach D (flüssig nach gasförmig) muss Wärme aufgenommen werden. Dabei wird aber nicht die Temperatur erhöht, sondern der zuvor behandelte Entropiesprung erreicht. Dies wird als latente Wärme bezeichnet. Pro \unit{\mole} wird die Wärme
\begin{align}
    \Delta q = T\Delta s = T(s_D-s_O) = \Delta h
\end{align}
aufgenommen. Sie ist hier gleich der molaren Enthalpie $\Delta q = \Delta h = T\diff S+v\diff P$, da $P=\mathrm{const}$. 

\paragraph*{Innere Energie}
Die molare innere Energieänderung, welche beim Phasenübergang vorliegt entspricht:
\begin{align*}
    \Delta u=T\Delta s+P\Delta v.
\end{align*}

\paragraph*{Hebelgesetz}
%\begin{figure}[htbp]
%    \centering
%    \tfiguglyHebelgesetz
%    \caption{Schematische Isotherme mit gekennzeichnetem instabilen Zustand Y.}
%    \label{fig:uglyHebelgesetz}
%\end{figure}
Das Molvolumen der koexistierenden Phasen an einem beliebigen Punkt Y des instabilen Bereiches %(siehe Abb. \ref{fig:uglyHebelgesetz}) 
setzt sich wie folgt zusammen:
\begin{align*}
    v_Y=x_Ov_O+x_Dv_D
\end{align*}
mit den jeweiligen Molbrüchen der Phasen:
\begin{align*}
    x_O=\frac{N_O}{N_O+N_D}, \quad x_O=\frac{N_O}{N_O+N_D} \quad \mathrm{wobei} \quad N_O+N_D=1.
\end{align*}
\begin{formal}
  Das sogenannte \formalemph{Hebelgesetz} beschreibt die daraus folgende Gleichheit zwischen dem umgekehrt proportionalem Verhältnis der Molbruchphasenanteile und der Isothermenstrecken $|YD|$ und $|OY|$. Die Relation wird auch als Hebelgleichgewicht bezeichnet:
    \begin{align*}
        \boxed{\frac{x_O}{x_D}=\frac{v_D-v_Y}{v_Y-v_O}=\frac{|YD|}{|OY|}}\;.
    \end{align*}
\end{formal}


\paragraph*{Zusammenfassung}
Durch die Integration des chemischen Potentials entlang der Isothermen, haben wir die Phasenübergänge näher charakterisiert. Alternativ kann auch die freie Energie betrachtet werden:
\begin{align*}
    \diff f=-s\diff T-P\diff v.
\end{align*}
Die Isotherme führt auf:
\begin{align*}
    f(T=const,v)=-\int P\diff v+\phi_2(T).
\end{align*}
Die Extrema von $-P(v)$ sind in dieser Darstellung Wendepunkte der freien Energie, denn es gilt:
\begin{align*}
    \frac{\partial(-P)}{\partial v}=0=\frac{\partial^2f}{\partial v^2}.
\end{align*}
Der instabile und metastabile Bereich nimmt damit eine konkave Form an (siehe Abb.[Abb][[Ref]]), welche physikalisch durch die anliegende Sekante (ebenfalls Tangente der Isothermen) ersetzt wird und die freie Energie minimiert. Die mestastabilen Zustände sind dadurch gekennzeichnet, dass sie lokal stabil sind. Die Beschreibung über die freie Energie ist konsistent mit dem vorherigen Zugang, da sie auf dieselben Koexistenzpunkte wie das chemische Potential führt.
\paragraph*{Spinodale und Binodale}
Weitere grundlegende Begriffe bilden die Spinodale und Binodale. 
Sie sind in Abb. [Abb][Ref] abgebildet. 
Die \emph{Spinodale} beschreibt die Fläche, welche von der Kurve eingeschlossenen wird, welche durch alle Extrema der Isothermen im $P$-$v$-Diagramm verläuft. Damit beschreibt die Spinodale alle instabilen Bereiche.

Die \emph{Binodale} beschreibt die Fläche, welche von der Kurve eingeschlossen wird, die durch alle Koexistenzpunkte der Isothermen im $P$-$v$-Diagramm verläuft, abzüglich der Spinodalen. Diese Fläche umfasst also alle mestastabilen Bereiche.

\paragraph*{Metastabilität}
Metastabile Bereiche können durch vorsichtige Kompression (Übersättigung) eines Gases oder durch Expansion einer Flüssigkeit (entlang der Isothermen) erreicht werden. Dann führen Störungen in aller Regel dazu, dass Phasenübergänge in Form von Kondensation oder Verdampfung stattfinden.
Gewöhnlich bilden lokale Dichteschwankungen (oder allgemeiner: Inhomogenitäten)  Keime für Phasenübergänge. Diese können durch Druckvariationen im Medium verursacht werden. Derartige Keime können natürlich auch durch äußere Einflüsse induziert werden, beispielsweise durch Schall, welcher als Dichtewelle propagierend Dichteschwankungen im Medium verursacht. Ein weiteres Beispiel sind ionisierende Teilchen in der Nebelkammer, welche Kondensation induzieren.

Betrachten wir nicht nur Isotherme, so ist auch die Temperatur natürlich eine wesentliche Kontrollvariable der Phasenübergänge. Überhitzung kann jedoch unter Umständen zu Siedeverzug (oder Unterkühlung zum Kondensationsverzug) führen, falls keine phasenübergangsinduzierende Störungen auftreten.

\section{Thermodynamische Beschreibung des Phasenüberganges 1. Ordnung im einkomponentigen System}
Wir wollen die Phasenübergänge wieder etwas allgemeiner, unabhängig vom Kontext der Van-der-Waals-Gleichung betrachten.
\paragraph*{Modellvorstellung}
Wir betrachten ein System, welches Untersysteme verschiedener Phasen beinhaltet, welche wiederum an ein Volum- und Wärmereservoir angeschlossen sind. 
Die Phasen sind durch wärmeleitende, bewegliche und materiedurchlässige Flächen getrennt (beispielsweise einer Flüssigkeitsoberfläche).

\paragraph*{Zwei-Phasen-Koexistenz}
Nehmen wir an, es gäbe nur zwei Untersysteme, also zwei zu betrachtende Phasen.
Betrachten wir die freie Enthalpie des Gesamtsystems (da $T$ und $P$ die Kontrollgrößen des Systems sind, ist dies naheliegend), so setzt diese sich additiv aus der freien Enthalpie der einzelnen Teilsysteme zusammen:
\begin{align*}
    G(T,P,N_1,N_2)=G_1(T,P,N_1)+G_2(T,P,N_2)=N_1\mu_1(T,P)+N_2\mu_2(T,P).
\end{align*}
Da Druck und Temperatur dank der Reservoire konstant gehalten sind, nimmt die freie Enthalpie ein Minimum an, sodass ihre Ableitung gleich null ist und unter Zuhilfenahme der Relation $N_2=N-N_1$ gilt:
\begin{align*}
    \diff G=\frac{\partial G}{\partial N_1}\diff N_1+\frac{\partial G}{\partial N_2}\diff N_2=\left[\mu_1(T,P)-\mu_2(T,P)\right]\diff N_1=0.
\end{align*}
Daraus folgt die Gleichheit der chemischen Potentiale.
Die durch $\mu_1(T,P)=\mu_2(T,P)$ beschriebene Phasenkoexistenz entspricht dem Schnitt zweier (Hyper-)Flächen. Dieser Schnitt charakterisiert die Koexistenzlinie $P=P(T)$.
\paragraph*{Drei-Phasen-Koexistenz}
Analog können wir ein Modellsystem mit drei Untersystemen verschiedener Phasen betrachten, dann folgt:
\begin{align*}
    G(T,P,N_1,N_2,N_3)=N_1\mu_1(T,P)+N_2\mu_2(T,P)+N_3\mu_3(T,P)
\end{align*}
und ferner wieder mit $N_1+N_2+N_3=N=const$ und der Konstanz von Temperatur und Druck:
\begin{align*}
    \diff G=\mu_1(-\diff N_2-\diff N_3)+\mu_2\diff N_2+\mu_3\diff N_3=0.
\end{align*}
Auch hier folgt wieder die erwartete Gleichheit der chemischen Potentiale ${\mu_1(T,P)=\mu_2(T,P)=\mu_3(T,P)}$ als Schnitt dreier Flächen. Dieser charakterisiert entsprechend keine Koexistenzlinie, sondern einen Koexistenzpunkt, den bereits eingeführten Tripelpunkt. 
\paragraph*{Beispiel}
Der Tripelpunkt von Wasser liegt beispielsweise bei $273,16$ Kelvin und $0.006$ atm.

Im Allgemeinen gibt es keine Quadrupelpunkte für einkomponentige Systeme, da für diese (drei Gleichungen und zwei Unbekannte) keine Lösung bestimmt werden kann. 

\section{Phasendiagramme einkomponentiger Systeme}
Wir wollen zu der Van-der-Waals-Beschreibung zurückkehren und mit ihr die weiteren Phasenübergänge diskutieren.
Wieder dienen die Isothermen als Ausgangspunkt und das Vorgehen nach der Maxwellkonstruktion führt auf die Koexistenzpunkte $P_\mathrm{Koex}(T)$ der unterschiedlichen Phasen.
Es ergibt sich ein $P$-$v$-Diagramm der Form: [Abb].
Erkennbar sind die Isothermen und ihre Phasenübergangsbereiche. Der Tripelpunkt ist eingezeichnet und durch eine Isobare repräsentiert, welche einen Berührungspunkt mit jeder Phase hat. 
Eine weitere Darstellungsform bietet das klassische ($P$-$T$-) Phasendiagramm ([Abb][Ref]). Es stellt alle Koexistenzkurven in Abhängigkeit der Temperatur dar. Wir fassen die wichtigsten Begriffe zusammen, welche hier zur Beschreibung der Kurven notwendig sind:
\begin{itemize}
    \item Die \emph{Sublimation} stellt die Koexistenzdrücke des \emph{fest-gasförmigen} Phasenübergangs dar.
    \item Das \emph{Schmelzen}, welches durch die \emph{Gefrierdruckkurve} repräsentiert wird, beinhaltet alle Koexistenzdrücke des \emph{fest-flüssigen} Phasenübergangs.
    \item Das \emph{Sieden}, welches durch die \emph{Dampfdruckkurve} repräsentiert wird, beinhaltet alle Koexistenzdrücke des \emph{flüssig-gasförmigen} Phasenübergangs. Die Kurve endet im \emph{kritischen Punkt}, ab welchem die Dichte des Gases und der Flüssigkeit nicht mehr unterscheidbar sind.
    \item Alle drei Kurven treffen sich im \emph{Tripelpunkt}.
\end{itemize}
\paragraph*{Clausius-Clapeyron-Gleichung}
Gegeben sei nun eine Zwei-Phasen-Koexistenzkurve $P(T)$.
Entlang dieser gilt:
\begin{itemize}
    \item Es liegen Entropiesprünge $\Delta s=s_2-s_1$ vor, welche durch die latente Wärme (Sublimations-, Schmelz- oder Verdampfungswärme) induziert werden.
    \item Es liegen Volumensprünge $\Delta v=v_2-v_1$ vor.
    \item Es liegen allerdings keine Sprünge im chemischen Potential vor: $\Delta \mu=\mu_2-\mu_1=0$.
\end{itemize}
Für jede Phase $i$ gilt die Gibbs-Duhem-Gleichung der Form:
\begin{align*}
    \diff \mu_i=-s_i\diff T+v_i\diff P.
\end{align*}
Da entlang der Koexistenzkurve keine Änderung des chemischen Potentials erfolgt, gilt auch:
\begin{align*}
    \diff \Delta \mu=0=-\Delta s\diff T+\Delta v\diff P=\left(-\Delta s+\Delta v\frac{\diff P}{\diff T}\right)\diff T.
\end{align*}
Da beliebige Temperaturänderungen physikalisch erlaubt sind folgt ferner:
\begin{align*}
    \frac{\diff P}{\diff T}=\frac{\Delta s}{\Delta v}.
\end{align*}
Mit Einsetzen der Relation für die Wärmezufuhr erhalten wir schließlich die Clausius-Clapeyron-Gleichung:
\begin{align*}
    \boxed{\frac{\diff P}{\diff T}=\frac{\Delta q}{T\Delta v}=\frac{\Delta Q}{T\Delta V}}\;.
\end{align*}
Die Gleichung ermöglicht folglich mittels der Kenntnis der Wärmezufuhr und der Volumenänderung die Koexistenzkurve zu erschließen und gilt für alle Koexistenzkurven.
Im Allgemeinen nimmt beim fest-flüssig-Phasenübergang die Entropie und das Volumen zu, sodass auch die Koexistenzkurve eine positive Steigung hat. Ausnahme zu dieser Regel bildet Wasser, dessen Kristallstruktur im festen Zustand für ein größeres Volumen sorgt als im flüssigen Zustand (die höchste Dichte liegt bei $4°$ C). Die praktischen Folgen dieser Anomalie sind nicht unerheblich - zum einen führt dies dazu, dass Eis und damit ganze Eisberge schwimmen, dass Seen von oben nach unten gefrieren, dass Eis eine enorme Sprengkraft besitzt, usw.
\section{Phasenübergänge 1. Ordnung in Mischungen}
Natürlich können wir auch mehrkomponentige Systeme betrachten. Zum Beispiel können wir Salz oder Ammoniak in Wasser lösen oder Alkohol-Wasser-Mischungen herstellen, deren Gefrierpunkt niedriger ist als bei reinem Wasser.
Wir wollen derartige Systeme auch aus thermodynamischer Sicht modellisieren.
\paragraph*{Modellvorstellung}
Wieder betrachten wir ein aus mehreren Untersystemen bestehendes System, welches an Wärme- und Volumenreservoir gekoppelt ist. Betrachten wir $M$ Phasen im Gleichgewicht, so werden diese mit $M$ Untersystemen repräsentiert. Jedes Untersystem hat wiederum unterschiedliche Komponenten ($r$ Stück), sodass jede Phase $\alpha$ die Einstellkomponenten $N_1^\alpha,...,N_r^\alpha$ besitzt.

Die in einer Phase vorliegende Gesamtmolzahl entspricht natürlich der Summe über alle Einzelmolzahlen, $N^\alpha=\sum_{k=1}^{r}N_k^\alpha$. Die Gesamtmolzahl einer einzelnen Komponente im Gesamtsystem entspricht natürlich der Summe über alle Untersysteme: $N_k=\sum_{\alpha=1}^{M}N_k^\alpha$.
Ein einzelner Molbruch einer chemischen Komponente entspricht:
\begin{align*}
    x_k^\alpha=\frac{N_k^\alpha}{N^\alpha}\quad \mathrm{mit}\quad \sum_{k=1}^{r}x_k^\alpha=1.
\end{align*}
Liegen insgesamt keine chemischen Reaktionen vor, so sind die Gesamtmolzahlen konstant. Damit folgt für die einzelnen Molzahländerungen einer Komponente:
\begin{align*}
    \sum_{\alpha=1}^{M}N_k^\alpha=0.
\end{align*}

Da Temperatur und Druck die Kontrollvariablen des Systems sind, betrachten wir die freie Enthalpie:
\begin{align*}
    G=\sum_{\alpha=1}^{M}G^\alpha \quad \mathrm{mit} \quad G^\alpha(T,P,N_1^\alpha,...,N_r^\alpha)=\sum_{k=1}^{r}\mu_k^\alpha N_k^\alpha.
\end{align*}
Die letzte Gleichheit folgt aus der Euler-Gleichung. Es stellt sich die Frage, welche die relevanten Variablen sind.
Zwar gilt
\begin{align*}
    \mu_k^\alpha=\mu_k^\alpha(T,P,N_1^\alpha,...,N_r^\alpha),
\end{align*}
jedoch kann durch Rausziehen der Gesamtmolzahl $N^\alpha$ eine Darstellung in Abhängigkeit der Molbrüche erzeugt werden:
\begin{align*}
    \mu_k^\alpha=\mu_k^\alpha(T,P,x_1^\alpha,...,x_{r-1}^\alpha, N_r^\alpha).
\end{align*}
Nun kann die letzte Variable, $N_r^\alpha$, durch $1-\sum_{k=1}^{r-1}x_k^\alpha$ ersetzt werden.

Diese Darstellung ist damit nur noch von $(r-1)M+2$ \emph{inneren} intensiven Variablen abhängig und ermöglicht eine vollständige qualitative Beschreibung der Phasenübergänge.

Die quantitativ exakte Beschreibung erfordert $rM+2$ \emph{äußere} Variablen.
\paragraph*{Phasen-Gleichgewicht}
Im Gleichgewicht nimmt das System ein Minimum bezüglich der freien Enthalpie an ($\left(\diff G\right)_{T,P}=0$), also gilt:
\begin{align*}
    \left(\diff G\right)_{T,P}=\sum_{\alpha=1}^{M}\sum_{k=1}^{r}\mu_k^\alpha\diff N_k^\alpha=\sum_{k=1}^{r}\sum_{\alpha=2}^{M}\left(\mu_k^\alpha-\mu_k^1\right)\diff N_k^\alpha=0.
\end{align*} 
Die zweite Gleichheit folgt aus der Stoffmengenkonstanz des Gesamtsystems. Aus dieser Relation folgt letztlich wieder die Gleichheit der chemischen Potentiale einer Komponente:
\begin{align}
    \label{eq:chemPotGG}
    \boxed{\mu_k^1=\mu_k^2=...=\mu_k^M}\;.
\end{align}
Dies gilt jeweils für alle Komponenten $k=1,...,r$ und ist dadurch physikalisch begründbar, dass die Stoffe zwischen den Phasen übertreten können.

Wir wollen uns nun der Frage widmen, wie viele Freiheitsgrade $f$ im Phasen-Gleichgewicht vorliegen. Dazu müssen wir die Frage beantworten, wie viele Parameter notwendig sind, um die Phasen und ihre Phasenkoexistenz zu beschreiben.
\paragraph*{Gibbs'sche Phasenregel} Diese Motivation führt uns zur Betrachtung einkomponentiger und mehrkomponentiger Systeme unter diesem Gesichtspunkt:
\begin{itemize}
    \item \emph{Einkomponentige Systeme} haben nur eine Komponente ($r=1$) und damit zwei ($\left(r-1\right)M+2$) innere intensive Variablen, nämlich $P$ und $T$.
    Im Phasendiagramm lassen sich die Freiheitsgrade sehr leicht ablesen. Für alle Phasenbereiche sind die zwei Variablen unabhängig voneinander wählbar, es folgt der Freiheitsgrad $f=2$. Auf den Koexistenzlinien ($P(T)$) beschränkt sich der Freiheitsgrad auf eine Variable. Im Drei-Phasenkoexistenzpunkt ist der Freiheitsgrad natürlich: $f=0$.
    \item \emph{Mehrkomponentige Systeme} haben $r$ Komponenten und damit ganz allgemein $\left(r-1\right)M+2$ innere Variablen. Der Freiheitsgrad ergibt sich hier mittels der Verallgemeinerung der Beobachtung am einkomponentigen System als Differenz der Zahl der inneren Variablen und der vorliegenden Bestimmungsgleichungen $r\left(M-1\right)$: 
    \begin{align*}
        f=\left[\left(r-1\right)M+2\right]-\left[r\left(M-1\right)\right].
    \end{align*}
    Der zweite Term folgt daraus, dass pro betrachtete Komponente im System exakt $M-1$ Gleichungen für die chemischen Potentiale der Komponente in den $M$ unterschiedlichen Phasen vorliegen (Vergleich Gleichung \ref{eq:chemPotGG}).
\end{itemize}
Aus dieser Verallgemeinerung folgt die 
\begin{formal}
    \formalemph{Gibbs'sche Phasenregel:}
    \begin{align*}
        \boxed{f=r-M+2}\;.
    \end{align*}
    Sie beschreibt die Zahl der thermodynamischen Freiheitsgrade eines Systems im Phasen-Gleichgewicht und entspricht der Zahl der frei wählbaren intensiven Variablen. Es lässt sich daraus folgern, dass maximal $M\leq r+2$ Phasen koexistieren können. 
\end{formal}
Ein einfacher Weg sich die Phasenregel zu merken besteht in der Varanschaulichung, dass $r+2$ die intensiven Variablen $T,P,\mu_1,...,\mu_r$ beschreibt und die Gibbs-Duhem-Gleichung für jede Phase zu einer Reduktion um eine Variable beiträgt. 

\section{Phasendiagramme binärer Mischungen}
Wir betrachten nun ein System mit zwei Komponenten ($r=2$). Dabei kann es sich zum Beispiel um Metalllegierungen, Flüssigkristalle oder Flüssigkeitsgemische handeln.
\paragraph*{Ein- und Zweiphasengebiete} Wir üben zuerst die Anwendung der Phasenregel. Betrachten wir das System  
\begin{itemize}
    \item im Einphasenraum mit $M=1$, so folgt mit der Gibbs'schen Phasenregel ein Freiheitsgrad $f=3$. Die frei wählbaren Größen sind $T,P,x_1$. 
    \item Entlang der Phasenkoexistenzebene ($M=2$) existiert ein Freiheitsgrad $f=2$ mittels der Größen $T$ und $P$. Die dritte Variable $x_1$ ist hier durch die geltende Gleicheit der chemischen Potentiale ($\mu_i^1=\mu_i^2$) festgelegt. 
    \item Koexistieren drei Phasen eintlang einer Kurve ($M=3$), so reduziert sich der Freiheitsgrad wieder auf $f=1$ mit der freien Variable $T$.
    \item Der Quadrupelpunkt mit $M=4$ hat einen Freiheitsgrad $f=0$.
\end{itemize}
\paragraph*{Phasendiagramm}
Wir wollen uns nun zwei exemplarische Phasenübergänge genauer ansehen.

Im $T$-$x_1$-Diagramm [Abb][ref] ist der gas-flüssig-Übergang eines binären Gemisches bei konstantem Druck dargestellt. Wieder wird ein Koexistenzbereich charakterisiert, der die Phasenübergänge vorgibt. Betrachtet man einen Destillationsprozess, so beobachtet man, dass die Temperaturerhöhung bis zum Koexistenzbereich und der anschließende Phasenübergang in die gasförmige Phase dazu führen, dass die Stoffmengenkonzentration $x_1$ abnimmt. Auf diese Phasenübergangseigenschaft beruht der Destillationsprozess.

Im nächsten $T$-$x_1$-Diagramm ([abb][ref]) ist ein fest-flüssig-Übergang schematisch dargestellt. Wieder definieren die markierten Bereiche die Koexistenzbereiche verschiedener Phasen; diesmal gibt es auch einen Koexistenzbereich zweier fester Phasen unterschiedlicher Stoffmengenzusammensetzungen ($\alpha, \beta$). 

Wird beispielsweise der feste Zustand $\beta$ erhitzt und in den flüssigen Zustand überführt, so hat sowohl der flüssige Zustand, als auch das ursprünglich erhitzte feste Medium eine andere stoffliche Zusammensetzung, sodass der Verflüssigungsprozess entlang der Koexistenzkurven nach unten abrutscht - Ausnahme bildet der eutektische Punkt. Dieser verbindet alle drei Phasen über eine Isotherme und ist im eigentlichen Sinne eine Kurve (wenn die Abhängigkeit vom Druck mitberücksichtigt wird). Der fest-flüssig-Phasenübergang am eutektischen Punkt stellt sicher, dass die Zusammensetzungen der flüssigen und festen Phasen an diesem Punkt gleich bleiben und kontrollierbarer ist.


\begin{summary}
    Es gibt Phasenübergänge erster und zweiter Ordnung. Erstere werden zum Beispiel durch Unstetigkeiten in der Dichte oder dem Molvolumen oder durch latente Wärme gebildet.\\
    \formalemph{Phasenübergänge erster Ordnung in Einkomponentensystemen}
    In einem Van-der-Waals-Gas gilt die reduzierte, universelle Van-der-Waals-Zustandsgleichung:
    \begin{align*}
        \left(p+\frac{3}{\tilde{v}^2}\right)\left(3\tilde{v}-1\right)=8t
    \end{align*}
    mit den reduzierten Größen:
    \begin{align*}
        p=\frac{P}{P_k}, \qquad t=\frac{T}{T_k}, \qquad \tilde{v}=\frac{v}{v_k},
    \end{align*}
    wobei die Größen am kritischen Punkt mit $k$ denotiert sind.
    Bedingt durch die Unabhängigkeit der Zustandsgleichung von den Gasparametern $a$ und $b$, verhalten sich alle Gase beim flüssig-gasförmig-Übergang gleich. Dies wird als das \emph{Ähnlichkeitsgesetz} bezeichnet.
    Die Zustandsgleichung beschreibt verschiedene Isotherme. Jene für welche $T<T_k$ gilt, besitzen ein ausgezeichnetes Minimum und Maximum. Zwischen diesen ist die Stabilitätsbedingung,
    \begin{align*}
        \kappa_T=-\frac{1}{V}\left(\frac{\partial V}{\partial P}\right)_T>0 \quad \leftrightarrow \quad \left(\frac{\partial P}{\partial v}\right)_T<0,
    \end{align*}
    allerdings verletzt. Damit ist ein Teil der Isotherme unphysikalisch und das System instabil. Es erfolgt dort eine Phasentrennung, bei der die zwei Zustände (flüssig und gasförmig, welche links und rechts der Extrema vorliegen) in verschiedenen Verhältnissen entlang einer Isobaren mit unerschiedlichen Dichten und Molvolumina koexistieren.
    
    Betrachtet man das chemische Potential, welches die Isotherme charakterisiert und das thermische Gleichgewicht durch ihr Minimum bestimmt, so beobachtet man eine Unstetigkeit im Molvolumen $v=\left(\partial \mu/\partial P\right)_T$ dort, wo die stabile Phase des Systems jeweils endet.

    Für die Bereiche, in denen $v(P)$ nicht mehr eindeutig ist, gilt zusammenfassend:
    \begin{itemize}
        \item Der physikalische Zustand ist unstabil, sobald $\partial v/\partial P>0$ gilt.
        \item Das Minimum des chemischen Potentials charakterisiert immer den realisierten Zustand. 
        \item Nicht instabile Zustände, welche ein größeres chemisches Potential haben als stabile Zustände, nennt man metastabil.
        \item An dem Punkt, an dem ein Knick im chemischen Potential der Isothermen vorliegt, findet ein Zweigwechsel statt und zwei Phasen koexistieren mit unterschiedlichen Molvolumina.
    \end{itemize}
Dieser Koexistenzpunkt lässt sich mittels der Gleichheit der chemischen Potentiale in diesem Punkt bestimmen. Die Gleichheit ist auch physikalisch gedeutet naheliegend, da an diesem Punkt ein Gleichgewichtszustand zweier koexistenter Phasen vorliegt. Diese Bedingung führt auf die sogenannte Maxwellkonstruktion,
\begin{align*}
    \int_D^F v\diff P-\int_K^F v\diff P=\int_M^K v\diff P-\int_M^O v\diff P.
\end{align*}
Sie besagt, dass die durch die Isotherme und die Isobare des Überganges von stabilem zu metastabilem Bereich eingeschlossenen Flächen gleich groß sind.
Am Koexistenzpunkt findet nun bedingt durch die Unstetigkeit im Molvolumen auch ein Volumensprung statt.  
Verallgemeinert gilt:
\begin{itemize}
    \item Ein unstetiger Wechsel von Eigenschaften entspricht dem Übergang zwischen zwei Phasen.
    \item Ein Phasenübergang erster Ordnung entspricht einer Unstetigkeit in der ersten Ableitung eines
    geeigneten thermodynamischen Potentials (z. B. $G = \mu N$ ).
\end{itemize}
Auch die Kompressibilität erfährt aufgrund ihrer Proportionalität zur Ableitung des Molvolumens einen Sprung. Entsprechend unserer Intuition, ist sie für Gase groß und für Flüssigkeiten klein. 

Ab dem kritischen Punkt findet kein Volumensprung mehr statt. Findet ein Übergang bei der kritischen Temperatur $T_k$ statt, so ist dies ein Übergang zweiter Ordnung. 

Auch die Entropie, $s(T,v)=-\left(\partial \mu/\partial T\right)_P$, weist durch ihre Abhängigkeit vom Molvolumen einen Sprung auf. Dies führt zu einer sprunghaften Zunahme der Entropie beim Phasenübergang von flüssig zu gasförmig.

Als \emph{latente Wärme} wird die Wärme bezeichnet, welche aufgenommen wird und den Entropiesprung zu ermöglichen (diese Wärme erhöht die Temperatur nicht!). Pro Mol wird die Wärme
\begin{align*}
    \Delta q = T\Delta s = T(s_D-s_O) = \Delta h
\end{align*}
aufgenommen. Sie entspricht der molaren freien Enthalpie, da der Druck konst ist.
Die Energieänderung entspricht:
\begin{align*}
    \Delta u=T\Delta s+P\Delta v.
\end{align*}
\end{summary}