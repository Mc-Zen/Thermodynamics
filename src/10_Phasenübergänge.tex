% !TeX root = Theo_IV.tex

\chapter{Phasenübergänge\label{sec:phasenuebergaenge}}

Als Phasenübergang wird die Umwandlung einer Phase in eine andere bezeichnet. Ein typisches Beispiel ist der Wechsel zwischen Aggregatzuständen, z.~B. Verdampfung oder Kondensation. In \Abbref{fig:Phasendiagramme} sind zwei Phasendiagramme für die Phasen fest, flüssig und gasförmig dargestellt. Markiert sind jeweils der Tripelpunkt und der kritische Punkt. Kurven stellen Phasenkoexistenzlinien dar, bei der zwei Phasen (beispielsweise die feste und die flüssige) gleichzeitig vorhanden sind. Am Tripelpunkt treffen sich die Phasenkoexistenzlinien und alle drei Phasen liegen zugleich vor. Der kritische Punkt liegt am Ende der Koexistenzlinie von flüssiger und gasförmiger Phase. Hier geht der Dichteunterschied der beiden Phasen gegen null und Phasenübergänge zweiter Ordnung finden statt. Rechts wird das Phasendiagramm von Wasser gezeigt \textendash{} hier offenbart sich eine Anomalie: unter höheren Druck sinkt die Schmelztemperatur.

Phasenübergänge werden in zwei Kategorien eingeteilt: Phasenübergänge erster und zweiter Ordnung.

\begin{figure}[htbp]
    \centering
    \tfigPhaseDiagramNoAnomaly

    \tfigPhaseDiagramWithAnomaly
    \caption{Phasendiagramme ohne (oben) und mit Anomalie (unten). }
    \label{fig:Phasendiagramme}
\end{figure}




\section{Übergänge erster Ordnung in Einkomponentensystemen}

Beispiele für Phasenübergänge erster Ordnung sind Unstetigkeiten in der Dichte oder dem Molvolumen sowie latente Wärme.

Wir wollen zuerst eine exemplarische Diskussion anhand der Van-der-Waals-Zustandsgleichung machen.


\paragraph*{Van-der-Waals-Zustandsgleichung}

Wir haben das Van-der-Waals-Gas bereits in Kapitel~\ref{sec:das van der waals gas}  kennengelernt und die Zustandsgleichung
\begin{align}
    \label{eq:van der waals 1}
    \left( P+\frac{a}{v^2}  \right) (v-b) =RT \\
    \label{eq:van der waals 2}
    \equivalence P=\frac{NRT}{V-Nb}-\frac{N^2a}{V^2},
\end{align}
die es beschreibt. Für konstante Temperatur erhalten wir aus \eqref{eq:van der waals 1}
\begin{align}
    \label{eq:van der waals isotherme}
    v^3-\left( b+\frac{RT}{P} \right)v^2+\frac{a}{P}v-\frac{ab}{p} = 0.
\end{align}
Die Gleichung hat entweder drei Lösungen (für kleinere $T$) oder eine (siehe \Abbref{fig:VanDerWaalsIsotherm}).

Am kritischen Punkt $\{T_\mathrm{k},P_\mathrm{k},v_\mathrm{k}\}$ ist ${\partial P}/{\partial v} = 0$ und ${\partial^2 P}/{\partial v^2}=0$ und somit
\begin{align*}
    T_\mathrm{k}=\frac{8a}{27Rb}, \quad P_\mathrm{k}=\frac{a}{27b^2},\quad v_\mathrm{k}=3b.
\end{align*}
Diese Relationen verwenden wir, um die Van-der-Waals-Gleichung in reduzierten Größen $p=P/P_\mathrm{k}$, $t=T/T_\mathrm{k}$ und $\Tilde{v}=v/v_\mathrm{k}$ zu formulieren:
\begin{align}
    \label{eq:van der waals reduziert}
    \left( p+\frac{3}{\Tilde{v}^2} \right)(3\Tilde{v}-1) = 8t.
\end{align}
Dies ist eine universelle Gleichung, denn sie hängt nicht von den Gasparametern $a$ und $b$ und damit dem Gas selbst ab. Folglich verhält sich der flüssig-gasförmig-Übergang bei allen Gasen ähnlich, was auch als Ähnlichkeitsgesetz bekannt ist und empirisch semi-quantitativ bestätigt wurde.

\begin{figure}[htbp]
    \centering
    \tfigVanDerWaalsIsotherm
    \caption{Verschiedene Isotherme bei unterschiedlichen Temperaturen $T$ im $v/v_\mathrm{k}$-$p/p_\mathrm{k}$-Diagramm nach der (universellen) Van-der-Waals-Gleichung. }
    \label{fig:VanDerWaalsIsotherm}
\end{figure}


\paragraph*{Stabilität und Phasentrennung}

Wir betrachten nun einzelne Isotherme $T<T_\mathrm{k}$, also solche mit drei Nullstellen, wie beispielhaft in \Abbref{fig:PhaseSeparation} (links) dargestellt. Rechts ist der Zusammenhang von $p$ und $v$ invertiert. Man sieht sofort, dass $v(P)$ im Bereich von B bis R nicht eindeutig ist und es stellt sich die Frage, welcher Wert dort nun wirklich angenommen wird.

\begin{figure}[htbp]
    \centering
    \tfigPhaseSeparation
    \tfigPhaseSeparationInverted
    \caption{Links: Schematische Isotherme im $v$-$P$-Diagramm zur Darstellung eines Phasenübergangs unterhalb des kritischen Punktes. Rechts: Gleiche Isotherme im invertierten $P$-$v$-Diagramm.
        Einzelne Punkte auf der Isotherme sind mit Buchstaben benannt. Die Lösungen der Gleichung \eqref{eq:van der waals isotherme} entsprechen den Punkten O, K und D. Die physikalische Isotherme ist in grün gekennzeichnet, während instabile Bereiche in rot und metastabile in blau markiert sind. }
    \label{fig:PhaseSeparation}
\end{figure}

Auch ist bemerkenswert, dass im roten Bereich MLKJF die Stabilitätsbedingung
\begin{align*}
    \kappa_T=-\frac{1}{V}\frac{\partial V}{\partial P}_T>0 \equivalence \left( \frac{\partial P}{\partial v} \right)_T<0
\end{align*}
verletzt ist (positive Steigung). Ein Teil der Isotherme ist also unphysikalisch und das System nicht stabil. Hier findet die Phasentrennung in eine Phase mit kleinem Volumen (flüssige Phase) und großen Volumen (gasförmige Phase) statt.
%??

\paragraph*{Chemisches Potential}

Wir wollen als nächstes die Gibbs-Duhem-Beziehung \eqref{eq:gibbs_duhem}
\begin{align*}
    \diff\mu=-s\diff T+v\diff P=\frac{1}{N}\diff G
\end{align*}
anwenden, um das chemische Potential zu untersuchen. Integration entlang der Isotherme ($\diff T=0$) liefert
\begin{align*}
    \mu(T=\mathrm{konst},P) = \int v\diff P +\phi(T)
\end{align*}
mit unbekannter aber hier nicht relevanter Integrationskonstante $\phi(T)$. Der Verlauf von $\mu$ ist in \Abbref{fig:PhaseSeparationChemPotential} abgebildet.

\begin{figure}[htbp]
    \centering
    \tfigPhaseSeparationChemPotential
    \caption{Schematische Isotherme aus \Abbref{fig:PhaseSeparation} im $p$-$\mu$-Diagramm. Wieder ist in grün die physikalisch realisierte Isotherme markiert. Im Punkt D,O weist diese einen Knick auf, da hier beide Phasen koexistieren und auf den anderen Zweig gewechselt wird. }
    \label{fig:PhaseSeparationChemPotential}
\end{figure}

Das thermische Gleichgewicht wird durch das Minimum von $\mu=G/N$ bestimmt. Das sind dann die stabilen Zustände.  Damit gibt es einen Knick von $\mu(P)$ bei $\mu(D)=\mu(O)$ und auch eine Unstetigkeit in $v=(\partial\mu/\partial P)_T$. In dem fraglichen Bereich koexistieren zwei Phasen (hier Gas und Flüssigkeit) mit unterschiedlicher Dichte und Molvolumen.


\paragraph*{Phasenverhalten}

Koppeln wir nun das System an ein Wärme- und ein Volumenreservoir, halten die Temperatur $T$ konstant und erlauben $P$ sich quasistatisch zu verändern.

Im ersten Bereich $P<P_B$ ist das Molvolumen $v$ eindeutig. Im Bereich $P_B<P<P_D$ ist $v(C)>v(L)>v(N)$. Wir wissen, dass der Zustand L instabil ist, da $\partial v/\partial P>0$. Das Potential $\mu(C)$ ist kleiner als $\mu(N)$ und C ist im thermischen Gleichgewicht der realisierte Zustand. N ist zwar ebenfalls nicht instabil, aber entspricht nicht dem Minimum von $G$ und ist daher metastabil.

Für $P_D=P_K=P_O$ ist $\mu(D) = \mu(O)$. Hier wird der Zweig gewechselt und zwei Phasen koexistieren (jeweils mit $v(D)$ und $v(O)$). Die übrigen Bereiche werden analog zu den ersten beiden behandelt. Die physikalische Isotherme ist folglich die grüne Kurve in den Abbildungen \Abbref{fig:PhaseSeparation} und \Abbref{fig:PhaseSeparationChemPotential}.


\paragraph*{Koexistenzpunkt}

Die genaue Lage des Koexistenzpunktes lässt sich mithilfe der Bedingung $\mu(O)=\mu(D)$ bestimmen:
\begin{align*}
    \mu(O)-\mu(D) & = \int_D^Ov(P)\diff P \overset{!}{=}0                                       \\
                  & = \int_D^F v\diff P+\int_F^K v\diff P+\int_K^M v\diff P+\int_M^O v\diff P=0 \\
    \implication \int_D^F v\diff P-\int_K^F v\diff P=\int_M^K v\diff P-\int_M^O v\diff P.
\end{align*}
Die Fläche unterhalb der grünen Gerade in \Abbref{fig:PhaseSeparation} zu der Isothermen von O bis K entspricht also der Fläche oberhalb der grünen Geraden zur Isothermen von K bis D (sogenannte Maxwell-Konstruktion).


Die Bedingung $\mu(O)=\mu(D)$ ist auch sinnvoll, denn an dem Phasenkoexistenzpunkt werden Teilchen zwischen zwei Phase/Systemen ausgetauscht. Damit dies aber im thermodynamischen Gleichgewicht geschieht, muss das chemische Potential gleich sein.


\paragraph*{Diskussion des Phasenübergangs}

Betrachten wir nun wieder das Molvolumen (siehe \Abbref{fig:PhaseSeparation}). Für $P<P_D=P_O$ ist $v(P)$ groß und für $P>P_O=P_D$ klein. Es kommt zu einem Volumensprung am Koexistenzpunkt, denn
\begin{align*}
    v=\frac{1}{N}\left( \frac{\partial G}{\partial P} \right)_T =\left( \frac{\partial \mu}{\partial P} \right)_T
\end{align*}
ist dort unstetig. Wir sehen, dass ein Volumensprung einer Unstetigkeit der ersten Ableitung von $\mu=G/N$ entspricht. Aus den gewonnenen Erkenntnissen leiten wir folgende Bezeichungen ab:
\begin{itemize}
    \item Ein unstetiger Wechsel von Eigenschaften entspricht dem Übergang zwischen zwei Phasen
    \item Ein Phasenübergang erster Ordnung entspricht einer Unstetigkeit in der ersten Ableitung eines geeigneten thermodynamischen Potentials (z.~B. $G=\mu N$).
\end{itemize}

Die Zweige im $\mu$-$P$-Diagramm stellen Phasen im Phasendiagramm dar. Der Schnittpunkt der Zweige ist ein Phasenübergang.

Auch die isotherme Kompressibilität, welche proportional zu $|\partial v/\partial P|$ ist, ist bei $P<P_D$ groß und bei $P>P_O$ klein (besser in \Abbref{fig:VanDerWaalsIsotherm} zu sehen, die Kurven verlaufen links vom Koexistenzpunkt steil und rechts flach), was unseren Vorstellungen von einem Gas (hohe Kompressibilität) und einer Flüssigkeit (geringe Kompressibilität) entspricht.


Ab dem kritischen Punkt (bei Temperaturen $T>T_\mathrm{K}$ bzw. Drücken $P>P_\mathrm{k}$) liegt allerdings kein Volumensprung mehr vor (nur eine Lösung von Gleichung~\eqref{eq:van der waals isotherme}). Wird der Prozess so gewählt, dass der Übergang bei $T_\mathrm{k}$ stattfindet, so handelt es sich um einen Phasenübergang zweiter Ordnung, der im nächsten Kapitel behandelt wird. Es ist aber auch ein Weg denkbar, bei dem zuerst die Temperatur auf über $T_\mathrm{k}$ erhöht wird, dann das Volumen vergrößert oder verkleinert wird und die Temperatur anschließend wieder auf ihren Ursprungswert abgesenkt wird. In diesem Fall findet kein Phasenübergang statt, denn die Koexistenzlinie endet im kritischen Punkt.


\paragraph*{Entropiesprung}

Auch die Entropie muss bei einem Phasenübergang erster Ordnung einen Sprung aufweisen, da sie ebenfalls eine Ableitung des unstetigen chemischen Potentials ist: $s=-(\partial\mu/\partial T)_P$. Wir wollen diesen Entropiesprung berechnen:
\begin{align*}
    \Delta s = s_D-s_O = \int_D^O \left( \frac{\partial s}{\partial v} \right)_T\diff v = \int_{O\rightarrow D} \left( \frac{\partial P}{\partial T} \right)_T\diff v,
\end{align*}
wobei im ersten Schritt verwendet wurde, dass $\diff T=0$ und im zweiten Schritt die Maxwellbeziehungen angewandt wurden. Die Ableitung lässt sich herausziehen und als 
\begin{align*}
    \Delta s = \lim_{\Delta T\rightarrow 0} \frac{1}{\Delta T} \int_{O\rightarrow D}\left\{ P(T+\Delta T,v) - P(T,v) \right\}\diff v.
\end{align*}
schreiben. Wegen der Maxwellkonstruktion ist das Integral von $P$ entlang der realen Isotherme gleich dem entlang der van-der-Waals-Isotherme:
\begin{align*}
    \int_{O\rightarrow D} P\diff v = \int_{OMKFD} P\diff v.
\end{align*}
% und damit 
% \begin{align*}
%     \lim_{\Delta T\rightarrow 0} \frac{1}{\Delta T}.
% \end{align*}
Unter Zuhilfenahme der van-der-Waals-Zustandsgleichung \eqref{eq:van der waals 2} kann $\Delta s$ direkt berechnet werden:
\begin{align}
    \Delta s = \int_{v_O}^{v_D} \left(\frac{\partial P }{\partial T}\right)_v\diff v = \int_{v_O}^{v_D} \frac{R}{v-b}\diff v = R \ln\left( \frac{v_D-b}{v_O-b} \right) > 0. 
\end{align}
Wegen $\Delta s>0$ nimmt die Entropie beim Übergang von der flüssigen zur gasförmigen Phase sprunghaft zu. 


\paragraph*{Latente Wärme}

Bei dem Übergang von O nach D (flüssig nach gasförmig) muss Wärme aufgenommen werden. Dabei wird aber nicht die Temperatur erhöht, sondern der zuvor behandelte Entropiesprung erreicht. Dies wird als latente Wärme bezeichnet. Pro \unit{\mole} wird die Wärme
\begin{align}
    \Delta q = T\Delta s = T(s_D-s_O) = \Delta h
\end{align}
aufgenommen. Sie ist hier gleich der molaren Enthalpie $\Delta q = \Delta h = T\diff S+v\diff P$, da $P=\mathrm{const}$. 

\paragraph*{Innere Energie}
Die molare innere Energieänderung, welche beim Phasenübergang vorliegt entspricht:
\begin{align*}
    \Delta u=T\Delta s+P\Delta v.
\end{align*}

\begin{summary}
    Es gibt Phasenübergänge erster und zweiter Ordnung. Erstere werden zum Beispiel durch Unstetigkeiten in der Dichte oder dem Molvolumen oder durch latente Wärme gebildet.\\
    \formalemph{Phasenübergänge erster Ordnung in Einkomponentensystemen}
    In einem Van-der-Waals-Gas gilt die reduzierte, universelle Van-der-Waals-Zustandsgleichung:
    \begin{align*}
        \left(p+\frac{3}{\tilde{v}^2}\right)\left(3\tilde{v}-1\right)=8t
    \end{align*}
    mit den reduzierten Größen:
    \begin{align*}
        p=\frac{P}{P_k}, \qquad t=\frac{T}{T_k}, \qquad \tilde{v}=\frac{v}{v_k},
    \end{align*}
    wobei die Größen am kritischen Punkt mit $k$ denotiert sind.
    Bedingt durch die Unabhängigkeit der Zustandsgleichung von den Gasparametern $a$ und $b$, verhalten sich alle Gase beim flüssig-gasförmig-Übergang gleich. Dies wird als das \emph{Ähnlichkeitsgesetz} bezeichnet.
    Die Zustandsgleichung beschreibt verschiedene Isotherme. Jene für welche $T<T_k$ gilt, besitzen ein ausgezeichnetes Minimum und Maximum. Zwischen diesen ist die Stabilitätsbedingung,
    \begin{align*}
        \kappa_T=-\frac{1}{V}\left(\frac{\partial V}{\partial P}\right)_T>0 \quad \leftrightarrow \quad \left(\frac{\partial P}{\partial v}\right)_T<0,
    \end{align*}
    allerdings verletzt. Damit ist ein Teil der Isotherme unphysikalisch und das System instabil. Es erfolgt dort eine Phasentrennung, bei der die zwei Zustände (flüssig und gasförmig, welche links und rechts der Extrema vorliegen) in verschiedenen Verhältnissen entlang einer Isobaren mit unerschiedlichen Dichten und Molvolumina koexistieren.
    
    Betrachtet man das chemische Potential, welches die Isotherme charakterisiert und das thermische Gleichgewicht durch ihr Minimum bestimmt, so beobachtet man eine Unstetigkeit im Molvolumen $v=\left(\partial \mu/\partial P\right)_T$ dort, wo die stabile Phase des Systems jeweils endet.

    Für die Bereiche, in denen $v(P)$ nicht mehr eindeutig ist, gilt zusammenfassend:
    \begin{itemize}
        \item Der physikalische Zustand ist unstabil, sobald $\partial v/\partial P>0$ gilt.
        \item Das Minimum des chemischen Potentials charakterisiert immer den realisierten Zustand. 
        \item Nicht instabile Zustände, welche ein größeres chemisches Potential haben als stabile Zustände, nennt man metastabil.
        \item An dem Punkt, an dem ein Knick im chemischen Potential der Isothermen vorliegt, findet ein Zweigwechsel statt und zwei Phasen koexistieren mit unterschiedlichen Molvolumina.
    \end{itemize}
Dieser Koexistenzpunkt lässt sich mittels der Gleichheit der chemischen Potentiale in diesem Punkt bestimmen. Die Gleichheit ist auch physikalisch gedeutet naheliegend, da an diesem Punkt ein Gleichgewichtszustand zweier koexistenter Phasen vorliegt. Diese Bedingung führt auf die sogenannte Maxwellkonstruktion,
\begin{align*}
    \int_D^F v\diff P-\int_K^F v\diff P=\int_M^K v\diff P-\int_M^O v\diff P.
\end{align*}
Sie besagt, dass die durch die Isotherme und die Isobare des Überganges von stabilem zu metastabilem Bereich eingeschlossenen Flächen gleich groß sind.
Am Koexistenzpunkt findet nun bedingt durch die Unstetigkeit im Molvolumen auch ein Volumensprung statt.  
Verallgemeinert gilt:
\begin{itemize}
    \item Ein unstetiger Wechsel von Eigenschaften entspricht dem Übergang zwischen zwei Phasen.
    \item Ein Phasenübergang erster Ordnung entspricht einer Unstetigkeit in der ersten Ableitung eines
    geeigneten thermodynamischen Potentials (z. B. $G = \mu N$ ).
\end{itemize}
Auch die Kompressibilität erfährt aufgrund ihrer Proportionalität zur Ableitung des Molvolumens einen Sprung. Entsprechend unserer Intuition, ist sie für Gase groß und für Flüssigkeiten klein. 

Ab dem kritischen Punkt findet kein Volumensprung mehr statt. Findet ein Übergang bei der kritischen Temperatur $T_k$ statt, so ist dies ein Übergang zweiter Ordnung. 

Auch die Entropie, $s(T,v)=-\left(\partial \mu/\partial T\right)_P$, weist durch ihre Abhängigkeit vom Molvolumen einen Sprung auf. Dies führt zu einer sprunghaften Zunahme der Entropie beim Phasenübergang von flüssig zu gasförmig.

Als \emph{latente Wärme} wird die Wärme bezeichnet, welche aufgenommen wird und den Entropiesprung zu ermöglichen (diese Wärme erhöht die Temperatur nicht!). Pro Mol wird die Wärme
\begin{align*}
    \Delta q = T\Delta s = T(s_D-s_O) = \Delta h
\end{align*}
aufgenommen. Sie entspricht der molaren freien Enthalpie, da der Druck konst ist.
Die Energieänderung entspricht:
\begin{align*}
    \Delta u=T\Delta s+P\Delta v.
\end{align*}
\end{summary}